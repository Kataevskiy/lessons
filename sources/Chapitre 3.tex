\documentclass[10pt,a4paper]{article}
\usepackage[utf8]{inputenc}
\usepackage[french]{babel}
\usepackage[T1]{fontenc}
\usepackage{amsmath}
\usepackage{amsfonts}
\usepackage{amssymb}
\usepackage{graphicx}
\usepackage[left=2cm,right=2cm,top=2cm,bottom=2cm]{geometry}
\usepackage{setspace}
\usepackage{ulem}
\usepackage{stmaryrd}
\usepackage{amsthm}
\usepackage{dsfont}
\usepackage{mathpazo}

\onehalfspacing

\theoremstyle{definition}
\newtheorem{proposition}{Proposition}[section]
\newtheorem{theorem}[proposition]{Théorème}
\newtheorem{corollaire}[proposition]{Corollaire}
\newtheorem{lemme}[proposition]{Lemme}
\newtheorem{definition}[proposition]{Définition}

\DeclareMathOperator{\re}{Re}
\DeclareMathOperator{\im}{Im}
\DeclareMathOperator{\sgn}{sgn}

\begin{document}
\renewcommand{\labelitemi}{$*$}
\renewcommand{\labelenumi}{(\roman{enumi})}
\begin{center}
{\Large \textbf{Chapitre 3: Nombres complexes}}
\end{center}

\section{Généralités et rappels}
\subsection{Définition}
\begin{definition}
On définit (provisoirement) l'ensemble des nombres complexes comme
\[ \left\{ \begin{pmatrix}
a & -b \\
b & a
\end{pmatrix} \,\middle|\, a, b \in \mathbb{R} \right\}\]
On identifie tout réel à un "nombre complexe" $aI_2 = \begin{pmatrix}
a & 0 \\
0 & a
\end{pmatrix}$ et on définit $i = \begin{pmatrix}
0 & -1 \\
1 & 0
\end{pmatrix}$ de telle sorte que tout nombre complexe s'écrit de manière unique sous la forme $a + ib$, pour un couple $(a, b) \in \mathbb{R}^2$
\end{definition}
\begin{proposition}
Soit $z_1, z_2, z_3 \in \mathbb{C}$
\begin{itemize}
\item L'addition est \uline{commutative}: $z_1 + z_2 = z_2 + z_1$
\item L'addition est \uline{associative}: $z_1 + (z_2 + z_3) = (z_1 + z_2) + z_3$
\item La multiplication est commutative: $z_1 z_2 = z_2 z_1$
\item La multiplication est associative: $z_1 (z_2 z_3) = (z_1 z_2) z_3$
\item La multiplication \uline{distribue sur l'addition}: $z_1 (z_2 + z_3) = z_1 z_2 + z_1 z_3$
\end{itemize}
\end{proposition}

\subsection{Conjugaison}
\begin{definition}
Soit $z = a + ib$ un nombre complexe sous forme algébrique. \\
On définit:
\begin{itemize}
\item Son \uline{conjugué}: $\overline{z} = a - ib \in \mathbb{C}$
\item Sa \uline{partie réelle}: $\re(z) = a \in \mathbb{R}$
\item Sa \uline{partie imaginaire}: $\im(z) = b \in \mathbb{R}$
\end{itemize}
\end{definition}
\begin{proposition}
Soit $z_1, z_2 \in \mathbb{C}$ et $n \in \mathbb{N}$ \\
On a:
\begin{itemize}
\item $\overline{z_1 + z_2} = \overline{z_1} + \overline{z_2}$
\item $\overline{z_1 z_2} = \overline{z_1} \times \overline{z_2}$
\item $\overline{\overline{z}} = z$
\item $\overline{z^n} = \overline{z}^n$
\item On a $z_1 \in \mathbb{R} \iff z_1 = \overline{z_1}$
\item On a
\[\re(z_1) = \frac{z_1 + \overline{z_1}}{2} \quad \text{ et } \quad \im(z_1) = \frac{z_1 - \overline{z_1}}{2i} \]
\item On a
\[\begin{cases}
\re(z_1 + z_2) = \re(z_1) + \re(z_2) \\
\forall t \in \mathbb{R},\, \re(t z_1) = t \re(z_1)
\end{cases}\]
\item Et
\[\begin{cases}
\im(z_1 + z_2) = \im(z_1) + \im(z_2) \\
\forall t \in \mathbb{R},\, \im(t z_1) = t \im(z_1)
\end{cases}\] ($\mathbb{R}$-linéarité de $\re$ et $\im$)
\end{itemize}
\end{proposition}

\pagebreak

\begin{proposition}
Soit $z \in \mathbb{C}$ \\
LASSÉ (Les assertions suivantes sont équivalentes):
\begin{enumerate}
\item $\exists b \in \mathbb{R}: z = ib$
\item $\overline{z} = -z$
\item $\re(z) = 0$
\end{enumerate}
Quand elles sont vraies, on dit que $z$ est \uline{imaginaire pur}.
\end{proposition}

\subsection{Module}
\begin{proposition}
Soit $z \in \mathbb{C}$ \\
Alors $z \overline{z} \in \mathbb{R}_+$ et on a $z \overline{z} = 0$ si et seulement si $z = 0$
\end{proposition}
\begin{corollaire}
Tout nombre complexe non nul a un inverse: \\
$\forall z \in \mathbb{C}^*$, $\exists w \in \mathbb{C}: zw = 0$
\end{corollaire}
\begin{corollaire}
On a la règle du produit nul: \\
$\forall z_1, z_2 \in \mathbb{C}$, $z_1 z_2 = 0 \iff (z_1 = 0 \text{ ou } z_2 = 0)$ \\
On dit aussi que $\mathbb{C}$ est \uline{intègre}.
\end{corollaire}
\begin{proposition}
Soit $z_1 \in \mathbb{C}$ et $z_2 \in \mathbb{C}^*$ \\
Alors le conjugué de $\frac{z_1}{z_2}$ est $\frac{\overline{z_1}}{\overline{z_2}}$
\end{proposition}
\begin{definition}
Soit $z \in \mathbb{C}$ \\
Le module de $z$ est $\left| z \right| = \sqrt{z \overline{z}}$
\end{definition}
\begin{definition}
On note $\mathbb{U} = \left\{ z \in \mathbb{C} \mid \left|z\right| = 1 \right\}$ le \uline{cercle unité} (ou \uline{trigonométrique})
\end{definition}
\begin{proposition}
Soit $z_1 \in \mathbb{C}$ et $z_2 \in \mathbb{C}$ \\
On a $\left| z_1 z_2 \right| = |z_1| \cdot |z_2|$ et si $z_2 \neq 0$, $\left|\frac{z_1}{z_2}\right| = \frac{|z_1|}{|z_2|}$
\end{proposition}

\subsection{Inégalité triangulaire}
\begin{proposition}
Soit $z, w \in \mathbb{C}$ \\
LASSÉ:
\begin{enumerate}
\item $\overline{z} w \in \mathbb{R}_+$
\item $z = 0$ ou $(z \neq 0 \text{ et } \frac{w}{z} \in \mathbb{R}_+)$
\item $\exists u \in \mathbb{C}$, $\exists \lambda, \mu \in \mathbb{R}_+: \begin{cases}
z = \lambda u \\
w = \mu u
\end{cases}$
\end{enumerate}
Si ces assertions sont vraies, on dit que $z$ et $w$ sont \uline{positivement colinéaires}.
\end{proposition}
\noindent \uline{Remarque}: Si $z = a + ib$ et $w = c + id$, $\re(\overline{z}w) = ac + bd$ est le produit scalaire de $\begin{pmatrix}
a \\
b
\end{pmatrix}$ et $\begin{pmatrix}
c \\
d
\end{pmatrix}$
\begin{theorem}
Soit $z, w \in \mathbb{C}$
\begin{itemize}
\item On a $\re(z) \leq |z|$ \\
Il y a égalité si et seulement si $z \in \mathbb{R}_+$
\item \uline{Inégalité de Cauchy-Schwarz}: On a $\re(\overline{z} w) \leq |z| \cdot |w|$ \\
"Le produit scalaire est inférieure au produit des normes".
\item \uline{Inégalité triangulaire}: $\left| z + w \right| \leq |z| + |w|$ \\
Dans les deux derniers cas, il y a égalité si et seulement si $z$ et $w$ sont positivement colinéaires.
\end{itemize}
\end{theorem}
\begin{corollaire}
Pour tous $z_1, ...\,, z_n \in \mathbb{C}$, on a:
\[\left|z_1 + ... + z_n\right| \leq |z_1| + ... + |z_n|\]
\end{corollaire}
\begin{corollaire}[Inégalité triangulaire "à l'envers"]
Pour tous $z, w \in \mathbb{C}$, on a:
\[ \left|z + w\right| \geq |z| - |w| \]
Encore mieux: $\left| z + w \right| \geq \left| |z| - |w| \right|$
\end{corollaire}

\subsection{Distance}
\noindent Si $A$ et $B$ sont deux points d'affixes $z_A$ et $z_B$, la distance entre $A$ et $B$ est $d(A, B) = \lVert \vec{AB} \rVert = \left| z_B - z_A \right|$ \\
On la notera aussi $d(z_A, z_B)$
\begin{proposition}[Inégalité triangulaire]
Soit $z_1, z_2, z_3 \in \mathbb{C}$ \\
Alors $d(z_1, z_3) \leq d(z_1, z_2) + d(z_2, z_3)$
\end{proposition}
\begin{definition}
Soit $z \in \mathbb{C}$ est $r \in \mathbb{R}_+^*$. \\
On définit:
\begin{itemize}
\item Le \uline{cercle} de centre $z$ et de rayon $r$: \\
$\Gamma(z, r) = \left\{ w \in \mathbb{C} \mid \left| z - w \right| = r \right\} =\left\{ w \in \mathbb{C} \mid \left| z - w \right|^2 = r^2 \right\}$
\item Le \uline{disque} (fermé) de centre $z$ et de rayon $r$: \\
$\Delta(z, r) = \left\{ w \in \mathbb{C} \mid \left| z - w \right| \leq r \right\} = \left\{ w \in \mathbb{C} \mid \left| z - w \right|^2 \leq r^2 \right\}$
\end{itemize}
\end{definition}

\section{Équation du second degré}
\subsection{Racines carrés d'un nombre complexe: résultat théorique}
\begin{theorem}[provisoirement admis]
Soit $\Delta \in \mathbb{C}^*$ \\
Alors il existe deux \uline{racines carrés} de $\Delta$, càd deux nombres complexes dont le carré vaut $\Delta$. \\
Ces deux carrés sont opposés l'une de l'autre.
\end{theorem}

\subsection{Racines carrés d'un nombre complexe: calcul en forme algébrique}
\begin{definition}
On définit la fonction signe
\[ \sgn: \begin{cases}
\mathbb{R} \to \mathbb{R} \\
x \mapsto \begin{cases}
1 \text{ si } x > 0 \\
0 \text{ si } x = 0 \\
-1 \text{ si } x < 0
\end{cases}
\end{cases} \]
\end{definition}

\pagebreak

\begin{theorem}
Soit $\Delta \in \mathbb{C}$ et $z = x + iy$ un nombre complexe sous forme algébrique. \\
On a alors:
\begin{align*}
z^2 = \Delta &\iff \begin{cases}
\re(z^2) = \re(\Delta) \\
|z^2| = |\Delta| \\
\sgn(\im(z^2)) = \sgn(\im(\Delta))
\end{cases} \\
&\iff \begin{cases}
x^2 - y^2 = \re(\Delta) \\
x^2 + y^2 = |\Delta| \\
\sgn(xy) = \sgn(\im(\Delta))
\end{cases}
\end{align*}
\end{theorem}

\subsection{Résolution de l'équation générale}
\begin{theorem}
Soit $a \in \mathbb{C}^*$, $b, c \in \mathbb{C}$ \\
On considère l'équation $az^2 + bz + c = 0$ (E) \\
Soit $\Delta = b^2 - 4ac$ le \uline{discriminant} de (E)
\begin{itemize}
\item \uline{Si $\Delta = 0$}: (E) a une unique solution:
\[-\frac{b}{2a}\]
\item \uline{Si $\Delta \neq 0$}: (E) a deux solutions:
\[ \frac{-b -\delta}{2a} \quad \text{ et } \quad \frac{-b +\delta}{2a} \]
où $\delta$ est \uline{une} racine de $\Delta$
\end{itemize}
\end{theorem}

\subsection{Relation coefficient-racines}
\begin{theorem}
Soit $a \in \mathbb{C}^*$, $b, c \in \mathbb{C}$. On note $z_1$ et $z_2$ les solutions de $ax^2 + bx + c = 0$ \\
(s'il n'y en a qu'une, notée $\zeta$, on pose $z_1 = z_2 = \zeta$)
\begin{itemize}
\item On a $\forall z \in \mathbb{C}$
\[az^2 + bz + c = a(z - z_1)(z - z_2)\]
\item Relation coefficients-racines (formules de Viète)
\[z_1 + z_2 = -\frac{b}{a} \quad \text{ et } \quad z_1 z_2 = \frac{c}{a}\]
\end{itemize}
\end{theorem}

\subsection{Système somme-produit}
\begin{theorem}
Soit $S, P \in \mathbb{C}$ \\
On considère le système $(\Sigma): \begin{cases}
x + y = S \\
xy = P
\end{cases}$ d'inconnue $(x, y) \in \mathbb{C}^2$ \\
On considère \uline{l'équation associée} $z^2 - Sz + P = 0$ (ÉA), d'inconnue $z \in \mathbb{C}$ \\
On note $z_1$ et $z_2$ les solutions de (ÉA) (en posant $z_1 = z_2 = \zeta$ s'il n'y en a qu'une). \\
Alors les solutions de $(\Sigma)$ sont $(z_1, z_2)$ et $(z_2, z_1)$
\end{theorem}

\section{Exponentielle complexe}
\subsection{Préliminaires: congruence modulo $T$}
\begin{definition}
Soit $T \in \mathbb{C}^*$ \\
On dit que $z_1$ et $z_2 \in \mathbb{C}$ sont \uline{congrus modulo $T$} si $\exists k \in \mathbb{Z}: z_2 - z_1 = kT$ \\
Si c'est la cas, on note $z_1 \equiv z_2 \quad (\text{mod }T)$
\end{definition}
\begin{proposition}
Soit $T \in \mathbb{C}^*$ \\
La congruence modulo $T$ est une \uline{relation d'équivalence}.
\begin{itemize}
\item Elle est \uline{réflexive}: $\forall z \in \mathbb{C}$, $z \equiv z \quad (\text{mod }T)$
\item Elle est \uline{symétrique}: $\forall z, z' \in \mathbb{C}$, $z \equiv z' (\text{mod }T) \implies z' \equiv z (\text{mod }T)$
\item Elle est \uline{transitive}: $\forall z, z', z'' \in \mathbb{C}$, $\left(z \equiv z' (\text{mod }T) \text{ et } z' \equiv z'' (\text{mod }T)\right) \implies z \equiv z'' (\text{mod }T)$
\end{itemize}
\end{proposition}
\begin{proposition}
Soit $T \in \mathbb{C}^*$
\begin{itemize}
\item Soit $z_1, z_2, z_3 \in \mathbb{C}$ \\
Si $\begin{cases}
z_1 \equiv z_2 (\text{mod }T) \\
z_3 \equiv z_4 (\text{mod }T)
\end{cases}$ on a $z_1 + z_3 \equiv z_2 + z_4 (\text{mod }T)$
\item Soit $z_1, z_2 \in \mathbb{C}$ et $\lambda \in \mathbb{C}^*$ \\
Si $z_1 \equiv z_2 (\text{mod }T)$ alors $\lambda z_1 \equiv \lambda z_2 (\text{mod }\lambda T)$
\end{itemize}
\end{proposition}

\subsection{Exponentielle complexe}
\begin{theorem}[provisoirement admis]
Il existe une fonction $\exp: \mathbb{C} \to \mathbb{C}$ vérifiant:
\begin{enumerate}
\item $\exp(0) = 1$
\item Propriété fondamentale: $\forall x, y \in \mathbb{C}$, $\exp(x + y) = \exp(x) \exp(y)$
\item On a $\forall z \in \mathbb{C}$, $\exp(\overline{z}) = \overline{\exp(z)}$
\item $\forall z \in \mathbb{C}$, $\frac{\exp(tz) - 1}{t} \xrightarrow[t \to 0]{} z$
\item Tout $w \in \mathbb{C}^*$ s'écrit $w = \exp(z)$ pour un certain $z \in \mathbb{C}$
\item Pour tous $z, z' \in \mathbb{C}$, on a $\exp(z) = \exp(z') \iff z \equiv z' (\text{mod }2i\pi)$
\end{enumerate}
\end{theorem}
\noindent \uline{Remarque}: À la fin de l'année, on définira l'exponentielle de $z$ comme
\[ \exp(z) = \lim_{n \to +\infty} \sum_{k = 0}^n \frac{z^k}{k!} \]
\begin{definition}
On définit les fonctions
\begin{align*}
&\cos : \begin{cases}
\mathbb{R} \to \mathbb{R} \\
\theta \mapsto \re(e^{i\theta})
\end{cases}
&\sin : \begin{cases}
\mathbb{R} \to \mathbb{R} \\
\theta \mapsto \im(e^{i\theta})
\end{cases}
\end{align*}
\end{definition}
\begin{proposition}
Soit $z \in \mathbb{C}$ \\
On a: $|e^z| = e^{\re(z)}$
\end{proposition}

\subsection{$\mathbb{U}$ et exponentielle}
\begin{theorem}
\hfill
\begin{itemize}
\item Pour tout $\theta \in \mathbb{R}$, on a $e^{i\theta} \in \mathbb{U}$
\item Pour tout $z \in \mathbb{U}$, on peut trouver $\theta \in \mathbb{R}$ tel que $z = e^{i\theta}$
\item Pour tout $\theta, \theta' \in \mathbb{R}$, on a $e^{i\theta} = e^{i\theta'} \iff \theta \equiv \theta' \quad (\text{mod }2\pi)$
\end{itemize}
\end{theorem}
\begin{theorem}[Formules d'Euler]
Soit $\theta \in \mathbb{R}$ \\
On a
\[ \cos(\theta) = \frac{e^{i\theta} + e^{-i\theta}}{2} \quad \text{ et } \quad \sin(\theta) = \frac{e^{i\theta} - e^{-i\theta}}{2i} \]
\end{theorem}
\begin{proposition}[Formules de De Moivre]
Soit $\theta \in \mathbb{R}$ et $n \in \mathbb{Z}$ \\
On a 
\[ (\cos(\theta) + i\sin(\theta))^n = \cos(n\theta) + i\sin(n\theta) \]
\end{proposition}
\begin{proposition}[Factorisation par l'arc moitié]
Soit $\alpha, \beta \in \mathbb{R}$ \\
On a
\[ e^{i\alpha} + e^{i\beta} = e^{i \frac{\alpha + \beta}{2}} \left( e^{i \frac{\alpha - \beta}{2}} + e^{i \frac{\beta - \alpha}{2}} \right) = 2\cos\left(\frac{\alpha - \beta}{2}\right) e^{i \frac{\alpha + \beta}{2}} \]
Et 
\[ e^{i\alpha} - e^{i\beta} = e^{i \frac{\alpha + \beta}{2}} \left( e^{i \frac{\alpha - \beta}{2}} - e^{i \frac{\beta - \alpha}{2}} \right) = 2i\sin\left(\frac{\alpha - \beta}{2}\right) e^{i \frac{\alpha + \beta}{2}} \]
\end{proposition}

\subsection{Arguments d'un nombre complexe}
\begin{proposition}
\hfill 
\begin{itemize}
\item Soit $z \in \mathbb{C}$ \\
On peut trouver $\theta \in \mathbb{R}$ tel que $z = |z|e^{i\theta}$
\item Pour tous $r, r' \in \mathbb{R}_+^*$ et tous $\theta, \theta' \in \mathbb{R}$, on a $re^{i\theta} = r'e^{i\theta'} \iff \begin{cases}
r = r' \\
\theta \equiv \theta' (\text{mod }2\pi)
\end{cases}$
\end{itemize}
\end{proposition}
\begin{definition}
Soit $z \in \mathbb{C}^*$
\begin{itemize}
\item Tout nombre $\theta \in \mathbb{R}$ tel que $z = |z|e^{i\theta}$ est \uline{un} argument de $z$
\item On appelle \uline{argument principal} de $z$ et on note $\arg(z)$ l'unique argument de $z$ qui appartient à $\left] -\pi, \pi \right[$
\item On appelle \uline{forme exponentielle} de $z$ toute écriture de la forme $z = |z|e^{i\theta}$, où $\theta$ est un argument de $z$
\end{itemize}
\end{definition}
\begin{proposition}
Soit $z_1, z_2 \in \mathbb{C}^*$ \\
On a:
\begin{itemize}
\item $\arg(z_1 z_2) \equiv \arg(z_1) + \arg(z_2) \quad (\text{mod }2\pi)$
\item $\forall n \in \mathbb{Z}$, $\arg(z_1^n) \equiv n \arg(z_1) \quad (\text{mod }2\pi)$
\item $\arg\left(\frac{z_1}{z_2}\right) \equiv \arg(z_1) - \arg(z_2) \quad (\text{mod }2\pi)$
\end{itemize}
\end{proposition}

\section{Compléments de trigonométrie}
\subsection{Valeurs}
\begin{proposition}
\hfill
\begin{itemize}
\item Soit $\theta \in \mathbb{R}$ \\
On a
\[\cos^2(\theta) + \sin^2(\theta) = 1\]
\item Soit $a, b \in \mathbb{R}$ tels que $a^2 + b^2 = 1$ \\
Alors on peut trouver $\theta \in \mathbb{R}$ tel que $\begin{cases}
a = \cos(\theta) \\
b = \sin(\theta)
\end{cases}$
\end{itemize}
\end{proposition}

\subsection{Périodicité et (im)parité}
\begin{proposition}
\hfill
\begin{itemize}
\item $\cos$ est $2\pi$-périodique et paire.
\item $\sin$ est $2\pi$-périodique et impaire.
\end{itemize}
\end{proposition}
\begin{proposition}
Soit $\theta_1, \theta_2 \in \mathbb{R}$ tels que $\begin{cases}
\cos(\theta_1) = \cos(\theta_2) \\
\sin(\theta_1) = \sin(\theta_2)
\end{cases}$ \\
Alors $\theta_1 \equiv \theta_2 \quad (\text{mod }2\pi)$
\end{proposition}

\subsection{Formules d'addition}
\begin{proposition}
Soit $\alpha, \beta \in \mathbb{R}$ \\
On a:
\begin{align*}
\cos(\alpha + \beta) &= \cos\alpha \cos\beta - \sin\alpha \sin\beta \\
\sin(\alpha + \beta) &= \cos\alpha \sin\beta + \sin\alpha \cos\beta \\
\cos(\alpha - \beta) &= \cos\alpha \cos\beta + \sin\alpha \sin\beta \\
\sin(\alpha - \beta) &= -\cos\alpha \sin\beta + \sin\alpha \cos\beta
\end{align*}
\end{proposition}
\begin{corollaire}
Soit $\theta \in \mathbb{R}$ \\
On a:
\begin{align*}
\cos(\theta + \pi) &= -\cos(\theta) & \sin(\theta + \pi) &= -\sin(\theta) \\
\cos(\pi - \theta) &= -\cos(\theta) & \sin(\pi - \theta) &= \sin(\theta) \\
\cos\left(\theta + \frac{\pi}{2}\right) &= -\sin(\theta) & \sin\left(\theta + \frac{\pi}{2}\right) &= \cos(\theta) \\
\cos\left(\frac{\pi}{2} - \theta\right) &= \sin(\theta) & \sin\left(\frac{\pi}{2} - \theta\right) &= \cos(\theta)
\end{align*}
\end{corollaire}
\begin{proposition}
Soit $\theta_1, \theta_2 \in \mathbb{R}$ \\
On a:
\begin{align*}
\cos(\theta_1) = \cos(\theta_2) \iff \theta_1 &\equiv \theta_2 (\text{mod }2\pi) \text{ ou } \theta_1 \equiv -\theta_2 (\text{mod }2\pi) \\
\sin(\theta_1) = \sin(\theta_2) \iff \theta_1 &\equiv \theta_2 (\text{mod }2\pi) \text{ ou } \theta_1 \equiv \pi - \theta_2 (\text{mod }2\pi)
\end{align*}
\end{proposition}
\begin{corollaire}[de formules d'addition]
Soit $\theta \in \mathbb{R}$ \\
On a:
\begin{align*}
cos(2\theta) &= \cos^2(\theta) - \sin^2(\theta) \\
&= 2\cos^2(\theta) - 1 \\
&= 1 - 2\sin^2(\theta) \\
\sin(2\theta) &= 2\cos(\theta)\sin(\theta)
\end{align*}
\end{corollaire}

\subsection{Transformation produit $\rightarrow$ somme (linéarisation)}
\begin{proposition}
Soit $\alpha, \beta \in \mathbb{R}$ \\
On a:
\begin{align*}
\cos(\alpha)\cos(\beta) &= \frac{1}{2}(\cos(\alpha + \beta) + \cos(\alpha - \beta)) \\
\sin(\alpha)\sin(\beta) &= \frac{1}{2}(\cos(\alpha - \beta) - \cos(\alpha + \beta)) \\
\sin(\alpha)\cos(\beta) &= \frac{1}{2}(\sin(\alpha + \beta) + \sin(\alpha - \beta))
\end{align*}
\end{proposition}
\noindent \uline{Remarque}: On peut vouloir "délinéariser" une expression. \\
La clef est la formule de De Moivre. \\
Par exemple, $\cos(3t) + i\sin(3t) = (\cos(t) + i\sin(t))^3$\\
On développe l'expression et on prend partie réelle / imaginaire.

\subsection{Transformation somme $\rightarrow$ produit (factorisation)}
\begin{proposition}
Soit $p, q \in \mathbb{R}$ \\
On a:
\begin{align*}
\cos(p) + \cos(q) &= 2\cos\left(\frac{p - q}{2}\right) \cos\left(\frac{p + q}{2}\right) \\
\sin(p) + \sin(q) &= 2\cos\left(\frac{p - q}{2}\right) \sin\left(\frac{p + q}{2}\right)
\end{align*}
\end{proposition}
\noindent \uline{Exemple de calcul important}: Soit $t \in \mathbb{R}$ \\
Calculons $\sum\limits_{k = 0}^n \cos(kt)$ et $\sum\limits_{k = 0}^n \sin(kt)$ \\
Si $e^{it} = 1$, on a $\sum\limits_{k = 0}^n \cos(kt) = \sum\limits_{k = 0}^n 1 = n + 1$ et $\sum\limits_{k = 0}^n \sin(kt) = \sum\limits_{k = 0}^n 0 = 0$ \\
Si $e^{it} \neq 1$ on va calculer
\begin{align*}
\sum_{k = 0}^n e^{ikt} &= \frac{e^{i(n + 1)t} - 1}{e^{it} - 1} \\
&= \frac{e^{i \frac{n + 1}{2}t} \left( e^{i\frac{n + 1}{2}t} - e^{-i\frac{n + 1}{2} t}\right)}{e^{i\frac{t}{2}}\left(e^{i\frac{t}{2}} - e^{-i\frac{t}{2}}\right)} \\
&= e^{i\frac{n}{2}t} \frac{2i\sin\left(\frac{n + 1}{2}t\right)}{2i\sin\left(\frac{t}{2}\right)} \\
&= e^{i\frac{n}{2}t} \frac{\sin\left(\frac{n + 1}{2}t\right)}{\sin\left(\frac{t}{2}\right)}
\end{align*}
On a donc
\[ \sum_{k = 0}^n \cos(kt) = \re\left(\sum_{k = 0}^n e^{ikt}\right) = \frac{\sin\left(\frac{n + 1}{2}t\right)}{\sin\left(\frac{t}{2}\right)} \cos\left(\frac{n}{2}t\right) \]
Et
\[ \sum_{k = 0}^n \sin(kt) = \im\left(\sum_{k = 0}^n e^{ikt}\right) = \frac{\sin\left(\frac{n + 1}{2}t\right)}{\sin\left(\frac{t}{2}\right)} \sin\left(\frac{n}{2}t\right) \]

\subsection{Déphasage}
\begin{proposition}
Soit $u, v \in \mathbb{R}$ non tous les deux nuls (càd $(u, v) \neq (0, 0)$) \\
On écrit le nombre complexe $u + iv$ sous forme exponentielle. \\
On a $u + iv = Ae^{i\psi}$, où $A = \sqrt{u^2 + v^2}$ et $\psi \in \mathbb{R}$ est un argument de $u + iv$ \\
On a alors $\forall x \in \mathbb{R}$, $u\cos(x) + v\sin(x) = A\cos(x - \psi)$
\end{proposition}

\section{Cyclotomie}
\begin{definition}
Soit $n \in \mathbb{N}$
\begin{itemize}
\item Un nombre $z \in \mathbb{C}$ est une racine $n$-ième de l'unité si $z^n = 1$
\item L'ensemble des racines $n$-ièmes de l'unité est noté $\mathbb{U}_n$
\end{itemize}
\end{definition}
\begin{theorem}
Soit $n \in \mathbb{N}^*$
On a
\[ \mathbb{U}_n = \left\{ e^{i2\pi\frac{k}{n}} \mid k \in \mathbb{Z}\right\} = \left\{ e^{i2\pi\frac{k}{n}} \mid k \in \llbracket 0, n - 1 \rrbracket \right\} \]
\end{theorem}

\subsection{Équations $z^n = a$}
\begin{theorem}
Soit $a \in \mathbb{C}^*$, que l'on écrit sous forme exponentielle $a = |a|e^{i\theta}$
\begin{itemize}
\item Alors $\sqrt[n]{|a|}e^{i\frac{\theta}{n}}$ est \uline{une} solution de $z^n = a$
\item Si $z_0$ est une solution de $z^n = a$, l'ensemble des solutions est
\[ \left\{ z_0 \omega \mid \omega \in \mathbb{U}_n \right\} = \left\{ z_0 e^{i2\pi\frac{k}{n}} \mid k \in \llbracket 0, n - 1 \rrbracket \right\} \]
\end{itemize}
\end{theorem}

\subsection{Somme}
\begin{proposition}
Soit $n \geq 2$ \\
Alors la somme des racines $n$-ièmes de l'unité est nulle.
\[ \sum_{\omega \in \mathbb{U}_n} \omega = 0 \]
\end{proposition}

\section{Géométrie plane}
\subsection{Rappels sur les angles}
\noindent Étant donné trois points $O, A, B$ du plan tels que $O \neq A$ et $O \neq B$, on dispose de l'angle géométrique $\widehat{AOB}$ entre les demi-droites $\left[ OA \right)$ et $\left[ OB \right)$. C'est un élément de $[0, \pi]$ \\
Deux vecteurs non nuls $\vec{u}$ et $\vec{v}$ du plan définissent un angle oriente $(\vec{u}, \vec{v})$ \\
Sa \uline{mesure principale} appartient à $\left] -\pi, \pi \right]$ \\
On la notera simplement $(\vec{u}, \vec{v})$
\begin{proposition}
Soit $\vec{u}, \vec{v}, \vec{w}$ trois vecteurs non nuls du plan. \\
On a:
\begin{itemize}
\item Antisymétrie: $(\vec{v}, \vec{u}) \equiv -(\vec{u}, \vec{v}) \quad (\text{mod }2\pi)$
\item Relation de Chasles: $(\vec{u}, \vec{w}) \equiv (\vec{u}, \vec{v}) + (\vec{v}, \vec{w}) \quad (\text{mod }2\pi)$
\end{itemize}
\end{proposition}

\subsection{Angles et arguments}
\noindent Dans toute la suite, on notera $\vec{Z}, \vec{W}$, etc... les vecteurs d'affixe $z, w$, etc... \\
On notera $\vec{H}$ le vecteur(horizontal) d'affixe $1$ \medskip

\noindent \uline{Point-clef}: "L'angle" $(\vec{H}, \vec{Z})$ est "l'argument" de $z$ \\
Plus précisement, toute mesure de $(\vec{H}, \vec{Z})$ est un argument de $z$
\begin{proposition}
\hfill
\begin{itemize}
\item Soit $z, w \in \mathbb{C}^*$ \\
On a $(\vec{Z}, \vec{W}) \equiv \arg\left(\frac{w}{z}\right) \quad (\text{mod }2\pi)$
\item Soit $A, B, C$ trois points distincts, d'affixes $a, b, c$ \\
Alors $(\vec{AB}, \vec{AC}) \equiv \arg\left(\frac{c - a}{b - a}\right) \quad (\text{mod }2\pi)$
\end{itemize}
\end{proposition}
\begin{corollaire}
Soit $z, w \in \mathbb{C}^*$
\begin{itemize}
\item Critère de colinéarité:
\begin{itemize}
\item On a $\vec{Z}$ et $\vec{W}$ colinéaires ssi $\frac{w}{z}$ est réel.
\item Soit $A, B, C$ trois points distincts. \\
Alors $A, B, C$ sont alignés ssi $\frac{c - a}{b - a}$ est un réel.
\end{itemize}
\item Critère d'orthogonalité: Soit $z, w \in \mathbb{C}^*$ \\
Alors $\vec{Z}$ et $\vec{W}$ sont orthogonaux ssi $\frac{w}{z}$ est imaginaire pur.
\end{itemize}
\end{corollaire}

\subsection{Similitudes directes: définition et classification}
\begin{definition}
\hfill
\begin{itemize}
\item On appelle \uline{similitude directe} toute application de la forme
\[f_{a, b}: \begin{cases}
\mathbb{C} \to \mathbb{C} \\
z \mapsto az + b
\end{cases} \]
où $a \in \mathbb{C}^*$ et $b \in \mathbb{C}$
\item On dit que $a$ est le \uline{rapport} de la similitude $f_{a,b}$
\item On note $sim_+(\mathbb{C})$ l'ensemble des similitudes directes.
\end{itemize}
\end{definition}
\begin{theorem}
Soit $a \in \mathbb{C}^*$ que l'on écrit $a = |a|e^{i\theta}$ pour un certain $\theta \in \mathbb{R}$ et $b \in \mathbb{C}$
\begin{itemize}
\item Si $a = 1$, $f_{a, b}$ est la translation de vecteur $b$
\item Si $a \neq b$, $f_{a, b}$ a un unique point fixe $w = \frac{b}{1 - a}$ et:
\begin{itemize}
\item $\forall z \in \mathbb{C}$, $f_{a, b}(z) - w = a(z - w)$
\item On peut obtenir $f_{a, b}$ en composant l'homothétie de centre $w$ et de rapport $|a|$ et la rotation de centre $w$ et d'angle $\theta$
\end{itemize}
\end{itemize}
\end{theorem}
\noindent \uline{Remarque}: Si $a \in \mathbb{R} \setminus \{1\}$, on dit que $f_{a,b}$ est une \uline{homothétie} de centre $w$ et de rapport $a$ \medskip

\noindent \uline{Remarque}: Deux similitudes de \uline{même centre $w$} commutent.
\begin{proposition}
Soit $a \in \mathbb{C}^*$ et $b \in \mathbb{C}$
\begin{itemize}
\item $f_{a, b}$ "dilate les distances d'un facteur $|a|$"
\[ \forall z_1, z_2 \in \mathbb{C} ,\, \left| f_{a, b}(z_2) - f_{a, b}(z_1) \right| = |a| |z_2 - z_1| \]
\item $f_{a, b}$ "préserve les angles": pour tous $z_1, z_2, z_3 \in \mathbb{C}$ distincts, on a
\[ \arg\left(\frac{f_{a, b}(z_3) - f_{a, b}(z_1)}{f_{a, b}(z_2) - f_{a, b}(z_1)}\right) \equiv \arg\left(\frac{z_3 - z_1}{z_2 - z_1}\right) \quad (\text{mod }2\pi) \]
\end{itemize}
\end{proposition}

\subsection{Structure de $sim_+(\mathbb{C})$}
\begin{proposition}[$sim_+(\mathbb{C})$ est un groupe]
\hfill
\begin{itemize}
\item $sim_+(\mathbb{C})$ est stable par composition: $\forall f, g \in sim_+(\mathbb{C})$, $g \circ f \in sim_+(\mathbb{C})$
\item Tout $f \in sim_+(\mathbb{C})$ est bijectif, et on a $f^{-1} \in sim_+(\mathbb{C})$
\end{itemize}
\end{proposition}
\begin{proposition}[$sim_+(\mathbb{C})$ agit exactement $2$-transitivement sur $\mathbb{C}$]
Soit $z_1 \neq z_2$ et $w_1 \neq w_2 \in \mathbb{C}$ \\
Alors il existe un unique $f \in sim_+(\mathbb{C})$ tel que $f(z_1) = w_1$ et $f(z_2) = w_2$
\end{proposition}
\end{document}