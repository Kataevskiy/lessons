\documentclass[10pt,a4paper]{article}
\usepackage[utf8]{inputenc}
\usepackage[french]{babel}
\usepackage[T1]{fontenc}
\usepackage{amsmath}
\usepackage{amsfonts}
\usepackage{amssymb}
\usepackage{graphicx}
\usepackage[left=2cm,right=2cm,top=2cm,bottom=2cm]{geometry}
\usepackage{setspace}
\usepackage{ulem}
\usepackage{stmaryrd}
\usepackage{amsthm}
\usepackage{dsfont}
\usepackage{mathpazo}

\onehalfspacing

\theoremstyle{definition}
\newtheorem{proposition}{Proposition}[section]
\newtheorem{theorem}[proposition]{Théorème}
\newtheorem{corollaire}[proposition]{Corollaire}
\newtheorem{lemme}[proposition]{Lemme}
\newtheorem{definition}[proposition]{Définition}

\DeclareMathOperator{\im}{im}

\begin{document}
\renewcommand{\labelitemi}{$*$}
\begin{center}
{\Large \textbf{Chapitre 10. Algèbre générale}}
\end{center}

\section{Conjugaison dans un groupe. Théorème de Lagrange}
\subsection{Morphisme de conjugaison}
\begin{definition}
Soit $H$ un groupe, $h \in G$ \\
Alors
\[ \varphi_h: \begin{cases}
G \to G \\
g \mapsto h g h^{-1}
\end{cases} \] est un morphisme de conjugaison (ou encore un automorphisme intérieur).
\end{definition}
\begin{proposition}
$\phi_h$ est bien un automorphisme et
\[ \Phi: \begin{cases}
G \to \text{Aut } G = \{ f: G \to G \mid f \text{ automorphisme} \} \\
h \mapsto \varphi_h
\end{cases}\]
est un morphisme de groupes.
\end{proposition}
\begin{definition}
Soit $G$ un groupe, $g, g' \in G$ \\
On dit que $g$ et $g'$ sont conjugués s'il existe $h \in G$ tel que $g' = h g h^{-1}$
\end{definition}
\begin{proposition}
Sur un groupe, la relation "être conjugué" est une relation d'équivalence. On appelle classe de conjugaison les classes pour cette relation.
\end{proposition}

\subsection{Classe à gauche, classe à droite selon un sous-groupe}
\begin{definition}
Soit $G$ un groupe, $H$ un sous-groupe. \\
On définit la congruence à gauche modulo $H$ par
\[ x \sim y \iff y \in x H \iff \exists h \in H ,\, y = x h \]
\end{definition}
\begin{proposition}
La congruence à gauche modulo $H$ est une relation d'équivalence et la classe de $x$ est $\overline{x} = x H$
\end{proposition}
\begin{theorem}[Théorème de Lagrange]
\hfill \\
Soit $G$ un groupe fini. Le cardinal d'un sous-groupe de $G$ divise celui de $G$
\end{theorem}

\subsection{Relations compatibles avec une loi. Groupe quotient}
\begin{definition}
Soit $(M, *)$ un monoïde, $\equiv$ une relation d'équivalence sur $M$ \\
On dit que $*$ et $\equiv$ sont compatibles si $\forall a, x, y \in M ,\, x \equiv M \implies \begin{cases}
 a * x \equiv a * y \\
 x * a \equiv y * a
\end{cases}$
\end{definition}
\begin{theorem}
Soit $(G, +)$ un groupe abélien. On note $G_{|H}$ le quotient de $G$ par la congruence modulo $H$ sur lequel on définit une loi quotient $+ : \overline{x} + \overline{y} = \overline{x + y}$ \\
Alors $(G_{|H}, +)$ est un groupe (abélien) appelé groupe quotient de $G$ par $H$ 
\end{theorem}
\begin{definition}
Un sous-groupe $H$ de $G$ tel que $\forall x \in G, x H x^{-1} \subset H$ est dit distingué.
\end{definition}
\begin{proposition}
Si $H$ est distingué de $G$ alors $\equiv$ (congruence modulo $H$) et $\times$ sont compatibles.
\end{proposition}
\begin{theorem}
$G_{|H}$ peut être muni d'une loi quotient qui en fait un groupe.
\end{theorem}
\begin{theorem}[Théorème d'isomorphisme]
Soit $f: G \to G'$ un morphisme de groupes. On pose
\[ \overline{f} : \begin{cases}
G_{|\ker f} \to \im f \\
\overline{x} \mapsto \overline{f}(\overline{x}) = f(x)
\end{cases} \]
$\overline{f}$ est bien définie et c'est un isomorphisme (canoniquement associé à $f$)
\[ G_{|\ker f} \simeq \im f\]
\end{theorem}

\subsection{Ordre d'un élément dans un groupe}
\begin{definition}
Soit $G$ un groupe, $a \in G$ \\
L'ordre de $a$ dans $G$ est le "cardinal" du sous-groupe engendré par $a$
\end{definition}
\begin{proposition}
Si $a$ est d'ordre fini $n$ dans un groupe $G$ alors $n$ est le plus petit entier $k \geq 1$ avec $a^k = 1$
\end{proposition}
\begin{theorem}[Théorème de Largange]
Soit $G$ un groupe fini de cardinal $n$, $a \in G$
\begin{itemize}
\item L'ordre de $a$ divise $n$
\item $a^n = 1$
\end{itemize}
\end{theorem}

\subsection{Le groupe symétrique}
\begin{definition}
Soit $a_1, ...,\, a_n \in E$ 2 à 2 distincts, $p \geq 2$ \\
On note \[(a_1, a_2, ...,\, a_p): x \mapsto \begin{cases}
a_{i + 1} \text{ si } x = a_i \text{ avec } i \in \llbracket 1, p - 1 \rrbracket \\
a_1 \text{ si } x = a_p \\
x \text{ si } x \not\in \{ a_1, ...,\, a_p \}
\end{cases} \]
$(a_1, ...,\, a_p)$ est appelé $p$-cycle, de support $\{ a_1, ...,\, a_p \}$ de longueur $p$ \\
Si $p = 2$ on parle de transpositions.
\end{definition}
\begin{proposition}
\hfill
\begin{itemize}
\item Un $p$-cycle $\sigma$ et $S_E$ est un élément d'ordre $p$ de $(S_E, \circ)$
\item Si $\sigma \in S_E$, $c = (a_1, ...,\, a_p)$ alors
\[ \sigma \circ c \circ \sigma^{-1} = \sigma \circ (a_1, ...,\, a_p) \circ \sigma^{-1} = (\sigma(a_1), ...,\, \sigma(a_p)) \]
\item $(a_1 a_2 ... a_p) = (a_1 a_2) \circ (a_2 a_3) \circ ... \circ (a_{p - 1} a_p)$
\item Si $\sigma$ et $\sigma'$ sont des cycles à support disjoints, alors $\sigma \circ \sigma' = \sigma' \circ \sigma$ 
\end{itemize}
\end{proposition}
\begin{theorem}[Théorème de décomposition en produit de cycles à support disjoint]
\hfill \\
Soit $\sigma \in S_E$ et soit $\Omega_1, ...,\, \Omega_n$ les orbites de $E$ sous l'action de $\sigma$ \\
Alors il existe $c_1, ...,\, c_n$ cycles à supports disjoints tels que
\[ \sigma = c_1 \circ c_2 \circ ... \circ c_n \]
\end{theorem}
\begin{corollaire}
\hfill \begin{itemize}
\item Les cycles engendrent $S_E$
\item Les transpositions engendrent $S_E$: tout $\sigma \in S_E$ est un produit de transpositions.
\end{itemize}
\end{corollaire}

\pagebreak

\subsection{Signature d'une permutation}
\begin{lemme}
Soit $\sigma, \sigma' \in S_n$ \\
On note $N(\sigma) = \text{Card} \left\{ (i, j) \in \llbracket 1, n \rrbracket \mid i < j \text{ et } \sigma(j) < \sigma(i) \right\}$ et $\varepsilon(\sigma) = (-1)^{N(\sigma)}$ \\
Alors $\varepsilon(\sigma' \circ \sigma) = \varepsilon(\sigma') \varepsilon(\sigma)$
\end{lemme}
\begin{theorem}
Il existe un unique morphisme de groupes non trivial
\[ \varepsilon: \begin{cases}
S_n \to \{ -1, 1 \} \\
\sigma \mapsto \varepsilon(\sigma)
\end{cases} \]
appelée signature. Si $\sigma = \tau_1 \circ ... \circ \tau_2$ avec $\tau_i$ transpositions alors $\varepsilon(\sigma) = (-1)^{r}$
\end{theorem}
\begin{definition}
$\sigma$ est dite paire si $\varepsilon(\sigma) = 1$ \\
$\ker \varepsilon$ est un sous-groupe de $S_n$ appelé groupe alternée d'ordre $n$ noté $\mathfrak{A}_n$ \\
On a $| \mathfrak{A}_n | = \frac{n!}{2}$
\end{definition}

\section{Congruence modulo un idéal}
\subsection{Anneaux quotients}
Ici les anneaux sont commutatifs.
\begin{theorem}
Soit $A$ un anneau et $I$ un idéal différent de $A$ \\
$(A_I, +, \times)$ est un anneau commutatif appelé anneau quotient de $A$ par $I$
\end{theorem}
\begin{theorem}[Corps de rupture d'un polynôme irréductible]
\hfill \\
Soit $A = K[X]$ et $\Pi$ un polynôme irréductible de $K[X]$
\[ L = \frac{K[X]}{\Pi K[X]} \]
est appelé corps de rupture du polynôme $\Pi$ et
\[ \underset{K}{\dim} \frac{K[X]}{\Pi K[X]} = \deg \Pi \]
\end{theorem}
\begin{theorem}[Théorème d'isomorphisme]
Soit $f : A \to B$ un morphisme d'anneaux. \\
Alors
\[ \overline{f}: \begin{cases}
A_{|\ker f} \to \im f \\
\overline{x} \mapsto \overline{f}(\overline{x}) = f(x)
\end{cases}\]
est un isomorphisme.
\[ A_{|\ker f} \simeq \im f \]
On dit que $\overline{f}$ est l'isomorphisme canoniquement associé à $f$
\end{theorem}

\subsection{Congruences modulo $a$}
\begin{definition}
Soit $A$ un anneau et $a \in A$ \\
La congruence modulo $a$ est la congruence modulo $a A$. On la note pour $x, y \in A$
\[ x \equiv y \mod a \iff y - x \in a A \iff a \mid y - x \]
\end{definition}

\subsection{Le petit théorème de Fermat}
\begin{theorem}
\hfill \begin{itemize}
\item Si $a \in \mathbb{Z}$ et $p \nmid a$ ($p \in \mathbb{P}$) alors \\
$a^{p - 1} \equiv 1 [p]$
\item Si $a \in \mathbb{Z}$ et $p \in \mathbb{P}$ alors \\
$a^p \equiv a [p]$ \\
$\overline{a}^p = \overline{a}$ dans $\mathbb{F}_p = \mathbb{Z}_{|p\mathbb{Z}}$
\end{itemize}
\end{theorem}
\begin{proposition}
Dans $\mathbb{F}_p[X] = \mathbb{Z}_{|p\mathbb{Z}}[X]$
\[ X^p - X = X(X - \overline{1}) ... (X - \overline{p - 1}) \]
\[ X^{p - 1} - 1 = (X - \overline{1}) ... (X - \overline{p - 1} \]
\[ \overline{(p - 1)!} = -1 \text{ (Théorème de Wilson) }\] 
\end{proposition}

\subsection{La caractérisation d'un anneau, d'un corps}
\noindent Soit $A$ un anneau commutatif et 
\[ f: \begin{cases}
\mathbb{Z} \to A \\
k \mapsto k 1_A = (1_A + ... + 1_A)
\end{cases} \]
$A_0 = \im f = \{k 1_A \}_{k \in \mathbb{Z}}$ \\
$\ker f = n \mathbb{Z}$ avec $n \in \mathbb{N}^*$ unique car $\ker f$ est un idéal de $\mathbb{Z}$ \\
Si $f$ est injective, alors $n = 1$ et $A_0 \simeq \mathbb{Z}$, on dit que $A$ est de caractéristique nulle \\
Sinon, $n \geq 2$ et $A_0 \simeq \mathbb{Z}_{|n\mathbb{Z}}$, on dit que $A$ est de caractéristique $n$

\subsection{Complément sur les corps}
\begin{proposition}
La caractéristique d'un corps est nulle ou finie égale à un nombre premier.
\end{proposition}

\subsubsection{Complément 1}
\begin{theorem}
Soit $P \in K[X]$, $P \neq 0$ \\
Alors il existe un surcorps de $K$ sur lequel $P$ est scindé.
\end{theorem}

\subsubsection{Complément 2: construction de corps fini}
\noindent En utilisant les corps de rupture on peut construire les corps de taille donné. \\
Par exemple: $\Pi = X^3 + X + 1$ est irréductible dans $\mathbb{F}_2[X]$, ainsi
\[ L = \mathbb{F}_8 = \frac{\mathbb{F}_2[X]}{(\Pi)} = \text{ Vect }(1, \alpha, \alpha^2) \]
avec $\alpha^3 + \alpha + 1 = 0$ et $(\Pi)$ l'idéal engendré par $\Pi$

\section{L'indicatrice d'Euler}
\subsection{Le théorème chinois}
\begin{theorem}[Théorème chinois]
Soit $m, n \geq 1$ premiers entre eux. \\
Alors 
\[ \overline{f} : \begin{cases}
\mathbb{Z}_{|mn\mathbb{Z}} \to \mathbb{Z}_{|m\mathbb{Z}} \to \mathbb{Z}_{|n\mathbb{Z}} \\
\overline{\overline{k}} \mapsto (\dot{k}, \overline{k})
\end{cases} \]
est un isomorphisme d'anneaux: 
\[ \mathbb{Z}_{|m\mathbb{Z}} \times \mathbb{Z}_{|n\mathbb{Z}} \simeq \mathbb{Z}_{|mn\mathbb{Z}} \]
en tant qu'anneaux.
\end{theorem}
\noindent \uline{Extension}: Si $n_1, ...,\, n_r$ sont premiers $2$ à $2$ alors
\[ \mathbb{Z}_{|n_1 \mathbb{Z}} \times ... \times \mathbb{Z}_{|n_r \mathbb{Z}} \simeq \mathbb{Z}_{|n_1 ... n_r \mathbb{Z}} \]

\subsection{Expression de l'indicatrice d'Euler}
\begin{definition}
On note $\varphi(1) = 1$ et pour $n \geq 2$ \\
$\varphi$ est le nombre d'entiers $k \in \llbracket 1, n \rrbracket$ tels que $k \wedge n = 1$ \\
$\varphi: \mathbb{N}^* \to \mathbb{N}$ est appelée indicatrice d'Euler.
\end{definition}
\begin{proposition}
Soit $n \geq 2$. On a:
\begin{itemize}
\item $\varphi(n) = \text{ Card} \{k \in \llbracket 0, n - 1 \rrbracket \mid k \wedge n = 1 \}$
\item $\varphi(n) = \text{ Card} (\mathbb{Z}_{|n \mathbb{Z}})^\times$
\item $\varphi(n) = \text{ Card} \{x \in \mathbb{Z}_{|n\mathbb{Z}} \mid x \text{ engendre } (\mathbb{Z}_{|n\mathbb{Z}}, +) \}$
\item $\varphi(n) =$ nombre de racines n-ièmes primitives de l'unité.
\item $\varphi(n) =$ nombre de générateurs de $(\mathbb{U}_n, \times)$
\end{itemize}
\end{proposition}
\begin{theorem}[Théorème d'Euler-Fermat]
Soit $n \geq 2$ \\
Si $a$ est premier avec $n$ alors
\[ a^{\varphi(n)} \equiv [n] \]
\end{theorem}
\begin{theorem}
Si $m, n \in \mathbb{N}^*$
\[ m \wedge n = 1 \implies \varphi(mn) = \varphi(m) \varphi(n) \]
\end{theorem}
\begin{lemme}
Si $A$ et $B$ anneaux, alors $(A \times B)^\times = A^\times \times B^\times$
\end{lemme}
\begin{theorem}
Soit $n = p_1^{\alpha_1} ... p_r^{\alpha_r}$ sa décomposition en facteurs premiers.
\[ \varphi(n) = \left(p_1^{\alpha_1} - p_1^{\alpha_1 - 1}\right) ... \left(p_r^{\alpha_r} - p_r^{\alpha_r - 1}\right) \]
\[ \varphi(n) = n \left( 1 - \frac{1}{p_1} \right) ... \left( 1 - \frac{1}{p_r} \right) \]
\end{theorem}

\subsection{Complément La formule sommatoire d'Euler}
\begin{theorem}[formule sommatoire d'Euler]
Pour $n \geq 1$ on a
\[ n = \sum_{d \mid n} \varphi(d) \]
\end{theorem}

\section{Exercices classiques}
\subsection{Théorème de Wilson}
\begin{enumerate}
\item Montrer que si $p$ est premier alors $(p - 1)! \equiv -1 [p]$
\item Si $n \geq 2$ et $(n - 1)! \equiv -1[n]$ montrer que $n$ est premier.
\end{enumerate}

\subsection{Groupes tels que $x^2 = 1$}
\noindent Soit $(G, \times)$ un groupe tel que $\forall x \in G$, $x^2 = 1_G$
\begin{enumerate}
\item Montrer que $G$ est abélien.
\item Si $G$ est fini, montrer que $G \simeq \left( (\mathbb{Z}_{|2\mathbb{Z}})^n, + \right)$
\end{enumerate}

\subsection{Carrés de $\mathbb{F}_p$}
\noindent Soit $p$ un nombre premier impair.
\begin{enumerate}
\item Dénombrer les carrés de $(\mathbb{Z}_{|p\mathbb{Z}})^*$
\item Montrer que si $x \in (\mathbb{Z}_{|p\mathbb{Z}})^*$ alors $x$ carré $\iff x^{\frac{p - 1}{2}} = 1$
\item Montrer que $-1$ carré dans $\mathbb{Z}_{|p\mathbb{Z}} \iff p \equiv 1 [4]$
\item En déduire qu'il existe une infinité de nombres premiers $\equiv 1 [4]$
\end{enumerate}

\subsection{Groupe diagonal $\mathbb{D}_{2n}$}
\noindent On note $\mathbb{D}_n$ le groupe des isométries affines du plan complexe qui laissent $\mathbb{U}_n$ globalement invariant.
\begin{enumerate}
\item Préciser les éléments de $\mathbb{D}_{2n}$ et son cardinal. \\
$\mathbb{D}_{2n}$ est appelé le groupe diédral d'ordre $2n$
\item Montrer que $\mathbb{D}_{2n}$ est engendré par deux éléments: \\
$R$ d'ordre $n$, $S$ d'ordre $n$ tels que $SR = R^{-1}S$
\item Réciproquement, si un groupe $G$, $G = <R, S>$ avec $R$ d'ordre $n$ et $S$ d'ordre $2$ et $SR = R^{-1}S$ montrer que $G \simeq \mathbb{D}_{2n}$
\item Montrer que tout sous-groupe fini du groupe des isométries du plan complexe est isomorphe à $(\mathbb{Z}_{|n\mathbb{Z}})$ (cyclique) ou à $\mathbb{D}_{2n}$ (avec $n \geq 3$)
\end{enumerate}
\end{document}