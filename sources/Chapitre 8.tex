\documentclass[10pt,a4paper]{article}
\usepackage[utf8]{inputenc}
\usepackage[french]{babel}
\usepackage[T1]{fontenc}
\usepackage{amsmath}
\usepackage{amsfonts}
\usepackage{amssymb}
\usepackage{graphicx}
\usepackage[left=2cm,right=2cm,top=2cm,bottom=2cm]{geometry}
\usepackage{setspace}
\usepackage{ulem}
\usepackage{stmaryrd}
\usepackage{amsthm}
\usepackage{dsfont}
\usepackage{mathpazo}

\onehalfspacing

\theoremstyle{definition}
\newtheorem{proposition}{Proposition}[section]
\newtheorem{theorem}[proposition]{Théorème}
\newtheorem{corollaire}[proposition]{Corollaire}
\newtheorem{lemme}[proposition]{Lemme}
\newtheorem{definition}[proposition]{Définition}

\DeclareMathOperator{\pgcd}{pgcd}
\DeclareMathOperator{\ppcm}{ppcm}
\DeclareMathOperator{\im}{im}
\DeclareMathOperator{\car}{car}
\DeclareMathOperator{\Frac}{Frac}

\begin{document}
\renewcommand{\labelitemi}{$*$}
\renewcommand{\labelenumi}{(\roman{enumi})}
\begin{center}
{\Large \textbf{Chapitre 8. Anneaux et corps}}
\end{center}

\section{Anneaux}
\subsection{Généralités}
\begin{definition}
Un \uline{anneau} est un ensemble $A$ muni de deux lois de composition interne $+$ et $\cdot$ tels que:
\begin{itemize}
\item $(A, +)$ est un groupe abélien, noté additivement (en particulier, son élément neutre est noté $0_A$)
\item La loi $\cdot$ est associative et possède un élément neutre $1_A$
\item La loi $\cdot$ \uline{distribue sur l'addition}: $\forall x, y, z \in A: \begin{cases}
x(y + z) = xy + xz \\
(y + z)x = yx + zx
\end{cases}$
\end{itemize}
Un anneau est dit \uline{commutatif} si sa multiplication est commutative.
\end{definition}

\subsection{Règles de calcul}
\begin{proposition}
\hfill
\begin{itemize}
\item $0_A$ est \uline{absorbant}: $\forall a \in A$, $0_A \cdot a = a \cdot 0_A = 0_A$
\item (règle de signes): $\forall a, b \in A$, $(-a)b = -(ab) = a \cdot (-b)$ \quad (où $(-a)$ est l'opposé de $a$ pour la loi $+$)
\end{itemize}
\end{proposition}
\begin{theorem}
Soit $A$ un anneau et $a, b \in A$ tels que $ab = ba$ ($a, b$ commutent) \\
Alors
\[\forall n \in \mathbb{N}: \begin{cases}
(a + b)^n = \sum\limits_{k = 0}^n \binom{n}{k} a^k b^{n - k} \text{ \quad (Binôme de Newton) } \\
a^{n + 1} - b^{n + 1} = (a - b) \sum\limits_{k = 0}^n a^k b^{n - k}
\end{cases}\]
\end{theorem}

\subsection{Groupe des inversibles}
\begin{definition}
Soit $A$ un anneau. \\
Le \uline{groupe des inversibles (ou des unités) de $A$} est $A^\times = \left\{ x \in A \mid \exists y \in A: xy = yx = 1_A \right\}$ \\
Comme son nom l'indique, $(A^\times, \cdot)$ est un groupe.
\end{definition}

\subsection{Sous-anneaux}
\begin{definition}
Soit $A$ un anneau. \\
Un \uline{sous-anneau de $A$} est une partie $B$ de $A$ telle que:
\begin{itemize}
\item $B$ soit un sous-groupe de $(A, +)$
\item $B$ soit stable par $\cdot$ et $1_A \in B$
\end{itemize}
\end{definition}
\begin{proposition}
Soit $A$ un anneau et $B \subseteq A$ \\
Pour que $B$ soit un sous-anneau de $A$, il faut et il suffit que:
\begin{itemize}
\item $1_A \in B$
\item $\forall x, y \in B$, $x - y \in B$
\item $\forall x, y \in B$, $xy \in B$
\end{itemize}
\end{proposition}

\subsection{Morphismes d'anneaux}
\begin{definition}
Soit $A, B$ deux anneaux. \\
Un morphisme (d'anneaux) $f: A \to B$ est une application telle que:
\begin{itemize}
\item $\forall x, y \in A$, $f(x + y) = f(x) + f(y)$ \quad ($f$ morphisme de groupes additifs)
\item $f(1_A) = 1_B$
\item $\forall x, y \in A$, $f(xy) = f(x) f(y)$
\end{itemize}
\end{definition}
\begin{proposition}
Soit $f: A \to B$ un morphisme d'anneaux. \\
Alors $\im(f)$ est un sous-anneau de $B$
\end{proposition}

\section{Corps}
\subsection{Anneau intègre}
\begin{definition}
Un anneau $A$ est dit \uline{intègre} s'il est commutatif, non nul (différent de l'anneau nul: $0_A \neq 1_A$) et que $\forall x, y \in A$, $xy = 0_A \implies (x = 0_A \text{ ou } y = 0_A)$
\end{definition}
\begin{proposition}
Soit $n \in \mathbb{N}^*$ \\
Alors $\mathbb{Z}/n\mathbb{Z}$ est intègre ssi $n$ est premier.
\end{proposition}

\section{Corps: généralités}
\begin{definition}
Un \uline{corps} est un anneau commutatif, non nul, et dans lequel tout élément non nul est inversible.
\end{definition}
\begin{proposition}
Tout anneau intègre fini est un corps.
\end{proposition}
\begin{definition}
Soit $L$ un corps. \\
Un \uline{sous-corps} de $L$ est un sous-anneau $K$ de $L$ tel que $\forall x \in K \setminus \{ 0 \}$, $x^{-1} \in K$ \\
On dit aussi que $L$ est un \uline{sur-corps} de $K$ ou que $L/K$ est une \uline{extension de corps}.
\end{definition}
\begin{definition}
Soit $K_1, K_2$ deux corps. \\
Un \uline{morphisme de corps} $K_1 \to K_2$ est un morphisme d'anneaux $K_1 \to K_2$
\end{definition}
\begin{proposition}
Tout morphisme de corps est injectif.
\end{proposition}

\subsection{Caractéristique d'un corps}
\begin{definition}
La \uline{caractéristique} $\car(K)$ d'un corps $K$ est l'ordre de $1_K$ dans le groupe additif $(K, +)$ s'il est fini, et $0$ sinon.
\end{definition}
\begin{theorem}
Soit $K$ un corps.
\begin{itemize}
\item Si $K$ est de caractéristique non nulle, alors sa caractéristique est un nombre premier, et $K$ continent un sous-corps isomorphe à $\mathbb{F}_p$
\item Si $K$ est de caractéristique nulle, il contient un sous-corps isomorphe à $\mathbb{Q}$
\end{itemize}
\end{theorem}

\subsection{Corps des fractions d'un anneau intègre}
\noindent On va construire un \uline{corps de fractions} $\Frac(A)$ "contenant $A$". \\
L'ensemble $\Frac(A)$ est le quotient de $A \times (A \setminus \{ 0 \})$ par la relation $\sim$ définie par \\
$\forall (a_1, b_1),(a_2, b_2) \in A \times (A \setminus \{ 0 \})$, $(a_1, b_1) \sim (a_2, b_2) \iff a_1 b_2 = a_2 b_1$ \\
dont on vérifie que c'est une relation d'équivalence. \\
On note simplement $\frac{a}{b}$ le classe d'équivalence de $(a, b) \in A \times (A \setminus \{ 0 \})$ \\
On munit alors $K = \Frac(A)$ de deux lois:
\begin{align*}
\frac{a_1}{b_1} + \frac{a_2}{b_2} &= \frac{a_1 b_2 + a_2 b_1}{b_1 b_2} \\
\frac{a_1}{b_1} \cdot \frac{a_2}{b_2} &= \frac{a_1 a_2}{b_1 b_2}
\end{align*}
(avec $b_1 b_2 \neq 0_A$)
\end{document}