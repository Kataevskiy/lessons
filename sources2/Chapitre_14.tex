\documentclass[10pt,a4paper]{article}
\usepackage[utf8]{inputenc}
\usepackage[french]{babel}
\usepackage[T1]{fontenc}
\usepackage{amsmath}
\usepackage{amsfonts}
\usepackage{amssymb}
\usepackage{graphicx}
\usepackage[left=2cm,right=2cm,top=2cm,bottom=2cm]{geometry}
\usepackage{setspace}
\usepackage{ulem}
\usepackage{stmaryrd}
\usepackage{amsthm}
\usepackage{dsfont}
\usepackage{mathpazo}
\usepackage{empheq}

\usepackage{mathabx}

\onehalfspacing{}
\theoremstyle{definition}
\newtheorem{proposition}{Proposition}[section]
\newtheorem{theorem}[proposition]{Théorème}
\newtheorem{corollary}[proposition]{Corollaire}
\newtheorem{lemma}[proposition]{Lemme}
\newtheorem{definition}[proposition]{Définition}

\DeclareMathOperator*{\mat}{Mat}
\DeclareMathOperator{\rg}{rg}
\DeclareMathOperator{\vect}{Vect}
\DeclareMathOperator{\Sp}{Sp}

\begin{document}
\renewcommand{\labelitemi}{\textbullet}

\begin{center}
{\Large \textbf{Chapitre 14. Produit scalaire, groupe orthogonal}}
\end{center}
\begin{proposition}
    Soit \(K\) un corps de caractéristique différent de 2 \\
    Si \(X = \begin{pmatrix}
        x_1 \\
        \vdots \\
        x_n
    \end{pmatrix}\), \(Y = \begin{pmatrix}
        y_1 \\
        \vdots \\
        y_n
    \end{pmatrix} \in K^n\) et \(A = (a_{ij})_{1 \leq i, j \leq n}\) alors
    \[\boxed{X^T A Y = \sum_{1 \leq i, j \leq n} a_{ij} x_i y_j}\]
    \[\boxed{e_i^T A e_j = a_{ij}}\]
    En particulier, si pour tout \(X, Y \in K^n\), \(X^T A Y = X^T B Y\) alors \(A = B\)
\end{proposition}

\section{Forme bilinéaire symétrique}
\subsection{Généralités}
\begin{definition}
    Soit \(E\) un \(K\)-ev et \(\varphi: E \times E \to K\) une fbs ( forme bilinéaire symétrique ) \\
    La forme quadratique associée à \(\varphi\) est
    \[q: \begin{cases}
        E \to K \\
        x \mapsto q(x) = \varphi(x, x)
    \end{cases}\]
    \(\varphi\) est la forme polaire de \(q\)
\end{definition}
\begin{proposition}[ Identité de polarisation ]
    Avec ces notations
    \begin{empheq}[box=\fbox]{align*}
        \varphi(x, y) &= \frac{1}{2}\left(q(x + y) - q(x) - q(y)\right) \\
        &= \frac{1}{2} \left(q(x) + q(y) - q(x - y)\right) \\
        &= \frac{1}{4} \left(q(x + y) - q(x - y)\right)
    \end{empheq}
    À une forme quadratique ne correspond qu'une unique forme polaire.
\end{proposition}

\subsection{Expression matricielle}
\begin{definition}
    Soit \(E\) un \(K\)-ev de dimension finie, \(\mathcal{B}=(e_1, ...,\, e_n)\) une base de \(E\) et \(\varphi: E \times E \to K\) fbs et \(q\) sa forme quadratique. \\
    On pose \(\mat\limits_{\mathcal{B}}(\varphi) = \mat\limits_{\mathcal{B}}(q) = (\varphi(e_i, e_j))_{1 \leq i, j \leq n} \in S_n(K)\)
\end{definition}
\begin{proposition}
    Avec ces notations, si \(x, y \in E\) de colonnes \(X\) et \(Y\) dans la base \(\mathcal{B}\) et \(A = \mat\limits_{\mathcal{B}}(\varphi)\) on a
    \[\boxed{\varphi(x, y) = X^T A Y = Y^T A X}\]
\end{proposition}
\begin{definition}
    Soit \(A \in S_n(K) = \{ M \in M_n(K) \mid M^T = M \}\) \\
    \(\varphi_A: (X, Y) \in K \times K \mapsto X^T A Y \) est une fbs sur \(K^n\) appelée fbs canoniquement associée à \(A\)
\end{definition}

\pagebreak

\begin{proposition}
    Soit \(E\) \(K\)-ev, \(\varphi\) une fbs sur \(E\) de dim. finie. \\
    Soit \(\mathcal{B},\, \mathcal{C}\) deux bases et \(A = \mat\limits_{\mathcal{B}}(\varphi),\, B = \mat\limits_{\mathcal{C}}(\varphi),\, P = \mat(\mathcal{B}, \mathcal{C})\) \\
    Alors \[\boxed{B = P^T A P}\]
\end{proposition}
\begin{definition}
    Soit \(A, B \in S_n(K)\) \\
    On dit que \(A\) et \(B\) sont congruentes s'il existe \(P \in GL_n(K)\) avec \(B = P^T A P\)
\end{definition}
\begin{definition}
    Soit \(\varphi\) une fbs sur \(E\) de dim. finie. \\
    Le rang de \(\varphi\) ( ou de \(q\) sa forme quadratique ) est \(\rg \varphi = \rg A\) où \(A = \mat\limits_{\mathcal{B}}(\varphi)\) avec \(\mathcal{B}\) une base de \(E\)
\end{definition}

\subsection{Orthogonalité selon un fbs}
\begin{definition}
    Soit \(\varphi\) une fbs sur \(E\) \(K\)-ev \\
    \(x, y \in E\) sont dits orthogonaux pour \(\varphi\) si \(\varphi(x, y) = 0\). On écrit \(x \perp y\) ou \(x \perp^\varphi y\) \\
    On dit que \(x \in E\) est isotrope si \(x \perp x\) ie. \(\varphi(x, x) = 0\) \\
    Si \(A \subset E\) 
    \[A^\perp = A^{\perp(\varphi)} = \{x \in E \mid \forall a \in A,\, \varphi(x, a) = 0 \}\]
    Le noyau de \(\varphi\) est 
    \[E^\perp = \{x \in E \mid \forall y \in E,\, \varphi(x, y) = 0\}\]
    On dit que \(\varphi\) est non dégénérée si \(E^\perp = \{ 0 \}\)
\end{definition}
\begin{proposition}
    Soit \(\varphi\) une fbs sur \(E\) de dim. finie et \(A = \mat\limits_{\mathcal{B}}(\varphi)\) avec \(\mathcal{B}\) une base de \(E\) \\
    Alors \[\boxed{\varphi \text{ non dégénérée } \iff A \text{ inversible }}\]
\end{proposition}
\begin{proposition}
    Soit \(E\) un \(K\)-ev de dim. finie et \(\varphi\) une fbs sur \(E\) \uline{non dégénérée}. \\
    Alors \[H: \begin{cases}
        E \to E^* \\
        x \mapsto \varphi(x, \cdot)
    \end{cases} \quad \text{ avec } \quad \varphi(x, \cdot): \begin{cases}
        E \to K \\
        y \mapsto \varphi(x, y)
    \end{cases}\]
    est un isomorphisme: \(\forall l \in E^*,\, \exists! x \in E,\, \forall y : l(y) = \varphi(x, y)\)
\end{proposition}

\subsection{Bases orthogonales}
\begin{definition}
    Soit \(\varphi\) une fbs sur \(E\) de dim. finie. \\
    \((e_1, ...,\, e_n) = \mathcal{B}\) base de \(E\) est dite orthogonale si \(\forall i \neq j,\, \varphi(e_i, e_j) = 0\) \\
    \(\mathcal{B}\) est une base orthogonale \(\iff\) \(\mat\limits_{\mathcal{B}}(\varphi) \in D_n(K)\) \\
    Dans ces conditions \(\rg(\varphi)\) est le nombre de coefficients différents de 0 sur la diagonale.
\end{definition}
\begin{theorem}
    Soit \(\varphi\) une fbs sur \(E\) de dimension finie. \\
    Alors \(\varphi\) possède une base orthogonale.
\end{theorem}
\begin{lemma}
    Si \(x \in E\) est non isotrope, alors \(E = Kx \oplus (Kx)^\perp\)
\end{lemma}

\pagebreak

\begin{corollary}
    \hfill
    \begin{enumerate}
        \item Soit \(E\) un \(K\)-ev de dim. finie \(n\), \(\varphi\) une fbs sur \(E\) \\
        Il existe une base \((e_1, ...,\, e_n)\) orthogonale de \(E\) telle que
        \[\varphi(x, y) = \lambda_1 x_1 y_1 + ... + \lambda_r x_r y_r \quad \text{ où } \quad x = \sum_{i = 1}^{n} x_i e_i,\, y = \sum_{i = 1}^{n} y_i e_i\]
        \[q(x) = \lambda_1 x_1^2 + ... + \lambda_r x_r^2 \]
        avec \(\lambda_1, ...,\, \lambda_r \in K^*\) et \(r = \rg \varphi\)
        \item Si \(A \in S_n(K)\) alors il existe \(P \in GL_n(K)\) tel que
        \[P^{-1}AP = \begin{pmatrix}
            \lambda_1 & & & & & & \\
            & \lambda_2 & & & & (0) & \\
            & & \ddots & & & & \\
            & & & \lambda_r & & & \\
            & & & & 0 & & \\
            & (0) & & & & \ddots & \\
            & & & & & & 0
        \end{pmatrix} \in D_n(K)\]
        avec \(\lambda_i \in K^*\) et \(r = \rg A\)
    \end{enumerate}
\end{corollary}

\section{Formes positives, produit scalaire}
\subsection{Matrices positives}
\noindent Ici \(E\) est \(\mathbb{R}\)-ev
\begin{definition}
    Soit \(A \in S_n(\mathbb{R})\) \\
    On dit que \(A\) est positive si \(\varphi_A\) est une fbs positive: \(\forall x \in \mathbb{R}^n,\ X^T A X \geq 0\) \\
    On note \[S_n^+(\mathbb{R}) = \{A \in S_n(\mathbb{R}) \mid A \text{ positive }\}\]
    On dit que \(A\) est définie positive si \(\varphi_A\) est un produit scalaire: \(\forall X \in \mathbb{R}^n \setminus \{0\},\, X^T A X > 0\) \\
    On note \[S_n^{++}(\mathbb{R}) = \{A \in S_n(\mathbb{R}) \mid A \text{ définie positive }\}\]
\end{definition}

\subsection{Exemples d'espaces préhilbertiens réels}
\subsubsection{Espaces préhilbertiens fonctionnels}
\noindent \(E = \mathcal{C}([a, b], \mathbb{R})\) muni de
\[ \langle f, g \rangle = \int_{a}^{b} fg\]

\subsubsection{Espace de Legendre}
\noindent \(E = \mathcal{C}([-1, 1], \mathbb{R})\) muni de 
\[\langle f, g \rangle = \int_{-1}^{1} fg\]

\subsubsection{Espace d'Hermite}
\noindent \(E = \left\{ f: \mathbb{R} \to \mathbb{R} \mid f \text{ continue et } x \mapsto f(x)^2 e^{-x^2} \text{ intégrable }\right\}\) muni de 
\[\langle f, g \rangle = \int_{\mathbb{R}} f(x) g(x) e^{-x^2} \,dx\]
\(E\) contient en particulier les fonctions polynomiales.
\begin{lemma}
    Si \(f, g \in L^2(I, \mathbb{K})\) alors \(fg \in L^1(I, \mathbb{K})\)
\end{lemma}

\subsubsection{Espace de Laguerre}
\noindent \(E = \left\{ f: \mathbb{R_+} \to \mathbb{R} \mid f \text{ continue et } x \mapsto f(x)^2 e^{-x} \text{ intégrable sur } \mathbb{R}_+ \right\}\) muni de
\[\langle f, g \rangle = \int_{\mathbb{R}_+} f(x) g(x) e^{-x} \,dx\]

\subsection{Théorème de représentation des formes linéaires}
\begin{corollary}
    Soit \(E\) un espace euclidien. \\
    Tout \(l \in E^*\) s'écrit de manière unique
    \[l: \begin{cases}
        x \mapsto \langle e, x \rangle \\
        E \to \mathbb{R}
    \end{cases}\]
    avec \(e \in E\)
\end{corollary}

\subsection{Orthogonalité dans les espaces préhilbertiens}
\begin{proposition}[ Pythagore ]
    Soit \(x, y \in E\) \\
    Alors 
    \[x \perp y \iff \lVert x + y \rVert^2 = \lVert x \rVert^2 + \lVert y \rVert^2\]
\end{proposition}
\begin{proposition}
    \hfill
    \begin{enumerate}
        \item \(A \subset B \subset C \implies B^\perp \subset A^\perp\)
        \item \(A \subset A^{\perp \perp}\)
        \item Si \(F = \vect A\) alors \(A^\perp = F^\perp\)
    \end{enumerate}
\end{proposition}
\begin{proposition}
    Si \((E_i)_{i \in I}\) famille de sev de \(E\) 2 à 2 \(\perp\) alors \(\sum\limits_{i \in I} E_i\) est directe, on la note
    \[\overset{\perp}{\bigoplus_{i \in I}} E_i\]
\end{proposition}
\begin{theorem}
    Soit \(E\) un espace préhilbertien et \(F\) un sev de \uline{dimension finie}. \\
    Alors:
    \begin{enumerate}
        \item \(F \oplus F^\perp = E\)
        \item \(F^{\perp \perp} = F\)
    \end{enumerate}
    On définit la projection orthogonale sur \(F\) par  \(p_F = p_{F, F^\perp}\) \\
    Si \((e_1, ...,\, e_p)\) est une BON de \(F\) alors
    \[\boxed{p_F(x) = \sum_{i = 1}^{p} \langle e_i, x \rangle e_i}\]
\end{theorem}
\begin{proposition}
    Si \(E\) est préhilbertien, \(F\) sev de dim. finie, on peut considérer: \\
    \(p_F = p_{F, F^\perp} =\) projection orthogonale sur \(F\) \\
    \(s_F = s_{F, F^\perp}: x = x_F + x_{F^\perp} \mapsto x_F - x_{F^\perp} =\) symétrie orthogonale par rapport à \(F\)
    \[\boxed{s_F = 2p_F - \text{ Id }}\]
\end{proposition}
\begin{proposition}
    Soit \(E\) un espace euclidien et \((e_1, ...,\, e_p)\) un système orthonormé ( SON ). \\
    On peut compléter \((e_1, ...,\, e_p)\) en une BON \((e_1, ...,\ e_n)\) de \(E\)
\end{proposition}
\begin{theorem}[ Orthonormalisation au sens de Gram-Schmidt ]
    \hfill \\
    Soit \(N = \llbracket 1, n \rrbracket\) ou \(\mathbb{N}^*\) et \((a_i)_{i \in \mathbb{N}}\) un système libre de \(E\), espace préhilbertien réel. \\
    Alors il existe un unique système \((e_i)_{i \in N}\) tel que:
    \begin{enumerate}
        \item \((e_i)_{i \in N}\) est un système orthonormé.
        \item \(\forall k \in N\), \(\vect(e_1, ...,\, e_k) = \vect(a_1, ...,\, a_k)\)
        \item \(\forall k \in N\), \(\langle e_k, a_k \rangle > 0\) 
    \end{enumerate}
    \((e_i)_{i \in N}\) est appelé orthonormalisé au sens de Gram-Schmidt des \(a_i\)
\end{theorem}

\subsection{Distance à un sous-espace de dimension finie}
\begin{theorem}[ Inégalité de Bessel ]
    Soit \(E\) un espace préhilbertien réel et \(F\) un sev de dimension finie de \(E\) \\
    Soit \((e_1, ...,\, e_p)\) une BON de \(F\), \(p = \dim F\) est \(x \in F\) \\
    Alors \(d(x, F)\) est atteinte en un unique point, le projeté orthogonal de \(x\) sur \(F\), \(p_F(x)\) \\
    De plus
    \[p_F(x) = \langle e_1, x \rangle e_1 + ... + \langle e_p, x \rangle e_p\]
    \[d(x, F) = \sqrt{\lVert x \rVert^2 - \sum_{i = 1}^{p} \langle e_i, x \rangle^2}\]
    Et en particulier
    \[\sum_{i = 1}^{p} \langle e_i, x \rangle^2 \leq \lVert x \rVert^2\]
\end{theorem}

\subsection{Adaptation aux espaces préhilbertiens complexes}
\begin{definition}
    Un produit scalaire hermitien sur \(E\) un \(\mathbb{C}\)-ev est:
    \begin{enumerate}
        \item linéaire à droite: \(\forall x,\, y \mapsto \langle x, y \rangle\) linéaire.
        \item semi-linéaire à gauche: \(\forall y ,\, \langle x + \lambda x', y \rangle = \langle x, y \rangle + \overline{\lambda} \langle x', y \rangle\) et \(\langle x, y \rangle = \overline{\langle y, x \rangle}\)
        \item \(\forall x \neq 0,\, \langle x, x \rangle > 0\)
    \end{enumerate}
\end{definition}
\begin{definition}
    \(M \in M_n(\mathbb{C})\) est dite hermitienne si \(M^* = M\), ce qui signifie \(\forall k, l :\, \overline{M_{k, l}} = M_{l, k}\)
\end{definition}
\begin{proposition}
    On garde les résultats suivants:
    \begin{enumerate}
        \item Existence du BON dans \(E\) hermitien.
        \item \(F\) sev de dim. finie de \(E\) préhilbertien complexe: \begin{itemize}
            \item \(F \oplus F^\perp = E\) et \(F^{\perp \perp} = F\)
            \item \(p_F(x) = \sum\limits_{k = 1}^{n} \langle e_k, x \rangle e_k\) si \((e_1, ...,\, e_n)\) BON de \(F\)
            \item \(\lVert p_F(x) \rVert^2 = \sum\limits_{k = 1}^{n} | \langle e_k, x \rangle |^2 \leq \lVert x \rVert^2\)
        \end{itemize}
        \item ON au sens de GS toujours valable.
        \item \(\langle \lambda x, \lambda x \rangle = \lambda \overline{\lambda} \langle x, x \rangle = |\lambda|^2 \langle x, x \rangle \)
    \end{enumerate}
\end{proposition}

\section{Isométries vectorielles et matrices orthogonales}
\subsection{Isométries vectorielles d'un espace euclidien}
\begin{definition}
    Soit \(E\) un espace euclidien et \(u \in \mathcal{L}(E)\) \\
    On dit que \(u\) est un isomorphisme orthogonal ou encore isométrie vectorielle si \(u\) conserve le produit scalaire:
    \[\forall x ,y \in E \quad \langle u(x), u(y) \rangle = \langle x, y \rangle\]
    ( ie. u conserve la norme )
    \[\lVert u(x) \rVert  = \lVert x \rVert\]
    On note \(O(E)\) l'ensemble des isométries vectorielles de \(E\)
\end{definition}
\begin{proposition}
    \(O(E) \subset GL(E)\) \\
    C'est un sous-groupe de \(GL(E)\) appelé groupe orthogonal de \(E\)
\end{proposition}
\begin{proposition}
    \[\boxed{u \in O(E) \implies \Sp u \subset \{-1, 1\}}\]
    De plus
    \[\boxed{\ker(u - \text{ Id }) \perp \ker(u + \text{ Id})}\]
\end{proposition}
\begin{proposition}
    Soit \(E\) euclidien, \(u \in O(E)\) \\
    Si \(F\) est un sev stable par \(u\) alors \(F^\perp\) est stable par \(u\)
\end{proposition}
\begin{proposition}
    Soit \(E\) un espace euclidien, \((e_1, ...,\, e_n)\) une BON et \(u \in \mathcal{L}(E)\) \\
    Alors \[\boxed{u \in O(E) \iff (u(e_1), ...,\, u(e_n)) \text{ BON }}\]
\end{proposition}

\subsection{Matrices orthogonales}
\begin{definition}
    Soit \(A \in M_n(\mathbb{R})\) \\
    ON dit que \(A\) est orthogonale si les colonnes de \(A\) forment une BON de \(\mathbb{R}^n\) pour le produit scalaire canonique. \\
    On note \(O(n)\) l'ensemble des matrices orthogonales de \(M_n(\mathbb{R})\)
\end{definition}
\begin{proposition}
    Si \(A \in M_n(\mathbb{R})\)
    \begin{empheq}[box=\fbox]{align*}
        A \in O(n) &\iff A^T A = I_n \\
        &\iff A A^T = I_n \\
        &\iff \text{ les lignes de } A \text{ forment une BON de } \mathbb{R}^n \\
        &\iff A \in GL_n(\mathbb{R}) \text{ et } A^{-1} = A^T
    \end{empheq}
    \(O(n)\) est un sous-groupe de \(GL_n(\mathbb{R})\) appelé groupe orthogonal d'ordre \(n\)
\end{proposition}
\begin{proposition}
    Soit \(u \in \mathcal{L}(E)\), \(E\) espace euclidien, \(\mathcal{B}\) une BON de \(E\) et \(A = \mat\limits_{\mathcal{B}}(u)\) \\
    Alors
    \[\boxed{u \in O(E) \iff A \in O(n)}\]
\end{proposition}
\begin{proposition}
    Si \(\mathcal{B}\) BON de \(E\) et \(P = \mat(\mathcal{B}, \mathcal{B}')\) alors
    \[\boxed{P \in O(n) \iff \mathcal{B}' \text{ BON }}\]
\end{proposition}

\pagebreak

\begin{proposition}
    Si \(A \in O(n)\), \(u_A \in O(\mathbb{R}^n)\) alors
    \[\boxed{\Sp A = \Sp u_A \subset \{-1, 1\}}\]
    et \[\boxed{\ker(A - I_n) \perp \ker(A + I_n)}\]
\end{proposition}

\subsection{Groupe spécial orthogonal}
\begin{proposition}
    Si \(E\) euclidien, \(u \in O(E)\) alors \(\det(u) = \pm 1\)
\end{proposition}
\begin{definition}
    Soit \(E\) euclidien et \(n \in \mathbb{N}^*\) \\
    On note \(SO(n) = SL_n(\mathbb{R}) \cap O(n) = \{A \in O(n) \mid \det A = 1\}\) le groupe spécial orthogonal d'ordre \(n\) \\
    \(SO(E) = SL_n(E) \cap O(E)\) est le groupe orthogonal de \(E\)
\end{definition}
\begin{definition}
    Si \(n \in SO(E)\) on dit que \(n\) est une rotation de \(E\) \\
    Si \(A \in SO(n)\) on dit que \(A\) est une matrice orthogonale positive. \\
    Si \(u \in O(E) \setminus SO(E)\), \(u\) est une "antirotation".
\end{definition}
\begin{definition}
    Soit \(E\) \(\mathbb{R}\)-ev de dim. finie. \\
    Choisir une orientation de \(E\) c'est décréter une base \(\mathcal{B}_0\) directe. \\
    Si \(\mathcal{B}\) est une autre base, \(\mathcal{B} \text{ directe } \iff \det \mat(\mathcal{B}_0, \mathcal{B}) > 0\)
\end{definition}
\begin{definition}
    Soit \(E\) un espace euclidien orienté et \(\mathcal{B}\) une base orthonormé directe, \(n = \dim E\) \\
    On appelle produit mixte de \((x_1, ...,\, x_n) \in E^n\) le déterminant \([x_1, ...,\, x_n] = \det\limits_{\mathcal{B}}(x_1, ...,\, x_n)\) \\
    Il est indépendant de la base \(\mathcal{B}\) orthonormée directe.
\end{definition}

\subsection{Groupe orthogonal d'un plan euclidien orienté}
\begin{definition}
    Si \(\theta \in \mathbb{R}\) on va noter 
    \[R_\theta = \begin{pmatrix}
        \cos \theta & - \sin \theta \\
        \sin \theta & \cos \theta
    \end{pmatrix}\]
    Si \(\theta, \theta' \in \mathbb{R}\), \(R_\theta R_{\theta'} = R_{\theta + \theta'} = R_{\theta'}R_\theta\)
\end{definition}
\begin{theorem}
    Soit \(f: \begin{cases}
        (\mathbb{R}, +) \to (SO(2), \times) \\
        \theta \mapsto R_\theta
    \end{cases}\) \\
    \(f\) est un morphisme surjectif de groupes de noyau \(\ker f = 2 \pi \mathbb{Z}\) \\
    En particulier \(SO(2) \approx \mathbb{R} / 2 \pi \mathbb{Z} \)
\end{theorem}
\begin{definition}
    Soit \((\vec{i}, \vec{j})\) une BON directe de \(P\) et \(\theta \in \mathbb{R}\) \\
    On rappelle rotation d'angle \(\theta\) de \(P\) l'endomorphisme \(r_\theta \in \mathcal{L}(P)\) tel que \(\mat\limits_{(\vec{i}, \vec{j})} r_\theta = R_\theta\) \\
    \(r_\theta\) est indépendant de la base orthonormée directe choisie.
\end{definition}
\begin{proposition}
    Soit \(P\) un plan euclidien orienté et \(u \in SO(P)\) \\
    Alors il existe \(\theta \in \mathbb{R}\) tel que \(u = r_\theta\) \\
    En particulier \(SO(P)\) est un groupe abélien et \(SO(P) \approx SO(2) \approx \mathbb{R} / 2 \pi \mathbb{Z}\)
\end{proposition}
\begin{proposition}
    Soit \(P\) un plan euclidien, \(u \in O(P) \setminus SO(P)\) ( "antirotation" ). \\
    Alors \(u\) est une réflexion ie. une symétrie orthogonale par rapport à une droite \(\Delta : u = s_\Delta\)
\end{proposition}
\begin{proposition}
    \[\boxed{O(P) = \{ \text{ rotation d'angle } \theta \in \mathbb{R},\, \text{ symétrie orthogonale } s_\Delta\}}\]
\end{proposition}

\section{Réduction des matrices orthogonales}
\subsection{Cas général}
\begin{theorem}
    Soit \(E\) un espace euclidien et \(u \in O(E)\) \\
    Si \(\theta \in \mathbb{R}\), \(R_\theta = \begin{pmatrix}
        \cos \theta & - \sin \theta \\
        \sin \theta & \cos \theta
    \end{pmatrix}\) alors il existe une base orthonormée de \(E\) telle que
    \[\mat_{\mathcal{B}}(u) = \begin{pmatrix}
        \boxed{R_{\theta_1}} & & & & & & & & \\
        & \ddots & & & & & & & \\
        & & \boxed{R_{\theta_p}} & & & & & & \\
        & & & -1 & & & & & \\
        & & & & \ddots & & & & \\
        & & & & & -1 & & & \\
        & & & & & & 1 & & \\
        & & & & & & & \ddots & \\
        & & & & & & & & 1
    \end{pmatrix} \text{ avec } \theta_i \not\equiv 0 [\pi]\]
\end{theorem}
\begin{corollary}
    Soit \(A \in O(n)\) \\
    Il existe \(P \in O(n)\) telle que
    \[P^{-1} A P = P^T A P = \begin{pmatrix}
        \boxed{R_{\theta_1}} & & & & & & & & \\
        & \ddots & & & & & & & \\
        & & \boxed{R_{\theta_p}} & & & & & & \\
        & & & -1 & & & & & \\
        & & & & \ddots & & & & \\
        & & & & & -1 & & & \\
        & & & & & & 1 & & \\
        & & & & & & & \ddots & \\
        & & & & & & & & 1
    \end{pmatrix} \text{ avec } \theta_i \not\equiv 0 [\pi]\]
\end{corollary}
\begin{proposition}
    Les réflexion engendrent \(O(E)\)
\end{proposition}

\subsection{Rotation en dimension 3}
\begin{theorem}
    Soit \(M \in SO(3)\) \\
    \(M\) est orthogonalement semblable à une matrice
    \[A_\theta = \begin{pmatrix}
        \cos \theta & - \sin \theta & 0 \\
        \sin \theta & \cos \theta & 0 \\
        0 & 0 & 1
    \end{pmatrix} \text{ avec } \theta \in \mathbb{R}\]
    \[\boxed{\exists P \in O(n) \quad P^{-1} M P = P^T M P = A_\theta}\]
\end{theorem}

\pagebreak

\begin{definition}
    Soit \(E\) un espace euclidien de dim. 3 \\
    Soit \(\vec{k}\) un vecteur unitaire, \(\Delta = \mathbb{R} \vec{k}\), \(p = \vec{k^\perp}\) orienté par \((\vec{i}, \vec{j})\) base de \(P\) avec \(\vec{i}n \vec{j}, \vec{k}\) directe orthonormée. \\
    La rotation d'axe \(\Delta\) orienté par \(\vec{k}\) et d'angle \(\theta\) est l'endomorphisme \(r_{\theta, \vec{k}}\) tel que
    \[\mat_{(\vec{i}, \vec{j}, \vec{k})}(r_{\theta, \vec{k}}) = A_\theta = \begin{pmatrix}
        \cos \theta & - \sin \theta & 0 \\
        \sin \theta & \cos \theta & 0 \\
        0 & 0 & 1
    \end{pmatrix}\]
\end{definition}
\begin{theorem}
    Soit \(u \in SO(E)\), \(E\) euclidien orienté de dimension 3. \\
    Il existe \(\theta \in \mathbb{R}\) et \(\vec{k}\) unitaire tel que
    \[\boxed{u = r_{\theta, \vec{k}}}\]
\end{theorem}

\section{Exercices classiques}
\subsection{Projecteur \(p\) tel que \(\vvvert p \vvvert = 1\)}
\noindent Soit \(E\) un espace euclidien, \(p\) un projecteur. On suppose que \(\forall x \in E\), \(\lVert p(x) \rVert \leq \lVert x \rVert\) \\
Montrer que \(p\) est un projecteur orthogonal.

\subsection{Décomposition \(QR\)}
\noindent Soit \(A \in GL_n(\mathbb{R})\) \\
Montrer que il existe un unique couple \((Q, R)\) avec: \\
\(\begin{cases}
    A = QR \\
    Q \in O(n) \\
    R \text{ est une matrice triangulaire à coefficients diagonaux} > 0
\end{cases}\)

\subsection{Décomposition de Cholevski - Inégalité d'Hadamard}
\begin{enumerate}
    \item Soit \(A \in S_n^{++}(\mathbb{R})\) \\
    Montrer qu'il existe une unique \(B \in M_n(\mathbb{R})\) triangulaire supérieure à coefficients diagonaux \(> 0\) tel que \(A = B^T B\) ( Cholevski )
    \item Montrer que si \(A \in S_n^{++}(\mathbb{R})\), \(A = (a_{ij})_{1 \leq i, j \leq n}\) alors \\
    \(0 \leq \det A \leq a_{11} a_{22} ... a_{nn}\) ( Hadamard )
\end{enumerate}

\subsection{Matrices de Gram - Inégalité d'Hadamard}
\noindent Soit \(E\) un espace préhilbertien, \(x_1, ...,\, x_n \in E\) \\
La matrice de Gram de \(x_1, ...,\, x_n\) est \(A = (\langle x_i, x_j \rangle)_{1 \leq i, j \leq n} = G(x_1, ...,\, x_n) \in S_n(\mathbb{R})\)
\begin{enumerate}
    \item Monter que \(A \in S_n^+(\mathbb{R})\) et même \((x_1, ...,\, x_n) \text{ libre } \implies A \in S_n^{++}(\mathbb{R})\) \\
    Soit \(F = \vect(x_1, ...,\, x_n)\), \((e_1, ...,\, e_r)\) une base orthonormée de \(F\) et \(P = \mat\limits_{(e_1, ...,\, e_r)}(x_1, ...,\, x_n)\)
    \item Montrer que \(A = G(x_1, ...,\, x_n) = P^T P\) et en déduire que \(\rg A = \rg(x_1, ...,\, x_n)\) \\
    On suppose que \((x_1, ...,\, x_n)\) libre \((r = n)\) et \((e_1, ...,\, e_n) = \mathcal{B}_0\) est l'ON au sens de Gram-Schmidt de \((x_1, ...,\, x_n)\)
    \item Montrer que \(|\det_{\mathcal{B}_0}(x_1, ...,\, x_n )| = \sqrt{\det G(x_1, ...,\, x_n)}\)
    \item Montrer que \(\det(G(x_1, ...,\, x_n)) \leq \lVert x_1 \rVert^2 ... \lVert x_n \rVert^2\) et préciser le cas d'égalité.
    \item Soit \(M \in M_n(\mathbb{R})\) \(M = \left( C_1 \mid ... \mid C_n\right)\) \\
    Montrer que \(|\det M| \leq \lVert C_1 \rVert \times ... \times \lVert C_n \rVert\) ( norme euclidienne canonique ) et le cas d’égalité quand \(M\) est inversible.
    \item Soit \(x \in E\). Montrer que \(d(x, F)^2 = \frac{\det G(x_1, ...,\, x_n, x)}{\det G(x_1, ...,\, x_n)}\)
\end{enumerate}

\subsection{Racines de polynômes orthogonaux}
\noindent Soit \(\mu: [a, b] \to \mathbb{R}_+^*\) avec \(a < b\). Sur \(\mathbb{R}[X]\) on pose \(\langle P, Q \rangle = \int\limits_{a}^{b} PQ \mu\)
\begin{enumerate}
    \item Monter qu'il existe une unique suite \((P_n)_{n \geq 0}\) de \(\mathbb{R}[X]\) orthonormés telle que \(\deg P_n = n\) \\
    Que dire de la suite \((Q_n)\) avec \(\deg Q_n = n\) et les \(Q_n\) 2 à 2 \(\perp\)?
    \item Monter que les \(P_n\) sont scindés à racines simples toutes dans \(\left] a, b \right[\)
    \item Il existe \((a_n),(b_n),(c_n)\) trois suites réelles avec \(\forall n \geq 0 :\, X P_{n + 1} = a_n P_{n + 2} + b_n P_{n + 1} + c_n P_n\)
\end{enumerate}
\end{document}