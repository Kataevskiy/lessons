\documentclass[10pt,a4paper]{article}
\usepackage[utf8]{inputenc}
\usepackage[french]{babel}
\usepackage[T1]{fontenc}
\usepackage{amsmath}
\usepackage{amsfonts}
\usepackage{amssymb}
\usepackage{graphicx}
\usepackage[left=2cm,right=2cm,top=2cm,bottom=2cm]{geometry}
\usepackage{setspace}
\usepackage{ulem}
\usepackage{stmaryrd}
\usepackage{amsthm}
\usepackage{dsfont}
\usepackage{mathpazo}

\onehalfspacing

\theoremstyle{definition}
\newtheorem{proposition}{Proposition}[section]
\newtheorem{theorem}[proposition]{Théorème}
\newtheorem{corollaire}[proposition]{Corollaire}
\newtheorem{lemme}[proposition]{Lemme}
\newtheorem{definition}[proposition]{Définition}

\DeclareMathOperator*{\negl}{o}
\DeclareMathOperator*{\dom}{O}
\DeclareMathOperator*{\eqv}{\sim}

\begin{document}
\renewcommand{\labelitemi}{$*$}
\begin{center}
{\Large \textbf{Chapitre 23: Séries}}
\end{center}

\section{Généralités}
\subsection{Définition}
\begin{definition}
Soit $(u_n)_{n \in \mathbb{N}} \in \mathbb{C}^{\mathbb{N}}$ \\
On dit que \uline{la série $\sum u_n$ converge} si la suite $\left( \sum\limits_{k = 0}^n u_k \right)_{n \in \mathbb{N}}$ converge, et qu'elle \uline{diverge} sinon.
\begin{itemize}
\item Le nombre $\sum\limits_{k = 0}^n u_k$ est la \uline{n-ième somme partielle} de la serie.
\item Le nombre $u_n$ est le \uline{terme général} de la série.
\item Si la série converge, la \uline{somme} de la série est $\sum\limits_{n = 0}^{+\infty} u_n = \lim\limits_{n \to +\infty} \sum\limits_{k = 0}^{n} u_k$
\end{itemize}
\end{definition}
\begin{proposition}[Linéarité de la somme]
Soit $\sum\limits_n u_n$ est $\sum\limits_n v_n$ deux séries convergentes. \\
Alors $\sum\limits_n (u_n + \lambda v_n)$ converge pour tout $\lambda \in \mathbb{C}$ et
\[\sum\limits_{n = 0}^{+\infty} (u_n + \lambda v_n) = \sum\limits_{n = 0}^{+\infty} + \lambda \sum\limits_{n = 0}^{+\infty} v_n\]
\end{proposition}
\begin{definition}
Soit $\sum\limits_n u_n$ est une série convergente. \\
Le \uline{$n$-ième reste} de la série est la somme $\sum\limits_{k = n + 1}^{+\infty} u_k$
\end{definition}
\begin{proposition}
La suite des restes d'une série convergente converge vers $0$.
\end{proposition}

\subsection{Divergence grossière}
\begin{proposition}
Soit $\sum\limits_n u_n$ une série convergente. \\
Alors $u_n \xrightarrow[n \to +\infty]{} 0$
\end{proposition}
 
\subsection{Critère spécial des séries alternées}
\begin{theorem}
Soit $(u_n)_{n \in \mathbb{N}}$ une suite réelle:
\begin{itemize}
\item décroissante
\item telle que $u_n \xrightarrow[n \to +\infty]{} 0$
\end{itemize}
Alors la série $\sum\limits_n (-1)^n u_n$ converge.
\end{theorem}

\section{Séries à termes positifs}
\begin{definition}
Soit $(u_n)_{n \in \mathbb{N}}$ une suite à valeurs réelles positives. \\
On dit alors que $\sum\limits_n u_n$ est \uline{une série à termes positifs} (SÀTP)
\end{definition}

\subsection{Théorèmes de comparaison}
\begin{theorem}
Soit $\sum\limits_n u_n$ est $\sum\limits_n v_n$ deux SÀTP telles que $u_n \leq v_n$ àpcr. \\
Alors $\sum\limits_n v_n$ converge $\implies$ $\sum\limits_n u_n$ converge. 
\end{theorem}

\pagebreak

\begin{corollaire}
\hfill
\begin{itemize}
\item Soit $\sum\limits_n u_n$ et $\sum\limits_n v_n$ deux SÀTP telles que $u_n = \dom\limits_{n \to +\infty} (v_n)$ \\
Alors $\sum v_n$ converge $\implies$ $\sum u_n$ converge.
\item C'est en particulier le cas si $u_n \eqv\limits_{n \to +\infty} v_n$ ou si $u_n = \negl\limits_{n \to +\infty} (v_n)$
\end{itemize}
\end{corollaire}

\subsection{Comparaison série-intégrale}
\begin{theorem}[Comparaison série/intégrale, cas décroissant]
Soit $f: \mathbb{R}_+^* \to \mathbb{R}$ décroissante et continue par morceaux. \\
Alors pour tout $n \in \mathbb{N}^*$
\[\int\limits_1^{n + 1}f \leq \sum_{k = 1}^n f(k) \leq \int\limits_1^{n + 1}f + f(1) - f(n+1)\]
\end{theorem}
\begin{theorem}
Soit $f: \mathbb{R}_+ \to \mathbb{R}$ continue par morceaux et croissante. \\
Alors pour tout $n \in \mathbb{N}$
\[\int\limits_{0}^{n + 1} f - ( f(n + 1) - f(0)) \leq \sum_{k = 0}^n f(k) \leq \int\limits_{0}^{n + 1} f\]
\end{theorem}
\begin{theorem}[Série de Riemann]
Soit $\alpha \in \mathbb{R}$ \\
Alors la série $\sum\limits_n \frac{1}{n^\alpha}$ converge ssi $\alpha > 1$
\end{theorem}
\noindent \uline{Remarque}: On définit le fonction zêta de Riemann
\[ \zeta: \begin{cases}
\left] 1, +\infty \right[ \to \mathbb{R} \\
s \mapsto \sum\limits_{n = 1}^{+\infty} \frac{1}{n^s}
\end{cases} \]
Par exemple, $\zeta(s) = \frac{\pi^2}{6}$, $\zeta(4) = \frac{\pi^4}{90}$ \medskip

\noindent \uline{Remarque}: On définit le $n$-ième nombre harmonique $H_n = \sum\limits_{k = 1}^n \frac{1}{k}$ \\
La démonstration du théorème précédent donne
\[ H_n = \ln(n) + o(1)\]
Plus précisément
\[ H_n = \ln(n) + \gamma + o(1)\]
avec $\gamma = \lim\limits_{n \to +\infty} H_n - ln(n)$

\section{Séries absolument convergentes}
\subsection{Convergence}
\begin{definition}
Soit $(u_n)_{n \in \mathbb{N}} \in \mathbb{C}^\mathbb{N}$ \\
On dit que $\sum\limits_n u_n$ \uline{converge absolument} si $\sum\limits_n |u_n|$ converge.
\end{definition}
\begin{theorem}
Soit $\sum\limits_n u_n$ une série (de terme général complexe) absolument convergente. \\
Alors $\sum\limits_n u_n$ converge.
\end{theorem}
\begin{theorem}
Soit $\sum\limits_n u_n$ et $\sum\limits_n v_n$ deux séries à valeurs complexes. \\
Si $u_n = \dom(v_n)$ et que $\sum\limits_n v_n$ converge absolument, alors $\sum u_n$ converge absolument.
\end{theorem}
\end{document}