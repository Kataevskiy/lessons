\documentclass[10pt,a4paper]{article}
\usepackage[utf8]{inputenc}
\usepackage[french]{babel}
\usepackage[T1]{fontenc}
\usepackage{amsmath}
\usepackage{amsfonts}
\usepackage{amssymb}
\usepackage{graphicx}
\usepackage[left=2cm,right=2cm,top=2cm,bottom=2cm]{geometry}
\usepackage{setspace}
\usepackage{ulem}
\usepackage{stmaryrd}
\usepackage{amsthm}
\usepackage{dsfont}
\usepackage{mathpazo}

\onehalfspacing

\theoremstyle{definition}
\newtheorem{proposition}{Proposition}[section]
\newtheorem{theorem}[proposition]{Théorème}
\newtheorem{corollaire}[proposition]{Corollaire}
\newtheorem{lemme}[proposition]{Lemme}
\newtheorem{definition}[proposition]{Définition}

\begin{document}
\renewcommand{\labelitemi}{$*$}
\renewcommand{\labelenumi}{(\roman{enumi})}
\begin{center}
{\Large \textbf{Chapitre 11. Compléments sur les nombres réels}}
\end{center}

\section{Partie entière}
\subsection{Caractère archimédien des réels et définitions}
\begin{proposition}[Caractère archimédien de $\mathbb{R}$]
On a $\forall x \in \mathbb{R}$, $\exists n \in \mathbb{Z} : n > x$
\end{proposition}
\begin{proposition}
Soit $x \in \mathbb{R}$ \\
Alors il existe un unique entier $n \in \mathbb{Z}$ tel que $n \leq x < n + 1$ \\
Cet entier $n$ est appelé \uline{partie entière (inférieure) de $x$} et noté $\lfloor x \rfloor$
\end{proposition}
\begin{corollaire}[du caractère archimédien]
\hfill
\begin{itemize}
\item On a $\forall \varepsilon > 0$, $\exists n \in \mathbb{N}^* : \frac{1}{n} < \varepsilon$
\item (Propriété d'Archimède): $\forall x, y \in \mathbb{R}_+^*$, $\exists n \in \mathbb{N}: nx > y$
\end{itemize}
\end{corollaire}

\subsection{Premières propriétés}
\begin{proposition}
\hfill
\begin{itemize}
\item On a $\forall x \in \mathbb{R}$, $\lfloor x \rfloor \leq x < \lfloor x \rfloor + 1 \text{ et } x - 1 < \lfloor x \rfloor \leq x$
\item $\forall n \in \mathbb{Z}$, $\forall x \in \mathbb{R}$, $\lfloor x + n \rfloor = \lfloor x \rfloor + n$
\item $\forall x \in \mathbb{R}$, $\forall n \in \mathbb{Z}$, $n \leq x \iff n \leq \lfloor x \rfloor$
\item La fonction $\lfloor \cdot \rfloor$ croît: $\forall x, y \in \mathbb{R}$, $x \leq y \implies \lfloor x \rfloor \leq \lfloor y \rfloor$
\end{itemize}
\end{proposition}

\subsection{Division euclidienne dans $\mathbb{R}$}
\begin{theorem}
Soit $x \in \mathbb{R}$ et $T \in \mathbb{R}_+^*$ \\
Il existe un unique couple $(q, r) \in \mathbb{Z} \times \left[0, T\right[$ tel que $x = qT + r$
\end{theorem}

\section{Vocabulaire de la proximité}
\subsection{Points $\delta$-proches}
\begin{proposition}
Soit $x, y \in \mathbb{R}$ et $\delta \geq 0$ \\
LASSÉ:
\begin{enumerate}
\item $\left| x - y \right| \leq \delta$
\item $x - \delta \leq y \leq x + \delta$
\item $y \in \left[ x - \delta, x + \delta \right]$
\item $y - \delta \leq x \leq y + \delta$
\item $x \in \left[ y - \delta, y + \delta \right]$
\end{enumerate}
Si ces assertions sont vraies, on dit que $x$ et $y$ sont \uline{$\delta$-proches}.
\end{proposition}
\begin{proposition}[Inégalité triangulaire]
Soit $x, y, z \in \mathbb{R}$ et $\delta, \eta \geq 0$ \\
Alors, si $x$ et $y$ sont $\delta$-proches et que $y$ et $z$ sont $\eta$-proches alors $x$ et $z$ sont $(\delta + \eta)$-proches.
\end{proposition}

\subsection{Points adhérents à une partie}
\begin{definition}
Soit $A \subseteq \mathbb{R}$ et $x \in \mathbb{R}$ \\
On dit que $x$ est \uline{adhérent} à $A$ si $\forall \delta > 0$, $\exists a \in A: \left| x - a \right| \leq \delta$ \\
On note $\overline{A}$ ou $\text{Adh}(A)$ et on appelle \uline{adhérence} de $A$ l'ensemble des points adhérents à $A$.
\end{definition}

\subsection{Densité}
\begin{proposition}
Soit $A \subseteq \mathbb{R}$ \\
LASSÉ:
\begin{enumerate}
\item Tout intervalle ouvert non vide rencontre $A$ (a une intersection non vide avec $A$)
\item $\forall x \in \mathbb{R}$, $\forall \delta > 0$, $\exists a \in A: \left| x - a \right| \leq \delta$
\item $\overline{A} = \mathbb{R}$
\end{enumerate}
Quand ces assertions sont vraies, on dit que $A$ est \uline{dense} (dans $\mathbb{R}$)
\end{proposition}
\begin{proposition}
$\mathbb{Q}$ est dense dans $\mathbb{R}$
\end{proposition}
\begin{corollaire}
$\mathbb{R} \setminus \mathbb{Q}$ est dense dans $\mathbb{R}$
\end{corollaire}

\section{Bornes supérieures et inférieures}
\subsection{Définition et existence}
\begin{definition}
\hfill
\begin{itemize}
\item Soit $A \subseteq \mathbb{R}$ une partie non vide et majorée. \\
La \uline{borne supérieure} de $A$ (si elle existe) est le plus petit des majorants de $A$. \\
On la note $\sup(A)$
\item De même, si $A \subseteq \mathbb{R}$ est non vide et minorée, \\
On appelle \uline{borne inférieure} de $A$ le plus grand des minorants de $A$. \\
On la note $\inf(A)$
\end{itemize}
\end{definition}
\begin{theorem}[Propriété de la borne supérieure]
\hfill \\
Toute partie non vide et majorée de $\mathbb{R}$ possède une borne supérieure.
\end{theorem}
\begin{proposition}
Si $A \subseteq \mathbb{R}$ admet un maximum, alors $\sup(A) = \max(A)$
\end{proposition}
\begin{corollaire}
Soit $A \subseteq \mathbb{R}$ non vide et majorée. \\
Si $\sup(A) \notin A$, alors $A$ n'a pas de maximum.
\end{corollaire}

\subsection{Caractérisation et utilisation pratique}
\begin{proposition}
Soit $A \subseteq \mathbb{R}$ non vide et majorée et $S \in \mathbb{R}$ \\
LASSÉ:
\begin{enumerate}
\item $S = \sup(A)$
\item (caractérisation epsilonesque): $S$ majore $A$ et $\forall \varepsilon > 0$, $\exists a \in A: a \geq S - \varepsilon$
\item $S$ majore $A$ et $S \in \overline{A}$
\item (caractérisation sequentielle): $S$ majore $A$ et il existe une suite $(a_n)_{n \in \mathbb{N}} \in A^\mathbb{N}$ telle que $a_n \xrightarrow[n \to +\infty]{} S$
\end{enumerate}
\end{proposition}
\begin{proposition}[Passage à la borne supérieure des inégalités larges]
\hfill \\
Soit $A \subseteq \mathbb{R}$ non vide et majorée et $M \in \mathbb{R}$ \\
Alors $\sup(A) \leq M \iff M$ majore $A$
\end{proposition}

\pagebreak

\section{Compléments}
\subsection{Sous-groupes de $\mathbb{R}$}
\begin{theorem}
Tout sous-groupe de $(\mathbb{R}, +)$ est monogène ou dense.
\end{theorem}

\subsection{Retour sur le caractère archimédien}
\noindent \uline{Rémarque}: la propriété de la borne supérieure entraîne le caractère archimédien de $\mathbb{R}$, càd \\
$\forall x \in \mathbb{R}$, $\exists n \in \mathbb{Z}: n > x$

\subsection{Classification des intervalles}
\begin{theorem}
Soit $I \subseteq \mathbb{R}$ un intervalle. \\
Alors il existe $a, b \in \mathbb{R}$ tel que
\[ I \in \{ \emptyset,\, \{ a \},\, [a, b],\, [a, b[ ,\, ]a, b],\, ]a ,b[,\, [a, +\infty[,\, ]a, +\infty[,\, \left]-\infty, b\right],\, \left]-\infty, b\right[,\, \mathbb{R} \}\]
\end{theorem}
\end{document}