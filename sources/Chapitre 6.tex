\documentclass[10pt,a4paper]{article}
\usepackage[utf8]{inputenc}
\usepackage[french]{babel}
\usepackage[T1]{fontenc}
\usepackage{amsmath}
\usepackage{amsfonts}
\usepackage{amssymb}
\usepackage{graphicx}
\usepackage[left=2cm,right=2cm,top=2cm,bottom=2cm]{geometry}
\usepackage{setspace}
\usepackage{ulem}
\usepackage{stmaryrd}
\usepackage{amsthm}
\usepackage{dsfont}
\usepackage{mathpazo}


\onehalfspacing

\theoremstyle{definition}
\newtheorem{proposition}{Proposition}[section]
\newtheorem{theorem}[proposition]{Théorème}
\newtheorem{corollaire}[proposition]{Corollaire}
\newtheorem{lemme}[proposition]{Lemme}
\newtheorem{definition}[proposition]{Définition}

\begin{document}
\renewcommand{\labelitemi}{$*$}
\begin{center}
{\Large \textbf{Chapitre 6: Relations}}
\end{center}

\begin{definition}
Une \uline{relation} (binaire) sur un ensemble est une partie $\mathrel{\mathcal{R}}$ de $E \times E$ \\
Étant donné $x, y \in E$, on notera $x \mathrel{\mathcal{R}} y$ si $(x, y)$ est élément de $\mathrel{\mathcal{R}}$ et $x \not\mathrel{\mathcal{R}} u$ sinon.
\end{definition}

\section{Relation d'ordre}
\subsection{Généralités}
\begin{definition}
Soit $E$ un ensemble. \\
Une \uline{relation d'ordre} sur $E$ est une relation $\mathrel{\mathcal{R}}$:
\begin{itemize}
\item  \uline{Réflexive}: $\forall x \in E$, $x \mathrel\mathcal{R} x$
\item \uline{Antisymétrique}: $\forall x, y \in E$, $(x \mathrel\mathcal{R} y \text{ et } y \mathrel\mathcal{R} x) \implies x = y$
\item \uline{Transitive}: $\forall x, y, z \in E$, $(x \mathrel\mathcal{R} y \text{ et } y \mathrel\mathcal{R} z) \implies x \mathrel\mathcal{R} z$
\end{itemize}
\end{definition}
\begin{definition}
Soit $(E, \preccurlyeq)$ un ensemble ordonné. \\
On dit que l'ordre $\preccurlyeq$ est \uline{total} si $\forall x, y \in E$, $(x \preccurlyeq y \text{ ou } y \preccurlyeq x)$
\end{definition}
\begin{definition}
Soit $(E, \preccurlyeq)$ un ensemble ordonné.
\begin{itemize}
\item Deux éléments $x, y \in E$ sont dits \uline{comparables} si $x \preccurlyeq y$ ou $y \preccurlyeq x$
\item Une partie $A \subseteq E$ est une \uline{chaîne} si deux éléments quelconques de A sont toujours comparables.
\item Une partie $A \subseteq E$ est une \uline{antichaîne} si deux éléments quelconques de $A$ ne sont jamais comparables.
\end{itemize}
\end{definition}
\begin{definition}
Soit $(E, \preccurlyeq)$ et $(F, \sqsubseteq)$ deux ensembles ordonnées. \\
Une application $f: E \to F$ est dite \uline{croissante} si $\forall x_1, x_2 \in E$, $x_1 \preccurlyeq x_2 \implies f(x_1) \sqsubseteq f(x_2)$
\end{definition}

\subsection{Éléments particuliers}
\begin{definition}
Soit $(E, \preccurlyeq)$ un ensemble ordonné. \\
Soit $A \subseteq E$. On dit que:
\begin{itemize}
\item $A$ est \uline{majoré} s'il existe $M \in E$ tel que $\forall a \in A$, $a \preccurlyeq M$
\item $A$ est \uline{minoré} s'il existe $m \in E$ tel que $\forall a \in A$, $m \preccurlyeq a$
\item $A$ \uline{admet un maximum} s'il existe $M \in A$ tel que $\forall a \in A$, $a \preccurlyeq M$
\item $A$ \uline{admet un minimum} s'il existe $m \in A$ tel que $\forall a \in A$, $m \preccurlyeq a$
\end{itemize}
\end{definition}
\begin{proposition}
Soit $(E, \preccurlyeq)$ un ensemble ordonné et $A \subseteq E$ \\
S'il existe, le maximum (resp. le minimum de $A$) est unique. \\
On le note $\max(A)$ (resp. $\min(A)$).
\end{proposition}
\begin{definition}
Soit $(E, \preccurlyeq)$ un ensemble ordonné et $A \subseteq E$ \\
Un élément $a \in A$ est dit:
\begin{itemize}
\item \uline{Maximal}, s'il n'y a pas d'élément de $A$ qui lui est strictement supérieur, càd si \\
$\forall a' \in A$, $a \preccurlyeq a' \implies a = a'$
\item \uline{Minimal}, si $\forall a' \in A$, $a' \preccurlyeq a \implies a' = a$
\end{itemize}
\end{definition}
\begin{proposition}
Soit $(E, \preccurlyeq)$ un ensemble ordonné et $A \subseteq E$ \\
Alors, si $A$ admet un maximum, $\max(A)$ est l'unique élément maximal de $A$
\end{proposition}

\pagebreak

\section{Relation d'équivalence}
\subsection{Généralités}
\begin{definition}
Une relation $\mathrel\mathcal{R}$ sur $E$ est une relation d'équivalence si elle est:
\begin{itemize}
\item \uline{Réflexive}: $\forall x \in E$, $x \mathrel\mathcal{R} x$
\item \uline{Symétrique}: $\forall x, y \in E$, $x \mathrel\mathcal{R} y \implies y \mathrel\mathcal{R} x$
\item \uline{Transitive}: $\forall x, y, z \in E$, $(x \mathrel\mathcal{R} y \text{ et } y \mathrel\mathcal{R} z) \implies x \mathrel\mathcal{R} z$
\end{itemize}
\end{definition}

\subsection{Classes d'équivalence}
\begin{definition}
Soit $E$ un ensemble muni d'une relation d'équivalence $\sim$ et $x \in E$ \\
On définit la \uline{classe d'équivalence de $x$}
\[ \left[x\right]_\sim = d(x) = \bar{x} = \dot{x} = \left\{ y \in E \mid x \sim y \right\} \]
\end{definition}
\begin{definition}
Soit $E$ un ensemble. \\
Une famille $(A_i)_{i \in I}$ de parties de $E$ est une \uline{partition de $E$} si:
\begin{itemize}
\item $\forall i \in I$, $A_i \neq \emptyset$
\item Les ensembles sont ($2$ à $2$) disjoints: $\forall i, j \in I$, $i \neq j \implies A_i \cap A_j \neq \emptyset$
\item Les ensembles recouvrent $E$, càd $\bigcup\limits_{i \in I} A_i = E$
\end{itemize}
\end{definition}
\begin{proposition}
Soit $E$ un ensemble et $\sim$ une relation d'équivalence sur $E$ \\
Les classes de $\sim$ forment une partition de $E$
\end{proposition}
\begin{lemme}
Soit $x, y \in E$ tels que $x \sim y$ \\
Alors $\left[ x \right] = \left[ y \right]$
\end{lemme}

\subsection{Ensemble quotient}
\begin{definition}
Soit $\sim$ une relation d'équivalence sur un ensemble $E$
\begin{itemize}
\item On appelle \uline{ensemble quotient} l'ensemble $E/\sim$ des classes d'équivalence de $\sim$
\item L'application $\begin{cases}
E \to E / \sim \\
x \mapsto \left[ x \right]_\sim
\end{cases}$ est appelée la \uline{surjection canonique}.
\end{itemize}
\end{definition}
\begin{definition}
Soit $E$ un ensemble muni d'une relation d'équivalence. Soit $f: E \to F$ une application. \\
Notons $\pi: E \to E / \sim$ la surjection canonique. \\
On dit que $f$ \uline{passe} (ou \uline{descend}) au quotient si $\forall x_1, x_2 \in E$, $x_1 \sim x_2 \implies f(x_1) = f(x_2)$ \\
Dans ce cas, il existe une unique application $\bar{f}: E / \sim \to F$ telle que $\bar{f} \circ \pi = f$
\end{definition}

\pagebreak

\subsection{Deux quotients importants}
\noindent \uline{"Construction" de $\mathbb{Q}$ à partir de $\mathbb{N}$ et $\mathbb{Z}$} \\
On munit l'ensemble $\mathbb{Z} \times \mathbb{N}^*$ de la relation $\sim$ définie par: \\
\[ \forall (a_1, b_1), (a_2, b_2) \in \mathbb{Z} \times \mathbb{N}^*,\, (a_1, b_1) \sim (a_2, b_2) \iff a_1 b_2 = a_2 b_1 \]
On "définit" $\mathbb{Q}$ comme l'ensemble quotient $\mathbb{Z} \times \mathbb{N}^* / \sim$ \\
On définit alors 2 lois:
\[ + : \begin{cases}
(\mathbb{Z} \times \mathbb{N}^*) / \sim \times (\mathbb{Z} \times \mathbb{N}^*) / \sim \to (\mathbb{Z} \times \mathbb{N}^*) / \sim \\
(\left[(a_1, b_1)\right]_\sim , \left[(a_2, b_2)\right]_\sim) \mapsto \left[(a_1 b_2 + a_2 + b_1, b_1 b_2)\right]_\sim
\end{cases}\]
\[\cdot: \begin{cases}
(\mathbb{Z} \times \mathbb{N}^*) / \sim \times (\mathbb{Z} \times \mathbb{N}^*) / \sim \to (\mathbb{Z} \times \mathbb{N}^*) / \sim \\
(\left[(a_1, b_1)\right]_\sim , \left[(a_2, b_2)\right]_\sim) \mapsto \left[(a_1 a_2, b_1 b_2)\right]_\sim
\end{cases}\] \medskip

\noindent \uline{"Construction" de $\mathbb{Z} / n\mathbb{Z}$} \\
Soit $n \in \mathbb{N}^*$ \\
On sait que la congruence modulo $n$ est une relation d'équivalence sur $\mathbb{Z}$ \\
On note $\left[ x \right]_n$ la classe d'équivalence de $x$ par cette relation et $\mathbb{Z} / n\mathbb{Z}$ l'ensemble quotient.
\end{document}