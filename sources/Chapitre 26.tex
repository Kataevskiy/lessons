\documentclass[10pt,a4paper]{article}
\usepackage[utf8]{inputenc}
\usepackage[french]{babel}
\usepackage[T1]{fontenc}
\usepackage{amsmath}
\usepackage{amsfonts}
\usepackage{amssymb}
\usepackage{graphicx}
\usepackage[left=2cm,right=2cm,top=2cm,bottom=2cm]{geometry}
\usepackage{setspace}
\usepackage{ulem}
\usepackage{stmaryrd}
\usepackage{amsthm}
\usepackage{dsfont}
\usepackage{mathpazo}


\onehalfspacing

\theoremstyle{definition}
\newtheorem{proposition}{Proposition}[section]
\newtheorem{theorem}[proposition]{Théorème}
\newtheorem{corollaire}[proposition]{Corollaire}
\newtheorem{lemme}[proposition]{Lemme}
\newtheorem{definition}[proposition]{Définition}

\newcommand{\vp}[2]{\left< #1 \mid #2 \right>}

\begin{document}
\renewcommand{\labelitemi}{$*$}
\begin{center}
{\Large \textbf{Chapitre 26: Espaces euclidiens}}
\end{center}
Dans tout le chapitre, le corps des scalaires est $\mathbb{R}$

\section{Généralités}
\subsection{Produit scalaire}
\begin{definition}
Soit $E$ un espace vectoriel (réel). \\
Un \uline{produit scalaire} sur $E$ est une application
\[ \vp{\cdot}{\cdot} : \begin{cases}
E^2 \to \mathbb{R} \\
(u, v) \to \vp{u}{v}
\end{cases}\]
\begin{itemize}
\item \uline{linéaire}
\item \uline{symétrique} (càd $\forall u, v \in E$, $\vp{u}{v} = \vp{v}{u}$)
\item et \uline{définie positive} (càd $\forall u \in E, \vp{u}{u} \geq 0$ et $\forall u \in E$, $\vp{u}{u} = 0 \implies u = 0_E$)
\end{itemize}
Un \uline{espace préhibertien} (réel) est la donnée d'une ev E et d'un produit scalaire sur $E$ \\
Un \uline{espace euclidien} est un espace préhibertien de dimension finie.
\end{definition}

\subsection{Norme euclidienne}
\begin{definition}
Soit $E$ un espace préhibertien.
\begin{itemize}
\item La \uline{norme (euclidienne)} de $u \in E$ est $\lVert u \rVert = \sqrt{\vp{u}{v}}$
\item Le \uline{distance} de $u$ à $v \in E$ est $\text{d}(u, v) = \lVert v - u \rVert$
\end{itemize}
\end{definition}
\begin{theorem}[Inégalité de Cauchy-Schwarz]
Soit $E$ un espace préhibertien et $u, v \in E$ \\
On a
\[\vp{u}{v} \leq \left| \vp{u}{v} \right| \leq \lVert u \rVert \cdot \lVert v \rVert\]
"Le produit scalaire est inférieur au produit des normes"
\end{theorem}
\begin{theorem}
La norme $\lVert \cdot \rVert$ est une \uline{norme}, càd qu'on a: \\
\uline{Positivité}: $\forall u \in E$, $\lVert u \rVert \geq 0$ \\
\uline{Séparation}: $\forall u \in E$, $\lVert u \rVert = 0 \implies u = 0_E$ \\
\uline{Homogénéité}: $\forall u \in E$, $\forall \lambda \in \mathbb{R}$, $\lVert \lambda u \rVert = |\lambda| \cdot \lVert u \rVert$ \\
\uline{Inégalité triangulaire}: $\forall u, v \in E$, $\lVert u + v \rVert \leq \lVert u \rVert + \lVert v \rVert$ \\
\end{theorem}
\end{document}