\documentclass[10pt,a4paper]{article}
\usepackage[utf8]{inputenc}
\usepackage[french]{babel}
\usepackage[T1]{fontenc}
\usepackage{amsmath}
\usepackage{amsfonts}
\usepackage{amssymb}
\usepackage{graphicx}
\usepackage[left=2cm,right=2cm,top=2cm,bottom=2cm]{geometry}
\usepackage{setspace}
\usepackage{ulem}
\usepackage{stmaryrd}
\usepackage{amsthm}
\usepackage{dsfont}
\usepackage{mathpazo}

\onehalfspacing

\theoremstyle{definition}
\newtheorem{proposition}{Proposition}[section]
\newtheorem{theorem}[proposition]{Théorème}
\newtheorem{corollaire}[proposition]{Corollaire}
\newtheorem{lemme}[proposition]{Lemme}
\newtheorem{definition}[proposition]{Définition}

\DeclareMathOperator{\re}{Re}
\DeclareMathOperator{\im}{Im}

\begin{document}
\renewcommand{\labelitemi}{$*$}
\renewcommand{\labelenumi}{(\roman{enumi})}
\begin{center}
{\Large \textbf{Chapitre 13. Limites et continuité}}
\end{center}

\section{Voisinage}
\begin{definition}
Soit $a \in\overline{\mathbb{R}}$ \\
Un \uline{voisinage de $a$}:
\begin{itemize}
\item Si $a \in \mathbb{R}$, est un ensemble qui contient $[a - \delta, a + \delta]$, pour un certain $\delta > 0$
\item Si $a = +\infty$, est un ensemble $\left[A, +\infty\right[$ pour un certain $A \in \mathbb{R}$
\item Si $a = -\infty$, est un ensemble $\left]-\infty, A\right]$ pour un certain $A \in \mathbb{R}$
\end{itemize}
\end{definition}
\begin{lemme}
Soit $V$ un voisinage de $+\infty$ \\
Alors il existe une suite $(v_n)_{n \in \mathbb{N}} \in V^{\mathbb{N}}$ telle que $v_n \xrightarrow[x \to +\infty]{} +\infty$ 
\end{lemme}
\begin{definition}
Soit $f: I \to \mathbb{R}$ et $a \in \overline{\mathbb{R}}$ \\
On dit qu'une propriété (de la fonction $f$) est vraie \uline{au voisinage de $a$} s'il existe un voisinage $V$ de $a$ tel que la propriété soit vraie sur $V \cap I$
\end{definition}

\section{Notion de limite}
\uline{Cadre}: Dans cette section, $f: I \to \mathbb{R}$ est une fonction définie sur une partie $I$ de $\mathbb{R}$ et $a$ est un élément de $I$ ou $\pm\infty$. En pratique, $I$ sera un intervalle et $a$ un point ou une borne de l'intervalle.
\subsection{Limites en $\pm\infty$}
\begin{definition}
Soit $I$ un ensemble non majoré et $f: I \to \mathbb{R}$
On dit que:
\begin{itemize}
\item \uline{$f$ converge vers $l \in \mathbb{R}$ en $+\infty$} si $\forall \varepsilon > 0$, $\exists H \in \mathbb{R}$: $\forall x \in I$, $x \geq H \implies |f(x) - l| \leq \varepsilon$
\item \uline{$f$ tend vers $+\infty$ en $+\infty$} si $\forall A \in \mathbb{R}$, $\exists H \in \mathbb{R}$: $\forall x \in I$, $x \geq H \implies f(x) \geq A$
\item \uline{$f$ tend vers $-\infty$ en $+\infty$} si $\forall A \in \mathbb{R}$, $\exists H \in \mathbb{R}$: $\forall x \in I$, $x \geq H \implies f(x) \leq A$
\end{itemize}
\end{definition}

\subsection{Limites en un réel}
\uline{Cadre}: $a \in \overline{I}$
\begin{definition}
Soit $a \in \overline{I}$ et $f: I \to \mathbb{R}$
\begin{itemize}
\item On dit que \uline{$f$ tend vers $l \in \mathbb{R}$ en $a$} si \\
$\forall \varepsilon > 0$, $\exists \lambda > 0$: $\forall x \in I$, $|x - a| \leq \delta \implies |f(x) - l| \leq \varepsilon$
\item On dit que \uline{$f$ tend vers $+\infty$ en $a$} si \\
$\forall A \in \mathbb{R}$, $\exists \lambda > 0$: $\forall x \in I$, $|x - a| \leq \delta \implies f(a) \geq A$
\item On dit que \uline{$f$ tend vers $-\infty$ en $a$} si \\
$\forall A \in \mathbb{R}$, $\exists \lambda > 0$: $\forall x \in I$, $|x - a| \leq \delta \implies f(a) \leq A$
\end{itemize}
\end{definition}
\begin{proposition}
Soit $a \in I$ \\
Si $f$ admet une limite (dans $\overline{\mathbb{R}}$) en $a$, cette limite est nécessairement $f(a)$
\end{proposition}

\subsection{Variantes}
\end{document}