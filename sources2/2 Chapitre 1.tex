\documentclass[10pt,a4paper]{article}
\usepackage[utf8]{inputenc}
\usepackage[french]{babel}
\usepackage[T1]{fontenc}
\usepackage{amsmath}
\usepackage{amsfonts}
\usepackage{amssymb}
\usepackage{graphicx}
\usepackage[left=2cm,right=2cm,top=2cm,bottom=2cm]{geometry}
\usepackage{setspace}
\usepackage{ulem}
\usepackage{stmaryrd}
\usepackage{amsthm}
\usepackage{dsfont}
\usepackage{mathpazo}


\onehalfspacing

\theoremstyle{definition}
\newtheorem{proposition}{Proposition}[section]
\newtheorem{theorem}[proposition]{Théorème}
\newtheorem{corollaire}[proposition]{Corollaire}
\newtheorem{lemme}[proposition]{Lemme}
\newtheorem{definition}[proposition]{Définition}

\DeclareMathOperator{\com}{com}
\DeclareMathOperator{\re}{Re}
\DeclareMathOperator{\im}{Im}

\begin{document}
\renewcommand{\labelitemi}{$*$}
\begin{center}
{\Large \textbf{Chapitre 1: Structures fondamentales}}
\end{center}
Dans la suite, $\mathbb{K}$ désigne $\mathbb{R}$ ou $\mathbb{C}$

\section{Groupes, anneaux, corps, espaces vectoriels}
\subsection{Structures algébriques usuelles}
\uline{lci} $*$: $\begin{cases}
E \times E \to E \\
(x, y) \mapsto x * y
\end{cases}$
\begin{definition}
Soit $M$ un ensemble muni d'une lci $*$ \\
$(M, *)$ est un monoïde si:
\begin{enumerate}
\item $*$ est associative.
\item $*$ possède un élément neutre $e_M$
\end{enumerate}
\end{definition}
\begin{definition}
Un groupe est un monoïde dont tous les éléments sont inversibles.
\end{definition}
\begin{definition}
Soit $A$ un ensemble avec 2 lci: $+$ et $*$ \\
$A$ est un anneau si:
\begin{enumerate}
\item $(A, +)$ est un groupe abélien.
\item $(A, *)$ est un monoïde.
\item $\forall a, x, y \in A \qquad \begin{cases}
a * (x + y) = a * x + a * y \\
(x + y) * a = x * a + y * a
\end{cases}$
\end{enumerate}
\end{definition}
\begin{definition}
Un anneau commutatif $\neq \{ 0 \}$ dont tous les éléments non nuls sont inversibles est un corps.
\end{definition}
\begin{definition}
Soit $(E, + , \bullet)$ avec $E$ ensemble, $*$ lci et $\bullet$: $\begin{cases}
\mathbb{K} \times E \to E \\
(\lambda, x) \mapsto \lambda \bullet x
\end{cases}$
(l.c. externe) \\
$(E, +, \bullet)$ est un $\mathbb{K}$ espace vectoriel si:
\begin{enumerate}
\item $(E, +)$ groupe abélien.
\item $\forall x \in E \qquad 1 \bullet x = x$
\item $\forall \lambda \in \mathbb{K}$, $\forall x, y \in E \qquad \lambda \bullet (x + y) = \lambda \bullet x + \lambda \bullet y$
\item $\forall \lambda, \mu \in \mathbb{K}$, $\forall x \in E \qquad (\lambda + \mu) \bullet x = \lambda \bullet x + \mu \bullet x$
\item $\forall \lambda, \mu \in \mathbb{K}$, $\forall x \in E \qquad (\lambda \bullet \mu) \bullet x = \lambda \bullet (\mu \bullet x) = \mu \bullet (\lambda \bullet x)$
\end{enumerate}
\end{definition}

\subsection{Sous-structures}
\noindent \uline{Rappels}:
\begin{enumerate}
\item $G$ groupe, $H \subset G$ \\
$H$ sous-groupe $\iff \begin{cases}
1_G \in H \\
\forall x, y \in H,\, xy \in H \\
\forall x \in H,\, x^{-1} \in H
\end{cases}$ \\
$H$ est un groupe aussi pour la restriction.
\item $A$ anneau, $B \subset A$ \\
$B$ sous-anneau $\iff \begin{cases}
\forall x, y \in B,\, x + y \in B,\, xy \in B \\
1_A \in B \\
\forall x \in B,\, -x \in B
\end{cases}$ \\
Le sous-anneau $B$ est en particulier un anneau.
\item $K$ un corps, $L \subset K$ \\
$L$ sous-corps de $K$ $\iff \begin{cases}
L \text{ sous-anneau de } K \\
\forall x \in L \setminus \{ 0 \},\, x^{-1} \in L
\end{cases}$
\item $E$ un $\mathbb{K}$-ev, $F \subset E$ \\
$F$ sous-espace vectoriel de $E$ $\iff \begin{cases}
\forall x, y \in F,\, x + y \in F \\
\forall \lambda \in \mathbb{K},\, \forall x \in F,\, \lambda x \in F \\
0 \in F
\end{cases}$
\end{enumerate} \medskip

\noindent \uline{Démarche}: Pour montrer qu'un ensemble est un machin \footnote{monoïde, groupe, anneau, corps ou $\mathbb{K}$-ev}, on pourra le réaliser comme un sous-machin d'un machin connu.

\begin{lemme}
Soit $M = \begin{pmatrix}
\alpha & \beta \\
\gamma & \delta
\end{pmatrix} \in GL_2(K)$ \\
Alors
\[ M^{-1} = \frac{1}{\det M} \begin{pmatrix}
\delta & - \beta \\
- \gamma & \alpha
\end{pmatrix}\]
\end{lemme}
\begin{proposition}
Soit $M$ un machin \footnotemark[1] et $(M_i)_{i \in I}$ une famille de sous-machins de $M$ \\
Alors $\bigcap\limits_{i \in I} M_i$ est un sous-machin de $M$
\end{proposition}
\end{document}