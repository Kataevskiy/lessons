\documentclass[10pt,a4paper]{article}
\usepackage[utf8]{inputenc}
\usepackage[french]{babel}
\usepackage[T1]{fontenc}
\usepackage{amsmath}
\usepackage{amsfonts}
\usepackage{amssymb}
\usepackage{graphicx}
\usepackage[left=2cm,right=2cm,top=2cm,bottom=2cm]{geometry}
\usepackage{setspace}
\usepackage{ulem}
\usepackage{stmaryrd}
\usepackage{amsthm}
\usepackage{dsfont}
\usepackage{mathpazo}

\onehalfspacing

\theoremstyle{definition}
\newtheorem{proposition}{Proposition}[section]
\newtheorem{theorem}[proposition]{Théorème}
\newtheorem{corollaire}[proposition]{Corollaire}
\newtheorem{lemme}[proposition]{Lemme}
\newtheorem{definition}[proposition]{Définition}

\DeclareMathOperator*{\eqv}{\thicksim}
\DeclareMathOperator*{\negl}{o}
\DeclareMathOperator*{\dom}{O}
\DeclareMathOperator{\dl}{DL}

\begin{document}
\renewcommand{\labelitemi}{$*$}
\renewcommand{\labelenumi}{(\roman{enumi})}
\begin{center}
{\Large \textbf{Chapitre 16. Analyse asymptotique}}
\end{center}
\uline{Cadre}: Dans tout le chapitre, on regardera des fonctions $f: I \to \mathbb{R}$ et un point $a \in \bar{I} \cup \{  \pm\infty \}$ en lequel il est pertinent de s'intéresser à la limite de $f$. En pratique, $I$ sera un intervalle (et $a$ un point ou une extrémité de $I$) ou $I$ sera $\mathbb{N}$ (et $a = +\infty$) dans le cas des suites.

\section{Équivalence}
\subsection{Définitions}
\begin{definition}
Soit $f,g: I \to \mathbb{R}$. On dit que $f$ et $g$ sont équivalentes au voisinage de $a$ et on note \\
$f(x) \eqv\limits_{x \to a} g(x)$ ou $f \eqv\limits_{a} g$ s'il existe un voisinage $V$ de $a$ dans $I$ et $\mu: V \to \mathbb{R}$ telle que:
\[ \begin{cases}
\forall x \in V ,\, g(x) = \mu(x) f(x) \\
\mu(x) \xrightarrow[x \to a]{} 1
\end{cases} \]
En pratique, on sera dans l'un des 2 cas suivants:
\begin{itemize}
\item $f$ et $g$ ne s'annulent pas au voisinage de $a$
\item $f(a) = g(a) = 0$ et il existe un voisinage de $a$ tel que $f$ et $g$ ne s'annulent pas sur $V \setminus \{ a \}$ \\
(On dit que $a$ est un  \uline{zéro isolé} de $f$ et $g$). \\
Dans ce cas, la définition devient simplement:
\[ f(x) \eqv\limits_{x \to a} g(x) \iff \frac{g(x)}{f(x)} \xrightarrow[x \to a]{} 1 \]
\end{itemize}
\end{definition}
\medskip
\noindent \uline{\textbf{Avertissement}}: En particulier: $f(x) \eqv\limits_{x \to a} 0$ signifie que $f$ est nulle au voisinage de $a$. \\
C'est une assertion que l'on lit souvent après des erreurs de calculs.
\begin{proposition}
$\sim$ est une relation d'équivalence: Soit $f,g,h : I \to \mathbb{R}$
\begin{itemize}
\item \uline{Réflexivité}: On a $f(x) \eqv\limits_{x \to a} f(x)$
\item \uline{Symétrie}: Si $f \eqv\limits_{a} g$, alors $g \eqv\limits_{a} f$
\item \uline{Transitivité}: Si $f \eqv\limits_{a} g$ et $g \eqv\limits_{a} h$, alors $f \eqv\limits_{a} h$
\end{itemize}
\end{proposition}

\subsection{Propriétés multiplicatives}
\begin{proposition}
Soit $f_1, f_2, g_1, g_2: I \to \mathbb{R}$ tels que: $\begin{cases}
g_1(x) \eqv\limits_{x \to a} f_1(x) \\
g_2(x) \eqv\limits_{x \to a} f_2(x)
\end{cases}$
\begin{itemize}
\item Alors $g_1 g_2 (x) \eqv\limits_{x \to a} f_1 f_2 (x)$
\item Si $g_1$ et $g_2$ ne sont pas nulles au voisinage de $a$, $\frac{g_1(x)}{g_2(x)} \eqv\limits_{x \to a} \frac{f_1(x)}{f_2(x)}$
\item On peut élever un équivalent à une puissance (constante). \\
Soit $\alpha \in \mathbb{R}$, on a $f_1(x)^\alpha \eqv\limits_{x \to a} g_1(x)^\alpha$ (à conditions que ces fonctions soient définis au voisinage de $a$).
\end{itemize}
\end{proposition}

\subsection{Propriétés de l'équivalence}
\begin{proposition}
Soit $f, g: I \to \mathbb{R}$
\begin{itemize}
\item Si $\begin{cases}
f(x) \eqv\limits_{x \to a} g(x) \\
f(x) \xrightarrow[x \to a]{} l \in \overline{\mathbb{R}}
\end{cases}$ alors $g(x) \xrightarrow[x \to a]{} l$
\item Si $l \in \mathbb{R}^*$ (ni $\pm\infty$, ni 0) et tel que $\begin{cases}
f(x) \xrightarrow[x \to a]{} l \\
g(x) \xrightarrow[x \to a]{} l
\end{cases}$ alors $f(x) \eqv\limits_{x \to a} g(x)$
\end{itemize}
\end{proposition}
\begin{proposition}
Soit $f,g: I \to \mathbb{R}$ telles que $f(x) \eqv\limits_{x \to a} g(x)$ \\
Alors $f$ et $g$ ont le même signe au voisinage de $a$. \\
Plus précisément: il existe un voisinage $V$ de $a$ dans $I$ tel que, pour tout $x \in V$, $f(x)$ et $g(x)$ ont le même signe ($<0$, nul, $>0$).
\end{proposition}

\section{Négligeabilité et domination}
\subsection{Définitions}
\begin{definition}
Soit $f,g: I \to \mathbb{R}$
\begin{itemize}
\item On dit que \uline{$f$ est négligeable devant $g$ au voisinage de $a$}, et on note $f(x) = \negl\limits_{x \to a} (g(x))$ (ou $f = \negl\limits_{a}(g)$ s'il existe un voisinage $V$ de $a$ dans $I$ et $\varepsilon(x) \xrightarrow[x \to a]{} 0$
\item On dit que \uline{$f$ est dominée par $g$ au voisinage de $a$} et on note $f(x) = \dom\limits_{x \to a}(g(x))$ (ou $f = \dom\limits_{a}(g)$) s'il existe un voisinage $V$ de $a$ dans $I$ et $c: V \to \mathbb{R}$ bornée tels que $\forall x \in V$, $f(x) = c(x) g(x)$
\end{itemize}
\end{definition}
\medskip
\noindent \uline{Remarque important}: Dans les cas usuels, on a
\begin{itemize}
\item $f(x) = \negl\limits_{x \to a}(g(x))$ ssi $\frac{f(x)}{g(x)} \xrightarrow[x \to a]{} 0$
\item $f(x) = \dom\limits_{x \to a}(g(x))$ ssi $x \mapsto \frac{f(x)}{g(x)}$ est bornée au voisinage de $a$
\end{itemize}

\subsection{Opérations}
\begin{proposition}
Soit $f,g: I \to \mathbb{R}$
\begin{itemize}
\item Si $f(x) \eqv\limits_{x \to a} g(x)$ alors $f(x) = \dom\limits_{x \to a}(g(x))$
\item Si $f(x) = \negl\limits_{x \to a}(g(x))$ alors $f(x) = \dom\limits_{x \to a}(g(x))$
\end{itemize}
\end{proposition}
\begin{proposition}
Soit $f,g,h: I \to \mathbb{R}$
\begin{itemize}
\item Si $f(x) = \dom\limits_{x \to a}(g(x))$ et $g(x) = \dom\limits_{x \to a}(h(x))$ alors $f(x) = \dom\limits_{x \to a}(h(x))$
\item Si l'une des relations de dominations est remplacée par une relation de négligeabilité, on obtient \\
$f(x) = \negl\limits_{x \to a}(h(x))$
\end{itemize}
\end{proposition}
\begin{proposition}
Soit $f_1, f_2, g: I \to \mathbb{R}$ et $\lambda, \mu \in \mathbb{R}$ \\
Si $\begin{cases}
f_1(x) = \negl\limits_{x \to a}(g(x)) \\
f_2(x) = \negl\limits_{x \to a}(g(x))
\end{cases}$ alors $\lambda f_1(x) + \mu f_2(x) = \negl\limits_{x \to a}(g(x))$
\end{proposition}
\begin{proposition}
Soit $f_1, f_2, g_1, g_2: I \to \mathbb{R}$ \\
Si $\begin{cases}
f_1(x) = \dom\limits_{x \to a}(g_1(x)) \\
f_2(x) = \dom\limits_{x \to a}(g_2(x))
\end{cases}$ alors $f_1(x) f_2(x) = \dom\limits_{x \to a}(g_1(x) g_2(x))$ \\
Si l'une de deux relations de domination est remplacée par une relation de négligeabilité, on a \\
$f_1(x) f_2(x) = \negl\limits_{x \to a}(g_1(x) g_2(x))$
\end{proposition}
\begin{proposition}
Soit $\varphi: J \to I$ une fonction telle que $\varphi(t) \xrightarrow[t \to b]{} a$ et $b \in \overline{J} \cup \{ \pm\infty \}$ et $f, g: I \to \mathbb{R}$ \\
Alors, si $f(x) \eqv\limits_{x \to a} g(x)$ on a $f(\varphi(t)) \eqv\limits_{t \to b} g(\varphi(t))$ \\
Idem avec $\negl$ ou $\dom$.
\end{proposition}

\pagebreak

\subsection{Notations de Landau}
On utilisera $\negl\limits_{x \to a}(g(x))$ et $\dom\limits_{x \to a}(g(x))$ dans des égalités pour designer une fonction non nommée, dont on garantit qu'elle est négligeable ou dominée par $g(x)$. \\
Par exemple, on écrira $\sin(x) = x - \frac{x^3}{6} + \negl\limits_{x \to 0}(x^4)$ pour dire $\sin(x) = x - \frac{x^3}{6} + f_1(x)$, où $f_1(x) = \negl\limits_{x \to 0}(x^4)$ \\
Attention! Cette pratique a des conséquences surprenantes:
\begin{itemize}
\item Par exemple, on ne peut pas simplifier $\negl\limits_{x \to 0}(x) - \negl\limits_{x \to 0}(x)$ \\
Cette expression veut dire $f_1(x) - f_2(x)$ où $f_1(x), f_2(x) = \negl\limits_{x \to 0}(x)$ \\
On remplacera par un unique $\negl\limits_{x \to 0}(x)$
\item On sait que $x^4 = \negl\limits_{x \to 0}(x^3)$. On en déduit que si $f_1(x) = \negl\limits_{x \to 0}(x^4)$ alors $f_1(x) = \negl\limits_{x \to 0}(x^3)$. \\
Dans un calcul:
\begin{align*}
\sin(x) &= x - \frac{x^3}{6} + \negl\limits_{x \to 0}(x^4) \\
        &= x - \frac{x^3}{6} + \negl\limits_{x \to 0}(x^3)
\end{align*}
On a en fait écrit $\negl\limits_{x \to 0}(x^4) = \negl\limits_{x \to 0}(x^3)$ mais cette égalité n'est pas symétrique: \\
on ne peut pas remplacer $\negl\limits_{x \to 0}(x^3)$ par un $\negl\limits_{x \to 0}(x^4)$
\end{itemize}
\begin{theorem}
Soit $f,g: I \to \mathbb{R}$ LASSÉ:
\begin{enumerate}
\item $f(x) \eqv\limits_{x \to a} g(x)$
\item $f(x) = g(x) + \negl\limits_{x \to a}(g(x))$
\item $g(x) = f(x) + \negl\limits_{x \to a}(f(x))$
\end{enumerate}
\end{theorem}

\section{Développements limités}
\subsection{Définition et premières propriétés}
\begin{definition}
Soit $f: I \to \mathbb{R}$ et $a \in I$ \\
On dit que $f$ admet \uline{un développement limité à l'ordre n en a}: $\dl_n(a)$ s'il existe $c_0, ...\,, c_n \in \mathbb{R}$ tels que \\
$f(x) = c_0 + c_1(x - a) + c_2(x - a)^2 + ... + c_n(x - a)^n + \negl\limits_{x \to a}((x - a)^n)$ \\
( ou $f(a + h) = c_0 + c_1 h + c_2 h^2 + ... + c_n h^n + \negl\limits_{h \to 0}(h^n)$ )
\end{definition}
\begin{proposition}[Troncature de DL]
Soit $m \leq n$ deux entiers naturels. \\
Si $f$ admet un $\dl_n(a): f(a + h) = c_0 + c_1 h + ... + c_n h^n +\negl(h^n)$ \\
alors elle admet un $\dl_m(a): f(a + h) = c_0 + c_1 h + ... + c_m h^m + \negl(h^m)$
\end{proposition}
\begin{proposition}
Soit $f: I \to \mathbb{R}$ admettant un $\dl_n(a): f(a + h) = c_0 + c_1 h + ... + c_n h^n + \negl(h^n)$ \\
Si $c_0, c_1, ...\,, c_n$ ne sont pas tous nuls, on note $\mu = \min \{ i \in \llbracket 0, n \rrbracket \mid c_i \neq 0 \}$ et on a: $f(a + h) \eqv\limits_{h \to 0} c_\mu h^\mu$
\end{proposition}
\begin{proposition}
Soit $f: I \to \mathbb{R}$ et $a \in I$
\begin{itemize}
\item $f$ est continue et $a$ ssi elle admet un $\dl_0(a)$ ( le DL est alors $f(a + h) = f(a) + \negl(1)$ )
\item $f$ est dérivable en $a$ ssi elle admet un $\dl_1(a)$ ( le DL est alors $f(a + h) = f(a) + f'(a)h + \negl(h)$ )
\end{itemize}
\end{proposition}
\begin{proposition}[Unicité du DL]
Soit $f: I \to \mathbb{R}$ et $b_0, ...\,, b_n$, $c_0, ...\,, c_n \in \mathbb{R}$ tels que
\begin{align*}
f(a + h) &= b_0 + b_1 h + ... + b_n h^n + \negl(h^n) \\
         &= c_0 + c_1 h + ... + c_n h^n + \negl(h^n)
\end{align*}
Alors $\forall i \in \llbracket 0, n \rrbracket$, $b_i = c_i$
\end{proposition}
\begin{corollaire}
Soit $f: I \to \mathbb{R}$ admettant un $\dl_n(0): f(h) = c_0 + c_1 h + ... + c_n h^n + \negl(h^n)$
\begin{itemize}
\item Si $f$ est paire, on a $c_1 = c_3 = ... = 0$
\item Si $f$ est impaire, alors $c_0 = c_2 = ... = 0$
\end{itemize}
\end{corollaire}

\subsection{Lemme de primitivation des DL}
\begin{lemme}
Soit $f: I \to \mathbb{R}$ dérivable et $a \in I$ \\
On suppose que $f'$ admet un $\dl_n(a): f'(a + h) = c_0 + c_1 h + ... + c_n h^n + \negl(h^n)$ \\
Alors $f$ admet un $\dl_{n + 1}(a): f(a + h) = f(a) + c_0 h + c_1 \frac{h^2}{2} + ... + c_n \frac{h^{n + 1}}{n + 1} + \negl(h^{n + 1})$
\end{lemme}

\subsection{Théorème de Taylor-Young}
\noindent \uline{Rappel}: Si $f: I \to \mathbb{R}$ est $n$ fois dérivable et $a \in I$, il existe un unique polynôme $P$ de $\mathbb{R}_n[X]$ tel que \\
$\forall k \in \llbracket 0, n \rrbracket$, $f^{(k)}(a) = P^{(k)}(a)$ c'est \uline{le $n$-ième polynôme de Taylor de $f$ en $a$}:
\[\sum\limits_{k = 0}^{n} \frac{f^{(k)}(a)}{k!} (X - a)^k\]
\begin{theorem}[Taylor-Young]
Soit $f \in C^n(I)$ et $a \in I$ \\
Alors
\[ f(x) = \sum_{k = 0}^n \frac{f^{(k)}(a)}{k!} (x - a)^k + \negl\limits_{x \to a}((x - a)^n)\]
càd
\[ f(a + h) = \sum_{k = 0}^n \frac{f^{(k)}(a)}{k!} h^k + \negl\limits_{h \to 0}(h^n)\]
Notamment, $f$ admet un $\dl_n(a)$
\end{theorem}
\begin{theorem}[Formulaire]
Pour tout $n \in \mathbb{N}$, on a:
\begin{align*}
\frac{1}{1 - h} &= 1 + h + h^2 + ... + h^n + \negl(h^n) \\
\frac{1}{1 + h} &= 1 - h + h^2 - h^3 + ... + (-1)^n h^n + \negl(h^n) \\
(1 + h)^\alpha &= 1 + \alpha h + \frac{\alpha (\alpha - 1)}{2!} h^2 + \frac{\alpha (\alpha - 1) (\alpha - 2)}{3!} h^3 + ... + \frac{\alpha (\alpha - 1) (\alpha - 2) ... (\alpha - n + 1)}{n!} h^n + \negl(h^n) \\
\ln(1 + h) &= h - \frac{h^2}{2} + \frac{h^3}{3} + ... + (-1)^{n - 1} \frac{h^n}{n} + \negl(h^n) \\
\arctan(h) &= h - \frac{h^3}{3} + \frac{h^5}{5} + ... + (-1)^n \frac{h^{2n + 1}}{2n + 1} + \negl(h^{2n + 1}) \\
\exp(h) &= 1 + h + \frac{h^2}{2!} + \frac{h^3}{3!} + ... + \frac{h^n}{n!} + \negl(h^n) \\
\cosh(h) &= 1 + \frac{h^2}{2!} + \frac{h^4}{4!} + ... + \frac{h^{2n}}{(2n)!} + \negl(h^{2n}) \\
\sinh(h) &= h + \frac{h^3}{3!} + \frac{h^5}{5!} + ... + \frac{h^{2n + 1}}{(2n + 1)!} + \negl(h^{2n + 1}) \\
\cos(h) &= 1 - \frac{h^2}{2!} + \frac{h^4}{4!} + ... + (-1)^n \frac{h^{2n}}{(2n)!} + \negl(h^{2n}) \\
\sin(h) &= h - \frac{h^3}{3!} + \frac{h^5}{5!} + ... + (-1)^n \frac{h^{2n + 1}}{(2n + 1)!}
\end{align*}
\end{theorem}

\section{Calculs pratiques}
\subsection{Somme et produit}
On retrouve le DL de $\cosh$:
\begin{align*}
e^h &= 1 + h + \frac{h^2}{2!} + ... + \frac{h^{2n - 1}}{(2n - 1)!} + \frac{h^n}{n!} + \negl(h^n) \\
e^{-h} &= 1 - h + \frac{h^2}{2!} + ... - \frac{h^{2n - 1}}{(2n -1)!} + \frac{h^n}{n!} + \negl(h^n)
\end{align*}
Donc 
\[ \cosh(h) = \frac{e^h + e^{-h}}{2} = 1 + \frac{h^2}{2!} + ... + \frac{h^{2n}}{(2n)!} + \negl(h^{2n}) \]
Cela marche plus généralement pour les parties paire ( $x \mapsto \frac{f(x) + f(-x)}{2}$ ) et impaire ( $x \mapsto \frac{f(x) - f(-x)}{2}$ ) d'une fonction $f$. \medskip

$\dl_3(0)$ de $x \mapsto e^x \sinh(x)$:
\begin{align*}
e^x \sinh(x) &= \left(1 + x + \frac{x^2}{2} + \negl(x^2)\right)\left(x + \frac{x^3}{6} + \negl(x^3)\right) \\
&= x + x^2 + \frac{x^3}{2} + \frac{x^3}{6} + \negl(x^3) \\
&= x + x^2 + \frac{2}{3} x^3 + \negl(x^3)
\end{align*}
Si un facteur a une valuation $> 0$, on peut développer l'autre à une précision moindre.

\subsection{Composition}
On peut facilement composer un DL avec une puissance de $x$: par exemple, donnons un $\dl_5(0)$ de \\
$x \mapsto \sin(x^2) \cos(x)$
\begin{align*}
\sin(x^2) \cos(x) &= (x^2 + \negl(x^5))(1 - \frac{x^2}{2} + \negl(x^3)) \\
&= x^2 - \frac{1}{2} x^4 + \negl(x^5)
\end{align*} \medskip

On peut aussi composer avec des fonction plus compliquées. Donnons  un $\dl_2(0)$ de $x \mapsto e^{\sin(x)}$
\begin{align*}
e^{\sin(x)} &= 1 + \sin(x) + \frac{(\sin(x))^2}{2} + \negl(x^2) \\
&= 1 + (x + \negl(x^2)) + \frac{(x + \negl(x))^2}{2} + \negl(x^2) \\
&= 1 + x + \frac{1}{2} x^2 + \negl(x^2)
\end{align*}

\pagebreak

\subsection{Quotient}
Pas besoin de nouvelle technique: on utilise le DL de $u \mapsto \frac{1}{1 + u}$ \medskip

$\dl_4(0)$ de $\frac{1}{\cos}$
\begin{align*}
\frac{1}{\cos(x)} &= \frac{1}{1 + (\cos(x) - 1)} \\
&= 1 - \left(- \frac{x^2}{2} + \frac{x^4}{24} + \negl(x^4)\right) + \left(- \frac{x^2}{2} + \negl(x^2)\right)^2 + \negl(x^4) \\
&= 1 + \frac{x^2}{2} - \frac{x^4}{24} + \frac{x^4}{4} + \negl(x^4) \\
&= 1 + \frac{1}{2} x^2 + \frac{5}{24} x^4 + \negl(x^4)
\end{align*} \medskip

$\dl_5(0)$ de $\tan$
\begin{align*}
\tan(x) = \frac{\sin(x)}{\cos(x)} &= \left(x - \frac{x^3}{6} + \frac{x^5}{120} + \negl(x^5)\right)\left(1 + \frac{x^2}{2} + \frac{5 x^4}{24} + \negl(x^4)\right) \\
&= x + \frac{x^3}{2} - \frac{x^3}{6} + \frac{5 x^5}{24} - \frac{x^5}{12} + \frac{x^5}{120} + \negl(x^5) \\
&= x + \frac{1}{3} x^3 + \frac{2}{15} x^5 + \negl(x^5)
\end{align*} \medskip

\noindent \uline{Remarque}: Selon le programme officiel,
\[\tan(x) = x + \frac{1}{3} x^3 + \negl(x^3)\]
est à connaître.

\section{Applications}
\subsection{Limites et équivalents}
Déterminons la limite en 0 de $\frac{1}{\sin^2(x)} - \frac{1}{x^2}$
\begin{align*}
\frac{1}{\sin^2(x)} - \frac{1}{x^2} &= \frac{x^2 - \sin^2(x)}{x^2 \sin^2(x)} \\
\intertext{Or}
x^2 - \sin^2(x) &= \frac{x^4}{3} + \negl(x^4) \\
x^2 - \sin^2(x) &\eqv\limits_{x \to 0} \frac{x^4}{3} \\
\intertext{Donc}
\frac{x^2 - \sin^2(x)}{x^2 \sin^2(x)} &\eqv\limits_{x \to 0} \frac{x^4}{3 x^4} = \frac{1}{3} \\
\intertext{Donc}
\frac{x^2 - \sin^2(x)}{x^2 \sin^2(x)} &\xrightarrow[x \to 0]{} \frac{1}{3}
\end{align*}

Équivalent en 0 de $x \mapsto (\cosh(x))^x - (\cos(x))^x$
\begin{align*}
(\cosh(x))^x &= \exp(x \ln(\cosh(x))) \\
&= \exp\left(x\left(\left(\frac{x^2}{2} + \negl(x^3)\right) + \negl(x^3)\right)\right) \\
&= 1 + \frac{x^3}{2} + \negl(x^3) \\
\intertext{De même}
(\cos(x))^x &= 1 - \frac{x^3}{2} + \negl(x^3) \\
\intertext{Donc}
(\cosh(x))^x - (\cos(x))^x &= x^3 + \negl(x^3) \\
&\eqv\limits_{x \to 0} x^3
\end{align*}

\subsection{Étude locale d'une fonction}
On rappelle que $f: I \to \mathbb{R}$ admet un $\dl_0(a)$ ssi $f$ est continue en $a$. \\
\phantom{\indent On rappelle que $f: I \to \mathbb{R}$ admet un} $\dl_1(a)$ ssi $f$ est dérivable en $a$. \medskip

Un DL de la forme $f(a + h) = f(a) + f'(a) h + \lambda h^\nu + \negl(h^\nu)$ permet de déterminer la position de $f$ par rapport à sa tangente:
\[f(a + h) - (f(a) + f'(a)h) \eqv\limits_{h \to 0} \lambda h^\nu\]
et deux fonctions équivalentes ont localement le même signe. \\
Cela montre notamment le résultat suivant:
\begin{proposition}
Soit $f \in C^2(I)$ et $a \in I$ un point intérieur.
\begin{itemize}
\item Si $f$ a un minimum local en $a$, alors $f'(a) = 0$ et $f''(a) \geq 0$
\item Si $f'(a) = 0$ et $f''(a) > 0$, $f$ admet un minimum local en a.
\end{itemize}
\end{proposition}

\pagebreak

\subsection{Asymptotes}
On peut calculer des \uline{développements asymptotiques} plus généraux que des DL. \medskip

Donnons un DA à la précision $\negl\limits_{x \to 0}(x)$ de $x \mapsto \frac{1}{e^x - 1}$
\begin{align*}
\frac{1}{e^x - 1} &= \frac{1}{x + \frac{x^2}{2} + \frac{x^3}{6} + \negl(x^3)} \\
&= \frac{1}{x} \frac{1}{1 + \frac{x}{2} + \frac{x^2}{6} + \negl(x^2)} \\
&= \frac{1}{x} \left(1 - \left( \frac{x}{2} + \frac{x^2}{6} + \negl(x^2) \right) + \left( \frac{x}{2} + \negl(x) \right)^2 + \negl(x^2) \right) \\
&= \frac{1}{x} \left(1 - \frac{x}{2} + \frac{x^2}{12} + \negl(x^2) \right) \\
&= \frac{1}{x} - \frac{1}{2} + \frac{x}{12} + \negl(x)
\end{align*} \medskip

\uline{Remarque}: On peut calculer certains DA quand $x \to +\infty$ en passant par la fonction $h \mapsto f(\frac{1}{h})$ \medskip

Considérons $x \mapsto x^2 \ln\left(\frac{x}{x - 1}\right)$
\begin{align*}
f\left(\frac{1}{h}\right) &= \frac{1}{h^2} \ln\left(\frac{\frac{1}{h}}{\frac{1}{h} - 1}\right) \\
&= \frac{1}{h^2} \ln\left(\frac{1}{1 - h}\right) \\
&= -\frac{1}{h^2} \ln(1 - h) \\
&= -\frac{1}{h^2} \left(-h -\frac{h^2}{2} - \frac{h^3}{3} + \negl(h^3)\right) \\
&= \frac{1}{h} + \frac{1}{2} + \frac{h}{3} + \negl\limits_{h \to 0}(h)
\end{align*}
Donc
\[f(x) = x + \frac{1}{2} + \frac{1}{3x} + \negl\limits_{x \to \pm\infty}\left(\frac{1}{x}\right)\]
Graphiquement, cela dit que la droite d'équation $y = x + \frac{1}{2}$ est une asymptote du graphe de $f$ et cela donne la position relative du graphe et de l'asymptote.
\end{document}