\documentclass[10pt,a4paper]{article}
\usepackage[utf8]{inputenc}
\usepackage[french]{babel}
\usepackage[T1]{fontenc}
\usepackage{amsmath}
\usepackage{amsfonts}
\usepackage{amssymb}
\usepackage{graphicx}
\usepackage[left=2cm,right=2cm,top=2cm,bottom=2cm]{geometry}
\usepackage{setspace}
\usepackage{ulem}
\usepackage{stmaryrd}
\usepackage{amsthm}
\usepackage{dsfont}
\usepackage{mathpazo}

\onehalfspacing

\theoremstyle{definition}
\newtheorem{proposition}{Proposition}[section]
\newtheorem{theorem}[proposition]{Théorème}
\newtheorem{corollaire}[proposition]{Corollaire}
\newtheorem{lemme}[proposition]{Lemme}
\newtheorem{definition}[proposition]{Définition}

\DeclareMathOperator{\re}{Re}
\DeclareMathOperator{\im}{Im}

\begin{document}
\renewcommand{\labelitemi}{$*$}
\renewcommand{\labelenumi}{(\roman{enumi})}
\begin{center}
{\Large \textbf{Chapitre 13. Limites et continuité}}
\end{center}

\section{Voisinage}
\begin{definition}
Soit $a \in\overline{\mathbb{R}}$ \\
Un \uline{voisinage de $a$}:
\begin{itemize}
\item Si $a \in \mathbb{R}$, est un ensemble qui contient $[a - \delta, a + \delta]$, pour un certain $\delta > 0$
\item Si $a = +\infty$, est un ensemble $\left[A, +\infty\right[$ pour un certain $A \in \mathbb{R}$
\item Si $a = -\infty$, est un ensemble $\left]-\infty, A\right]$ pour un certain $A \in \mathbb{R}$
\end{itemize}
\end{definition}
\begin{lemme}
Soit $V$ un voisinage de $+\infty$ \\
Alors il existe une suite $(v_n)_{n \in \mathbb{N}} \in V^{\mathbb{N}}$ telle que $v_n \xrightarrow[x \to +\infty]{} +\infty$ 
\end{lemme}
\begin{definition}
Soit $f: I \to \mathbb{R}$ et $a \in \overline{\mathbb{R}}$ \\
On dit qu'une propriété (de la fonction $f$) est vraie \uline{au voisinage de $a$} s'il existe un voisinage $V$ de $a$ tel que la propriété soit vraie sur $V \cap I$
\end{definition}

\section{Notion de limite}
\uline{Cadre}: Dans cette section, $f: I \to \mathbb{R}$ est une fonction définie sur une partie $I$ de $\mathbb{R}$ et $a$ est un élément de $I$ ou $\pm\infty$. En pratique, $I$ sera un intervalle et $a$ un point ou une borne de l'intervalle.
\subsection{Limites en $\pm\infty$}
\begin{definition}
Soit $I$ un ensemble non majoré et $f: I \to \mathbb{R}$
On dit que:
\begin{itemize}
\item \uline{$f$ converge vers $l \in \mathbb{R}$ en $+\infty$} si $\forall \varepsilon > 0$, $\exists H \in \mathbb{R}$: $\forall x \in I$, $x \geq H \implies |f(x) - l| \leq \varepsilon$
\item \uline{$f$ tend vers $+\infty$ en $+\infty$} si $\forall A \in \mathbb{R}$, $\exists H \in \mathbb{R}$: $\forall x \in I$, $x \geq H \implies f(x) \geq A$
\item \uline{$f$ tend vers $-\infty$ en $+\infty$} si $\forall A \in \mathbb{R}$, $\exists H \in \mathbb{R}$: $\forall x \in I$, $x \geq H \implies f(x) \leq A$
\end{itemize}
\end{definition}

\subsection{Limites en un réel}
\uline{Cadre}: $a \in \overline{I}$
\begin{definition}
Soit $a \in \overline{I}$ et $f: I \to \mathbb{R}$
\begin{itemize}
\item On dit que \uline{$f$ tend vers $l \in \mathbb{R}$ en $a$} si \\
$\forall \varepsilon > 0$, $\exists \lambda > 0$: $\forall x \in I$, $|x - a| \leq \delta \implies |f(x) - l| \leq \varepsilon$
\item On dit que \uline{$f$ tend vers $+\infty$ en $a$} si \\
$\forall A \in \mathbb{R}$, $\exists \lambda > 0$: $\forall x \in I$, $|x - a| \leq \delta \implies f(a) \geq A$
\item On dit que \uline{$f$ tend vers $-\infty$ en $a$} si \\
$\forall A \in \mathbb{R}$, $\exists \lambda > 0$: $\forall x \in I$, $|x - a| \leq \delta \implies f(a) \leq A$
\end{itemize}
\end{definition}
\begin{proposition}
Soit $a \in I$ \\
Si $f$ admet une limite (dans $\overline{\mathbb{R}}$) en $a$, cette limite est nécessairement $f(a)$
\end{proposition}

\pagebreak

\subsection{Variantes}
\begin{definition}
Soit $f: I \to \mathbb{R}$, $J$ une partie de $I$ et $a \in \overline{I} \cup \{ \pm\infty \}$. On suppose que $a$ est arbitrairement proche d'éléments de $J$ ( càd $a \in \overline{J}$ ou ($a = +\infty$ et $J$ n'est pas majoré) ou ($a = -\infty$ et $J$ n'est pas minoré) ) \\
On dit alors que $f(x) \xrightarrow[\substack{x \to a \\ x \in J}]{} l \in \mathbb{R}$ si:
\[\forall \varepsilon > 0, \, \exists \delta > 0: \forall x \in J, \, |x - a| \leq \delta \implies |f(x) - l| \leq \varepsilon\ \quad (\text{cas } a \in \mathbb{R})\]
\[\forall \varepsilon > 0, \, \exists H \in \mathbb{R}: \forall x \in J, \, x \geq H \implies |f(x) - l| \leq \varepsilon \quad (\text{cas } a = +\infty)\]
etc...
\end{definition}
\begin{proposition}
Soit $J_1, J_2$ deux parties de $I$, $a \in I \cup \{ \pm\infty \}$ et $l \in \overline{R}$. On suppose que $a$ est arbitrairement proche d'éléments de $J_1$ et de $J_2$ \\
Alors
\[ f(x) \xrightarrow[\substack{x \to a \\ x \in J_1 \cup J_2}]{} l \iff \begin{cases}
f(x) \xrightarrow[\substack{x \to a \\ x \in J_1}]{} l \\ f(x) \xrightarrow[\substack{x \to a \\ x \in J_2}]{} l \end{cases}\]
\end{proposition}

\section{Propriétés de la limite}
\subsection{Caractère local}
\begin{proposition}
Soit $f, g: I \to \mathbb{R}$ et $a \in \overline{I} \cup \{ \pm\infty \}$ arbitrairement proches d'éléments de $I$ \\
Si $f$ et $g$ coïncident au voisinage de $a$, alors $f$ admet une limite en $a$ ssi $g$ en admet une. Dans ce cas, ces limites sont les mêmes.
\end{proposition}

\subsection{Propriétés des fonctions convergentes}
\begin{proposition}
Les fonctions convergentes sont localement bornés: \\
Soit $f: I \to \mathbb{R}$, $a \in \overline{I} \cup \{ \pm\infty \}$ tel que $f(x) \xrightarrow[x \to a]{} l \in \mathbb{R}$ \\
Alors $f$ est bornée au voisinage de $a$.
\end{proposition}
\begin{proposition}[$\mathbb{R}_+^*$ est ouvert]
Soit $f: I \to \mathbb{R}$ et $a \in \overline{I} \cup \{ \pm\infty \}$ \\
Si $f(x) \xrightarrow[x \to a]{} l \in \mathbb{R}_+^*$, alors $f$ est $>0$ au voisinage de $a$.
\end{proposition}

\subsection{Caractérisation séquentielle de la limite}
\begin{theorem}
Soit $f: I \to \mathbb{R}$ et $a \in \overline{I} \cup \{ \pm\infty \}$. Soit $l \in \overline{\mathbb{R}}$ \\
On a $f(x) \xrightarrow[x \to a]{} l$ si et seulement si, pour toute suite $(\xi_n)_{n \in \mathbb{N}} \in I^\mathbb{N}$ telle que $\xi_n \xrightarrow[x \to +\infty]{} a$, on a $f(\xi_n) \xrightarrow[x \to +\infty]{} l$
\end{theorem}

\subsection{Composition des limites}
\begin{theorem}[À retenir mais mal énoncé]
Si $f(x) \xrightarrow[x \to a]{} b$ et $g(y) \xrightarrow[y \to b]{} l$, alors $g(f(x)) \xrightarrow[x \to a]{} l$
\end{theorem}
\begin{theorem}[Plus précis]
Soit $f: I \to J$ et $a \in \overline{I} \cup \{ \pm\infty \}$ et $b \in \overline{\mathbb{R}}$ tels que $f(x) \xrightarrow[x \to a]{} b$
\begin{itemize}
\item Déjà, $b \in \overline{J} \cup \{ \pm\infty \}$
\item Pour toute fonction $g: J \to \mathbb{R}$ telle que $g(y) \xrightarrow[y \to b]{} l \in \overline{\mathbb{R}}$, on a $g(f(x)) \xrightarrow[x \to a]{} l$
\end{itemize}
\end{theorem}

\subsection{Théorème de la limite monotone}
\begin{theorem}
Soit $f: I \to \mathbb{R}$ une fonction monotone.
\begin{itemize}
\item Si $I$ n'est pas majoré, $f$ admet une limite $l \in \overline{\mathbb{R}}$ et $+\infty$
\item Si $I$ n'est pas minoré, $f$ admet une limite $l \in \overline{\mathbb{R}}$ en $-\infty$
\item Si $a$ est un réel tel que $a \in \overline{I \cap \left]-\infty, a\right[}$, $f$ admet une limite $l \in \overline{\mathbb{R}}$ à gauche de $a$
\item Si $a$ est un réel tel que $a \in \overline{I \cap \left]a, +\infty\right[}$, $f$ admet une limite $l \in \overline{\mathbb{R}}$ à droite de $a$
\end{itemize}
\end{theorem}

\section{Continuité}
\subsection{Continuité en un point}
Cadre: $f: I \to \mathbb{R}$ et $a \in I$
\begin{definition}
$f$ est \uline{continue en $a$} si $\forall \varepsilon > 0$, $\exists \delta > 0: \forall x \in I$, $|x - a| \leq \delta \implies |f(x) - f(a)| \leq \varepsilon$
\end{definition}
\begin{proposition}[Caractère local de la continuité]
Soit $f, g: I \to \mathbb{R}$ deux fonctions. \\
Si $f$ et $g$ coïncident au voisinage de $a$, alors $f$ est continue en $a$ ssi $g$ l'est.
\end{proposition}
\begin{definition}
$f: I \to \mathbb{R}$ est \uline{continue à gauche} (resp. \uline{à droite}) et $a$ si $f(x) \xrightarrow[\substack{x \to a \\ x \leq a}]{} f(a)$ (resp. $f(x) \xrightarrow[\substack{x \to a \\ x \geq a}]{} f(a)$
\end{definition}

\subsection{Continuité globale}
\begin{definition}
Une fonction $f: I \to \mathbb{R}$ est continue si elle est continue en tout point de $I$ \\
On note $C^0(I) = C^0(I; \mathbb{R})$ l'ensemble des fonctions $f: I \to \mathbb{R}$ continues.
\end{definition}

\subsection{Opérations}
\begin{theorem}
Soit $f, g: I \to \mathbb{R}$ et $\lambda \in \mathbb{R}$
\begin{itemize}
\item Soit $a \in I$. Si $f$ et $g$ sont continues en $a$, alors $\lambda f$, $|f|$, $\max(f, g)$, $f + g$, $f g$ sont continues en $a$.
\item Si $f, g \in C^0(I)$, alors $\lambda f$, $|f|$, $\max(f, g)$, $f + g$, $f g \in C^0(I)$
\end{itemize}
\end{theorem}
\begin{theorem}
Soit $I, J \subseteq \mathbb{R}$ et $f: I \to J$, $g: J \to \mathbb{R}$
\begin{itemize}
\item Soit $a \in I$. Si $f$ est continue en $a$ et que $g$ est continue en $f(a)$, alors $g \circ f$ est continue en $a$.
\item Si $f$ et $g$ sont continues, $g \circ f$ l'est aussi.
\end{itemize}
\end{theorem}
\begin{theorem}["Théorème"]
Les fonctions usuelles vues au chapitre $5$ (exponentielle, logarithme, fonctions trigonométriques, trigonométriques réciproques, trigonométriques hyperboliques) sont continues.
\end{theorem}

\subsection{Prolongement par continuité}
\begin{theorem}
Soit $I \subseteq R$, $a \in I$ tel que $a \in \overline{I \setminus \{ a \}}$ et $f: I \setminus \{ a \} \to \mathbb{R}$ une fonction continue. \\
Alors il existe un prolongement continu $\tilde{f}:I \to \mathbb{R}$ de $f$ si et seulement si $f$ admet une limite finie en $a$. \\
Dans ce cas, un tel prolongement est unique, c'est
\[\tilde{f}:\begin{cases}
I \to \mathbb{R} \\ x \mapsto \begin{cases}
f(x) \text{ si } x \neq a \\ \lim\limits_{x \to a} f(x) \text{ si } x = a
\end{cases}
\end{cases}\]
\end{theorem}

\subsection{Prolongement des identités}
\begin{theorem}[Prolongement des identités, version continue]
Soit $f, g: I \to \mathbb{R}$ continues et $A \subseteq I$
\begin{itemize}
\item Si $f$ et $g$ coïncident en $A$, alors elles coïncident sur $\overline{A} \cap I$
\item En particulier, si $f$ et $g$ coïncident sur $A$ et que $A$ est dense dans $I$, alors $f = g$
\end{itemize}
\end{theorem}

\section{Fonctions continues sur un intervalle: propriétés globales}
Dans toute cette section, $I$ est un intervalle.
\subsection{Théorème des valeurs intermédiaires}
\begin{theorem}
Soit $I$ un intervalle et $f \in C^0(I)$. Soit $a < b \in I$. \\
Soit $y \in \mathbb{R}$ compris entre $f(a)$ et $f(b)$ (càd $y \in \left[f(a), f(b)\right]$ ou $y \in \left[f(b), f(a)\right]$) \\
Alors il existe $c \in [a, b]$ tel que $y = f(c)$
\end{theorem}
\begin{corollaire}
Soit $f \in C^0(I)$ et $J \subseteq I$ un intervalle. \\
Alors $f[J]$ est un intervalle. \\
"L'image continu d'un intervalle est un intervalle."
\end{corollaire}
\begin{corollaire}
Soit $I$ un intervalle et $f \in C^0(I)$. On suppose que $f$ ne s'annule pas. \\
Alors $f$ est de signe constant. On a $f > 0$ ou $f < 0$
\end{corollaire}
\begin{corollaire}[TVI généralisé]
Soit $f:]a, b[ \to \mathbb{R}$ continue et $y$ strictement compris entre $\lim\limits_{x \to a} f(x)$ et $\lim\limits_{x \to b} f(x)$ (dont on suppose qu'elles existent). \\
Alors il existe $c \in ]a, b[$ tel que $f(c) = y$
\end{corollaire}

\subsection{Fonctions continues bijectives}
\begin{theorem}[de la bijection monotone]
Soit $a < b$ deux réels et $f \in C^0([a ,b])$ strictement monotone. \\
Alors $f$ induit une bijection entre $[a, b]$ et le segment joignant $f(a)$ et $f(b)$
\end{theorem}
\begin{theorem}
Soit $I$ et $J$ deux intervalles et $f: I \to J$ une bijection continue.
Alors:
\begin{itemize}
\item $f$ est strictement monotone.
\item $f^{-1}:J \to I$ est encore continue.
\end{itemize}
\end{theorem}
\begin{proposition}
Soit $I$ un intervalle et $f: I \to \mathbb{R}$ continue et injective. \\
Alors $f$ est strictement monotone.
\end{proposition}
\begin{lemme}
Soit $g: J \to I$ une application bijective strictement monotone entre intervalles. \\
Alors $g$ est continue.
\end{lemme}

\subsection{Théorème des bornes atteintes}
\begin{theorem}[de bornes atteintes]
Soit $f: [a, b] \to \mathbb{R}$ une fonction continue sur un \uline{segment}. \\
Alors il existe $\sigma, \tau \in [a, b]$ tels que $\forall x \in [a, b]$, $f(\sigma) \leq f(x) \leq f(\tau)$ \\
"Une fonction continue sur un segment est bornée et atteint ses bornes."
\end{theorem}
\begin{corollaire}
"L'image continue d'un segment est un segment" \\
Plus précisément, soit $f: [a, b] \to \mathbb{R}$ continue. \\
Alors $f[[a, b]]$ est un segment.
\end{corollaire}

\subsection{Uniforme continuité}
\begin{definition}
$f: I \to \mathbb{R}$ est \uline{uniformément continue} si \\
$\forall \varepsilon > 0$, $\exists \delta > 0 : \forall x, y \in I$, $|x - y| \leq \delta \implies |f(x) - f(y)| \leq \varepsilon$
\end{definition}
\begin{theorem}[Heine]
Soit $f: [a, b] \to \mathbb{R}$ une fonction continue sur un segment. \\
Alors $f$ est uniformément continue.
\end{theorem}

\section{Brève extension aux fonctions à valeurs complexes}
On considère des fonction $I \to \mathbb{C}$ où $I$ est une partie de $\mathbb{R}$ (le plus souvent un intervalle). Comme dans le cas des suites, on définit $f(x) \xrightarrow[x \to a]{} l \in \mathbb{C}$ soit avec la définition usuelle (interprétée avec des modules) soit avec
\[\begin{cases}
\re(f(x)) \xrightarrow[x \to a]{} \re(l) \\ \im(f(x)) \xrightarrow[x \to a]{} \im(l)
\end{cases}\]
En particulier, si $a \in I$, $f$ est continue en $a$ \\
(ce qui signifie $\forall \varepsilon > 0$, $\exists \delta > 0: \forall x \in I$, $|x - a| \leq \delta \implies |f(x) - f(a)| \leq \varepsilon$) \\
ssi $\re f$ et $\im f: I \to \mathbb{R}$ sont continues.\medskip

\noindent On abandonne: les fonctions tendant vers $\pm\infty$, gendarmes et compagnie, limite monotone, le TVI, les bijections monotones. \\
On garde: les théorèmes d'opération, la caractérisation séquentielle, le continuité uniforme et le théorème de Heine. \\
Pour le théorème de bornes atteintes, on peut appliques la version réelle à $|f|$.
\begin{theorem}
Soit $f \in C^0([a, b]; \mathbb{C})$ une fonction continue sur un segment. \\
Alors $f$ est bornée. \\
Plus précisément, on peut trouver $\tau \in [a, b]$ tel que $\forall x \in [a, b]$, $|f(x)| \leq |f(\tau)|$
\end{theorem}
\end{document}