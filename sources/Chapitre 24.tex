\documentclass[10pt,a4paper]{article}
\usepackage[utf8]{inputenc}
\usepackage[french]{babel}
\usepackage[T1]{fontenc}
\usepackage{amsmath}
\usepackage{amsfonts}
\usepackage{amssymb}
\usepackage{graphicx}
\usepackage[left=2cm,right=2cm,top=2cm,bottom=2cm]{geometry}
\usepackage{setspace}
\usepackage{ulem}
\usepackage{stmaryrd}
\usepackage{amsthm}
\usepackage{dsfont}
\usepackage{mathpazo}


\onehalfspacing

\theoremstyle{definition}
\newtheorem{proposition}{Proposition}[section]
\newtheorem{theorem}[proposition]{Théorème}
\newtheorem{corollaire}[proposition]{Corollaire}
\newtheorem{lemme}[proposition]{Lemme}
\newtheorem{definition}[proposition]{Définition}

\DeclareMathOperator{\re}{Re}
\DeclareMathOperator{\im}{Im}

\begin{document}
\renewcommand{\labelitemi}{$*$}
\begin{center}
{\Large \textbf{Chapitre 24: Familles sommables}}
\end{center}
\section{Familles de nombres positifs}
\subsection{Généralités}
\begin{definition}
Soit $x = (x_j)_{j \in J}$ une famille de réels $\geq 0$ indexés par une ensemble $J$ \\
On définit
\[\sum\limits_{j \in J} x_j = \sup \left\{ \sum\limits_{j \in J_0} x_j \mid J_0 \in \mathcal{P}_f(J) \right\} \in [0, +\infty]\]
Avec la convention que la forme supérieure vaut $+\infty$ si l'ensemble n'est pas majoré. \\
(Ici, $\mathcal{P}_f$ désigne l'ensemble des parties finies de $J$)
\end{definition}
\begin{proposition}
Soit $x, y \in \mathbb{R}_+^J$, des familles indexées par $J$ \\
On a:
\begin{itemize}
\item \uline{Restriction}: Si $K \subseteq J$, $\sum\limits_{j \in K} x_j \leq \sum\limits_{j \in J} x_j$
\item \uline{Linéarité}: $\forall \lambda \in \mathbb{R}_+^*$, $\sum\limits_{j \in J} (x_j + \lambda y_j) = \sum\limits_{j \in J} x_j + \lambda \sum_{j \in J} y_j$
\item \uline{Croissance}: Si $\forall j \in J$, $x_j \leq y_j$, alors $\sum\limits_{j \in J} x_j \leq \sum\limits_{j \in J} y_j$
\end{itemize}
\end{proposition}
\begin{corollaire}
Supposons $J = \bigsqcup\limits_{k = 1}^n J_k$ \\
Alors, pour toute famille $x \in \mathbb{R}_+^J$, on a 
\[\sum\limits_{j \in J} x_j = \sum\limits_{k = 1}^n \sum\limits_{j \in J_k} x_j\]
\end{corollaire}

\subsection{Commutativité}
\begin{proposition}
\hfill
\begin{itemize}
\item Soit $\sigma: I \to J$ une bijection et $x \in \mathbb{R}_+^J$ \\
Alors 
\[\sum\limits_{i \in I} x_{\sigma(I)} = \sum\limits_{j \in J} x_j\]
\item En particulier, si $\sigma: J \to I$ est bijective
\[\sum\limits_{j \in J} x_j = \sum\limits_{j \in J} x_{\sigma(j)}\]
\end{itemize}
\end{proposition}

\subsection{Sommation par paquets}
\begin{theorem}[Sommation par paquets]
Soit $x \in \mathbb{R}_+^J$ et un recouvrement disjoint $J = \bigsqcup\limits_{\lambda \in \Lambda} J_\lambda$ \\
Alors
\[\sum\limits_{j \in J} x_j = \sum\limits_{\lambda \in \Lambda} \sum\limits_{j \in J_\lambda} x_j\]
\end{theorem}
\begin{corollaire}[Théorème de Fubini]
Soit $(x_{i, j})_{(i, j) \in I \times J}$ une famille de réels positifs. \\
Alors
\[ \sum\limits_{i \in I} \sum\limits_{j \in J} x_{i, j} = \sum\limits_{(i, j) \in I \times J} x_{i, j} = \sum\limits_{j \in J} \sum\limits_{i \in I} x_{i, j}\]
\end{corollaire}

\pagebreak

\section{Familles sommables de nombres complexes}
\subsection{Généralités}
\begin{definition}
\hfill
\begin{itemize}
\item Une famille $x = (x_j)_{j \in J} \in \mathbb{C}^J$ est dite \uline{sommable} si $\sum\limits_{j \in J} |x_j| < +\infty$
\item On note $l^1(J; \mathbb{C})$ l'ensemble des familles sommables indexées par $J$
\item Si $x \in \mathbb{R}^J$ est sommable, on définit sa \uline{somme} $\sum\limits_{j \in J} x_j = \sum\limits_{j \in J} x_j^+ - \sum\limits_{j \in J} x_j^-$
\item Si $x \in \mathbb{C}^J$ est sommable, on définit sa \uline{somme} $\sum\limits_{j \in J} x_j = \sum\limits_{j \in J} \re(x_j) + i \sum\limits_{j \in J} \im(x_j)$
\end{itemize}
\end{definition}
\begin{lemme}
Soit $x \in l^1(J; \mathbb{C})$. Pour tout $\varepsilon > 0$, il existe une partie $J_0 \subseteq J$ finie telle que
\[\forall K \in \mathcal{P}_f(J), \, J_0 \in K \implies \left| \sum\limits_{j \in J} x_j - \sum\limits_{j \in K} x_j \right| \leq \varepsilon\]
\end{lemme}
\begin{corollaire}
\hfill
\begin{itemize}
\item En appliquant le lemme à $\varepsilon = 2^{-n}$, on peut trouver une suite $(J_n)_{n \in \mathbb{N}}$ de parties finies de $J$ telles que $\sum\limits_{j \in J} x_j \xrightarrow[n \to +\infty]{} \sum\limits_{j \in J} x_j$ (en remplaçant $J_n$ par $J_0 \cup ... \cup J_n$ on peut même imposer qu'elle soit croissante)
\item Étant donné $x, y \in l^1(J; \mathbb{C})$ et $\varepsilon > 0$, on peut trouver $J_0$ et $K_0$ comme dans le lemme et l'ensemble fini $L_0 = J_0 \cup K_0$ vérifie
\[\left| \sum\limits_{j \in J} x_j - \sum\limits_{j \in L_0} x_j \right| \leq \varepsilon \text{\quad et \quad}
\left| \sum\limits_{j \in J} y_j - \sum\limits_{j \in L_0} y_j \right| \leq \varepsilon\]
Naturellement, cela s'étend à $\Gamma$ familles sommables.
\end{itemize}
\end{corollaire}

\subsection{Propriétés}
\begin{proposition}
Soit $x, y \in l^1(J; \mathbb{C})$ \\
On a:
\begin{itemize}
\item \uline{Restriction}: Si $K \subseteq J$, $(x_k)_{k \in K}$ est sommable.
\item \uline{Linéarité}: Pour toute $\lambda \in \mathbb{C}$, $(x_j + \lambda y_j)_{j \in J} \in l^1(J; \mathbb{C})$ et $\sum\limits_{j \in J} (x_j + \lambda y_j) = \sum\limits_{j \in J} + \sum\limits_{j \in J} y_j$
\item \uline{Croissance}: Si $x, y$ sont à valeurs réelles et que $\forall j \in J$, $x_j \leq y_j$, alors $\sum\limits_{j \in J} \leq \sum\limits_{j \in J} y_j$
\end{itemize}
\end{proposition}
\begin{proposition}[Inégalité triangulaire]
Soit $x \in l^1(J; \mathbb{C})$ \\
On a
\[ \left| \sum\limits_{j \in J} x_j \right| \leq \sum\limits_{j \in J} \left| x_j \right|\]
\end{proposition}

\subsection{Commutativité}
\begin{proposition}
Soit $x \in l^1(J; \mathbb{C})$
\begin{itemize}
\item Si $\sigma: I \to J$ est une bijection, $(x_{\sigma(i)})_{i \in I}$ est sommable et $\sum\limits_{i \in I} x_{\sigma(i)} = \sum\limits_{j \in J} x_j$
\item En particulier, si $\sigma: J \to J$, $(x_{\sigma(j)})_{j \in J}$ est sommable et $\sum\limits_{j \in J} x_{\sigma(j)} = \sum\limits_{j \in J} x_j$
\end{itemize}
\end{proposition}
\begin{corollaire}[sur les séries]
Si $\sum\limits_n x_n$ est une série absolument convergente et que $\sigma: \mathbb{N} \to \mathbb{N}$ est une bijection, alors $\sum\limits_n x_{\sigma(n)}$ est encore absolument convergente et $\sum\limits_{n = 0}^{+\infty} = \sum\limits_{n = 0}^{+\infty} x_{\sigma(n)}$
\end{corollaire}

\subsection{Sommation par paquets}
\begin{theorem}
Soit $(x_j)_{j \in J} \in \mathbb{C}^J$ et on considère un recouvrement disjoint $J = \bigsqcup\limits_{\lambda \in \Lambda}$ \\
Alors $(x_j)_{j \in J}$ est sommable ssi, pour tout $\lambda \in \Lambda$, $(x_j)_{j \in J_\lambda}$ est sommable et que $(\sum\limits_{j \in J_\lambda})_{\lambda \in \Lambda}$ est sommable. \\
Et, si c'est le cas, on a
\[ \sum_{j \in J} x_j = \sum_{\lambda \in \Lambda} \sum_{j \in J_\lambda} x_j\]
\end{theorem}
\begin{corollaire}[Théorème de Fubini]
Soit $(x_{i, j})_{i, j} \in I \times J \in \mathbb{C}^{I \times J}$ \\
Alors $(x_{i, j})_{(i, j) \in I \times J}$ est sommable ssi les familles $(x_{i, j})_{j \in J} \, (i \in I)$ le sont et que $(\sum\limits_{j \in J} x_{i, j})_{i \in I}$ soit sommable. \\
Si c'est la cas, on a
\[\sum_{(i, j) \in I \times J} x_{i, j} = \sum_{i \in I} \sum_{j \in J} x_{i, j}\]
En particulier, si $x$ est sommable
\[ \sum_{i \in I} \sum_{j \in J} x_{i, j} = \sum_{j \in J} \sum_{i \in I} x_{i, j}\]
\end{corollaire}

\subsection{Produit}
\begin{theorem}
Soit $a \in l^1(I; \mathbb{C})$ et $b \in l^1(J; \mathbb{C})$ \\
Alors la famille $(a_i b_j)_{(i, j) \in I \times J}$ est sommable et
\[\sum_{(i, j) \in I \times J} a_i b_j = \left(\sum_{i \in \mathbb{N}} a_i \right) \left(\sum_{j \in J} b_j\right)\]
\end{theorem}
\begin{definition}
Soit $\sum\limits_n a_n$ et $\sum\limits_n b_n$ deux suites (à valeurs complexes). \\
Leur \uline{produit de Cauchy} est la série $\sum\limits_n c_n$ où, pour tout $t \in \mathbb{N}$, $c_n = \sum\limits_{\substack{i, j \in \mathbb{N} \\ i + j = n}} a_i b_j = \sum\limits_{k = 0}^n a_k b_{n - k}$
\end{definition}
\begin{corollaire}
Soit $\sum\limits_n a_n$ et $\sum\limits_n b_n$ deux séries absolument convergentes. \\
Alors leur produit de Cauchy $\sum\limits_n c_n$ est une série absolument convergente et
\[\sum_{n = 0}^{+\infty} c_n = \left( \sum_{n = 0}^{+\infty} a_n \right) \left( \sum_{n = 0}^{+\infty} b_n \right)\]
\end{corollaire}
\end{document}