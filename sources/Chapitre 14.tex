\documentclass[10pt,a4paper]{article}
\usepackage[utf8]{inputenc}
\usepackage[french]{babel}
\usepackage[T1]{fontenc}
\usepackage{amsmath}
\usepackage{amsfonts}
\usepackage{amssymb}
\usepackage{graphicx}
\usepackage[left=2cm,right=2cm,top=2cm,bottom=2cm]{geometry}
\usepackage{setspace}
\usepackage{ulem}
\usepackage{stmaryrd}
\usepackage{amsthm}
\usepackage{dsfont}
\usepackage{mathpazo}

\onehalfspacing

\theoremstyle{definition}
\newtheorem{proposition}{Proposition}[section]
\newtheorem{theorem}[proposition]{Théorème}
\newtheorem{corollaire}[proposition]{Corollaire}
\newtheorem{lemme}[proposition]{Lemme}
\newtheorem{definition}[proposition]{Définition}

\DeclareMathOperator{\re}{Re}
\DeclareMathOperator{\im}{Im}

\begin{document}
\renewcommand{\labelitemi}{$*$}
\renewcommand{\labelenumi}{(\roman{enumi})}
\begin{center}
{\Large \textbf{Chapitre 14. Dérivation}}
\end{center}
\uline{Cadre}: Dans tout le chapitre, $I$ est un intervalle. \\
(même si les résultats généraux s'étendent à des parties de $\mathbb{R}$ quelconques, pourvu que $a \in I$ ne soit pas \uline{isolé} dans $I$, càd pourvu que $a \in \overline{I \setminus \{ a \}}$)

\section{Généralités}
\subsection{Définition}
\begin{definition}
Soit $a \in I$ et $f: I \to \mathbb{R}$
\begin{itemize}
\item On définit la fonction \uline{taux d'accroissement de f en a}:
\[\tau_{[f; a]} : \begin{cases}
I \setminus \{ a \} \to \mathbb{R} \\
x \mapsto \frac{f(x) - f(a)}{x - a}
\end{cases}\]
\item La fonction $f$ est \uline{dérivable en $a$} si $\tau_{[f; a]}$ possède une limite finie en a. \\
Si c'est le cas, on définit \uline{le nombre dérivé}:
\[f'(a) = \lim\limits_{x \to a} \frac{f(x) - f(a)}{x - a}\]
\item La fonction est \uline{dérivable} si elle est dérivable en tout point de $I$
\end{itemize}
On note $D^1(I) = D^1(I; \mathbb{R})$ l'ensemble des fonctions dérivables.
\end{definition}
\begin{proposition}
Soit $f:I \to \mathbb{R}$ et $a \in I$. LASSÉ:
\begin{enumerate}
\item $f$ est dérivable en a.
\item Il existe une fonction $\kappa: I \to \mathbb{R}$ continue en $a$ et telle que $\forall x \in I$, $f(x) = f(a) + \kappa(x)(x - a)$
\item Il existe $\nu \in \mathbb{R}$ et une fonction $\eta: I \to \mathbb{R}$ telle que $\begin{cases}
\forall x \in I ,\, f(x) = f(a) + \nu(x - a) + (x - a) \eta(x) \\
\eta(x) \xrightarrow[x \to a]{} 0 
\end{cases}$
\item Il existe $\nu \in \mathbb{R}$ et une fonction $\varepsilon: I_a \to \mathbb{R}$ telle que $\begin{cases}
\forall h \in I_a ,\, f(a + h) = f(a) + \nu h + h \varepsilon(h) \\
\varepsilon(h) \xrightarrow[h \to 0]{} 0
\end{cases}$
\end{enumerate}
Si c'est le cas, on a $f'(a) = \kappa(a) = \nu$
\end{proposition}
\begin{proposition}
Soit $f: I \to \mathbb{R}$ \\
Si $f$ est dérivable en $a \in I$ alors elle est continue en $a$. On a donc $D^1(I) \subseteq C^0(I)$
\end{proposition}

\subsection{Caractère local de la dérivabilité}
\begin{proposition}
Soit $f, g: I \to \mathbb{R}$ qui coïncident au voisinage de $a \in I$ \\
Alors $f$ est dérivable en $a$ ssi $g$ l'est. Si c'est le cas, $f'(a) = g'(a)$

\subsection{Dérivées à gauche et à droite}
\begin{definition}
Soit $f: I \to \mathbb{R}$ et $a \in I$
\begin{itemize}
\item (Si $a$ n'est pas l'extrémité gauche de $I$), on dit que $f$ est \uline{dérivable à gauche} en $a$ si $\tau_{[f;a]}$ admet une limite finie par valeurs inférieures en $a$. Si c'est le cas, on note
\[f'_g(a) = \lim\limits_{\substack{x \to a \\ x < a}} \frac{f(x) - f(a)}{x - a}\]
\item Idem à droite, avec
\[f'_d(a) = \lim\limits_{\substack{x \to a \\ x > a}} \frac{f(x) - f(a)}{x - a}\]
(Si cette limite existe)
\end{itemize}
\end{definition}
\end{proposition}

\subsection{Opérations}
\begin{proposition}
Soit $f,g: I \to \mathbb{R}$, $a \in I$ et $\lambda \in \mathbb{R}$
\begin{itemize}
\item Si $f$ est dérivable en $a$, $\lambda f$ aussi et $(\lambda f)' (a) = \lambda f'(a)$
\item Si $f$ est dérivable en $a$, $f + g$ aussi et $(f + g)'(a) = f'(a) + g'(a)$
\item Si $f$ et $g$ sont dérivables en $a$, $fg$ aussi et
\[(fg)'(a) = f'(a)g(a) + f(a)g'(a)\]
\item Si $f$ et $g$ sont dérivables en $a$ et que $g(a) \neq 0$, $\frac{f}{g}$ aussi et
\[\left(\frac{f}{g}\right)'(a) = \frac{f'(a)g(a) - f(a)g'(a)}{g(a)^2}\]
\end{itemize}
\end{proposition}
\begin{corollaire}
Si $f,g : I \to \mathbb{R}$ sont dérivables et que $\lambda \in \mathbb{R}$
\begin{itemize}
\item $\lambda f$, $f + g$, $fg$ sont dérivables.
\item Si en outre, $g$ ne s'annule pas sur $I$, $\frac{f}{g}$ est dérivable.
\end{itemize}
\end{corollaire}
\begin{corollaire}
$D^1(I)$ est un sous-algèbre de $C^0(I)$ (ou de $\mathbb{R}^I$), càd un sous-anneau stable par opération linéaire.
\end{corollaire}
\begin{proposition}[Dérivation des fonctions composées, ou "chain rule"]
\hfill \\
Soit $I, J \subseteq \mathbb{R}$ deux intervalles et $f: I \to J$, $g: J \to \mathbb{R}$
\begin{itemize}
\item Soit $a \in I$. Si $f$ est dérivable en $a$ et que $g$ est dérivable en $f(a)$, alors $g \circ f$ est dérivable en $a$ et \\
$(g \circ f)'(a) = g'(f(a)) f'(a)$
\item Si $f$ et $g$ sont dérivables, $g \circ f$ aussi et $(g \circ f)' = (g' \circ f)f'$
\end{itemize}
\end{proposition}

\subsection{Critère de dérivabilité des fonctions réciproques}
\begin{proposition}
Soit $I, J \subseteq \mathbb{R}$ deux intervalles, $f: I \to J$ une bijection dérivable, $a \in I$ et $b = f(a) \in J$ \\
Alors $f^{-1}$ est dérivable en $b$ ssi $f'(a) \neq 0$ \\
Si c'est le cas, on a
\[(f^{-1})'(b) = \frac{1}{f'(a)} = \frac{1}{f'(f^{-1}(b))}\]
\end{proposition}

\section{Théorèmes principaux}
\subsection{Extrema locaux}
\begin{definition}
Soit $f: I \to \mathbb{R}$ et $a \in I$
\begin{itemize}
\item On dit que $f$ \uline{possède un minimum local en $a$} s'il existe $\delta > 0$ tel que \\
$\forall x \in I$, $|x - a| \leq \delta \implies f(x) \geq f(a)$
\item On dit que $f$ \uline{possède un maximum local en $a$} s'il existe $\delta > 0$ tel que \\
$\forall x \in I$, $|x - a| \leq \delta \implies f(x) \leq f(a)$
\end{itemize}
On dit que $f$ possède un \uline{extremum local} en $a$ si elle possède un minimum ou un maximum local en $a$.
\end{definition}
\begin{definition}
Un élément $a \in I$ est dit \uline{intérieur} s'il n'est pas une extrémité de $I$.
\end{definition}
\begin{theorem}
Soit $f: I \to \mathbb{R}$ dérivable. \\
Soit $a \in I$ un point intérieur en lequel $f$ admet un extremum local. Alors $f'(a) = 0$ 
\end{theorem}
\begin{theorem}
Soit $f: I \to \mathbb{R}$ et $a \in I$ intérieur. \\
Si $f$ est dérivable en $a$ et qu'elle admet un extremum local en $a$, alors $f'(a) = 0$
\end{theorem}
\begin{definition}
Un point $a \in I$ où $f: I \to \mathbb{R}$ est dérivable et tel que $f'(a) = 0$ s'appelle un \uline{point critique} (ou un \uline{point stationnaire}) pour $f$.
\end{definition}

\subsection{Théorème de Rolle et des accroissements finis}
\begin{theorem}[Théorème de Rolle]
Soit $f: [a, b] \to \mathbb{R}$ continue et dérivable en tout point de $]a, b[$ \\
Si $f(a) = f(b)$ alors il existe $c \in \left]a, b\right[$ tel que $f'(c) = 0$
\end{theorem}
\begin{theorem}[Théorème des accroissements finis]
Soit $f: [a, b] \to \mathbb{R}$ continue, \\
dérivable en tout point de $]a, b[$ \\
Alors il existe $c \in \left]a, b\right[$ tel que
\[f'(c) = \frac{f(b) - f(a)}{b - a}\]
\end{theorem}
\medskip
\noindent \uline{Interprétation cinématique}: Dans un mouvement rectiligne, la vitesse instantanée vaut, à un certain moment, la vitesse moyenne.

\subsection{Monotonie et signe de la dérivée}
\begin{theorem}
Soit $f: I \to \mathbb{R}$ dérivable. Alors:
\begin{itemize}
\item $f$ est constante ssi $f' = 0$
\item $f$ est croissante ssi $f' \geq 0$
\item $f$ est strictement croissante ssi $f' \geq 0$ et que la restriction de $f'$ à tout intervalle non trivial est non nulle.
\end{itemize}
\end{theorem}
\begin{corollaire}
Soit $f \in C^0(I)$ \\
Si $f' > 0$ sur $I$, sauf éventuellement en un nombre fini de points, alors $f$ est strictement croissante.
\end{corollaire}

\subsection{Inégalité des accroissements finis}
\begin{proposition}
Soit $f: I \to \mathbb{R}$ dérivable et $k \in \mathbb{R}$ tel que $\forall x \in I$, $|f'(x)| \leq k$ \\
Alors $\forall x, y \in I$, $|f(y) - f(x)| \leq k |y - x|$
\end{proposition}
\begin{definition}
Soit $f: I \to \mathbb{R}$
\begin{itemize}
\item Pour $k \in \mathbb{R}_+$, on dit que $f$ est \uline{$k$-lipschitzienne} si $\forall x, y \in I$, $|f(y) - f(x)| \leq k|y - x|$
\item On dit que $f$ est \uline{lipschitzienne} s'il existe $k \in \mathbb{R}_+$ telle que $f$ soit $k$-lipschitzienne.
\item On dit que $f$ est une \uline{contraction} s'il existe $k \in [0, 1[$ telle que $f$ soit $k$-lipschitzienne.
\end{itemize}
\end{definition}
\begin{proposition}["Reformulation" de l'inégalité des accroissements finis]
Soit $f: I \to \mathbb{R}$ dérivable.
\begin{itemize}
\item Pour tout $k \in \mathbb{R}_+$, $f$ est $k$-lipschitzienne ssi $\forall x \in I$, $|f'(x)| \leq k$
\item $f$ est lipschitzienne ssi $f'$ est bornée.
\end{itemize}
\end{proposition}

\subsection{Théorème de la limite de la dérivée}
\begin{theorem}
Soit $f: I \to \mathbb{R}$ continue et $a \in I$. \\
On suppose que:
\begin{itemize}
\item $f$ est dérivable en tout point de $I \setminus \{ a \}$
\item $f'(x) \xrightarrow[\substack{x \to a \\ x \neq a}]{} l \in \mathbb{R}$
\end{itemize}
Alors $f$ est dérivable en $a$, et $f'(a) = l$
\end{theorem}
\begin{theorem}
Soit $f \in C^0(I)$, $a \in I$ tel que $f$ dérivable sur $I \setminus \{ a \}$ et $f'(x) \xrightarrow[\substack{x \to a \\ x \neq a}]{} \pm\infty$ \\
Alors
\[\frac{f(a + h) - f(a)}{h} \xrightarrow[h \to 0]{} \pm\infty\]
En particulier, $f'$ n'est pas dérivable.
\end{theorem}

\section{Fonction de classe $C^n$}
\subsection{Généralités}
On rappelle que, pour $n \geq 1$, on peut considérer l'ensemble $D^n(I) = D^n(I; \mathbb{R})$ des fonctions $n$ fois dérivables. Si $f \in D^n(I)$, on note $f^{(n)}$ la \uline{dérivée $n$-ième} de $f$.
\begin{definition}
\hfill
\begin{itemize}
\item $f: I \to \mathbb{R}$ est dite \uline{de classe $C^n$} si elle est $n$ fois dérivable et que $f^{(n)}$ est continue.
\item $f: I \to \mathbb{R}$ est dite \uline{lisse} ou \uline{de classe $C^\infty$} si elle est $n$ fois dérivable pour tout $n \geq 1$.
\end{itemize}
On note $C^n(I) = C^n(I; \mathbb{R})$ et $C^\infty(I) = C^\infty(I; \mathbb{R})$ les ensembles continues de ces fonctions. \\
Comme une fonction dérivable est continue, on a:
\[ ... \subseteq D^3(I) \subseteq C^2(I) \subseteq D^2(I) \subseteq C^1(I) \subseteq D^1(I) \subseteq C^0(I) \subseteq \mathbb{R}^I\]
et on a $C^\infty(I) = \bigcap\limits_{n \in \mathbb{N}^*} D^n(I) = \bigcap\limits_{n \in \mathbb{N}} C^n(I)$
\end{definition}

\subsection{Fonctions continûment dérivables}
\uline{Remarque}: "Continûment dérivable" = "de classe $C^1$"
\begin{proposition}[Théorème de la limite de la dérivée, version $C^1$]
Soit $f \in C^0(I)$ et $a \in I$ \\
Si $f_{|I \setminus \{ a \}}$ est de classe $C^1$ et que $f'(x) \xrightarrow[\substack{x \to a \\ x \neq a}]{} l$ alors $f \in C^1(I)$ et $f'(a) = l$
\end{proposition}
\begin{proposition}
Soit $f \in C^1([a, b])$ \\
Alors $f$ est lipschitzienne.
\end{proposition}
\begin{proposition}
Soit $f \in C^1(I; \mathbb{R})$ et $a \in I$ tel que $f'(a) > 0$ \\
Alors $f$ est strictement croissante au voisinage de $a$.
\end{proposition}

\subsection{Opérations algébriques}
\begin{proposition}
Soit $f, g: I \to \mathbb{R}$ de classe $C^n$ (resp. $n$ fois dérivables) et $\lambda \in \mathbb{R}$
\begin{itemize}
\item Alors $\lambda f$ est de classe $C^n$ (resp. $n$ fois dérivable) et $(\lambda f)^{(n)} = \lambda f^{(n)}$
\item Alors $f + g$ est de classe $C^n$ est $(f + g)^{(n)} = f^{(n)} + g^{(n)}$
\item (Formule de Leibniz) $fg$ est de classe $C^n$ et
\[(fg)^{(n)} = \sum\limits_{k = 0}^n \binom{n}{k} f^{(k)} g^{(n - k)}\]
\end{itemize}
\end{proposition}
\begin{corollaire}
$C^n(I)$ et $D^n(I)$ sont des sous-algèbres de $\mathbb{R}^I$ \\
(rappel: des sous-anneaux stables par combinaison linéaire) \\
Par intersection, il en va de même de $C^\infty(I)$
\end{corollaire}

\subsection{Composition et réciproque}
\begin{theorem}
Soit $I$ et $J$ deux intervalles de $\mathbb{R}$ et $f: I \to J$ et $g: J \to \mathbb{R}$ de classe $C^n$ (resp. $n$ fois dérivables). \\
Alors $g \circ f$ est de classe $C^n$.
\end{theorem}
\begin{theorem}
Soit $n \in \mathbb{N}^*$ et $f \in C^n(I)$ telle que $f'$ ne s'annule pas. \\
Alors $f$ induit une bijection de $I$ sur un intervalle $J$ et $f^{-1}: J \to I$ est de classe $C^n$.
\end{theorem}

\section{Brève extension aux fonctions à valeurs complexes}
\subsection{Généralités}
\begin{definition}
Soit $f: I \to \mathbb{C}$ et $a \in I$ \\
La fonction $f$ est dérivable en $a$ si $x \mapsto \frac{f(x) - f(a)}{x - a}$ possède une limite ($\in \mathbb{C}$) quand $x \to a$ \\
Si c'est le cas, cette limite est $f'(a) \in \mathbb{C}$
\end{definition}
\begin{proposition}
Soit $f: I \to \mathbb{C}$ et $c \in I$ \\
La fonction $f$ est dérivable en $a$ ssi $\re f, \im f : I \to \mathbb{R}$ le sont. \\
Si c'est le cas, $(f')(a) = \re(f)'(a) + i \im(f)'(a)$ (autrement dit: $\re f'(a) = (\re f)'(a)$, etc...).
\end{proposition}
S'étendent sans difficulté au cadre complexe: le caractère local, les théorèmes d'opération et la notion de fonction de classe $C^n$: on obtient des ensembles $D^n(I, \mathbb{C})$, $C^n(I, \mathbb{C})$, $C^\infty(I, \mathbb{C})$. \medskip

\noindent \uline{Rappel}: On a vu au chapitre $5$ que $\begin{cases}
\mathbb{R} \to \mathbb{C} \\
x \mapsto e^{\alpha x} \quad (\alpha \in \mathbb{C})
\end{cases}$
est dérivable, de dérivée $x \mapsto \alpha e^{\alpha x}$. \\
Par une récurrence immédiate, c'est juste une fonction lisse. \\
En revanche, notre section B s'écroule:
\begin{itemize}
\item La notion d'extremum n'a plus de sens
\item L'énoncé du théorème de Rolle aurait un sens, mais il est faux.
\end{itemize}
Par exemple: $f: \begin{cases}
\mathbb{R} \to \mathbb{C} \\
t \mapsto e^{it}
\end{cases}$
est lisse, vérifie $f(0) = f(2 \pi)$ et pourtant $f': t \mapsto ie^{it}$ ne s'annule jamais \\
(on a même $|f'| = 1$) \\
Rolle et TAF sont fondamentalement des théorèmes en dimension $1$.

\subsection{Inégalité des accroissements finis}
Le programme officiel énonce l'inégalité des accroissements finis pour $f \in C^1(I, \mathbb{C})$ avec la démo suivante (qu'on comprendra plus tard).
\begin{proposition}
Si $\forall t \in [a, b]$, $|f'(t)| \leq k$, on a
\[|f(b) - f(a)| = \left| \int\limits_a^b f'(t) \, dt \right| \leq \int\limits_a^b \left|f'(t)\right| \, dt \leq \int\limits_a^b k \, dt \leq k |b - a|\]
En fait, l'inégalité des accroissements finis reste vraie, pour $f \in D^1(I, \mathbb{C})$
\end{proposition}
\end{document}