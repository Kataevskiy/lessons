\documentclass[10pt,a4paper]{article}
\usepackage[utf8]{inputenc}
\usepackage[french]{babel}
\usepackage[T1]{fontenc}
\usepackage{amsmath}
\usepackage{amsfonts}
\usepackage{amssymb}
\usepackage{graphicx}
\usepackage[left=2cm,right=2cm,top=2cm,bottom=2cm]{geometry}
\usepackage{setspace}
\usepackage{ulem}
\usepackage{stmaryrd}
\usepackage{amsthm}
\usepackage{dsfont}
\usepackage{mathpazo}

\onehalfspacing

\theoremstyle{definition}
\newtheorem{proposition}{Proposition}[section]
\newtheorem{theorem}[proposition]{Théorème}
\newtheorem{corollaire}[proposition]{Corollaire}
\newtheorem{lemme}[proposition]{Lemme}
\newtheorem{definition}[proposition]{Définition}

\begin{document}
\renewcommand{\labelitemi}{$*$}
\renewcommand{\labelenumi}{(\roman{enumi})}
\begin{center}
{\Large \textbf{Chapitre 2: Sommes et produits}}
\end{center}

\section{Notation $\Sigma$}
\subsection{Définition et premières propriétés}
\begin{definition}
Soit $(a_i)_{i \in I}$ une famille de nombres réels indexée par un ensemble fini $I$ \\
On note $\sum\limits_{i \in I} a_i$ la somme des éléments de la famille. \\
Dans le cas où $I = \llbracket p, q \rrbracket$ on notera
\[ \sum_{i = p}^q a_i = \sum_{i \in \llbracket p, q \rrbracket} a_i = a_p + a_{p + 1} + ... + a_q \]
\end{definition}
\begin{proposition}
Soit $I$ un ensemble fini, $(a_i)_{i \in I}$ et $(b_i)_{i \in I} \in \mathbb{R}^I$ et $\lambda \in \mathbb{R}$ \\
On a: \\
\uline{Linéarité}: On a $\sum\limits_{i \in I} (a_i + b_i) = \sum\limits_{i \in I} a_i + \sum\limits_{i \in I} b_i$ \\
\uline{Linéarité}: On a $\sum\limits_{i \in I} (\lambda a_i) = \lambda \sum\limits_{i \in I} a_i$ \\
\uline{Positivité}: Si $\forall i \in I$, $a_i \leq b_i$, alors $\sum\limits_{i \in I} a_i \leq \sum\limits_{i \in I} b_i$
\end{proposition}
\begin{proposition}[Encadrement grossier d'une somme]
Soit $I$ un ensemble fini non vide et $(a_i)_{i \in I} \in \mathbb{R}^I$ \\
Alors
\[ |I| \times \min\{ a_i \mid i \in I \} \leq \sum_{i \in I} a_i \leq |I| \times \max\{ a_i \mid i \in I \} \]
\end{proposition}
\begin{proposition}
Soit $a, b \in \mathbb{R}$ et $n \in \mathbb{N}$ \\
Alors 
\[ b + (a + b) + (2a + b) + ... + (na + b) = \sum_{k = 0}^n (ka + b) = (n + 1) \frac{na + 2b}{2} \]
"La moyenne des termes d'une suite arithmétique est la moyenne des termes extrêmes". \\
"La somme des termes d'une suite arithmétique vaut le nombre de termes fois la moyenne des termes dans la suite arithmétique".
\end{proposition}
\begin{proposition}
\hfill
\begin{itemize}
\item Soit $p \leq a < r$ trois entiers et $(a_i)_{i = p}^r$ une famille de nombres réels. \\
Alors
\[ \sum_{i = p}^r a_i = \sum_{i = p}^q a_i + \sum_{i = q + 1}^r a_i \]
(Relation de Chasles)
\item Soit $(J_1, ...\,, J_s)$ un recouvrement disjoint d'un ensemble $I$ et $(a_i)_{i \in I} \in \mathbb{R}^I$ \\
Alors
\[\sum_{i \in I} a_i = \sum_{i \in J_1} a_i + \sum_{ i \in J_2} a_i + ... + \sum_{i \in J_s} = \sum_{k = 1}^s \sum_{i \in J_k} a_i \]
\end{itemize}
\end{proposition}
\noindent \uline{Remarque cruciale}: L'addition est commutative. \\
En particulier, si deux familles $(a_i)_{i \in I}$ et $(b_j)_{j \in J}$ prennent les mêmes valeurs, le même nombre de fois, alors
\[ \sum_{i \in I} a_i = \sum_{j \in J} b_j \]
Notamment, si $(a_i)_{i = p}^q$ est une famille de réels:
\[ \sum_{i = p}^q a_i = \sum_{j = 0}^{q - p} a_{j + p} = \sum_{k = 0}^{q - p} a_{q - k} \]
on vient d'effectuer \uline{un changement d'indice}.
\begin{proposition}[Télescopage]
Soit $(a_i)_{i = p}^{q + 1}$ une famille de réels. \\
Alors
\[ \sum_{i = p}^q (a_{i + 1} - q_i) = a_{q + 1} - a_p \]
\end{proposition}
\noindent \uline{Exemples importants}:
\[ \sum_{k = 1}^n \frac{1}{k(k + 1)} = \sum_{k = 1}^n \left(\frac{1}{k} - \frac{1}{k + 1}\right) = \frac{n}{n + 1} \]
\[ \sum_{k = 0}^n [(k + 1)^3 - k^3] = (n + 1)^3 \quad \text{ et } \quad \sum_{k = 0}^n [(k + 1)^3 - k^3] = 3\sum_{k = 0}^n k^2 + 3\sum_{k = 0}^n k + \sum_{k = 0}^n 1 \]
D'où
\[ \sum_{k = 0}^n k^2 = \frac{n(n+1)(2n + 1)}{6} \]

\section{Produits}
\begin{definition}
Si $(a_i)_{i \in I}$ est une famille de réels indexée par un ensemble fini $I$, \\
on note $\prod\limits_{i \in I} a_i$ le produit des éléments de cette famille.
\end{definition}
\begin{proposition}
Soit $(a_i)_{i \in I}$ et $(b_i)_{i \in I}$ deux familles indexées par le même ensemble fini $I$
\begin{itemize}
\item On a
\[\prod\limits_{i \in I}(a_i b_i) = \prod\limits_{i \in I} a_i \prod\limits_{i \in I} b_i\]
\item On a, pour tout $n \in \mathbb{N}$
\[\prod\limits_{i \in I}(a_i^n) = \left(\prod\limits_{i \in I} a_i \right)^n\]
\item Règle du produit nul: $\prod\limits_{i \in I} a_i = 0 \iff \exists i \in I: a_i = 0$
\item Si $\forall i \in I$, $a_i \geq 0$, alors $\prod\limits_{i \in I} a_i \geq 0$
\item Si $\forall i \in I$, $0 \leq a_i \leq b_i$, on a $\prod\limits_{i \in I} a_i \leq \prod\limits_{i \in I} b_i$
\item Si $(J_1, ...\,, J_s)$ est un recouvrement disjoint de $I$, alors
\[\prod_{i \in I} a_i = \left( \prod_{i \in J_1} a_i \right) ... \left( \prod_{i \in J_s} a_i \right) = \prod_{k = 1} \prod_{i \in J_k} a_i \]
\end{itemize}
\end{proposition}
\begin{definition}
Soit $n \in \mathbb{N}$ \\
On définit \uline{factorielle $n$}
\[ n! = \prod_{i = 1}^n i \]
\end{definition}
\begin{proposition}
Soit $(a_i)_{i = k}^{q + 1}$ une famille de réels \uline{non nuls}. \\
Alors
\[ \prod_{i = p}^q \frac{a_{i + 1}}{a_i} = \frac{a_{q + 1}}{a_p} \]
\end{proposition}

\section{Somme géométrique}
\subsection{Formule de base}
\begin{theorem}
Soit $q \in \mathbb{R}$ et $n \in \mathbb{N}$ \\
On a
\[ (q - 1)(q^n + q^{n - 1} + ... + q + 1) = (q - 1)\sum_{k = 0}^n q^k = q^{n + 1} - 1 \]
En particulier, si $q \neq 1$
\[ \sum_{k = 0}^n q^k = \frac{q^{n + 1} - 1}{q - 1} \]
\end{theorem}
\noindent \uline{Remarque}: De même, si $q \neq 1$
\[ \sum_{k = a}^b q^k  = \frac{q^{b + 1} - q^a}{q - 1} \]
\[\frac{\text{après-dernier terme } - \text{ premier terme}}{q - 1}\]

\subsection{Variants}
\begin{proposition}
Soit $a, b \in \mathbb{R}$ et $n \in \mathbb{N}^*$ \\
On a
\begin{align*}
a^n - b^n &= (a - b)(a^{n - 1} + a^{n - 2}b + ... + ab^{n - 2} + b^{n - 1})\\
&= (a - b)\left( \sum_{k = 0}^{n - 1} a^k b^{n - 1 -k} \right)
\end{align*}
Si $n$ est impair
\begin{align*}
a^n + b^n &= (a + b)(a^{n - 1} - a^{n - 2}b + ... - ab^{n - 2} + b^{n - 1})\\
&= (a + b)\left( \sum_{k = 0}^{n - 1} (-1)^k a^k b^{n - 1 -k} \right)
\end{align*}
\end{proposition}

\section{Formule de binôme de Newton}
\subsection{Coefficients binomiaux}
\noindent On rappelle que pour tous $n, k \in \mathbb{N}$, $\binom{n}{k}$ est le cardinal de $\mathcal{P}(\llbracket 1, n \rrbracket)$ \\
En particulier $\binom{n}{k} = 0$ dès que $k > n$ \\
On étend la notion en posant $\binom{n}{k} = 0$ si $k < 0$ \\
On obtient ainsi des coefficients binomiaux $\binom{n}{k}$ pour $k \in \mathbb{Z}$ et $n \in \mathbb{N}$
\begin{proposition}[Formule de Pascal]
Soit $n \in \mathbb{N}$ et $k \in \mathbb{Z}$ \\
On a
\[ \binom{n + 1}{k + 1} = \binom{n}{k} + \binom{n}{k + 1} \]
\end{proposition}
\begin{theorem}
Soit $n \in \mathbb{N}$ et $k \in \mathbb{Z}$ \\
On a
\[\binom{n}{k} = \begin{cases}
\frac{n!}{k!(n - k)!} = \frac{n(n - 1)...(n - k + 1)}{k} \text{ si } k \in \llbracket 0, n \rrbracket \\
0 \text{ sinon }
\end{cases} \]
\end{theorem}
\noindent \uline{Remarque}: Le théorème montre que $!k$ divise le produit de $k$ entiers consécutifs.
\begin{proposition}[Symétrie des binomiaux]
Soit $k \in \mathbb{Z}$ et $n \in \mathbb{N}$ \\
On a
\[ \binom{n}{k} = \binom{n}{n - k} \]
\end{proposition}

\subsection{Binôme de Newton}
\begin{theorem}
Soit $a, b \in \mathbb{R}$ et $n \in \mathbb{N}$ \\
On a
\[ (a + b)^n = \sum_{k = 0}^n \binom{n}{k} a^k b^{n - k} = \sum_{k = 0}^n \binom{n}{k} a^{n - k} b^k \]
\end{theorem}
\noindent \uline{Exemples importants}:
\[ \sum_{k = 0}^n \binom{n}{k} = \sum_{k = 0}^n \binom{n}{k} 1^k 1^{n - k} = (1 + 1)^n = 2^n \]
\[ \sum_{k = 0}^n (-1)^k \binom{n}{k} = \sum_{k = 0}^n \binom{n}{k} (-1)^k 1^{n - k} = (1 - 1)^n = 0^n = \begin{cases}
1 \text{ si } n = 0 \\
0 \text{ si } n > 0
\end{cases} \]
Donc si $n > 0$, on a
\[ \sum_{\substack{k \in \llbracket 0, n \rrbracket \\ k \text{ paire }}} \binom{n}{k} - \sum_{\substack{k \in \llbracket 0, n \rrbracket \\ k \text{ impaire }}} \binom{n}{k} = 0 \]

\section{Sommes doubles}
\begin{proposition}
Soit $(a_{i, j})_{(i, j) \in \llbracket p, q \rrbracket \times \llbracket r, s \rrbracket}$ une famille de nombres réels. \\
Alors
\[ \sum_{(i, j) \in \llbracket p, q \rrbracket \times \llbracket r, s \rrbracket} a_{i, j} = \sum_{i = p}^q \sum_{j = r}^s a_{i, j} = \sum_{j = r}^s \sum_{i = p}^q a_{i, j} \]
\end{proposition}
\begin{proposition}
Soit $(a_i)_{i = p}^q$ et $(b_j)_{j = r}^s$ deux familles de réels. \\
Alors
\[ \left(\sum_{i = p}^q a_i\right) \left(\sum_{j = r}^s b_j\right) = \sum_{i = p}^q \sum_{j = r}^s a_i b_j = \sum_{j = r}^s \sum_{i = p}^q a_i b_j \]
\end{proposition}
\begin{corollaire}
Soit $\left(a_i^{(1)}\right)_{i_1 \in I_1}, ...\,, \left( a_i^{(d)}\right)_{i_d \in I_d}$ $d$ familles de nombres réels indexées par des ensembles finis $I_1, ...\,, I_d$ \\
Alors
\[ \prod_{j = 1}^d \sum_{i_j \in I_j} a_{i_j}^{(j)} = \left(\sum_{i_1 \in I_1} a_{i_1}^{(1)} \right) ... \left( \sum_{i_d \in I_d} a_{i_d}^{(d)}\right) = \sum_{(i_1, ...\,, i_d) \in I_1 \times ... \times I_d} a_{i_1}^{(1)} ... a_{i_d}^{(d)}\]
\end{corollaire}

\subsection{Sommes triangulaires}
\begin{proposition}
Soit $(a_{i, j})_{1 \leq i \leq j \leq n}$ une famille indexée par $\left\{ (i, j) \in \llbracket 1, n \rrbracket^2 \mid i \leq j \right\}$ \\
Alors
\[ \sum_{1 \leq i \leq j \leq n} a_{i, j} = \sum_{i = 1}^n \sum_{j = i}^n a_{i, j} = \sum_{j = 1}^n \sum_{i = 1}^j a_{i, j} \]
\end{proposition}

\subsection{Carré d'une somme}
\begin{proposition}
Si $(a_i)_{i = 1}^n$ est une famille de nombres, on a
\[ \left( \sum_{i = 1}^n a_i \right)^2 = \left(\sum_{i = 1}^n a_i \right) \left( \sum_{j = 1}^n a_j \right) = \sum_{1 \leq i, j \leq n} a_i a_j = \sum_{i = 1}^n a_i^2 + 2\sum_{1 \leq i < j \leq n} a_i a_j \]
\end{proposition}
\end{document}
