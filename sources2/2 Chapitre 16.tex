\documentclass[10pt,a4paper]{article}
\usepackage[utf8]{inputenc}
\usepackage[french]{babel}
\usepackage[T1]{fontenc}
\usepackage{amsmath}
\usepackage{amsfonts}
\usepackage{amssymb}
\usepackage{graphicx}
\usepackage[left=2cm,right=2cm,top=2cm,bottom=2cm]{geometry}
\usepackage{setspace}
\usepackage{ulem}
\usepackage{stmaryrd}
\usepackage{amsthm}
\usepackage{dsfont}
\usepackage{mathpazo}

\onehalfspacing{}
\theoremstyle{definition}
\newtheorem{proposition}{Proposition}[section]
\newtheorem{theorem}[proposition]{Théorème}
\newtheorem{corollary}[proposition]{Corollaire}
\newtheorem{lemme}[proposition]{Lemme}
\newtheorem{definition}[proposition]{Définition}

\DeclareMathOperator{\re}{Re}
\DeclareMathOperator*{\vect}{Vect}

\begin{document}
\renewcommand{\labelitemi}{\textbullet}

\renewcommand{\labelenumi}{(\roman{enumi})}

\begin{center}
{\Large \textbf{Chapitre 16. Intégrales à paramètre}}
\end{center}

\noindent \(I, J\) intervalles d'intérieur non vide, \(E\) un evn.

\renewcommand{\labelenumi}{\arabic{enumi}.}

\section{Le théorème de convergence dominée}
\subsection{Le théorème de convergence dominée}
\begin{theorem}[ Théorème de convergence monotone ]
    Soit \(g_n: I \to [0, +\infty]\) mesurables et \(g_n: I \to [0, +\infty]\) \\
    On suppose que:
    \begin{enumerate}
        \item \((g_n)_{n \in \mathbb{N}}\) converge simplement vers \(g\)
        \item \(\forall n \in \mathbb{N}\), \(0 \leq g_n \leq g_{n + 1}\)
    \end{enumerate}
    Alors dans \([0, +\infty]\)
    \[\lim_{n \to +\infty}\int_{I} g_n = \int_{I} g \in [0, +\infty]\]
\end{theorem}
\begin{corollary}
    Soit \(F_0 : I \to \mathbb{R}_+\) intégrable et \(F_n: I \to \mathbb{R}_+\) \((n \geq 1)\) mesurable. \\
    On suppose:
    \begin{enumerate}
        \item \(\forall n \in \mathbb{N}\), \(0 \leq F_{n + 1} \leq F_n\)
        \item \((F_n)\) converge simplement vers 0
    \end{enumerate}
    Alors
    \[\int_{I} F_n \xrightarrow[n \to +\infty]{} 0\]
\end{corollary}

\subsection{Énoncé du théorème de convergence dominée}
\begin{theorem}[ Théorème de convergence dominée ]
    Soit \(f_n: I \to \mathbb{K}\) continue par morceaux \((n \in \mathbb{N})\) \\
    On suppose:
    \begin{enumerate}
        \item \((f_n)_{n \geq 0}\) converge simplement vers \(f: I \to \mathbb{K}\) continue par morceaux sur \(I\)
        \item Il existe \(\varphi: I \to \mathbb{R}_+\) \uline{intégrable} telle que \(\forall n \in \mathbb{N}, \forall t \in I \quad |f_n(x)| \leq \varphi(t)\) ( Hypothèse de domination )
    \end{enumerate}
    Alors les \(f_n\) sont intégrables et \(f\) aussi et
    \[\boxed{\int_{I}f_n \xrightarrow[n \to +\infty]{} \int_{I}f}\]
    On a même
    \[\lVert f_n - f \rVert_1 = \int_{I}|f_n - f| \xrightarrow[n \to +\infty]{} 0\]
\end{theorem}

\subsection{Premiers exemples d'application}
\noindent Quelques exercices classiques:
\begin{enumerate}
    \item Montrer que \[I_n = \int_{0}^{\frac{\pi}{2}} \sin^{n}t \,dt \xrightarrow[n \to +\infty]{} 0\]
    \item Soit \(f: [0, 1] \to \mathbb{K}\) continue. \\ Montrer que \[I_n = \int_{0}^{1} f\left(\frac{x}{n}\right) dx \xrightarrow[n \to +\infty]{} f(0)\]
    \item Soit \(f: [0, 1] \to \mathbb{K}\) continue avec \(\lim\limits_{+\infty}f = 0\). \\ Montrer que \[I_n = \int_{0}^{1} f(nx) dx \xrightarrow[n \to +\infty]{} f(0)\]
    \item Soit \(n \geq 1\) et \[I_n = \int_{1}^{+\infty}e^{-x^{n}} dx\] Montrer que \(I_n \xrightarrow[n \to +\infty]{} 0\) et \(I_n \underset{+\infty}{\sim} \frac{\alpha}{n}\) avec \(\alpha\) à exprimer avec une intégrale.
    \item Montrer que \[I_n = \int_{0}^{n}\left(1 - \frac{x^2}{n^2}\right)^{n^2} dx \xrightarrow[n \to +\infty]{} \int_{0}^{+\infty}e^{-x^2} dx\]
\end{enumerate}

\subsection{Théorème de convergence dominée appliquée à l'interversion série / suite}
\begin{corollary}[ Théorème de convergence dominée ]
    Soit \(f_n: I \to \mathbb{K}\) continues par morceaux \((n \in \mathbb{N})\) \\
    On suppose que:
    \begin{enumerate}
        \item \(\sum f_n\) converge simplement vers \(\sum\limits_{n = 0}^{+\infty} f_n\) \(\mathcal{C}^0\) par morceaux.
        \item Il existe \(\varphi: I \to \mathbb{R}_+\) intégrable telle que \(\forall n \in \mathbb{N}\, \left| \sum\limits_{k = 0}^{n} f_k \right| \leq \varphi\) ( Domination )
    \end{enumerate}
    Alors les \(f_n\) sont intégrables, \(\sum f_n\) est intégrable et
    \[\boxed{\int_{I} \sum_{n = 0}^{+\infty} f_n = \sum_{n = 0}^{+\infty}\int_{I} f_n}\]
\end{corollary}

\subsection{Le théorème d'intégration terme à terme}
\begin{theorem}[ Théorème d'intégration terme à terme pour les fonctions positives ]
    \hfill \\
    Soit \(f_n: I \to \mathbb{R}_+\) intégrable \((n \in \mathbb{N})\) \\
    On suppose que \(\sum f_n\) converge simplement et que \(\sum\limits_{n = 0}^{+\infty} f_n\) est continue par morceaux. \\
    Alors dans \([0, +\infty]\) on a
    \[\int_{I}\sum_{n = 0}^{+\infty}f_n = \sum_{n = 0}^{+\infty} \int_{I} f_n\]
    En particulier
    \[\sum_{n = 0}^{+\infty} f_n \text{ intégrable } \iff \sum_{n \in \mathbb{N}} \int_{I} f_n < +\infty\]
    Dans ces conditions, dans \(\mathbb{R}_+^*\)
    \[\int_{I}\sum_{n = 0}^{+\infty}f_n = \sum_{n = 0}^{+\infty} \int_{I} f_n\]
\end{theorem}
\begin{theorem}[ Théorème d'intégration terme à terme ]
    Soit \(f_n: I \to \mathbb{K}\) intégrables \((n \in \mathbb{N})\) \\
    On suppose que:
    \begin{enumerate}
        \item \(\sum f_n\) converge simplement vers une fonction continue par morceaux.
        \item \(\sum\limits_{n \in \mathbb{N}}\int\limits_{I}|f_n| < +\infty\)
    \end{enumerate}
    Alors \(\sum\limits_{n = 0}^{+\infty} f_n\) est intégrable et
    \[\boxed{\int_{I}\sum_{n = 0}^{+\infty}f_n = \sum_{n = 0}^{+\infty}\int_{I} f_n}\]
    De plus
    \[\left| \int_{I}\sum_{n = 0}^{+\infty}f_n \right| \leq \sum_{n = 0}^{+\infty} \int_{I}|f_n|\]
\end{theorem}

\section{Continuité et dérivabilité des intégrales à paramètre}
\subsection{Convergence dominée avec un paramètre continue}
\begin{corollary}
    Soit \(f:(x, t) \in A \times I \mapsto f(x, t) \in \mathbb{K}\) avec \(A \subset E\) ( \(E\) evn ) et \(a\) adhérent à \(A\) \\
    Soit \(g: I \to \mathbb{K}\) continue par morceaux. \\
    On suppose:
    \begin{enumerate}
        \item Pour tout \(t \in I\), \(f(x, t) \xrightarrow[x \to a]{}g(t)\)
        \item Pour tout \(x \in A\), \(t \mapsto f(x, t)\) est \(\mathcal{C}^0\) par morceaux.
        \item Il existe \(\varphi: I \to \mathbb{R}_+\) \uline{intégrable} telle que \(\forall (x, t) \in A \times I\), \(|f(x, t)| \leq \varphi\) ( Domination )
    \end{enumerate}
    Alors pour tout \(x \in A\), \(f(x, \cdot)\) ( ie. \(t \mapsto f(x, t)\) ) est intégrable, \(g\) aussi et
    \[\boxed{\int_{I} f(x, t) \,dt \xrightarrow[x \to a]{} \int_{I} g(t) \,dt}\]
\end{corollary}

\subsection{Continuité}
\begin{theorem}
    Soit \(A \subset E\) ( \(E\) evn ) et \(f: (x, t) \in A \times I \mapsto f(x, t) \in \mathbb{K}\) \\
    On suppose:
    \begin{enumerate}
        \item Pour tout \(x \in A\), \(t \mapsto f(x, t)\) est continue par morceaux.
        \item Pour tout \(t \in I\), \(x \mapsto f(x, t)\) est continue.
        \item Il existe \(\varphi: I \to \mathbb{R}_+\) intégrable telle que \(\forall (x, t) \in A \times I\), \(|f(x, t)| \leq \varphi\) ( Domination )
    \end{enumerate}
    Alors \[F: x \in A \mapsto \int_{I} f(x, t) \,dt\] est bien définie et continue en \(A\)
\end{theorem}

\subsection{Dérivation sous le signe intégral}
\begin{theorem}[ Formule de Leibniz ]
    Soit \(f:\begin{cases} J \times I \to \mathbb{K} \\ (x, t) \mapsto f(x, t)\end{cases}\) \\
    On suppose:
    \begin{enumerate}
        \item À \(x\) fixé \(t \mapsto f(x, t)\) est intégrable sur \(I\)
        \item À \(t\) fixé \(x \mapsto f(x, t)\) est \(\mathcal{C}^1\) sur \(J\)
        \item À \(x\) fixé \(t \mapsto \frac{\partial f}{\partial x}(x, t)\) est continue par morceaux.
        \item Il existe \(\varphi: I \to \mathbb{R}_+\) intégrable telle que \(\forall (x, t) \in J \times I\), \(\left| \frac{\partial f}{\partial x}(x, t) \right| \leq \varphi(t)\) ( Domination )
    \end{enumerate}
    Alors \[F: x \in J \mapsto \int_{I} f(x, t) \,dt\] est de classe \(\mathcal{C}^1\) et \(\forall x \in J\)
    \[F'(x) = \int_{I} \frac{\partial f}{\partial x}(x, t) \,dt\]
\end{theorem}
\begin{proposition}
    Soit \(f:\begin{cases} J \times I \to \mathbb{K} \\ (x, t) \mapsto f(x, t)\end{cases}\) et \(k \in \mathbb{N}^*\) \\
    On suppose:
    \begin{enumerate}
        \item À \(t\) fixé \(x \mapsto f(x, t)\) est \(\mathcal{C}^k\) sur \(J\)
        \item Pour tout \(0 \leq i \leq k - 1\), à \(x\) fixé \(t \mapsto \frac{\partial^i f}{\partial x^i}(x, t)\) est intégrable sur \(I\)
        \item À \(x\) fixé \(t \mapsto \frac{\partial^k f}{\partial x^k}(x, t)\) est continue par morceaux.
        \item Il existe \(\varphi: I \to \mathbb{R}_+\) intégrable telle que \(\forall (x, t) \in J \times I\), \(\left| \frac{\partial^k f}{\partial x^k}(x, t) \right| \leq \varphi(t)\) ( Domination )
    \end{enumerate}
    Alors \[F: x \in J \mapsto \int_{I} f(x, t) \,dt\] est de classe \(\mathcal{C}^k\) et \(\forall i \in \llbracket 1, k \rrbracket,\, x \in J\)
    \[\boxed{F^{(i)}(x) = \int_{I} \frac{\partial^i f}{\partial x^i}(x, t) \,dt}\]
\end{proposition}
\begin{corollary}
    Soit \(f:\begin{cases} J \times I \to \mathbb{K} \\ (x, t) \mapsto f(x, t)\end{cases}\) \\
    On suppose:
    \begin{enumerate}
        \item À \(t\) fixé \(x \mapsto f(x, t)\) est \(\mathcal{C}^{+\infty}\) sur \(J\)
        \item Pour tout \(k \in \mathbb{N}\), à \(x\) fixé \(t \mapsto \frac{\partial^i f}{\partial x^i}(x, t)\) est continue par morceaux.
        \item Pour tout \(k \in \mathbb{N}\) il existe \(\varphi_k: I \to \mathbb{R}_+\) intégrable avec \(\forall (x, t) \in J \times I\), \(\left| \frac{\partial^k f}{\partial x^k}(x, t) \right| \leq \varphi_k(t)\) ( Domination )
    \end{enumerate}
    Alors \[F: x \in J \mapsto \int_{I} f(x, t) \,dt\] est de classe \(\mathcal{C}^{+\infty}\) et \(\forall k \geq 0 ,\, x \in J\)
    \[\boxed{F^{(i)}(x) = \int_{I} \frac{\partial^i f}{\partial x^i}(x, t) \,dt}\]
\end{corollary}

\subsection{La fonction \(\Gamma\) d'Euler (HP)}
\begin{definition}
    Pour \(x > 0\) on pose
    \[\Gamma(x) = \int_{0}^{+\infty}t^{x - 1}e^{-t} \,dt\]
    C'est la fonction Gamma d'Euler.
\end{definition}
\noindent \uline{Extension}: Si \(z \in \mathbb{C}\) avec \(\re(z) > 0\) alors on peut définir \(\Gamma(z)\) de la même façon.
\begin{proposition}
    \hfill
    \begin{itemize}
        \item Pour \(x > 0\), \(\Gamma(x + 1) = x\Gamma(x)\) \\ En particulier \(\forall n \in \mathbb{N}\), \(\Gamma(n + 1) = n!\)
        \item \[\Gamma\left(\frac{1}{2}\right) = \int_{\mathbb{R}} e^{-x^2} \,dx = \sqrt{\pi}\]
    \end{itemize}
\end{proposition}
\begin{proposition}
    La fonction \(\Gamma\) est de classe \(\mathcal{C}^{+\infty}\) et pour \(x > 0,\, k \in \mathbb{N}\)
    \[\Gamma^{(k)}(x) = \int_{0}^{+\infty}(\ln t)^k t^{x - 1} e^{-t} \,dt\]
    \(\Gamma\) est convexe et \(\lim\limits_{0^+} \Gamma = \lim\limits_{+\infty} \Gamma = +\infty\)
\end{proposition}

\subsection{Fonctions à support compact}
\begin{definition}
    Soit \(f: \mathbb{R}^n \to \mathbb{K}\) \\
    Le support de \(f\) est \(S = \overline{\{ x \in \mathbb{R}^n \mid f(x) \neq 0\}}\) \\
    On sit que \(f\) est à support compact si \(S\) est compact ie. borné, autrement dit:
    \[\exists R > 0 \, \forall x \in \mathbb{R}^n,\, \lVert x\rVert > R \implies f(x) = 0\]
\end{definition}
\begin{definition}
    Une suite de fonctions régularisantes est une suite de fonctions \\
    \(\varphi: \mathbb{R} \to \mathbb{R}_+\) \(\mathcal{C}^{\infty}\) à support contenu dans \([-a_n, a_n]\) où:
    \begin{itemize}
        \item \(a_n\) est une suite de \(\mathbb{R}_+^*\)
        \item \(a_n\) décroît.
        \item \(a_n\) tend vers \(0\)
    \end{itemize}
    et vérifiant \(\int\limits_{\mathbb{R}} \varphi_n = 1\)
\end{definition}

\subsection{Intégrales doubles sur un pavé}
\begin{theorem}[ Théorème de Fubini ]
    Soit \(a \leq b,\, v \leq d\) et \(f:[a, b] \times [c, d] \to \mathbb{K}\) \\
    Alors
    \[\int_{a}^{b} \left(\int_{c}^{d} f(x, y) \,dy\right) \,dx = \int_{c}^{d} \left(\int_{a}^{b} f(x, y) \,dx\right) \,dy\]
    Cette valeur commune est appelée
    \[\iint\limits_{[a, b] \times [c, d]}f(x, y) \, dx dy\]
\end{theorem}

\subsection{Fonctions intégrables sur \(I \times J\)}
\begin{proposition}
    Soit \(f: I \times J \to \uline{\mathbb{R}_+}\) \\
    Sous reserve de régularité de fonction on a dans \[0, +\infty\]
    \[\int_{I} \left(\int_{J} f(x, y) \,dy\right) \,dx = \int_{J} \left(\int_{I} f(x, y) \,dx\right) \,dy\]
    On note la valeur commune
    \[\iint\limits_{I \times J} f \in [0, +\infty]\]
    Si \(\iint\limits_{I \times J} f < +\infty\) on dit que \(f\) est intégrable.
\end{proposition}

\section{Convergence en moyenne quadratique}
\subsection{La convergence en moyenne}
\begin{definition}
    Soit \(f_n \in L^1(I, \mathbb{K})\) et \(f \in L^1(I, \mathbb{K})\) \(n \in \mathbb{N}\) \\
    On dit que \(f_n\) converge en moyenne vers \(f\) si \[\lVert f_n - f \rVert_1 \to 0\] \\
    On dit que \(f_n\) converge en moyenne quadratique vers \(f\) si \[\lVert f_n - f \rVert_2 = \sqrt{\int_{I}|f_n - f|^2} \to 0\]
    On note \(L^\infty(I, \mathbb{K})\) l'ensemble des fonctions continues par morceaux sur \(I\) bornées. \\
    Alors \[\boxed{L^\infty \subset L^2(I) \subset L^1(I)}\]
    Ces inclusions sont strictes si \(I\) n'est pas un segment.
\end{definition}

\subsection{Complément: Base hilbertienne}
\begin{theorem}[ Inégalité de Bessel-Parseval ]
    Soit \(E\) préhilbertien, \((e_n)_{n \in \mathbb{N}}\) un système orthonormé et \(x \in E\) \\
    Alors \(\left( \langle e_n, x \rangle^2 \right)_{n \in \mathbb{N}}\) est sommable ( ie. \(\left(\langle e_n, x \rangle \right)_{n \in \mathbb{N}} \in l^2\) ) et
    \[\sum_{n \in \mathbb{N}} \langle e_n, x \rangle^2 \leq \lVert x \rVert^2\]
\end{theorem}
\begin{definition}
    Soit \(E\) préhilbertien et \((e_n)_{n \in \mathbb{N}}\) un système orthonormé. \\
    On dit que\((e_n)_{n \in \mathbb{N}}\) est une base hilbertienne si \(\overline{\vect\limits_{n \in \mathbb{N}}(e_n)} = E\)
\end{definition}
\end{document}