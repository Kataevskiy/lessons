\documentclass[10pt,a4paper]{article}
\usepackage[utf8]{inputenc}
\usepackage[french]{babel}
\usepackage[T1]{fontenc}
\usepackage{amsmath}
\usepackage{amsfonts}
\usepackage{amssymb}
\usepackage{graphicx}
\usepackage[left=2cm,right=2cm,top=2cm,bottom=2cm]{geometry}
\usepackage{setspace}
\usepackage{ulem}
\usepackage{stmaryrd}
\usepackage{amsthm}
\usepackage{dsfont}
\usepackage{mathpazo}

\usepackage{array}
\newcolumntype{M}[1]{>{\centering\arraybackslash}m{#1}}

\onehalfspacing

\theoremstyle{definition}
\newtheorem{proposition}{Proposition}[section]
\newtheorem{theorem}[proposition]{Théorème}
\newtheorem{corollaire}[proposition]{Corollaire}
\newtheorem{lemme}[proposition]{Lemme}
\newtheorem{definition}[proposition]{Définition}

\DeclareMathOperator{\re}{Re}
\DeclareMathOperator{\im}{Im}

\begin{document}
\renewcommand{\labelitemi}{$*$}
\begin{center}
{\Large \textbf{Chapitre 21: Intégration}}
\end{center}
Dans tout le chapitre: $I$ désigne un intervalle de $\mathbb{R}$

\section{Fonctions en escalier, fonctions continues par morceau}
\subsection{Subdivisions d'un segment}
\begin{definition}
\hfill
\begin{itemize}
\item Une \uline{subdivision} du segment $[a, b]$ est une famille $\sigma = (x_{0}, ... , x_{n})$, où \\ $a = x_{0} < x_{1} < ... < x_{n} = b$
\item les $x_{i}$ sont les \uline{points} de la subdivision
\item les intervalles $\left[x_{i}, x_{i+1}\right]$ ( resp. $\left]x_{i}, x_{i+1}\right[$ ) sont les \uline{composantes fermées} (resp. \uline{ouverte}) de $\sigma$
\item le \uline{pas} de la subdivision $\sigma$ est $\max{\{x_{i+1} - x_{i} \mid i \in \llbracket 0, n-1 \rrbracket\}}$
\end{itemize}
\end{definition}
\begin{definition}
Soit $\sigma$, $\sigma'$ deux subdivisions d'un segment $[a, b]$. \\
On dit que $\sigma'$ \uline{raffine} $\sigma$ (ou: est plus fine que $\sigma$) si toute composante (ouverte) de $\sigma'$ est incluse dans une composante (ouverte) de $\sigma$.
\end{definition}
\begin{proposition}
Deux subdivisions $\sigma_{1}$, $\sigma_{2}$ de $[a, b]$ possèdent toujours un raffinement commun.
\end{proposition}

\subsection{Fonctions en escalier}
\begin{definition}
\hfill
\begin{itemize}
\item Une fonction $\varphi: [a, b] \to \mathbb{R}$ sera dite \uline{en escalier} s'il existe une subdivision $\sigma$ de $[a, b]$ telle que $\varphi$ soit constante sur chaque composante de $\sigma$
\item On dit alors que $\sigma$ est \uline{adaptée} à $\varphi$
\end{itemize}
\end{definition}
\begin{proposition}
L'ensemble $\mathcal{E}([a, b])$ des fonctions en escalier sur $[a, b]$ est une sous-algèbre de $\mathbb{R}^{[a, b]}$ \\
et, $\forall f \in \mathcal{E}([a, b])$, $|f| \in \mathcal{E}([a, b])$
\end{proposition}

\subsection{Fonctions continues par morceaux}
\begin{definition}
Une fonction $f: [a,b] \to \mathbb{R}$ est \uline{continue par morceaux} s'il existe une subdivision $\sigma = (x_{0}, ... ,x_{n})$ de $[a, b]$ telle que:
\begin{itemize}
\item la restriction $f_{|]x_{i}, x_{i+1}[}$ de $f$ à chaque composante ouverte est continue
\item $f$ admet des limites à gauche (resp. à droite) en tout point de la subdivision, sauf $a = x_{0}$ (resp. $b = x_{0}$)
\end{itemize}
\end{definition}
\begin{lemme}
L'ensemble $C^{0}_{pm}([a, b])$ des fonctions continues par morceaux est la somme $C^{0}([a, b]) + \mathcal{E}([a, b])$.
\end{lemme}
\begin{corollaire}
Toute fonction continue par morceaux est bornée.
\end{corollaire}

\section{Convergence uniforme}
\subsection{Convergence simple}

\begin{definition}
Soit $(f_n)_{n \in \mathbb{N}}$ une suite de fonctions $I \rightarrow \mathbb{R}$. \\
On dit que $(f_n)_{n \in \mathbb{N}}$ \uline{converge simplement} vers $f: I \to \mathbb{R}$ si $\forall x \in I$, $f_n(x) \xrightarrow[n \to +\infty]{} f(x)$ \\
On notera
\[f_n \xrightarrow[n \to +\infty]{CS} f\]
\end{definition}
\begin{definition}
Si $f: I \to \mathbb{R}$ est bornée, on définit sa \uline{norme uniforme}: $ \lVert f \rVert_\infty$ = $\sup{\{ |f(t)| \mid t \in I \}}$
\end{definition}
\begin{proposition}
La norme uniforme $\lVert . \rVert_\infty$ est une \uline{norme} sur l'espace vectoriel $L^\infty(I)$ des fonctions $I \to \mathbb{R}$ bornées:
\begin{itemize}
\item Positivité: $\forall f \in L^\infty(I)$, $\lVert f \rVert_\infty \geq 0$
\item Séparation: $\forall f \in L^\infty(I)$, $\lVert f \rVert_\infty = 0 \implies f = 0$
\item Homogénéité: $\forall f \in L^\infty(I)$, $\forall \lambda \in \mathbb{R}$, $\lVert \lambda f \rVert_\infty = |\lambda| \cdot \lVert f \rVert_\infty$
\item Inégalité triangulaire: $\forall f, g \in L^\infty(I)$, $\lVert f + g \rVert_\infty \leq \lVert f \rVert_\infty + \lVert g \rVert_\infty$
\end{itemize}
\end{proposition}
\begin{definition}
Soit $(f_n)_{n \in \mathbb{N}}$ une suite de fonctions $I \to \mathbb{R}$ bornées et $f: I \to \mathbb{R}$ bornée. \\
On dit que $(f_n)_{n \in \mathbb{N}}$ \uline{converge uniformément} vers $f$ si $\lVert f_n - f \rVert_\infty \xrightarrow[n \to +\infty]{} 0$ \\
On note alors
\[f_n \xrightarrow[n \to +\infty]{CU} f\]
\end{definition}
\begin{proposition}
Soit $(\varphi_n)_{n \in \mathbb{N}}$ et $(\psi_n)_{n \in \mathbb{N}}$  deux suites de fonctions bornées sur $I$ et $f, g: I \to \mathbb{R}$ bornées telles que
\[\begin{cases}\varphi_n \xrightarrow[n \to +\infty]{CU} f \\
\psi_n \xrightarrow[n \to +\infty]{CU} g
\end{cases}\]
Alors:
\begin{itemize}
\item $\forall \lambda \in \mathbb{R}$, $\varphi_n + \lambda \psi_n \xrightarrow[n \to +\infty]{CU} f + \lambda g$
\item $|\varphi_n| \xrightarrow[n \to +\infty]{CU} f$ 
\end{itemize} 
\end{proposition}
\begin{theorem}
Soit $(\varphi_n)_{n \in \mathbb{N}}$ une suite de fonctions bornées et $f: I \to \mathbb{R}$ bornée telle que $\varphi_n \xrightarrow[n \to +\infty]{CU} f$. \\
Alors, si pour tout $n \in \mathbb{N}$, $\varphi_n$ est continue, $f$ l'est aussi. 
\end{theorem}

\subsection{Approximation uniforme}
\begin{theorem}
Soit $f \in C^{0}_{pm}([a, b])$. \\
Alors il existe une suite $(\varphi_n)_{n \in \mathbb{N}}$ de fonctions en escalier telle que $\varphi_n \xrightarrow[n \to +\infty]{CU} f$
\end{theorem}

\section{Définition de l'intégrale}
\subsection{Intégrale des fonctions en escalier}
\begin{definition}
Soit $\varphi \in \mathcal{E}([a, b])$ et $\sigma = (a = x_0, ... , x_n = b)$ une subdivision adaptée à $\varphi$. \\
On peut donc écrire 
\[\varphi = \sum\limits_{i=0}^{n} \lambda_i \mathds{1}_{x_i} + \sum\limits_{j=0}^{n-1} \mu_j \mathds{1}_{]x_j, x_{j+1}[}\]
On définit alors \uline{l'intégrale de $\varphi$}: 
\[\int\limits_a^b \varphi = \sum\limits_{j=0}^{n-1} \mu_j (x_{j+1} - x_j)\]
\end{definition}

\pagebreak

\begin{proposition}
\hfill
\begin{itemize}
\item Cette intégrale est bien définie.
\item L'intégrale est une forme linéaire $\int\limits_a^b : \mathcal{E}([a, b]) \to \mathbb{R}$
\item (Inégalité triangulaire \textnormal{\&} contrôle uniforme): $ \forall f \in \mathcal{E}([a, b])$,
\[\left|\int\limits_a^b \varphi \right| \leq \int\limits_a^b |\varphi| \leq (b - a)\lVert \varphi \rVert_\infty\]
\item Relation de Chasles: si $a < b < c$, on a $\forall \varphi \in \mathcal{E}([a, b])$, $\int\limits_a^c \varphi = \int\limits_a^b \varphi + \int\limits_b^c \varphi$
\end{itemize}
\end{proposition}

\subsection{Lemme fondamental et définition}
\begin{theorem}
Soit $f \in C_{pm}^{0}([a, b])$ et $(\varphi_n)_{n \in \mathbb{N}}$ une suite d'éléments de $\mathcal{E}([a, b])$ \\
convergent uniformément vers $f$. \\
Alors:
\begin{itemize}
\item La suite $(\int\limits_a^b \varphi_n)_{n \in \mathbb{N}}$ converge.
\item Si $(\psi)_{n \in \mathbb{N}} \in \mathcal{E}([a, b])^{\mathbb{N}}$ vérifie également $\psi_n \xrightarrow[n \to +\infty]{CU} f$, alors $\lim\limits_{n \to +\infty} \int\limits_a^b \varphi_n = \lim\limits_{n \to +\infty} \int\limits_a^b \varphi_n$
\end{itemize}
\end{theorem}
\begin{definition}
Soit $f \in C_{pm}^{0}([a, b])$ \\
On définit \uline{l'intégrale de $f$}: 
$\int\limits_a^b f = \int\limits_a^b f(t)dt$ comme la limite $\lim\limits_{n \to +\infty} \int\limits_a^b \varphi_n$ où $(\varphi_n)_{n \in \mathbb{N}}$ est une suite de fonctions en escalier convergeant uniformément vers f.
\end{definition}

\subsection{Propriétés de base}
\begin{theorem}
\hfill
\begin{itemize}
\item L'intégrale est une forme linéaire $\int\limits_a^b : C_{pm}^{0}([a, b]) \to \mathbb{R}$
\item Inégalité triangulaire et contrôle uniforme: $\forall f \in C_{pm}^{0}([a, b])$,
\[\left|\int\limits_a^b f \right| \leq \int\limits_a^b |f| \leq (b-a) \lVert f \rVert_\infty\]
\item si $f, g \in C_{pm}^{0}([a, c])$ et que $f$ et $g$ coïncident sur le complémentaire d'un ensemble fini, alors $\int\limits_a^b f = \int\limits_a^b g$
\item \uline{Positivité}: Soit $f \in C_{pm}^{0}([a, b])$ positive. Alors $\int\limits_a^b f \geq 0$
\item \uline{Croissance}: Soit $f, g \in C_{pm}^{0}([a, b])$ telles que $f \leq g$. Alors $\int\limits_a^b f \leq \int\limits_a^b g$
\end{itemize}
\end{theorem}
\begin{theorem}[stricte positivité]
\hfill
\begin{itemize}
\item Soit $f \in C_{pm}^{0}([a, b])$ à valeurs $\geq 0$. Soit $x_0 \in [a, b]$ un point en lequel $f$ est continue et $f(x_0) > 0$ \\
Alors $\int\limits_{a}^{b} f > 0$
\item Soit $f \in C_{pm}^{0}([a, b])$ à valeurs $\geq 0$ \\
Si $\int\limits_{a}^{b} f = 0$, alors $f = 0$
\end{itemize}
\end{theorem}
\begin{proposition}
Soit $(f_n)_{n \in \mathbb{N}}$ une suite de fonctions continues par morceaux sur $[a, b]$ qui converge uniformement vers $f \in C_{pm}^{0}([a, b])$ \\
Alors $\int\limits_{a}^{b} f_n \xrightarrow[n \to +\infty]{} \int\limits_{a}^{b} f$
\end{proposition}

\subsection{Brève extension aux fonctions à valeurs complexes}
\begin{definition}
\hfill
\begin{itemize}
\item Une fonction $f: [a, b] \to \mathbb{C}$ est dite \uline{continue par morceaux} si $\re f$ et $\im f: [a, b] \to \mathbb{R}$ sont continues par morceaux.
\item Pour une telle fonction, on définit
\[\int\limits_{a}^{b} f = \int\limits_{a}^{b} \re(f) + i \int\limits_{a}^{b} \im(f)\]
\end{itemize}
\end{definition}
Les propriétés "algébriques" (linéarité, Chasles, etc...) s'étendent sans difficulté.
\begin{proposition}
Soit $f \in C_{pm}^{0}([a, b]; \mathbb{C})$ \\
On a
\[\left| \int\limits_{a}^{b} f \right| \leq \int\limits_{a}^{b} |f|\]
\end{proposition}

\section{Calcul intégral}
\subsection{"Théorème fondamental"}
\begin{theorem}
Soit $f \in C_{pm}^{0}([a, b])$
\begin{itemize}
\item Alors $x \to \int\limits_{a}^{x} f$ est une \uline{primitive} de $f$ (càd une fonction dérivable $F$ telle que $F' = f$
\item \uline{Les} primitives de $f$ sont les fonctions de la forme $x \to \int\limits_{a}^{x} f + \kappa$, où $\kappa \in \mathbb{R}$
\end{itemize}
\end{theorem}
\begin{corollaire}
Toute fonction continue sur un intervalle $I$ admet une primitive/
\end{corollaire}
\begin{corollaire}
Soit $f \in C_{pm}^{0}([a, b])$ et $F$ une primitive de $f$. \\
Alors $\int\limits_{a}^{b} f = F(b) - F(a) = \left[ F \right]_a^b = \left[F(x)\right]_{x = a}^b$
\end{corollaire}
\begin{corollaire}[IAF pour $f \in C^1(I; \mathbb{C})$]
Soit $f \in C^1(I; \mathbb{C})$ et $M \in \mathbb{R}_+$ tel que $\forall x \in I$, $|f'(x)| \leq M$ \\
Alors $f$ est $M$-lipschitzienne: $\forall a, b \in I$, $|f(b) - f(a)| \leq M |b - a|$
\end{corollaire}

\subsection{Formulaire}
\begin{center}
\begin{tabular}{ M{14em} M{14em} M{14em} }
Sur un intervalle inclus dans & les primitives de & sont: \\
\hline
$\mathbb{R}$ & $\exp$ & $\exp + \kappa$ \\
\hline
$\mathbb{R}^*$ & $x \mapsto \frac{1}{x}$ & $x \mapsto \ln|x| + \kappa$ \\
\hline
$\mathbb{R}$ (si $\alpha \in \mathbb{N}$) & & \\
$\mathbb{R}^*$ (si $\alpha \in \mathbb{Z}$) & $x \mapsto x^\alpha$ & $x \mapsto \frac{x^{\alpha + 1}}{\alpha + 1} + \kappa$ \\
$\mathbb{R}_+^*$ (si $\alpha \in \mathbb{R}$) & & \\
\hline
$\mathbb{R}$ & $\sin$ & $-\cos + \kappa$ \\
& $\cos$ & $\sin + \kappa$ \\
& $\sinh$ & $\cosh + \kappa$ \\
& $\cosh$ & $\sinh + \kappa$ \\
\hline
$\mathbb{R}$ & $x \mapsto \frac{1}{1 + x^2}$ & $\arctan + \kappa$ \\
\hline
$\left] 1, 1 \right[$ & $x \mapsto \frac{1}{\sqrt{1 - x^2}}$ & $\arcsin + \kappa$
\end{tabular}
\end{center}
C'est un tableau de dérivées à l'envers.

\subsection{Intégration par parties}
\begin{theorem}
Soit $f, g \in C^1([a, b])$ \\
On a:
\[ \int_a^b f g' = \underbrace{f(b)g(b) - f(a)g(a)}_{\left[fg\right]_a^b} - \int_a^b f' g\]
\end{theorem}

\subsection{Changement de variables}
\begin{theorem}
Soit $f \in C^0(I)$ et $\varphi : [a, b] \mapsto I$ de classe $C^1$ \\
On a alors:
\[ \int_a^b f( \varphi(t)) \varphi'(t) \, dt = \int_{\varphi(a)}^{\varphi(b)} fu \, du\]
\end{theorem}

\subsection{Exponentielle fois (co)sinus}
\noindent \uline{Exemple}: Cherchons les primitives de $x \mapsto e^{2x} \cos(3x)$ \medskip

\noindent On a $\forall x \in \mathbb{R}$, $e^{2x} \cos(3x) = \re\underbrace{(e^{2x} e^{3x})}_{= e^{(2 + 3i)x}}$ \\
Une primitive de $x \mapsto e^{(2 + 3i)x}$ est $x \mapsto \frac{1}{2 + 3i} e^{(2 + 3i)x}$ \\
Donc une primitive de $x \mapsto e^{2x} \cos(3x) = \re(e^{(2 + 3i)x})$ est
\begin{align*}
x &\mapsto \re\left(\frac{1}{2 + 3i}e^{(2 + 3i)x}\right) = \re\left(\frac{2 - 3i}{13} e^{(2 + 3i)x} \right)\\
&= \frac{2}{13}\re\left(e^{(2 + 3i)x}\right) + \frac{3}{13}\im\left(e^{(2 + 3i)x}\right) \\
&= \frac{2}{13}e^{2x}\cos(3x) + \frac{3}{13}e^{2x}\sin(3x) \\
&= \frac{e^{2x}}{13}(2 \cos(3x) + 3 \sin(3x))
\end{align*}

\pagebreak

\subsection{Polynômes trigonométriques}
Il est facile d'interpréter un polynôme trigonométrique une fois linéarisé: \medskip

\noindent \uline{Exemple}: Trouvont une primitive de $x \mapsto \cos^3(x)$ \\
Pour $x \in \mathbb{R}$,
\begin{align*}
\cos^3(x) &= \left(\frac{e^{ix} + e^{-ix}}{2}\right)^3 \\
&= \frac{1}{8}(e^{i3x} + 3e^{ix} + 3e^{-ix} + e^{-i3x}) \\
&= \frac{\cos(3x)}{4} + \frac{3}{4}\cos(x)
\end{align*}
Donc une primitive de $x \mapsto \cos^3(x)$ est $x \mapsto \frac{\sin(3x)}{12} + \frac{3}{4}\sin(x)$

\subsection{Fractions rationnelles}
La méthode standard pour calculer une intégrale de $F \in \mathbb{R}(X)$ est de décomposer en éléments simples et d'intégrer les différents éléments simples. \medskip

\noindent \uline{Première espèce}: Une primitive de $x \mapsto \frac{1}{(x - a)^n}$ est $x \mapsto \ln|x - a|$ \\
Une primitive de $x \mapsto \frac{1}{(x - a)^n}$ est $x \mapsto \frac{1}{1 - n} \frac{1}{(x - a)^{n - 1}}$ \medskip

\noindent \uline{Deuxième espèce, exposant 1}: \\
Il y a deux briques de base:
\begin{itemize}
\item $\frac{Q'}{Q}$ qui donnera une primitive $x \mapsto \ln|Q(x)|$
\item $\frac{1}{Q}$, que l'on ramène à $\frac{1}{X^2	+ 1}$, qui va donner des primitives de $\arctan$
\end{itemize}
Par CL, on obtient tous les $\frac{P}{Q}$, $P \in \mathbb{R}_1[X]$ \medskip

\noindent \uline{Deuxième espèce, exposant quelconque}: On obtient une relation de récurrence par intégration par parties.

\section{Sommes de Riemann}
\begin{theorem}
Soit $f \in C_{pm}^0([a, b])$ \\
Alors
\[ \frac{b - a}{n} \sum_{k = 0}^{n - 1} f\left( a + k \frac{b - a}{n}\right) \xrightarrow[n \to +\infty]{} \int_a^b f\]
\end{theorem}

\section{Formules de Taylor globales}
\subsection{Formule de Taylor avec reste intégrale}
\begin{theorem}
Soit $f \in C^{n + 1}(I)$ et $a \in I$ \\
On a $\forall x \in I$
\[f(x) = \sum_{k = 0}^n \frac{f^{(k)}(a)}{k!}(x - a)^k + \int_a^x \frac{(x - t)^n}{n!} f^{(n + 1)}(t) \, dt\]
\end{theorem}

\subsection{Inégalité de Taylor-Lagrange}
\begin{theorem}
Soit $f \in \mathbb{C}^{n + 1}(I)$ et $a \in I$ tel que $f^{(n + 1)}$ soit bornée. \\
On a $\forall x \in I$
\[\left| f(x) - \sum_{k = 0}^n \frac{f^{(k)}(a)}{k!}(x - a)^k \right| \leq \frac{|x - a|^{n + 1}}{(n + 1)!} ||f^{(n + 1)}||_\infty\]
\end{theorem}
\end{document}