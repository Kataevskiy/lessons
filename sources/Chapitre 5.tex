\documentclass[10pt,a4paper]{article}
\usepackage[utf8]{inputenc}
\usepackage[french]{babel}
\usepackage[T1]{fontenc}
\usepackage{amsmath}
\usepackage{amsfonts}
\usepackage{amssymb}
\usepackage{graphicx}
\usepackage[left=2cm,right=2cm,top=2cm,bottom=2cm]{geometry}
\usepackage{setspace}
\usepackage{ulem}
\usepackage{stmaryrd}
\usepackage{amsthm}
\usepackage{dsfont}
\usepackage{mathpazo}

\onehalfspacing

\theoremstyle{definition}
\newtheorem{proposition}{Proposition}[section]
\newtheorem{theorem}[proposition]{Théorème}
\newtheorem{corollaire}[proposition]{Corollaire}
\newtheorem{lemme}[proposition]{Lemme}
\newtheorem{definition}[proposition]{Définition}

\usepackage{array}
\newcolumntype{M}[1]{>{\centering\arraybackslash}m{#1}}

\DeclareMathOperator{\gr}{gr}
\DeclareMathOperator{\re}{Re}
\DeclareMathOperator{\im}{Im}
\DeclareMathOperator{\sgn}{sgn}

\begin{document}
\renewcommand{\labelitemi}{$*$}
\begin{center}
{\Large \textbf{Chapitre 5: Fonctions réelles}}
\end{center}

\section{Intervalles}
\subsection{Segments}
\begin{definition}
Soit $a \leq b \in \mathbb{R}$ \\
On définit le \uline{segment} $[a, b] = \left\{ x \in \mathbb{R} \mid a \leq x \leq b \right\}$
\end{definition}
\begin{proposition}
Soit $a \leq b \in \mathbb{R}$ \\
On a $[a, b] = \left\{ (1 - \lambda) a + \lambda b \mid \lambda \in [0, 1] \right\}$
\end{proposition}

\subsection{Intervalles}
\begin{definition}
Une partie $I \subseteq \mathbb{R}$ est un \uline{intervalle} si $\forall x, y \in I$, $x \leq y \implies [x, y] \subseteq I$
\end{definition}

\section{Généralités sur les fonctions réelles}
\begin{definition}
Soit $D \subseteq \mathbb{R}$, $f, g: D \to \mathbb{R}$ et $\lambda \in \mathbb{R}$ \\
On définit:
\begin{itemize}
\item Le produit $\lambda f: \begin{cases}
D \to \mathbb{R} \\
x \mapsto \lambda f(x)
\end{cases}$
\item La somme $f + g: \begin{cases}
D \to \mathbb{R} \\
x \mapsto f(x) + g(x)
\end{cases}$
\item Le produit $f g: \begin{cases}
D \to \mathbb{R} \\
x \mapsto f(x) g(x)
\end{cases}$
\end{itemize}
\end{definition}

\subsection{Symétrie}
\begin{definition}
Soit $T > 0$
\begin{itemize}
\item On appelle \uline{domaine $T$-périodique} une partie $D \subseteq \mathbb{R}$ telle que $\forall x \in D$, $(x + T \in D \text{ et } x - T \in D)$
\item Soit $D$ un domaine $T$-périodique. \\
Une fonction $f: D \to \mathbb{R}$ est dite $T$-périodique si $\forall x \in D$, $f(x + T) = f(x)$
\end{itemize}
\end{definition}

\begin{proposition}
Soit $T > 0$
\begin{itemize}
\item Une partie $D \subseteq \mathbb{R}$ est $T$-périodique si et seulement si $\forall x_0 \in D$, $\forall x_1 \in \mathbb{R}$, $x_0 \equiv x_1 (\text{mod }T) \implies x_1 \in D$
\item Soit $D$ un domaine $T$-périodique et $f: D \to \mathbb{R}$ \\
Alors $f$ est $T$-périodique si et seulement si $\forall x_0, x_1 \in D$, $x_0 \equiv x_1 (\text{ mod}T) \implies f(x_0) = f(x_1)$
\end{itemize}
\end{proposition}
\begin{proposition}
Soit $T > 0$
\begin{itemize}
\item La somme et le produit de deux fonctions $T$-périodiques est $T$-périodique.
\item Si $f$ est $T$-périodique, toute composée $g \circ f$ est également $T$-périodique.
\end{itemize}
\end{proposition}
\begin{definition}
\hfill
\begin{itemize}
\item Une partie $D \subseteq \mathbb{R}$ est dite \uline{symétrique} (par rapport à $0$) si $\forall x \in D$, $-x \in D$
\item Soit $D \subseteq \mathbb{R}$ symétrique et $f: D \to \mathbb{R}$ \\
On dit que $f$ est:
\begin{itemize}
\item \uline{paire} si $\forall x \in D$, $f(-x) = f(x)$
\item \uline{impaire} si $\forall x \in D$, $f(-x) = -f(x)$
\end{itemize}
\end{itemize}
\end{definition}

\pagebreak

\begin{proposition}
\hfill
\begin{itemize}
\item La somme de deux fonctions $\begin{cases}
\text{ paires } \\
\text{ impaires }
\end{cases}$ est $\begin{cases}
\text{ paire } \\
\text{ impaire }
\end{cases}$
\item Le produit de deux fonctions $\begin{cases}
\text{paires} \\
\text{impaires}
\end{cases}$ est paire.
\item Le produit d'une fonction paire et d'une impaire est impaire.
\item Une composée $g \circ f$ où $f$ est paire est paire.
\item Si les deux fonctions sont paires ou impaires, $g \circ f$ a la parité suivante:
\begin{center}
\begin{tabular}{M{3em} | M{3em} | M{3em}}
$f$ \textbackslash  $\,\,g$ & p & i \\
\hline
p & p & p \\
\hline
i & p & i
\end{tabular}
\end{center}
\end{itemize}
\end{proposition}

\subsection{Monotonie}
\begin{definition}
Soit $D \subseteq \mathbb{R}$ et $f: D \to \mathbb{R}$ \\
On dit que:
\begin{itemize}
\item $f$ est \uline{croissante} si $\forall x, y \in D$, $x \leq y \implies f(x) \leq f(y)$
\item $f$ est \uline{strictement croissante} si $\forall x, y \in D$, $x < y \implies f(x) < f(y)$
\item $f$ est \uline{décroissante} si $\forall x, y \in D$, $x \leq y \implies f(x) \geq f(y)$
\item $f$ est \uline{strictement décroissante} si $\forall x, y \in D$, $x < y \implies f(x) > f(y)$
\item $f$ est \uline{(strictement) monotone} si elle est (strictement) croissante ou (strictement) décroissante.
\end{itemize}
\end{definition}
\begin{proposition}
\hfill
\begin{itemize}
\item La somme de deux fonctions $\begin{cases}
\text{ croissantes } \\
\text{ décroissantes }
\end{cases}$ est $\begin{cases}
\text{ croissante } \\
\text{ décroissante }
\end{cases}$
\item La somme d'une fonction $\begin{cases}
\text{ croissante } \\
\text{ décroissante }
\end{cases}$ est d'une fonction $\begin{cases}
\text{ croissante } \\
\text{ décroissante }
\end{cases}$ \\ 
est strictement $\begin{cases}
\text{ croissante } \\
\text{ décroissante }
\end{cases}$
\item La composée de deux fonctions (strictement) monotones est (strictement) monotone, et la monotonie de $g \circ f$ est donnée par:
\begin{center}
\begin{tabular}{M{3em} | M{3em} | M{3em}}
$f$ \textbackslash  $\,\,g$ & $\nearrow$ & $\searrow$ \\
\hline
$\nearrow$ & $\nearrow$ & $\searrow$ \\
\hline
$\searrow$ & $\searrow$ & $\nearrow$
\end{tabular}
\end{center}
\end{itemize}
\end{proposition}
\begin{proposition}[Injectivité des fonctions strictement monotones]
Soit $D \subseteq \mathbb{R}$ est $f: D \to \mathbb{R}$
\begin{itemize}
\item Si $f$ croît strictement, alors: $\forall x, y \in D$, $f(x) \leq f(y) \implies x \leq y$
\item Si $f$ décroît strictement, alors: $\forall x, y \in D$, $f(x) \leq f(y) \implies x \geq y$
\item Dans les deux cas, $f$ est injective.
\end{itemize}
\end{proposition}
\begin{proposition}
Soit $D, E \subseteq \mathbb{R}$ et $f: D \to E$
\begin{itemize}
\item Si $f$ est bijective et croissante, alors $f$ est strictement croissante et $f^{-1}: E \to D$ aussi.
\item Si $f$ est bijective et décroissante, alors $f$ est strictement décroissante et $f^{-1} : E \to D$ aussi.
\end{itemize}
\end{proposition}

\pagebreak

\subsection{Bornes et extrema}
\begin{definition}
Soit $D \subseteq \mathbb{R}$ et $f: D \to \mathbb{R}$ \\
On dit que $f$:
\begin{itemize}
\item Est \uline{majorée} si $\exists M \in \mathbb{R}: \forall x \in D$, $f(x) \leq M$
\item Est \uline{minorée} si $\exists m \in \mathbb{R}: \forall x \in D$, $f(x) \geq m$
\item Est \uline{bornée} si elle est minorée est majorée.
\item \uline{Admet un maximum} si $\exists c \in D: \forall x \in D$, $f(x) \leq f(c)$
\item \uline{Admet un minimum} si $\exists d \in D: \forall x \in D$, $f(x) \geq f(d)$
\end{itemize}
\end{definition}
\begin{proposition}
Soit $D \subseteq \mathbb{R}$ et $f: D \to \mathbb{R}$ \\
Alors $f$ est bornée si et seulement si $\exists c \in \mathbb{R}_+: \forall x \in D$, $\left| f(x) \right| \leq c$
\end{proposition}
\begin{proposition}
\hfill
\begin{itemize}
\item La somme de deux fonctions $\begin{cases}
\text{ minorées } \\
\text{ majorées }
\end{cases}$ est $\begin{cases}
\text{ minorée } \\
\text{ majorée }
\end{cases}$
\item La somme et le produit de deux fonctions bornées sont bornés.
\item Si $g$ est $\begin{cases}
\text{ minorée } \\
\text{ majorée } \\
\text{ bornée }
\end{cases}$ alors toute composée de la forme $g \circ f$ est $\begin{cases}
\text{ minorée } \\
\text{ majorée } \\
\text{ bornée}
\end{cases}$
\end{itemize}
\end{proposition}

\subsection{Transformations d'un graphe}
\noindent Étant donné $f: D \to \mathbb{R}$
\begin{itemize}
\item Pour $a \in \mathbb{R}$, le graphe de $f + a: \begin{cases}
D \to \mathbb{R} \\
x \mapsto f(x) + a
\end{cases}$ \\
est l'image de $\gr(f)$ par la translation de vecteur $\begin{pmatrix}
0 \\
a
\end{pmatrix}$
\item Pour $a \in \mathbb{R}$, le graphe de $f( \cdot + a) : \begin{cases}
D - a \to \mathbb{R} \\
x \mapsto f(x + a)
\end{cases}$ \\
est l'image de $\gr(f)$ par la translation de vecteur $\begin{pmatrix}
-a \\
0
\end{pmatrix}$
\item Pour $a \in \mathbb{R}$, le graphe de $f(a - \cdot) : \begin{cases}
D' \to \mathbb{R} \\
x \mapsto f(a - x)
\end{cases}$ où $D' = \left\{ x \in \mathbb{R} \mid a - x \in D \right\}$ \\
est l'image de $\gr(f)$ par la réflexion d'axe, la droite d'équation $x = \frac{a}{2}$
\item Pour $\lambda \neq 0$, le graphe de $\lambda f: \begin{cases}
D \to \mathbb{R} \\
x \mapsto \lambda f(x)
\end{cases}$ \\
est l'image de $\gr(f)$ par $\begin{pmatrix}
x \\
y
\end{pmatrix} \mapsto \begin{pmatrix}
x \\
\lambda y
\end{pmatrix}$
\item Pour $\lambda \neq 0$, le graphe de $f( \lambda \cdot): \begin{cases}
D' \to \mathbb{R} \\
x \mapsto f( \lambda x)
\end{cases}$ où $D' = \left\{ x \in \mathbb{R} \mid \lambda x \in D \right\}$ \\
est l'image de $\gr(f)$ par $\begin{pmatrix}
x \\
y
\end{pmatrix} \mapsto \begin{pmatrix}
\frac{x}{\lambda} \\
y
\end{pmatrix}$
\end{itemize}

\pagebreak

\subsection{Limites}
À partir de maintenant, $I$ est un intervalle non trivial et $x_0$ est un élément ou une borne de $I$
\begin{definition}
Soit $l \in \mathbb{R}$ \\
On dit que $f$ \uline{converge (ou tend)} vers $l$ en $x_0$ et on note $f(x) \xrightarrow[x \to x_0]{} l$ si \\
$\forall \varepsilon > 0$, $\exists \delta > 0: \forall x \in I$, $\left| x - x_0 \right| \leq \delta \implies \left| f(x) - l \right| \leq \varepsilon$
\end{definition}

\subsection{Continuité}
Ici, $I$ est un intervalle ,$x_0 \in I$ et $f: I \to \mathbb{R}$
\begin{definition}
\hfill
\begin{itemize}
\item La fonction $f$ est \uline{continue en $x_0$} si $f(x) \xrightarrow[x \to x_0]{} f(x_0)$
\item La fonction $f$ est \uline{continue} si elle est continue en tout point de $I$
\end{itemize}
On note $C^0(I) = C^0(I; \mathbb{R})$ l'ensemble des fonctions $I \to \mathbb{R}$ continues.
\end{definition}
\begin{theorem}
L'ensemble des fonctions continues est stable par somme, par produit, par quotient (si le dénominateur ne s'annule pas), par composition...
\end{theorem}

\subsection{Dérivabilité}
Ici, $I$ est un intervalle, $x_0 \in I$, $f: I \to \mathbb{R}$
\begin{definition}
\hfill
\begin{itemize}
\item On appelle le \uline{taux d'accroissement de $f$ en $x_0$} la fonction
\[\tau_{\left[f, x_0\right]}: \begin{cases}
I \setminus \{ x_0 \} \to \mathbb{R} \\
x \mapsto \frac{f(x) - f(x_0)}{x - x_0}
\end{cases}\]
\item On dit que $f$ est \uline{dérivable en $x_0$} si $\tau_{\left[f, x_0 \right]}$ admet une limite finie en $x_0$
\item Si c'est le cas, on note
\[f'(x_0) = \lim\limits_{x \to x_0} \tau_{ \left[ f, x_0 \right] }(x) = \lim\limits_{x \to x_0} \frac{f(x) - f(x_0)}{x - x_0}\]
le \uline{nombre dérivé de $f$ en $x_0$}
\item On dit que $f$ est \uline{dérivable} si elle est dérivable en tout point de $I$
\item Si c'est le cas, on note
\[f': \begin{cases}
I \to \mathbb{R} \\
x \mapsto f'(x)
\end{cases}\]
\end{itemize}
\end{definition}
\begin{definition}
Si $f: I \to \mathbb{R}$ est dérivable en $x_0$, la tangente à $\gr(f)$ en $x_0$ est la droite passant par $\begin{pmatrix}
x_0 \\
f(x_0)
\end{pmatrix}$ et de pente $f'(x_0)$, càd la droite d'équation $y = f'(x_0)(x - x_0) + f(x_0)$
\end{definition}
\begin{proposition}
Si $f$ est dérivable en $x_0$, elle est continue en $x_0$ \\
On note $D^1(I) = D^1(I; \mathbb{R})$ l'ensemble des fonctions dérivables sur $I$ \\
D'après la proposition précédente, $D^1(I) \subseteq C^0(I)$
\end{proposition}
\begin{theorem}
L'ensemble des fonctions dérivables est stable par somme, produit, quotient (si le dénominateur ne s'annule pas).
\end{theorem}
\begin{theorem}
Soit $I$ et $J$ deux intervalles de $\mathbb{R}$ et $f:I \to J$ et $g: J \to \mathbb{R}$ deux fonctions dérivables. \\
Alors $g \circ f$ est dérivable et $(g \circ f)' = (g' \circ f) \times f'$
\end{theorem}

\subsection{Tableau de variations}
Dans toute la section, $I \subseteq \mathbb{R}$ est un intervalle non trivial et $f: I \to \mathbb{R}$
\begin{theorem}
Supposons $f: I \to \mathbb{R}$ dérivable.
\begin{itemize}
\item La fonction $f$ est constante ssi $\forall x \in I$, $f'(x) = 0$
\item La fonction $f$ est croissante ssi $\forall x \in I$, $f'(x) \geq 0$
\item Si $f'$ est $> 0$ sur $I$, à l'exception éventuelle d'un nombre fini de points, alors $f$ est strictement croissante.
\end{itemize}
\end{theorem}
\begin{theorem}[des valeurs intermédiaires]
Soit $a \leq b$ deux éléments de $I$ \\
Notons $J$ le segment joignant $f(a)$ et $f(b)$ (donc $\left[ f(a), f(b) \right]$ ou $\left[ f(b), f(a) \right]$ suivant le cas) \\
Supposons $f$ continue. \\
Alors $\forall y \in J$, $\exists x \in \left[ a, b \right]: f(x) = y$
\end{theorem}
\begin{theorem}[de bijection monotone, version segments]
Soit $a \leq b$ deux éléments de $I$
\begin{itemize}
\item On suppose $f$ dérivable et $\forall x \in \left] a, b \right[$, $f'(x) > 0$ \\
Alors $f$ induit une bijection strictement croissante $[a, b] \to \left[f(a), f(b)\right]$
\item On suppose $f$ dérivable et $\forall x \in \left] a, b \right[$, $f'(x) < 0$ \\
Alors $f$ induit une bijection strictement décroissante $\left[ a, b \right] \to \left[ f(a), f(b) \right]$
\end{itemize}
\end{theorem}
\begin{theorem}
Soit $a \leq b$ deux éléments de $I$. \\
On suppose $f: I \to \mathbb{R}$ continue et strictement monotone sur $[a, b]$ \\
Alors $f$ induit une bijection $[a, b] \to \begin{cases}
\left[ f(a), f(b) \right] \\
\left[ f(b), f(a) \right]
\end{cases}$ suivant les cas.
\end{theorem}
\begin{theorem}[de la bijection monotone, version intervalles ouverts]
\hfill \\
Soit $a < b$ deux réels et $f: \left] a, b \right[ \to \mathbb{R}$ une fonction dérivable, de dérivée $> 0$ \\
Alors $f$ admet des limites en $a$ et $b$ et induit une bijection strictement croissante $\left] a, b \right[ \to \left] \lim\limits_a f, \lim\limits_b f \right[$
\end{theorem}

\subsection{Fonctions réciproques}
Ici, $I$ et $J$ sont deux intervalles de $\mathbb{R}$ et $f: I \to J$ est une bijection. \medskip

Graphiquement, $\gr(f)$ et $\gr(f^{-1})$ sont symétriques par rapport à la droite d'équation $y = x$
\begin{theorem}[Continuité de la réciproque]
\hfill \\
Si $f: I \to J$ est bijective et continue, alors $f^{-1}: J \to I$ est continue.
\end{theorem}
\begin{theorem}[Critère de dérivabilité des réciproques]
\hfill \\
Supposons $f: I \to J$ bijective et dérivable. Soit $x_0 \in I$ et $y_0 =f(x_0) \in J$ \\
Alors $f^{-1}$ est dérivable en $y_0$ si et seulement si $f'(x_0) \neq 0$ \\
Si c'est la cas,
\[ \left(f^{-1}\right)'(y_0) = \frac{1}{f'(x_0)	} = \frac{1}{f'(f^{-1}(y_0))}\]
\end{theorem}

\pagebreak

\subsection{Étude d'une fonction: le plan}
\noindent Étant donné une fonction, on peut l'étudier en six (ou sept) étapes. \medskip

\noindent (0. Si sule une expression est donnée, on détermine un domaine de définition. \\
On a alors une fonction $f: D \to \mathbb{R}$) \\
1. On examine les propriétés de symétrie de la fonction. Si on peut, on l'étudie sur un domaine plus petit. \\
2. On examine la régularité de la fonction: continuité, dérivabilité? \\
3. Là où c'est possible, on calcule la dérivée (en cherchant la forme la plus "multiplicative" possible) \\
4. Tableau de variations. \\
5. Limites. \\
6. Esquisse de graphe.
\begin{proposition}
On a $\forall x \in \left] -1, +\infty \right[$, $\ln( 1 + x ) \leq x$
\end{proposition}

\section{Fonctions usuelles}
\subsection{Exponentielle}
On a vu que $\forall z \in \mathbb{C}$, $\exp(\bar{z}) = \overline{\exp(z)}$. En particulier, $\forall x \in \mathbb{R}$, $\exp(x) \in \mathbb{R}$
\begin{definition}
On appelle \uline{exponentielle (réelle)} la fonction $\exp: \mathbb{R} \to \mathbb{R}$ induite par l'exponentielle complexe.
\end{definition}
\begin{proposition}
L'exponentielle $\mathbb{R} \to \mathbb{R}$
\begin{itemize}
\item Est strictement positive: $\forall x \in \mathbb{R}$, $\exp(x) > 0$
\item Est dérivable et $\exp' = \exp$
\item Admet des limites $\begin{cases}
\exp(x) \xrightarrow[x \to -\infty]{} 0 \\
\exp(x) \xrightarrow[x \to +\infty]{} +\infty
\end{cases}$
\item Vérifie la propriété fondamentale: $\forall x, y \in \mathbb{R}: \exp(x + y) = \exp(x) \exp(y)$
\end{itemize}
\end{proposition}
\begin{lemme}
On a $\forall x \in \mathbb{R}$, $\exp(x) \geq x + 1$
\end{lemme}
\begin{corollaire}
$\exp$ induit une bijection $\mathbb{R} \to \mathbb{R}_+^*$
\end{corollaire}

\subsection{Logarithme}
\begin{definition}
On note $\ln: \mathbb{R}_+^* \to \mathbb{R}$ la réciproque de $\exp$
\end{definition}
\begin{proposition}
\hfill
\begin{itemize}
\item $\ln$ est une bijection strictement croissante $\mathbb{R}_+^* \to \mathbb{R}$
\item $\ln$ est dérivable et $\forall y \in \mathbb{R}_+^*$, $\ln'(y) = \frac{1}{y}$
\item On a $\ln(y) \xrightarrow[y \to 0]{} -\infty$ et $\ln(y) \xrightarrow[y \to +\infty]{} +\infty$
\item On a $\forall y_1, y_2 \in \mathbb{R}_+^*$, $\ln(y_1 y_2) = \ln(y_1) + \ln(y_2)$
\end{itemize}
\end{proposition}
\begin{proposition}
$\forall x \in \left] -1, +\infty \right[$, $\ln(1 + x) \leq x$
\end{proposition}

\pagebreak

\subsection{Puissances}
\begin{definition}
Soit $r \in \mathbb{R}_+^*$ et $a \in \mathbb{R}$ \\
On définit $r^a$ = $\exp(a \ln(r))$
\end{definition}
\begin{proposition}
Soit $r, s \in \mathbb{R}_+^*$ et $a, b \in \mathbb{R}$ \\
On a:
\begin{itemize}
\item $r^{a + b} = r^a r^b$
\item $\left(r^a\right)^b = r^{a^b}$
\item $\left(rs\right)^q = r^q s^q$
\end{itemize}
Soit $r \in \mathbb{R}_+^*$: L'exponentielle de base $r$: $x \mapsto r^x$ \\
Elle est:
\begin{itemize}
\item Strictement décroissante si $r < 1$
\item Constante (égale à 1) si $r = 1$
\item Strictement croissante si $r > 1$
\end{itemize}
La fonction "puissance $a$-ième": $x \mapsto x^a$ est:
\begin{itemize}
\item Strictement décroissante si $a < 0$
\item Constante (égale à 1) si $a = 0$
\item Strictement croissante si $a > 0$
\end{itemize}
\end{proposition}
\begin{definition}
Soit $r \in \left] 0, 1 \right[ \cup \left] 1, +\infty \right[$ \\
On définit le logarithme en base $r$: $\log_r : \mathbb{R}_+^* \to \mathbb{R}$ comme la réciproque de $\begin{cases}
\mathbb{R} \to \mathbb{R}_+^* \\
x \mapsto r^x
\end{cases}$
\end{definition}
\begin{proposition}
Soit $r \in \left] 0, 1 \right[ \cup \left] 1, +\infty \right[$ et $x \in \mathbb{R}_+^*$ \\
On a
\[ \log_r(x) = \frac{\ln(x)}{\ln(r)} \]
\end{proposition}

\subsection{Croissances comparées}
\begin{theorem}
La fonction $x \mapsto x$ est \uline{négligeable devant $\exp$ au voisinage de $+\infty$}:
\[ \forall \varepsilon, A > 0,\quad \frac{x^A}{\exp(x)^\varepsilon} \xrightarrow[x \to +\infty]{} 0 \]
\[ \forall \varepsilon, A > 0,\quad \frac{\ln(x)^A}{x^\varepsilon} \xrightarrow[x \to +\infty]{} 0 \]
\[ \forall \varepsilon, A > 0,\quad \frac{\left| \ln(x) \right|^A}{\left(\frac{1}{x}\right)^\varepsilon} = x^\varepsilon \left| \ln(x) \right|^A \xrightarrow[x \to 0]{} 0\]
\end{theorem}

\subsection{Trigonométrie hyperbolique}
\begin{definition}
On définit les fonctions \uline{(co)sinus hyperbolique}:
\begin{align*}
&\cosh: \begin{cases}
\mathbb{R} \to \mathbb{R} \\
x \mapsto \frac{e^x + e^{-x}}{2} 
\end{cases} &\sinh: \begin{cases}
\mathbb{R} \to \mathbb{R} \\
x \mapsto \frac{e^x - e^{-x}}{2}
\end{cases}
\end{align*}
\end{definition}

\pagebreak

\begin{proposition}
\hfill
\begin{itemize}
\item $\cosh$ est paire, dérivable, de dérivée
\[\cosh' = \sinh\]
 et possède les limites $\cosh(x) \xrightarrow[x \to \pm\infty]{} +\infty$
\item $\sinh$ est impaire, dérivable, de dérivée
\[\sinh' = \cosh\]
et $\sinh(x) \xrightarrow[x \to -\infty]{} -\infty$ et $\sinh(x) \xrightarrow[x \to +\infty]{} +\infty$
\item On a $\cosh^2 - \sinh^2 = 1$
\end{itemize}
\end{proposition}
\begin{definition}
On définit la fonction tangente hyperbolique:
\[ \tanh: \begin{cases}
\mathbb{R} \to \mathbb{R} \\
x \mapsto \frac{\sinh(x)}{\cosh(x)}
\end{cases} \]
\end{definition}
\begin{proposition}
La fonction $\tanh$ est impaire, dérivable, de dérivée
\[\tanh' = 1 - \tanh^2 = \frac{1}{\cosh^2}\]
et vérifie $\tanh(x) \xrightarrow[x \to \pm\infty]{} \pm1$
\end{proposition}

\subsection{Fonctions trigonométriques réciproques}
\begin{definition}
On appelle \uline{arc cosinus} $\arccos: [-1, 1] \to [0, \pi]$ la réciproque de la bijection induite $\cos_{|[0, \pi]}^{|[-1, 1]}$ \\
On appelle \uline{arc sinus} $\arcsin : [-1, 1] \to [-\frac{\pi}{2}, \frac{\pi}{2}]$ la réciproque de la bijection induite $\sin_{|[-\frac{\pi}{2}]}^{|[-1, 1]}$
\end{definition}
\begin{proposition}
On a $\forall y \in [-1, 1]$, $\cos(\arcsin(y)) = \sin(\arccos(y)) = \sqrt{1 - y^2}$
\end{proposition}
\begin{proposition}
$\arccos$ et $\arcsin$ sont non dérivables en $-1$ et $1$, mais dérivables en tout $y \in \left] -1, 1 \right[$ et \\
$\forall y \in \left] -1, 1 \right[$:
\begin{align*}
&\arcsin'(y) = \frac{1}{\sqrt{1 - y^2}} &\arccos'(y) = \frac{-1}{\sqrt{1 - y^2}}
\end{align*}
\end{proposition}

\subsection{Tangente}
\begin{definition}
On note $D_\text{tan} = \left\{ x \in \mathbb{R} \mid x \not\equiv \frac{\pi}{2} (\text{mod } \pi) \right\} = \cos^{-1}[\mathbb{R}^*]$ \\
On définit la fonction \uline{tangente}
\[\tan: \begin{cases}
D_\text{tan} \to \mathbb{R} \\
x \mapsto \frac{\sin(x)}{\cos(x)}
\end{cases}\]
\end{definition}
\begin{proposition}
$\tan$ est impaire, $\pi$-périodique, dérivable de dérivée
\[\tan' = 1 + \tan^2 = \frac{1}{\cos^2}\]
et on a $\tan(x) \xrightarrow[\substack{x \to \frac{\pi}{2} \\ x < \frac{\pi}{2}}]{} + \infty$ et $\tan(x) \xrightarrow[\substack{x \to \frac{\pi}{2} \\ x > \frac{\pi}{2}}]{} -\infty$
\end{proposition}

\pagebreak

\begin{proposition}
Soit $x, y \in D_\text{tan}$
\begin{itemize}
\item Si $x + y \in D_\text{tan}$, on a
\[\tan(x + y) = \frac{\tan(x) + \tan(y)}{1 - \tan(x) \tan(y)}\]
\item Si $x - y \in D_\text{tan}$, on a
\[tan(x - y) = \frac{\tan(x) - \tan(y)}{1 + \tan(x) \tan(y)}\]
\end{itemize}
\end{proposition}
\begin{proposition}
Soit $x \in \mathbb{R}$ tel que $x \not\equiv \pi (\text{mod } 2\pi)$ \\
On peut exprimer $\cos(x)$ et $\sin(x)$ en fonction de $t = \tan(\frac{x}{2})$
\begin{align*}
&\cos(x) = \frac{1 - t^2}{1 + t^2} &\sin(x) = \frac{2t}{1 + t^2}
\end{align*}
\end{proposition}

\subsection{Arc tangente}
Par le théorème de la bijection monotone, $\tan$ induit une bijection $\left] -\frac{\pi}{2}, \frac{\pi}{2} \right[ \to \mathbb{R}$
\begin{definition}
On appelle arc tangente la fonction $\arctan: \mathbb{R} \to \left] -\frac{\pi}{2}, \frac{\pi}{2} \right[$, \\
réciproque de la bijection induite $\tan_{|\left] - \frac{\pi}{2}, \frac{\pi}{2} \right[}$
\end{definition}
\begin{proposition}
$\arctan$ est une fonction impaire, dérivable, de dérivée
\[\arctan': y \mapsto \frac{1}{1 + y^2}\]
et telle que $\arctan(x) \xrightarrow[x \to \pm\infty]{} \pm \frac{\pi}{2}$
\end{proposition}
\begin{proposition}
Soit $z = a + ib \in \mathbb{C}^*$ que l'on écrit $z = re^{i \theta}$, où $r > 0$, $\theta \in \mathbb{R}$ \\
On a $\begin{cases}
a = \re(z) = r \cos(\theta) \\
b = \im(z) = r \sin(\theta)
\end{cases}$ \\
Supposons $a \neq 0$ (càd $z \not\in i \mathbb{R}$) \\
On a alors
\[\frac{b}{a} = \frac{r \cos(\theta)}{r \sin(\theta)} = \tan(\theta)\]
Donc $\theta$ est un antécédent de $\frac{b}{a}$ par $\tan$
\[\theta \equiv \arctan \left(\frac{b}{a}\right) (\text{mod } \pi)\]
\end{proposition}
\begin{proposition}
On a $\forall x \in \mathbb{R}^*$
\[\arctan(x) + \arctan \left( \frac{1}{x} \right) = \sgn(x) \frac{\pi}{2} \]
\end{proposition}

\section{Brève extension aux fonctions à valeurs complexes}
Soit $I \subseteq \mathbb{R}$ un intervalle.
\begin{definition}
Soit $f: I \to \mathbb{C}$ \\
On dit que $f$ est dérivable si les fonctions $\begin{cases}
\re(f): I \to \mathbb{R} \\
\im(f): I \to \mathbb{R}
\end{cases}$ sont dérivables. \\
Si c'est la cas, on définit la dérivée de $f$
\[f' = \re(f)' + i \im(f)'\]
\end{definition}

\pagebreak

\begin{proposition}
Soit $f, g: I \to \mathbb{C}$ dérivables et $\lambda \in \mathbb{C}$ \\
Alors:
\begin{itemize}
\item $\lambda f$ est dérivable et $(\lambda f)' = \lambda f'$
\item $f + g$ est dérivable et $(f + g)' = f' + g'$
\item $fg$ est dérivable et $(fg)' = f'g + fg'$
\item Si $g$ ne s'annule pas, $\frac{f}{g}$ est dérivable et $\left(\frac{f}{g}\right)' = \frac{f'g - fg'}{g^2}$
\item $\exp \circ f$ est dérivable et $(\exp \circ f)' = f' \cdot (\exp \circ f)$
\end{itemize}
\end{proposition}

\section{Dérivée d'ordre supérieur}
Ici, $I$ est un intervalle de $\mathbb{R}$ non trivial.
\subsection{Définition}
\begin{definition}
Soit $f: I \to \mathbb{R}$
\begin{itemize}
\item On dit que $f$ est \uline{deux fois dérivable} si $f$ est dérivable et que $f'$ est dérivable. \\
On note alors $f'' = (f')'$
\item Par récurrence, pour tout $n \geq 2$, \uline{$f$ est $n$ fois dérivable} si $f$ est $(n - 1)$ fois dérivable et que $f^{(n -1)}$ est dérivable. \\
On note alors $f^{(n)} = \left( f^{(n - 1)} \right)'$ la dérivée $n$-ième.
\item On note $D^n(I) = D^n(I; \mathbb{R})$ l'ensemble des fonctions $n$ fois dérivables.
\item On dit que $f$ est \uline{lisse} ou \uline{de classe $C^\infty$} si elle est $n$ fois dérivable pour tout $n \in \mathbb{N}$ \\
On note $C^\infty(I) = C^\infty(I; \mathbb{R})$ l'ensemble des fonctions lisses.
\end{itemize} \medskip

\noindent \uline{Convention}: Toute fonction est "$0$ fois dérivable" et $f^{(0)} = f$
\end{definition}

\subsection{Propriétés de stabilité}
\begin{proposition}
Soit $f, g: I \to \mathbb{R}$ $n$ fois dérivable et $\lambda \in \mathbb{R}$ \\
Alors $\lambda f$ est $n$ fois dérivable et $(\lambda f)^{(n)} = \lambda f^{(n)}$ \\
Et $f + g$ est $n$ fois dérivable et $(f + g)^{(n)} = f^{(n)} + g^{(n)}$
\end{proposition}
\begin{corollaire}
Toute combinaison linéaire de fonctions lisses est lisse.
\end{corollaire}
\begin{theorem}[Formule de Leibniz]
Soit $f, g: I \to \mathbb{R}$ $n$ fois dérivables. \\
Alors $fg$ est $n$ fois dérivable et
\[ (fg)^{(n)} = \sum_{k = 0}^n \binom{n}{k} f^{(k)} g^{(n - k)} \]
\end{theorem}
\begin{corollaire}
Le produit de deux fonctions lisses est lisse.
\end{corollaire}
\begin{theorem}
La composée de deux fonctions $n$ fois dérivables est $n$ fois dérivables.
\begin{corollaire}
La composée de deux fonctions lisses est lisse.
\end{corollaire}
\end{theorem}
\end{document}