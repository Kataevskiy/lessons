\documentclass[10pt,a4paper]{article}
\usepackage[utf8]{inputenc}
\usepackage[french]{babel}
\usepackage[T1]{fontenc}
\usepackage{amsmath}
\usepackage{amsfonts}
\usepackage{amssymb}
\usepackage{graphicx}
\usepackage[left=2cm,right=2cm,top=2cm,bottom=2cm]{geometry}
\usepackage{setspace}
\usepackage{ulem}
\usepackage{stmaryrd}
\usepackage{amsthm}
\usepackage{dsfont}
\usepackage{mathpazo}

\onehalfspacing

\theoremstyle{definition}
\newtheorem{proposition}{Proposition}[section]
\newtheorem{theorem}[proposition]{Théorème}
\newtheorem{corollaire}[proposition]{Corollaire}
\newtheorem{lemme}[proposition]{Lemme}
\newtheorem{definition}[proposition]{Définition}

\begin{document}
\renewcommand{\labelitemi}{$*$}
\begin{center}
{\Large \textbf{Chapitre 20: Fractions rationnelles}}
\end{center}
Dans tout ce chapitre, on fixe un corps $K$ des scalaires.

\section{Généralités}
\subsection{Corps K(X)}
\begin{definition}
\uline{Le corps $K(X)$ des fractions rationnelles} est le corps des fractions de l'anneau intègre $K[X]$, $K(X) = Frac \; K[X]$
\end{definition}
\begin{definition}
Soit $P, Q \in K[X]$, où $Q \neq 0$.
\begin{itemize}
\item Si $P \perp Q$, on dit que la fraction rationnelle $\frac{P}{Q}$ est sous \uline{forme irréductible}.
\item Si $P \perp Q$ et que $Q$ est unitaire, la fraction rationnelle est dite sous \uline{forme irréductible unitaire}.
\end{itemize}
\end{definition}
\begin{proposition}
Si $F = \frac{P}{Q}$ est sous forme irréductible, alors les autres formes de la fraction sont les $\frac{PD}{QD}$, \\
où $D \in K[X]$
\end{proposition}
\begin{definition}
Soit $F = \frac{P}{Q} \in K(X)$. On définit son \uline{degré} $\deg{F} = \deg{P} - \deg{Q} \in \mathbb{Z} \cup \{-\infty\}$
\end{definition}
\begin{proposition}
Soit $F_1, F_2 \in K(X)$
\begin{itemize}
\item On a $\deg{(F_1 + F_2)} \leq \max{(\deg{F_1},\; \deg{F_2})}$
\item On a $\deg{(F_1 F_2)} = \deg{F_1} + \deg{F_2}$
\end{itemize}
\end{proposition}

\subsection{Racines et pôles}
\begin{definition}
Soit $F = \frac{P}{Q} \in K(X)$ \uline{sous forme irréductible}.
\begin{itemize}
\item Une \uline{racine} de $F$ est une racine de P.
\item Une \uline{pôle} de $F$ est une racine de Q.
\end{itemize}
\end{definition}
\begin{definition}
Soit $F = \frac{P}{Q} \in K$ sous forme irréductible dont les pôles forment l'ensemble Pôles(F). \\
Alors la \uline{fonction rationnelle associée} à $F$ est: $\begin{cases}
K \setminus \text{Pôles(F)} \to K \\
z \to \frac{P(z)}{Q(z)}
\end{cases}$
\end{definition}
\begin{proposition}[Ctirère radical de nullité]
Une fraction rationnelle possédant une infinité de racines est nulle.
\end{proposition}
\begin{corollaire}
Si les fonctions rationnelles associées à $F_1$ et $F_2 \in K(X)$ coïncident sur un ensemble infini, \\
alors $F_1 = F_2$
\end{corollaire}

\subsection{Autres opérations}
On peut définir la dérivée d'une fonction rationnelle $F = \frac{P}{Q} \in K(X)$ en posant: $F' = \frac{P'Q - PQ'}{Q^2}$
\begin{proposition}
Si $G = \frac{P}{Q}$ et $F \in K(X)$ sont deux fractions rationnelles, on peut "évaluer" $P$ et $Q$ en $F$. Par exemple, si $Q = \sum\limits_{k = 0}^{n} q_k X^k$, on a $Q(F) = \sum\limits_{k = 0}^{n} q_k F^k \in K(X)$. Dès que $Q(F) \neq 0$, on peut considérer la composition
\[G \circ F = G(F) = \frac{P(F)}{Q(F)} \in K(X)\]
\end{proposition}

\subsection{Partie entière}
\begin{proposition}
Soit $F \in K(X)$. \\
Il existe un unique couple $(E, F_0) \in K[X] \times K(X)$ tel que: $\begin{cases}
F = E + F_0 \\
\deg{F_0} < 0
\end{cases}$ \\
On dit que le polynôme $E$ est la \uline{partie entière} de $F$.
\end{proposition}

\section{Décomposition en éléments simples}
\subsection{Le résultat}
\begin{theorem}
Soit $F = \frac{P}{Q} \in K(X)$ sous forme irréductible unitaire. Soit $Q = Q_1^{\alpha_1} ... Q_r^{\alpha_r}$ la décomposition en facteurs irréductibles de $Q$. \\
Il existe alors un unique $E \in K[X]$ et une unique famille $(R_{i, j})_{\substack{1 \leq i \leq r \\ 1 \leq j \leq \alpha_i}}$ de polynômes telles que \\
$\forall i \in \llbracket 1, \alpha_i \rrbracket$, $\deg{R_{ij}} < \deg{Q_i}$ et
\[F = E + \sum\limits_{i = 1}^r \sum\limits_{j = 1}^{\alpha_i} \frac{R_{ij}}{Q_i^j}\]
Cette écriture est la \uline{décomposition en éléments simples} de $F$.
\end{theorem}
\begin{corollaire}[DES sur $\mathbb{C}$]
Soit $F = \frac{P}{Q} \in \mathbb{C}(X)$ sous forme irréductible unitaire et
\[Q = \prod\limits_{i = 1}^r (X - z_i)^{\alpha_i}\]
la DFI de $Q$. \\
Alors il existe un unique $E \in K[X]$ et une unique famille de complexes $(\lambda_{i,j})_{\substack{1 \leq i \leq r \\ 1 \leq j \leq \alpha_i}}$ telle que
\[F = E + \sum\limits_{i = 1}^r \sum\limits_{j = 1}^{\alpha_i} \frac{\lambda_{i,j}}{(X - z_i)^j}\]
\end{corollaire}
\begin{corollaire}[DES sur $\mathbb{R}$]
Soit $F = \frac{P}{Q} \in \mathbb{R}(X)$ sous forme irréductible et $Q = \prod\limits_{i = 1}^r (X - t_i)^{\alpha_i} \prod\limits_{k = 1}^s Q_k^{\beta_k}$ la DFI de $Q$, où $t_1, ... ,t_r$ sont les racines réelles de $Q$ et $Q_1, ... ,Q_s$ sont ses facteurs irréductibles de degré 2. \\
Alors il existe:
\begin{itemize}
\item Un unique $E \in K[X]$
\item Une unique famille de réels $(\lambda_{i,j})_{\substack{1 \leq i \leq r \\ 1 \leq j \leq \alpha_i}}$
\item d'uniques familles $(\mu_{k,j})_{\substack{1 \leq k \leq s \\ 1 \leq j \leq \beta_k}}$ et $(\nu_{k,j})_{\substack{1 \leq k \leq s \\ 1 \leq j \leq \beta_k}}$
\end{itemize}
telles que
\[F = E + \sum\limits_{i = 1}^r \sum\limits_{j = 1}^{\alpha_i} \frac{\lambda_{i,j}}{(X - z_i)^j} + \sum\limits_{k = 1}^s \sum\limits_{j = 1}^{\beta_k} \frac{\mu_{k,j} X + \nu_{k, j}}{Q_k^j}\]
\end{corollaire}

\subsection{Techniques de calcul}
\begin{proposition}
Soit $F = \frac{P}{Q}$ sous forme irréductible et $z \in K$ un pôle simple de $F$, càd une racine simple de $Q$. On écrit $Q = (X - z) Q_0$ \\
La partie polaire associée au pôle $z$ est $\frac{\lambda}{X - z}$, où $\lambda = \frac{P(z)}{Q(z)} = \frac{P(z)}{Q'(z)}$
\end{proposition}
\begin{proposition}
Si $F = \frac{P}{Q}$ (forme irréductible unitaire) et que $z$ est un pôle d'ordre $\alpha$, on écrit $Q = (X - z)^\alpha Q_0$ \\
La partie polaire associée au pôle $z$ est $\sum\limits_{j = 1}^\alpha \frac{\lambda_j}{(X - z)^j}$, où
\[\lambda_\alpha = \frac{P(z)}{Q_0(z)} = \frac{\alpha! P(z)}{Q^{(\alpha)}(z)}\]
\end{proposition}

\subsection{Applications}
\begin{proposition}
Soit $P \in K[X]$ scindé: $P = u \prod\limits_{i = 1}^n (X - z_i)^{\alpha_i}$ \\
Alors
\[\frac{P'}{P} = \sum\limits_{i = 1}^n \frac{\alpha_i}{X - z_i}\]
\end{proposition}
\begin{corollaire}[Théorème de Gauss-Lucas]
Soit $P \in \mathbb{C}$ non constant. Alors les racines de $P'$ sont \uline{combinaisons convexes} de racines de P.
\end{corollaire}
\end{document}