\documentclass[10pt,a4paper]{article}
\usepackage[utf8]{inputenc}
\usepackage[french]{babel}
\usepackage[T1]{fontenc}
\usepackage{amsmath}
\usepackage{amsfonts}
\usepackage{amssymb}
\usepackage{graphicx}
\usepackage[left=2cm,right=2cm,top=2cm,bottom=2cm]{geometry}
\usepackage{setspace}
\usepackage{ulem}
\usepackage{stmaryrd}
\usepackage{amsthm}
\usepackage{dsfont}
\usepackage{mathpazo}

\onehalfspacing

\theoremstyle{definition}
\newtheorem{proposition}{Proposition}[section]
\newtheorem{theorem}[proposition]{Théorème}
\newtheorem{corollaire}[proposition]{Corollaire}
\newtheorem{lemme}[proposition]{Lemme}
\newtheorem{definition}[proposition]{Définition}

\DeclareMathOperator*{\Vect}{Vect}
\DeclareMathOperator{\im}{im}
\DeclareMathOperator{\Sp}{Sp}
\DeclareMathOperator*{\Mat}{Mat}
\DeclareMathOperator{\Tr}{Tr}

\begin{document}
\renewcommand{\labelitemi}{\textbullet}
\renewcommand{\labelenumi}{(\roman{enumi})}
\begin{center}
{\Large \textbf{Chapitre 12. Réduction des endomorphismes}}
\end{center}

\section{Sous-espaces stables. Polynômes d'endomorphisme}
\subsection{Exemples de sous-espaces stables}
\begin{definition}
Soit $u \in \mathcal{L}(E)$, $F$ un sev de $E$ \\
On dit que $F$ est stable sous $u$ su $u(F) \subset F$ \\
On note alors $u_F$ l'induit de $u$ sur $F$
\end{definition}
\begin{proposition}
Si $P \in K[X]$, $P(u)$ laisse stable $F$ et $\boxed{P(u)_F = P(u_F)}$
\end{proposition}

\subsection{Exemples de sous-espaces stables}
\noindent \begin{itemize}
\item Premier type: Soit $E$ un $K$-ev, $u \in \mathcal{L}(E)$, $e \in E$ \\
Alors $F_e = \Vect\limits_{k \in \mathbb{N}}(u^k(e))$ est un sev stable par $u$, c'est même le plus petit sev stable contenant $e$
\item Deuxième type: $\ker P(u)$ et $\im P(u)$
\end{itemize}
\begin{proposition}
Soit $u, v \in \mathcal{L}(E)$ aveec $u \circ v = v \circ u$ \\
Alors $\ker v$ et $\im v$ sont stables par $u$
\end{proposition}
\begin{corollaire}
Soit $E$ un $K$-ev, $u \in \mathcal{L}(E)$ et $P \in K[X]$ \\
Alors $\ker P(u)$ et $\im P(u)$ sont stables par $u$
\end{corollaire}

\subsection{Théorème de décomposition des noyaux}
\begin{theorem}[ Théorème de décomposition des noyaux ]
\hfill \\
Soit $E$ un $K$-ev, $u \in \mathcal{L}(E)$ et $P, Q \in K[X]$ \uline{Premiers entre eux}. \\
Alors
\[ \boxed{\ker PQ(u) = \ker P(u) \oplus \ker Q(u)} \]
\end{theorem}
\begin{corollaire}
Soit $E$ un $K$-ev, $u \in \mathcal{L}(E)$ et $P_1, ...,\, P_r \in K[X]$ premiers entre eux $2$ à $2$ \\
Alors 
\[ \boxed{ \ker P_1 P_2 ... P_r(u) = \bigoplus_{i = 1}^r \ker P_i(u)} \]
\end{corollaire}

\subsection{Polynôme minimal d'un endomorphisme}
\begin{theorem}
Soit $E$ est de dimension finie et $\Phi: \begin{cases}
K[X] \to \mathcal{L}(E) \\
P \mapsto P(u) \end{cases}$ un morphisme d'algèbres. \\
Alors $\ker \Phi \neq \{ 0 \}$ et il existe un unique polynôme unitaire $\mu_u$ ( ou $\pi_u$ ) tel que $\ker \Phi = \mu_u K[X] $ \\
Si $P \in K[X]$ alors $P(u) = 0 \iff \mu_n \mid P$ \\
$\mu_u$ est donc le polynôme unitaire de plus petit degré ( non nul ) qui annule $u$ \\
Par ailleurs $\im \Phi = K[u] = \Vect\limits_{k \in \mathbb{N}}(u^k)$ est une sous-algèbre de $\mathcal{L}(E)$ ( commutative ) \\ de dimension $\deg \mu_u = d$ et  de base $( \text{Id}, u, ...,\, u^{d - 1} )$
\end{theorem}
\begin{definition}
Avec ces notations, $\mu_u$ s'appelle polynôme minimal de $u$
\end{definition}
\begin{proposition}
Si $E$ de dimension finie \begin{itemize}
\item $\boxed{\mu_u = 1 \iff E = \{ 0 \}}$
\item $\boxed{\mu_u = X - \lambda \iff u = \lambda \text{Id}_E,\, E \neq \{ 0 \}}$
\end{itemize}
\end{proposition}
\begin{theorem}
Soit $A \in M_n(K)$ \\
Alors $\Phi : \begin{cases}
K[X] \to M_n(K) \\
P \mapsto P(A)
\end{cases}$ est un morphisme d'algèbres non injectif. \\
Donc $\ker \Phi$ est un idéal différent de $\{ 0 \}$ qui s'écrit $\mu_A K[X]$ \\
Si $P \in K[X]$, $P(A) = 0 \iff \mu_A = P$ \\
et $\mu_A$ est donc le polynôme unitaire différent de $0$ de plus petit degré annulant $A$ \\
Par ailleurs, si $d = \deg \mu_A$, $K[A]$ est une sous-algèbre commutative de $M_n(K)$ de dimension $d$, de base $( \text{Id}, A, ...,\, A^{d - 1})$
\end{theorem}
\begin{definition}
$\mu_A$ est appelé polynôme minimal de $A$ ( aussi noté $\mu_A$ )
\end{definition}

\subsection{Racines de polynôme minimal}
\begin{proposition}
Soit $E$ un $K$-ev, $u \in \mathcal{L}(E)$, $Q \in K[X]$ \\
Si $(e, \lambda)$ un couple propre de $u$ alors 
\[ \boxed{Q(u)(e) = Q(\lambda)e} \]
\end{proposition}
\begin{proposition}
\hfill \begin{itemize}
\item Soit $E$ un $K$-ev de dimension finie, $u \in \mathcal{L}(E)$, $P$ un polynôme annulateur de $u$, $\lambda \in \Sp(u)$ \\
Alors $\lambda$ est racine de $P$ : $\Sp_u \in Z(P)$
\item Soit $A \in M_n(K)$, $\lambda \in \Sp(A)$, $P \in K[X]$ avec $P(A) = 0$ \\
Alors $\lambda$ est racine de $P$
\end{itemize}
\end{proposition}
\begin{proposition}
\hfill \begin{itemize}
\item Soit $E$ un $K$-ev de dim finie, $u \in \mathcal{L}(E)$ \\
Les racines de $\mu_u$ sont exactement les valeurs propres de $u$
\[ \boxed{\Sp u = Z(\mu_u)} \]
\item Soit $A \in M_n(K)$ \\
Les racines de $\mu_A$ sont exactement les valeurs propres de $A$
\[ \boxed{\Sp A = Z(\mu_A)} \]
\end{itemize}
\end{proposition}

\section{Diagonalisabilité}
\subsection{Endomorphismes diagonalisables}
\begin{definition}
Soit $E$ un $K$-ev de dim finie, $u \in \mathcal{L}(E)$ \\
On dit que $u$ est diagonalisable s'il existe une base $\mathcal{B}$ de $E$ telle que 
\[ \Mat_\mathcal{B}(u) = \begin{pmatrix}
\lambda_1 & & & (0) \\
& \lambda_2 & & \\
& & \ddots & \\
(0) & & & \lambda_n
\end{pmatrix} \in D_n(K) \]
Autrement dit, s'il existe une base de vecteurs propres.
\end{definition}

\pagebreak
\begin{theorem}
Soit $E$ un $K$-ev de dimension finie $n$, $u \in \mathcal{L}(E)$ \\
Les $5$ conditions suivantes sont équivalentes :
\begin{enumerate}
\item $u$ est diagonalisable.
\item Il existe $\lambda_1, ...,\, \lambda_r \in K$ $2$ à $2$ distincts tels que
\[ E = \bigoplus_{i = 1}^r \ker \left( u - \lambda_i \text{Id}_E \right) \]
\item Il existe $\lambda_1, ...,\, \lambda_r \in K$ $2$ à $2$ distincts tels que
\[ \prod_{i = 1}^r \left( u - \lambda_i \text{Id}_E \right) = 0 \]
\item Il existe $P \in K[X]$ scindé à racines simples annulant $u$
\item $\mu_u$ est scindé à racines simples.
\end{enumerate}
Dans ces conditions
\[ \boxed{E = \bigoplus_{\lambda \in \Sp u} \ker \left( u - \lambda \text{Id}_E \right)} \]
\[ \boxed{\mu_u = \prod_{\lambda \in \Sp u} (X - \lambda)} \]
( On dit que "la somme des sev propres rejoint $E$" )
\end{theorem}
\begin{proposition}
Soit $E$ un $K$-ev de dimension finie, $u \in \mathcal{L}(E)$ diagonalisable et $F$ un sev de $E$ stable par $u$ \\
Alors $u_F$ est aussi diagonalisable et $\boxed{\mu_{u_F} \mid \mu_u}$
\end{proposition}

\subsection{Matrices carrés diagonalisables}
\begin{definition}
Soit $A \in M_n(K)$ \\
$A$ est diagonalisable si $u_A : \begin{cases}
K^n \to K^n \\
X \mapsto AX
\end{cases}$ est diagonalisable.
\end{definition}
\begin{proposition}
Soit $A \in M_n(K)$ \\
Alors $A$ diagonalisable $\iff$ $A$ est semblable à une matrice diagonalisable.
\end{proposition}
\begin{proposition}
Soit $A \in M_n(K)$ \\
Les $5$ conditions suivantes sont équivalents :
\begin{enumerate}
\item $A$ est diagonalisable.
\item Il existe $P \in GL_n(K)$ tel que $P^{-1}AP \in D_n(K)$
\item Il existe $\lambda_1, ...,\, \lambda_r \in K$ $2$ à $2$ distincts tels que
\[ K^n = \bigoplus_{i = 1}^r \ker \left( A - \lambda_i I_n \right) \]
\item Il existe $Q \in K[X]$ scindé à racines simples annulant $A$
\item $\mu_A$ est scindé à racines simples.
\end{enumerate}
Dans ces conditions
\[ \boxed{K^n = \bigoplus_{\lambda \in \Sp A} \ker \left( A - \lambda I_n \right)} \]
\[ \boxed{\mu_A = \prod_{\lambda \in \Sp A} (X - \lambda)} \]
\end{proposition}
\begin{proposition}
Soit $E$ un $K$-ev de dim finie, $u \in \mathcal{L}(E)$, $\mathcal{B}$ base de $E$ et $A = \Mat\limits_{\mathcal{B}}(u)$ \\
Alors $\boxed{u \text{ diagonalisable } \iff A \text{ diagonalisable }}$
\end{proposition}
\begin{proposition}
Soit 
\[ M = \begin{pmatrix}
\boxed{A_1} & & (0) \\
& \ddots &  \\
(0) & & \boxed{A_r}
\end{pmatrix} \]
Alors \[ \boxed{M \text{ diagonalisable } \iff A_1, ...,\, A_r \text{ diagonalisable }} \]
\end{proposition}

\subsection{Diagonalisabilité du polynôme caractéristique}
\begin{proposition}
\hfill \begin{itemize}
\item Soit $E$ un $K$-ev de dim finie, $u \in \mathcal{L}(E)$ \\
Si $\chi_u$ est scindé à racines simples, $u$ est diagonalisable.
\item Soit $A \in M_n(K)$ et $\chi_A$ scindé à racines simples \\
Alors $A$ est diagonalisable.
\end{itemize}
\end{proposition}
\begin{proposition}
Soit $E$ un $K$-ev de dimension finie, $u \in \mathcal{L}(E)$ et $\lambda$ une valeur propre de $u$ \\
On note $\alpha$ l'ordre de $\lambda$ comme racine de $\chi_u$ : multiplicité algébrique de $\lambda$ comme valeur propre de $u$ \\
On note $\beta$ la dimension de $E_\lambda = \ker \left( u - \lambda \text{ Id } \right)$ : multiplicité géométrique de $\lambda$ \\
Alors $\boxed{1 \leq \beta \leq \alpha}$
\end{proposition}
\begin{theorem}
Soit $u \in \mathcal{L}(E)$, $E$ $K$-ev de dim finie et $\Sp u = \{ \lambda_1, ...,\, \lambda_r \}$ avec $\lambda_i$ $2$ à $2$ distincts. \\
Pour $1 \leq i \leq r$ on note: \\
$\beta_i = \dim \ker \left( u - \lambda \text{ Id } \right)$ : multiplicité géométrique\\
$\alpha_i$ l'ordre de $\lambda_i$ comme racine de $\chi_u$ : multiplicité algébrique\\
Alors 
\[ \boxed{u \text{ diagonalisable } \iff \begin{cases}
\chi_u \text{ scindé } \\
\forall i \in \llbracket 1, r \rrbracket,\, \beta_i = \alpha_i
\end{cases}} \]
\end{theorem}
\begin{definition}
\hfill \\
Diagonaliser un endomorphisme c'est trouver une base de vecteurs propres et les valeurs propres associés. \\
diagonaliser une matrice $A$ c'est trouvé $P \in GL_n(K)$ et $D \in D_n(K)$ tels que $P^{-1}AP = D$
\end{definition}
\begin{proposition}
Si $C_1, ...,\, C_n$ sont une base de vecteurs propres pour $A$ et $A C_i = \lambda_i C_i$\\
Si 
\[ P = \left(C_1 \mid \cdots \mid C_n \right) = \Mat \left( \text{ b.c. },  (C_1 \mid \cdots \mid C_n) \right) \]
Alors
\[ P^{-1} A P = \Mat_{(C_1, ...,\, C_n)}(u_A) = \begin{pmatrix}
\lambda_1 & & (0) \\
& \ddots & \\
(0) & & \lambda_n
\end{pmatrix} = D \]
\end{proposition}

\pagebreak

\section{Exercices classiques (1\textsuperscript{ère} série)}
\renewcommand{\labelenumi}{\arabic{enumi}.}
\subsection{Diagonalisation simultanée}
\noindent Soit $E$ un $K$-ev de dimension finie.
\begin{enumerate}
\item Soit $A \in \mathcal{L}(E)$ d'éléments co-diagonalisables ie. qui admettent une base commune de diagonalisation. \\
Montrer que les éléments de $A$ commutent.
\item Soit $u, v \in \mathcal{L}(E)$ diagonalisables avec $u \circ v = v \circ u$ \\
Montrer que $u$ et $v$ sont co-diagonalisables.
\item Soit $u_1, ...,\, u_p \in \mathcal{L}(E)$ diagonalisables commutant $2$ à $2$ \\
Montrer que $u_1, ...,\, u_p$ sont co-diagonalisables.
\item Montrer que c'est le cas pour $A \in \mathcal{L}(E)$ formé d'éléments diagonalisables, commutant $2$ à $2$
\item Soit $A, B \in M_n(K)$ diagonalisables. Si $AB = BA$ montrer qu'il existe $P \in GL_n(K)$ tel que \\
$P^{-1}AP$ et $P^{-1}BP$ sont diagonales.
\end{enumerate}

\subsection{Semi-simplicité des endomorphismes diagonalisables}
\noindent Soit $E$ un $K$-ev de dim finie, $u \in \mathcal{L}(E)$ diagonalisable.
\begin{enumerate}
\item Montrer que tout système libre de vecteurs propres de $u$ se complète en une base de vecteurs propres.
\item Soit $F \in E$ un sev stable par $u$ \\
Montrer que $F$ possède un supplémentaire stable par $u$ ( semi-simplicité )
\item Décrire les sous-espaces stables de $E$ par $u$ \\
À quelle condition sont-ils en nombre fini? ( avec $K$ infini )
\end{enumerate}

\subsection{Commutant d'un endomorphisme diagonalisable}
\begin{enumerate}
\item Soit $E$ un $K$-ev de dim finie $n$, $u \in \mathcal{L}(E)$ diagonalisable, \\
$\chi_u = \prod\limits_{i = 1}^r (X - \lambda_i)^{m_i}$ avec $\lambda_i \in K$ $2$ à $2$ distincts, $m_i \geq 1$ \\
Montrer que $\dim \mathcal{C}(u) = \sum\limits_{i = 1}^r m_i^2$, décrire les éléments de $\mathcal{C}(u)$ ( commutant de $u$ )
\item Soit $A \in M_n(K)$ diagonalisable et $\chi_A = \prod\limits_{i = 1}^r (X - \lambda_i)^{m_i}$
\begin{enumerate}
\item Montrer que $\dim \mathcal{C}(A) = \sum\limits_{i = 1}^r m_i^2$
\item Montrer que $\mathcal{C}(A) = K[A] \iff r = n \iff \chi_A \text { scindé à racines simples }$
\end{enumerate}
\end{enumerate}

\section{Endomorphismes trigonalisables}
\subsection{Généralités}
\begin{definition}
\hfill \begin{itemize}
\item Soit $E$ un $K$-ev de dimension finie et $u \in \mathcal{L}(E)$ \\
On dit que $u$ est trigonalisable s'il existe $\mathcal{B}$ base de $E$ avec $\Mat\limits_{\mathcal{B}}(u)$ triangulaire supérieure.
\item Soit $A \in M_n(K)$ \\
On dit que $A$ est trigonalisalbe si $u_A$ l'est.
\end{itemize}
\end{definition}
\begin{proposition}
Soit $A \in M_n(K)$ \\
Alors $A \text{ trigonalisable } \iff A \text{ semblable à une matrice triangulaire }$
\end{proposition}
\begin{proposition}
Soit $E$ un $K$-ev de dim finie, $u \in \mathcal{L}(E)$, $\mathcal{B}$ une base de $E$ et $A = \Mat\limits_{\mathcal{B}}(u)$
Alors 
\[ \boxed{ u \text{ trigonalisable } \iff A \text{ trigonalisable }} \]
\end{proposition}
\begin{proposition}
\hfill \begin{itemize}
\item Soit $E$ un $K$-ev de dim finie, $u \in \mathcal{L}(E)$ trigonalisable, $\lambda_1, ...,\, \lambda_n$ les valeurs propres comptées avec  \\
multiplicités et $P \in K[X]$ \\
Alors $P(u)$ est trigonalisable et $P(\lambda_1), ...,\, P(\lambda_n)$ sont ses valeurs propres comptées avec multiplicité.
\item Soit $A \in M_n(K)$ trigonalisable, $\chi_A = \prod\limits_{i = 1}^r (X - \lambda_i)$ et $P \in K[X]$ \\
Alors $P(A)$ est trigonalisable et $P(\lambda_1), ...,\, P(\lambda_n)$ sont ses valeurs propres comptés avec multiplicité.
\end{itemize}
\end{proposition}

\subsection{Le théorème de Cayley-Hamilton}
\begin{theorem} [ Théorème de Cayley-Hamilton ]
\hfill \begin{itemize}
\item Soit $E$ un $K$-ev de dimension finie $n$ et $u \in \mathcal{L}(E)$ \\
Alors
\[ \boxed{\chi_u(u) = 0} \quad \text{ et } \quad \boxed{\mu_u \mid \chi_u} \]
En particulier $\boxed{\deg \mu_u \leq n}$
\item Soit $A \in M_n(K)$ \\
Alors 
\[ \boxed{\chi_A(A) = 0} \quad \text{ et } \quad \boxed{\mu_A \mid \chi_A} \]
Et $\boxed{\deg \mu_A \leq n}$
\end{itemize}
\end{theorem}

\subsection{Caractérisation des endomorphismes trigonalisables}
\renewcommand{\labelenumi}{(\roman{enumi})}
\begin{theorem}
Soit $A \in M_n(K)$ \\
Les $4$ conditions suivantes sont équivalentes:
\begin{enumerate}
\item $A$ est trigonalisable.
\item $\chi_A$ est scindé.
\item $\mu_A$ est scindé.
\item Il existe $Q \neq 0$ dans $K[X]$ scindé avec $Q(A) = 0$
\end{enumerate}
\end{theorem}
\begin{corollaire}
Soit $E$ un $K$-ev de dim finie et $u \in \mathcal{L}(E)$ \\
Les $4$ conditions suivantes sont équivalentes:
\begin{enumerate}
\item $u$ est trigonalisable.
\item $\chi_u$ est scindé.
\item $\mu_u$ est scindé.
\item $u$ admet un polynôme annulateur scindé.
\end{enumerate}
\end{corollaire}
\begin{corollaire}
\hfill \begin{itemize}
\item Toute matrice carrée complexe est trigonalisable.
\item Tout endomorphisme d'un $\mathbb{C}$-ev de dimension finie est trigonalisable.
\end{itemize}
\end{corollaire}

\pagebreak

\begin{proposition}
\hfill \begin{itemize}
\item Soit $E$ un $K$-ev de dim finie $n$ et $u \in \mathcal{L}(E)$ \\
Alors 
\[ u \text{ nilpotent } \iff \chi_u = X^n \iff u \text{ trigonalisable et } \Sp(u) = \{ 0 \} \]
Dans ces conditions il existe $\mathcal{B}$ base de $E$ telle que
\[ \Mat_{\mathcal{B}}(u) = \begin{pmatrix}
0 & & (*) \\
 & \ddots & \\
 (0) & & 0
\end{pmatrix} \]
et $\boxed{u^n = 0}$
\item Soit $A \in M_n(K)$ \\
Alors
\[ A \text{ nilpotente } \iff \chi_A = X^n \iff \begin{cases}
A \text{ trigonalisable } \\
\Sp(A) = \{ 0 \}
\end{cases} \]
Dans ces conditions $A$ est semblable à $\begin{pmatrix}
0 & & (*) \\
& \ddots & \\
(0) & & 0
\end{pmatrix}$ et $A^n = 0$
\end{itemize}
\end{proposition}

\section{Sous-espace caractéristique}
\subsection{Présentation}
\begin{definition}
Soit $E$ un $K$-ev de dim finie, $u \in \mathcal{L}(E)$, $\lambda$ un valeur propre de $u$ et $\alpha$ sa multiplicité algébrique \\ ( son ordre comme racine de $\chi_u$ ) \\
Le sous-espace caractéristique de $u$ associée à $\lambda$ est
\[ \boxed{F_\lambda(u) = \ker\left( u - \lambda \text{Id}_E \right)^\alpha} \]
\end{definition}
\begin{theorem}
Soit $E$ un $K$-ev de dim finie $u \in \mathcal{L}(E)$, $\lambda \in \Sp(u)$ de multiplicité algébrique $\alpha$ \\
et racine d'ordre $p$ de $\mu_u$ \\
Alors 
\[ F_\lambda(u) = \ker\left(u - \text{Id}\right)^\alpha = \ker\left(u - \text{Id}\right)^p = \ker\left(u - \text{Id}\right)^n \]
De plus $p$ est le rang à partir duquel les noyaux itérés de $u - \lambda \text{Id}$ stationnent.
\end{theorem}
\begin{theorem}
Soit $E$ un $K$-ev de dimension finie, $u \in \mathcal{L}(E)$ trigonalisable et \\
$\chi_u = \prod\limits_{i = 1}^r (X - \lambda_i)^{\alpha_i}$ avec les $\lambda_i \in K$ $2$ à $2$ différents et $\alpha_i \in \mathbb{N}^*$ \\
Alors
\[ \boxed{E = \bigoplus_{i = 1}^r F_{\lambda_i} = \bigoplus_{i = 1}^r \ker(u - \lambda_i \text{Id})^{\alpha_i}} \]
De plus $\boxed{\dim F_{\lambda_i}(u) = \alpha_i}$
\end{theorem}

\pagebreak

\subsection{Théorème de réduction par sous-espace caractéristique}
\begin{theorem}[ Théorème de réduction par sev caractéristique ]
Soit $E$ un $K$-ev de dimension finie $n$, $u \in \mathcal{L}(E)$ trigonalisable et
$\chi_u = \prod\limits_{i = 1}^r (X - \lambda_i)^{\alpha_i}$ avec les $\lambda_i \in K$ $2$ à $2$ distincts et $\alpha_i \in \mathbb{N}^*$ \\
Alors il existe une base $\mathcal{B}$ de $E$ telle que
\[ \Mat\limits_{\mathcal{B}} = \begin{pmatrix}
\boxed{T_1} & & & (0) \\
& \boxed{T_2} & & \\
& & \ddots & \\
(0) & & & \boxed{T_r}
\end{pmatrix} \]
avec
\[ T_i = \begin{pmatrix}
\lambda_i & & (*) \\
& \ddots & \\
(0) & & \lambda_i
\end{pmatrix} \] de taille $\alpha_i$ pour tout $i \in \llbracket 1, r \rrbracket$
\end{theorem}

\subsection{Réduction des matrices de taille $2$}
\noindent Soit $A \in M_2(K)$ trigonalisable donc $\chi_A = (X - \lambda)(X - \mu)$ avec $\lambda, \mu \in K$
\begin{itemize}
\item Si $\lambda \neq \mu$ alors $\chi_A$ est scindé à racines simples. \\
$A$ est diagonalisable et $A$ semblable à $\begin{pmatrix}
\lambda & 0 \\
0 & \mu
\end{pmatrix}$
\item Si $\lambda = \mu$ alors $\chi_A = (X - \lambda)^2$ et \\
$A \text{ est diagonalisable } \iff A = \lambda I_2 $ \\
Si $A$ n'est pas diagonalisable alors $A$ est trigonalisable et $A$ est semblable $\begin{pmatrix}
\lambda & 1 \\
0 & \lambda
\end{pmatrix}$
\end{itemize}
Si $K = \mathbb{R}$ et $\chi_A = (X - \rho e^{i \theta})(X - \rho e^{-i \theta})$ \\
Alors $A$ est semblable sur $\mathbb{R}$ à $\rho R_\theta = \rho \begin{pmatrix}
\cos \theta & - \sin \theta \\
\sin \theta & \cos \theta
\end{pmatrix}$

\section{Exercices classiques}
\renewcommand{\labelenumi}{\arabic{enumi}.}
\subsection{Trigonalisation simultanée}
\noindent Soit $E$ un $\mathbb{C}$-ev de dim finie $n$ et $u, v \in \mathcal{L}(E)$ avec $u \circ v = v \circ u$
\begin{enumerate}
\item Montrer que $u$ et $v$ possèdent un vecteur propre commun.
\item Montrer que $u$ et $v$ sont cotrigonalisables ( il existe une base commune dans laquelle les matrices de \\
$u$ et $v$ sont triangulaires supérieures.
\item Soit $A, B \in M_n(\mathbb{C})$ \\
Montrer que $AB = BA \implies \exists P \in GL_n(\mathbb{C})$, $P^{-1}AP \in T_n(\mathbb{C})$ et $P^{-1}BP \in T_n(\mathbb{C})$ \\
La réciproque est-elle vraie?
\end{enumerate}

\subsection{Caractérisation de matrices nilpotents avec la trace}
\noindent Soit $A \in M_n(\mathbb{C})$. On suppose que $\Tr A = \Tr A^2 = ... = \Tr A^n = 0$ \\
Montrer que $A$ est nilpotente.

\subsection{Sous-espaces stables}
\noindent Soit $n \geq 1$
\begin{enumerate}
\item Montrer que si $A \in M_n(\mathbb{C})$ alors $A$ possède un sec stable de dimension $k$ avec $k \in \llbracket 0, n \rrbracket$ quelconque.
\item Soit $A \in M_n(\mathbb{R})$ \\
Montrer que $A$ possède une droite ou un plan stable.
\item Soit $A \in M_n(K)$ et $H$ un hyperplan de $K^n$. Montrer que: \\
$H$ stable par $A$ $\iff$ Il existe $L$ un vecteur propre de $A^T$ tel que $L^T X = 0$ est une équation de $H$
\end{enumerate}

\subsection{Réduction de matrices par blocs}
\noindent Soit $A \in M_n(\mathbb{C})$ et $B = \begin{pmatrix}
A & 2A \\
-1 & 2A
\end{pmatrix}$ \\
Montrer que $A$ diagonalisable $\iff$ $B$ diagonalisable
\end{document}