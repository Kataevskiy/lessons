\documentclass[10pt,a4paper]{article}
\usepackage[utf8]{inputenc}
\usepackage[french]{babel}
\usepackage[T1]{fontenc}
\usepackage{amsmath}
\usepackage{amsfonts}
\usepackage{amssymb}
\usepackage{graphicx}
\usepackage[left=2cm,right=2cm,top=2cm,bottom=2cm]{geometry}
\usepackage{setspace}
\usepackage{ulem}
\usepackage{stmaryrd}
\usepackage{amsthm}
\usepackage{dsfont}
\usepackage{mathpazo}

\onehalfspacing

\theoremstyle{definition}
\newtheorem{proposition}{Proposition}[section]
\newtheorem{theorem}[proposition]{Théorème}
\newtheorem{corollaire}[proposition]{Corollaire}
\newtheorem{lemme}[proposition]{Lemme}
\newtheorem{definition}[proposition]{Définition}

\DeclareMathOperator{\re}{Re}
\DeclareMathOperator{\im}{Im}

\begin{document}
\renewcommand{\labelitemi}{$*$}
\renewcommand{\labelenumi}{(\roman{enumi})}
\begin{center}
{\Large \textbf{Chapitre 13. Limites et continuité}}
\end{center}

\section{Voisinage}
\begin{definition}
Soit $a \in\overline{\mathbb{R}}$ \\
Un \uline{voisinage de $a$}:
\begin{itemize}
\item Si $a \in \mathbb{R}$, est un ensemble qui contient $[a - \delta, a + \delta]$, pour un certain $\delta > 0$
\item Si $a = +\infty$, est un ensemble $\left[A, +\infty\right[$ pour un certain $A \in \mathbb{R}$
\item Si $a = -\infty$, est un ensemble $\left]-\infty, A\right]$ pour un certain $A \in \mathbb{R}$
\end{itemize}
\end{definition}
\begin{lemme}
Soit $V$ un voisinage de $+\infty$ \\
Alors il existe une suite $(v_n)_{n \in \mathbb{N}} \in V^{\mathbb{N}}$ telle que $v_n \xrightarrow[x \to +\infty]{} +\infty$ 
\end{lemme}
\begin{definition}
Soit $f: I \to \mathbb{R}$ et $a \in \overline{\mathbb{R}}$ \\
On dit qu'une propriété (de la fonction $f$) est vraie \uline{au voisinage de $a$} s'il existe un voisinage $V$ de $a$ tel que la propriété soit vraie sur $V \cap I$
\end{definition}

\section{Notion de limite}
\uline{Cadre}: Dans cette section, $f: I \to \mathbb{R}$ est une fonction définie sur une partie $I$ de $\mathbb{R}$ et $a$ est un élément de $I$ ou $\pm\infty$. En pratique, $I$ sera un intervalle et $a$ un point ou une borne de l'intervalle.
\subsection{Limites en $\pm\infty$}
\begin{definition}
Soit $I$ un ensemble non majoré et $f: I \to \mathbb{R}$
On dit que:
\begin{itemize}
\item \uline{$f$ converge vers $l \in \mathbb{R}$ en $+\infty$} si $\forall \varepsilon > 0$, $\exists H \in \mathbb{R}$: $\forall x \in I$, $x \geq H \implies |f(x) - l| \leq \varepsilon$
\item \uline{$f$ tend vers $+\infty$ en $+\infty$} si $\forall A \in \mathbb{R}$, $\exists H \in \mathbb{R}$: $\forall x \in I$, $x \geq H \implies f(x) \geq A$
\item \uline{$f$ tend vers $-\infty$ en $+\infty$} si $\forall A \in \mathbb{R}$, $\exists H \in \mathbb{R}$: $\forall x \in I$, $x \geq H \implies f(x) \leq A$
\end{itemize}
\end{definition}

\subsection{Limites en un réel}
\uline{Cadre}: $a \in \overline{I}$
\begin{definition}
Soit $a \in \overline{I}$ et $f: I \to \mathbb{R}$
\begin{itemize}
\item On dit que \uline{$f$ tend vers $l \in \mathbb{R}$ en $a$} si \\
$\forall \varepsilon > 0$, $\exists \lambda > 0$: $\forall x \in I$, $|x - a| \leq \delta \implies |f(x) - l| \leq \varepsilon$
\item On dit que \uline{$f$ tend vers $+\infty$ en $a$} si \\
$\forall A \in \mathbb{R}$, $\exists \lambda > 0$: $\forall x \in I$, $|x - a| \leq \delta \implies f(a) \geq A$
\item On dit que \uline{$f$ tend vers $-\infty$ en $a$} si \\
$\forall A \in \mathbb{R}$, $\exists \lambda > 0$: $\forall x \in I$, $|x - a| \leq \delta \implies f(a) \leq A$
\end{itemize}
\end{definition}
\begin{proposition}
Soit $a \in I$ \\
Si $f$ admet une limite (dans $\overline{\mathbb{R}}$) en $a$, cette limite est nécessairement $f(a)$
\end{proposition}

\subsection{Variantes}
\begin{definition}
Soit $f: I \to \mathbb{R}$, $J$ une partie de $I$ et $a \in \overline{I} \cup \{ \pm\infty \}$. On suppose que $a$ est arbitrairement proche d'éléments de $J$ ( càd $a \in \overline{J}$ ou ($a = +\infty$ et $J$ n'est pas majoré) ou ($a = -\infty$ et $J$ n'est pas minoré) ) \\
On dit alors que $f(x) \xrightarrow[\substack{x \to a \\ x \in J}]{} l \in \mathbb{R}$ si:
\[\forall \varepsilon > 0, \, \exists \delta > 0: \forall x \in J, \, |x - a| \leq \delta \implies |f(x) - l| \leq \varepsilon\ \quad (\text{cas } a \in \mathbb{R})\]
\[\forall \varepsilon > 0, \, \exists H \in \mathbb{R}: \forall x \in J, \, x \geq H \implies |f(x) - l| \leq \varepsilon \quad (\text{cas } a = +\infty)\]
etc...
\end{definition}
\begin{proposition}
Soit $J_1, J_2$ deux parties de $I$, $a \in I \cup \{ \pm\infty \}$ et $l \in \overline{R}$. On suppose que $a$ est arbitrairement proche d'éléments de $J_1$ et de $J_2$ \\
Alors
\[ f(x) \xrightarrow[\substack{x \to a \\ x \in J_1 \cup J_2}]{} l \iff \begin{cases}
f(x) \xrightarrow[\substack{x \to a \\ x \in J_1}]{} l \\ f(x) \xrightarrow[\substack{x \to a \\ x \in J_2}]{} l \end{cases}\]
\end{proposition}

\section{Propriétés de la limite}
\subsection{Caractère local}
\begin{proposition}
Soit $f, g: I \to \mathbb{R}$ et $a \in \overline{I} \cup \{ \pm\infty \}$ arbitrairement proches d'éléments de $I$ \\
Si $f$ et $g$ coïncident au voisinage de $a$, alors $f$ admet une limite en $a$ ssi $g$ en admet une. Dans ce cas, ces limites sont les mêmes.
\end{proposition}

\subsection{Propriétés des fonctions convergentes}
\begin{proposition}
Les fonctions convergentes sont localement bornés: \\
Soit $f: I \to \mathbb{R}$, $a \in \overline{I} \cup \{ \pm\infty \}$ tel que $f(x) \xrightarrow[x \to a]{} l \in \mathbb{R}$ \\
Alors $f$ est bornée au voisinage de $a$.
\end{proposition}
\begin{proposition}[$\mathbb{R}_+^*$ est ouvert]
Soit $f: I \to \mathbb{R}$ et $a \in \overline{I} \cup \{ \pm\infty \}$ \\
Si $f(x) \xrightarrow[x \to a]{} l \in \mathbb{R}_+^*$, alors $f$ est $>0$ au voisinage de $a$.
\end{proposition}

\subsection{Caractérisation séquentielle de la limite}
\begin{theorem}
Soit $f: I \to \mathbb{R}$ et $a \in \overline{I} \cup \{ \pm\infty \}$. Soit $l \in \overline{\mathbb{R}}$ \\
On a $f(x) \xrightarrow[x \to a]{} l$ si et seulement si, pour toute suite $(\xi_n)_{n \in \mathbb{N}} \in I^\mathbb{N}$ telle que $\xi_n \xrightarrow[x \to +\infty]{} a$, on a $f(\xi_n) \xrightarrow[x \to +\infty]{} l$
\end{theorem}

\subsection{Composition des limites}
\begin{theorem}[À retenir mais mal énoncé]
Si $f(x) \xrightarrow[x \to a]{} b$ et $g(y) \xrightarrow[y \to b]{} l$, alors $g(f(x)) \xrightarrow[x \to a]{} l$
\end{theorem}
\begin{theorem}[Plus précis]
Soit $f: I \to J$ et $a \in \overline{I} \cup \{ \pm\infty \}$ et $b \in \overline{\mathbb{R}}$ tels que $f(x) \xrightarrow[x \to a]{} b$
\begin{itemize}
\item Déjà, $b \in \overline{J} \cup \{ \pm\infty \}$
\item Pour toute fonction $g: J \to \mathbb{R}$ telle que $g(y) \xrightarrow[y \to b]{} l \in \overline{\mathbb{R}}$, on a $g(f(x)) \xrightarrow[x \to a]{} l$
\end{itemize}
\end{theorem}

\subsection{Théorème de la limite monotone}
\begin{theorem}
Soit $f: I \to \mathbb{R}$ une fonction monotone.
\begin{itemize}
\item Si $I$ n'est pas majoré, $f$ admet une limite $l \in \overline{\mathbb{R}}$ et $+\infty$
\item Si $I$ n'est pas minoré, $f$ admet une limite $l \in \overline{\mathbb{R}}$ en $-\infty$
\item Si $a$ est un réel tel que $a \in \overline{I \cap \left]-\infty, a\right[}$, $f$ admet une limite $l \in \overline{\mathbb{R}}$ à gauche de $a$
\item Si $a$ est un réel tel que $a \in \overline{I \cap \left]a, +\infty\right[}$, $f$ admet une limite $l \in \overline{\mathbb{R}}$ à droite de $a$
\end{itemize}
\end{theorem}

\section{Continuité}
\subsection{Continuité en un point}
\end{document}