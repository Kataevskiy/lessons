\documentclass[10pt,a4paper]{article}
\usepackage[utf8]{inputenc}
\usepackage[french]{babel}
\usepackage[T1]{fontenc}
\usepackage{amsmath}
\usepackage{amsfonts}
\usepackage{amssymb}
\usepackage{graphicx}
\usepackage[left=2cm,right=2cm,top=2cm,bottom=2cm]{geometry}
\usepackage{setspace}
\usepackage{ulem}
\usepackage{stmaryrd}
\usepackage{amsthm}
\usepackage{dsfont}
\usepackage{mathpazo}

\onehalfspacing

\theoremstyle{definition}
\newtheorem{proposition}{Proposition}[section]
\newtheorem{theorem}[proposition]{Théorème}
\newtheorem{corollaire}[proposition]{Corollaire}
\newtheorem{lemme}[proposition]{Lemme}
\newtheorem{definition}[proposition]{Définition}

\begin{document}
\renewcommand{\labelitemi}{$*$}
\begin{center}
{\Large \textbf{Chapitre 24: Familles sommables}}
\end{center}
\section{Familles de nombres positifs}
\subsection{Généralités}
\begin{definition}
Soit $x = (x_j)_{j \in J}$ une famille de réels $\geq 0$ indexés par une ensemble $J$ \\
On définit
\[\sum\limits_{j \in J} x_j = \sup \left\{ \sum\limits_{j \in J_0} x_j \mid J_0 \in \mathcal{P}_f(J) \right\} \in [0, +\infty]\]
Avec la convention que la forme supérieure vaut $+\infty$ si l'ensemble n'est pas majoré. \\
(Ici, $\mathcal{P}_f$ désigne l'ensemble des parties finies de $J$)
\end{definition}
\begin{proposition}
Soit $x, y \in \mathbb{R}_+^J$, des familles indexées par $J$ \\
On a:
\begin{itemize}
\item \uline{Restriction}: Si $K \subseteq J$, $\sum\limits_{j \in K} x_j \leq \sum\limits_{j \in J} x_j$
\item \uline{Linéarité}: $\forall \lambda \in \mathbb{R}_+^*$, $\sum\limits_{j \in J} (x_j + \lambda y_j) = \sum\limits_{j \in J} x_j + \lambda \sum_{j \in J} y_j$
\item \uline{Croissance}: Si $\forall j \in J$, $x_j \leq y_j$, alors $\sum\limits_{j \in J} x_j \leq \sum\limits_{j \in J} y_j$
\end{itemize}
\end{proposition}
\begin{corollaire}
Supposons $j = \bigsqcup\limits_{k = 1}^n J_k$ \\
Alors, pour toute famille $x \in \mathbb{R}_+^J$, on a $\sum\limits_{j \in J} x_j = \sum\limits_{k = 1}^n \sum\limits_{j \in J_k} x_j$
\end{corollaire}

\subsection{Commutativité}
\begin{proposition}
\hfill
\begin{itemize}
\item Soit $\sigma: I \to J$ une bijection et $x \in \mathbb{R}_+^J$ \\
Alors $\sum\limits_{i \in I} x_{\sigma(I)} = \sum\limits_{j \in J} x_j$
\item En particulier, si $\sigma: J \to I$ est bijective \\
$\sum\limits_{j \in J} x_j = \sum\limits_{j \in J} x_{\sigma(j)}$
\end{itemize}
\end{proposition}
\end{document}