\documentclass[10pt,a4paper]{article}
\usepackage[utf8]{inputenc}
\usepackage[french]{babel}
\usepackage[T1]{fontenc}
\usepackage{amsmath}
\usepackage{amsfonts}
\usepackage{amssymb}
\usepackage{graphicx}
\usepackage[left=2cm,right=2cm,top=2cm,bottom=2cm]{geometry}
\usepackage{setspace}
\usepackage{ulem}
\usepackage{stmaryrd}
\usepackage{amsthm}
\usepackage{dsfont}
\usepackage{mathpazo}

\onehalfspacing

\theoremstyle{definition}
\newtheorem{proposition}{Proposition}[section]
\newtheorem{theorem}[proposition]{Théorème}
\newtheorem{corollaire}[proposition]{Corollaire}
\newtheorem{lemme}[proposition]{Lemme}
\newtheorem{definition}[proposition]{Définition}

\begin{document}
\renewcommand{\labelitemi}{$*$}
\renewcommand{\labelenumi}{(\roman{enumi})}
\begin{center}
{\Large \textbf{Chapitre 15. Convexité}}
\end{center}
Dans tout le chapitre, $I$ désigne un intervalle non trivial.

\section{Généralités}
\subsection{Définitions}
\begin{definition}
\hfill
\begin{itemize}
\item Une fonction $f: I \to \mathbb{R}$ est \uline{convexe} si $\forall a, b \in I$, $\forall \lambda \in [0, 1]$ 
\[f((1 - \lambda)a + \lambda b) \leq (1 - \lambda)f(a) + \lambda f(b)\]
\item Elle est dite \uline{concave} si $\forall a, b \in I$, $\forall \lambda \in [0, 1]$
\[f((1 - \lambda)a + \lambda b) \geq (1 - \lambda)f(a) + \lambda f(b)\]
\end{itemize}
\end{definition}

\subsection{Opérations}
\begin{proposition}
Soit $f, g: I \to \mathbb{R}$
\begin{itemize}
\item Si $f$ et $g$ sont convexes (resp. concaves), alors $f + g$ aussi.
\item Si $f$ et $g$ sont convexes, $\max(f, g)$ est convexe.
\item Si $f$ et $g$ sont concaves, $\min(f, g)$ est concave.
\item L'opposé d'une fonction convexe est concave et réciproquement.
\end{itemize}
\end{proposition}

\subsection{Inégalité de Jensen}
\begin{definition}
Soit $x_1, ...\, , x_n \in \mathbb{R}$ \\
Une \uline{combinaison convexe} de $x_1, ...\, , x_n$ est un nombre de la forme $\lambda_1 x_1 + ... + \lambda_n x_n$ où $\begin{cases}
\lambda_1, ...\, , \lambda_n \in \mathbb{R}_+ \\
\lambda_1 + ... + \lambda_n = 1
\end{cases}$  \\
On parle aussi de \uline{barycentre} des $a_1, ...\, , a_n$ (et dans le cas $\lambda_1 = ... = \lambda_n = \frac{1}{n}$ on parle d'\uline{isobarycentre}).
\end{definition}
\begin{theorem}[Inégalité de Jensen / de convexité]
Soit $f: I \to \mathbb{R}$ convexe, $a_1, ...\, , a_n \in \mathbb{R}$ et $\lambda_1, ...\, , \lambda_n \in \mathbb{R}_+$ tels que $\lambda_1 + ... + \lambda_n = 1$ \\
Alors
\[f\left(\sum_{i = 1}^n \lambda_i a_i\right) \leq \sum_{i = 1}^n \lambda_i f(a_i)\]
\end{theorem}

\subsection{Position de sécantes}
\begin{proposition}
Soit $f:I \to \mathbb{R}$ convexe et $a < b \in I$. Alors:
\begin{itemize}
\item Sur $I \cap \left]-\infty, a\right]$ le graphe de $f$ est au-dessus (au sens large) de la droite reliant $\begin{pmatrix} a \\ f(a) \end{pmatrix}$ et $\begin{pmatrix} b \\ f(b) \end{pmatrix}$
\item Sur $I \cap [b, +\infty[$ idem.
\item Sur $[a, b]$ le graphe est en-dessous de la droite.
\end{itemize}
\end{proposition}

\pagebreak

\subsection{Pentes}
\begin{proposition}[Inégalité de trois pentes]
Soit $f: I \to \mathbb{R}$ et $s < t < u \in I$ \\
Alors
\[\frac{f(t) - f(s)}{t - s} \leq \frac{f(u) - f(s)}{u - s} \leq \frac{f(u) - f(t)}{u - t}\]
\end{proposition} \medskip

\noindent \uline{Remarque}: Chacune des trois inégalités contenues dans le théorème est équivalent à deux autres \\
et au faut que $\begin{pmatrix} t \\ f(t) \end{pmatrix}$ est sous la courbe joignant $\begin{pmatrix} s \\ f(s) \end{pmatrix}$ et $\begin{pmatrix} u \\ f(u) \end{pmatrix}$  \\
Les trois inégalités équivalent à
\[ (u - s)f(t) \leq (u - t)f(s) + (t - s)f(u) \]
ou encore
\[ f(t) \leq \underbrace{\frac{u - t}{u - s}}_{1 - \lambda}f(s) + \underbrace{\frac{t - s}{u - s}}_{\lambda}f(u) \]
avec $\lambda$ tel que $t = (1 - \lambda) s + \lambda u$

\begin{proposition}
Une fonction $f: I \to \mathbb{R}$ est convexe si et seulement si, pour tout $a \in I$, la fonction \\
taux d'accroissement
\[ \tau_{[f, a]} \begin{cases}
I \setminus \{ a \} \to \mathbb{R} \\
x \mapsto \frac{f(x) - f(a)}{x - a}
\end{cases} \]
est croissante.
\end{proposition}

\section{Convexité et régularité}
\subsection{Régularité automatique}
\begin{proposition}
Soit $f: I \to \mathbb{R}$ convexe et $a \in I$ intérieur. \\
Alors $f$ est dérivable à gauche et à droite en $a$ et $f'_g(a) \leq f'_d(a)$
\end{proposition}
\begin{theorem}
Une fonction convexe est continue en tout point intérieur de son intervalle de définition.
\end{theorem}
\begin{corollaire}
Si $I$ est un intervalle ouvert, toute fonction $f: I \to \mathbb{R}$ convexe est continue.
\end{corollaire}

\subsection{Caractérisation des fonctions convexes (deux fois) dérivables}
\begin{theorem}
Soit $f: I \to \mathbb{R}$ dérivable. \\
Alors $f$ est convexe ssi $f'$ croît.
\end{theorem}
\begin{corollaire}
Soit $f: I \to \mathbb{R}$ deux fois dérivable. \\
Alors $f$ est convexe ssi $f'' \geq 0$
\end{corollaire}
\begin{corollaire}
Soit $f \in D^1(I)$ convexe et $a < b \in I$ \\
On a
\[ f'(a) \leq \frac{f(b) - f(a)}{b - a} \leq f'(b) \]
\end{corollaire}

\pagebreak

\subsection{Applications}
Le critère de convexité pour les fonctions deux fois dérivables montre directement que:
\begin{itemize}
\item $\exp$ est convexe.
\item $\ln$ est concave.
\item $\sin$ est concave sur $[0, - \pi]$
\end{itemize}
\begin{proposition}[Inégalité arithmético-géométrique générale]
Soit $\alpha_1, ...\, , \alpha_n \in \mathbb{R}_+$ de somme 1 et $\left.x_1, ...\, , x_n \in \mathbb{R}_+^*\right.$ \\
Alors
\[x_1^{\alpha_1} x_2^{\alpha_2} ... x_n^{\alpha_n} \leq \alpha_1 x_1 + \alpha_2 x_2 + ... + \alpha_n x_n\]
\end{proposition}

\subsection{Position par rapport à la tangente}
\begin{proposition}
Soit $f \in D^1(I)$ convexe et $a \in I$ \\
Alors $\forall x \in I$, $f(x) \geq f(a) + f'(a)(x - a)$
\end{proposition}
\noindent Cette proposition permet de retrouver des inégalités classiques:
\begin{itemize}
\item $\forall x \in \mathbb{R}$, $e^x \geq 1 + x$
\item $\forall x \in \mathbb{R}_+^*$, $\ln(x) \leq x + 1$
\item $\forall h \in \left]-1, +\infty\right[$, $\ln(1 + h) \leq h$
\end{itemize}
La concavité sur $[0, \pi]$ de $\sin$ donne: \\
$\forall x \in [0, \pi]$, $\sin(x) \leq x$ (on a même $\forall x \in \mathbb{R}$, $|\sin(x)| \leq |x|$) \\
La position par rapport aux sécantes montre aussi \\
$\forall x \in \left[0, \frac{\pi}{2}\right]$, $\sin(x) \geq \frac{2}{\pi} x$
\end{document}