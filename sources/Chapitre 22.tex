\documentclass[10pt,a4paper]{article}
\usepackage[utf8]{inputenc}
\usepackage[french]{babel}
\usepackage[T1]{fontenc}
\usepackage{amsmath}
\usepackage{amsfonts}
\usepackage{amssymb}
\usepackage{graphicx}
\usepackage[left=2cm,right=2cm,top=2cm,bottom=2cm]{geometry}
\usepackage{setspace}
\usepackage{ulem}
\usepackage{stmaryrd}
\usepackage{amsthm}
\usepackage{dsfont}
\usepackage{mathpazo}

\onehalfspacing

\theoremstyle{definition}
\newtheorem{proposition}{Proposition}[section]
\newtheorem{theorem}[proposition]{Théorème}
\newtheorem{corollaire}[proposition]{Corollaire}
\newtheorem{lemme}[proposition]{Lemme}
\newtheorem{definition}[proposition]{Définition}

\DeclareMathOperator{\vect}{Vect}

\begin{document}
\renewcommand{\labelitemi}{$*$}
\begin{center}
{\Large \textbf{Chapitre 22: Équations différentielles linéaires}}
\end{center}

\section{Généralités}
\subsection{Définitions}
\begin{definition}
\hfill \begin{itemize}
\item Soit $n \in \mathbb{N}^*$
Une \uline{équation linéaire résolue d'ordre $n$} est une équation de la forme
\[ y^{(n)} + a_{n - 1}(x)y^{(n - 1)} + ... + a_1(x)y' + a_0(x)y = b(x) \tag{É} \]
où $a_0, a_1, ...\,, a_{n - 1}$ et $b: I \to \mathbb{C}$ sont des fonctions continues.
\item Résoudre cette équation différentielle, c'est trouver toutes les fonctions $f \in D^n(I; \mathbb{C})$ telles que
\[ \forall x \in I,\, f^{(n)}(x) + a_{n - 1}(x)f^{(n - 1)}(x) + ... + a_1(x)f'(x) + a_0(x)f(x) = b(x) \]
L'équation différentielle (É) est dite \uline{homogène} si le second membre $b$ est nul.
\item \uline{L'équation homogène associée à (É)} est
\[ y^{(n)} + a_{n - 1}(x)y^{(n - 1)} + ... + a_1(x)y' + a_0(x)y = 0 \tag{ÉH} \]
\end{itemize}
\end{definition}
\noindent \uline{Remarque}: L'équation (É) est dite linéaire car elle ne fait intervenir que des combinaisons linéaires des dérivées de $f$ et résolue car le terme faisant apparaître la plus grande dérivée de $y$ n'est pas multiplié par un coefficient.
\begin{proposition}
\hfill
\begin{itemize}
\item Toute solution $f: I \to \mathbb{C}$ de l'équation (É) est automatiquement de classe $C^n$
\item Si les coefficients $a_0, a_1, ...\,, a_{n - 1}$ et le second membre $b$ sont des solution lisses, toute solution est automatiquement lisse.
\end{itemize}
\end{proposition}
\begin{proposition}[Principe de superposition]
\hfill \\
Soit $f, g \in D^n(I; \mathbb{C})$ deux solutions des équations différentielles
\begin{align*}
&y^{(n)} + a_{n - 1}(x)y^{(n - 1)} + ... + a_1(x)y' + a_0(x)y = b(x) \\
\text{ et } \quad &y^{(n)} + a_{n - 1}(x)y^{(n - 1)} + ... + a_1(x)y' + a_0(x)y = c(x)
\end{align*}
respectivement. \\
Soit $\lambda, \mu \in \mathbb{C}$. Alors $\lambda f + \mu g$ est une solution de l'équation différentielle
\[ y^{(n)} + a_{n - 1}(x)y^{(n - 1)} + ... + a_1(x)y' + a_0(x)y = \lambda b(x) + \mu c(x) \]
\end{proposition}
\begin{corollaire}
\hfill
\begin{itemize}
\item L'ensemble $\mathcal{S}_\text{hom}$ des solutions de (ÉH) est un sous-espace vectoriel de $D^n(I; \mathbb{C})$
\item Si $f_0 \in D^n(I; \mathbb{C})$ est une solution de (É), alors l'ensemble des solution de (É) est
\[ \mathcal{S} = \{ f_0 + h \mid h \in \mathcal{S}_\text{hom}\} \]
c'est-à-dire un sous-espace affine de $D^n(I; \mathbb{C})$ de direction $\mathcal{S}_\text{hom}$ \\
"Les solutions s'obtiennent comme somme d'une solution particulière et des solutions de l'équation homogène".
\end{itemize}
\end{corollaire}

\subsection{Complément: théorème de Cauchy linéaire}
\begin{theorem}
\hfill
\begin{itemize}
\item Quels que soient $x_0 \in I$ et les nombres complexe $s_0, ...\,, s_{n - 1} \in \mathbb{C}$, il existe une unique solution $f \in D^n(I; \mathbb{C})$ de (É) telle que
\[ f(x_0) = s_0 \quad  \text{ et } \quad f'(x_0) = s_1 \quad \text{ et } \quad f''(x_0) = s_2 \quad ... \quad \text{ et } \quad f^{(n - 1)}(x_0) = s_{n - 1} \]
\item Le sous-espace vectoriel $\mathcal{S}_\text{hom}$ des solutions de (ÉH) est de dimension $n$
\item L'ensemble des solution $\mathcal{S}$ est donc
\[ S = f_0 + \mathcal{S}_\text{hom} = \{ f_0 + \lambda_1 f_1 + ... + \lambda_n h_n \mid (\lambda_1, ...\,, \lambda_n) \in \mathbb{C}^n \} \]
où $f_0$ est une solution de (É) et où $(h_1, ...\,, h_n)$ est une base de solutions de (ÉH).
\end{itemize}
\end{theorem}

\section{Équation différentielles linéaires d'ordre $1$}
\noindent Dans cette section, on considère une équation
\[ y' + a(x)y = b(x) \tag{É} \]
et son équation homogène associé
\[ y' + a(x)y = 0 \tag{ÉH} \]
par deux fonctions continues $a, b: I \to \mathbb{C}$

\subsection{Équation homogène}
\begin{theorem}
\hfill
\begin{itemize}
\item Soit $a: I \to \mathbb{C}$ une fonction continue. \\
On note $A: I \to \mathbb{C}$ une primitive de $a$ et $h_0: \begin{cases}
I \to \mathbb{C} \\
x \mapsto e^{-A(x)}
\end{cases}$ \\
Alors les solutions de (ÉH) forment l'ensemble
\[ \mathcal{S}_\text{hom} = \vect(h_0) = \{ \lambda h_0 \mid \lambda \in \mathbb{C} \} \]
\item L'unique solution $f: I \to \mathbb{C}$ de (ÉH) valant $s_0$ en $x_0 \in I$ est la fonction
\[ \begin{cases}
I \to \mathbb{R} \\
x \mapsto s_0 \exp\left( - \int_{x_0}^x a(t) \, dt \right)
\end{cases} \]
\end{itemize}
\end{theorem}

\subsection{Solution particulière: quelques heuristiques}
\noindent Une fois l'équation homogène (ÉH) résolue, il s'agit de trouver une solution particulière à (É). \\
On verra dans la section ultérieure une méthode générale, mais une première idée est de chercher des solutions "du même type" que le second membre $b$ de (É). \medskip

\pagebreak

\noindent \uline{Exemple important}:
\[ y' + ay = Ae^{rx} \]
avec $a, r \in \mathbb{R}$, dont les solutions homogènes sont les $x \mapsto e^{-ax}$ \\
On peut avoir l'idée de chercher les solutions particulières sous la forme $Be^{rx}$. Cela marche presque tout le temps, mais on voit déjà que dans le cas $a = r = 0$, la solution particulière est linéaire. \\
En faite, deux cas se présentent, suivant que $x \mapsto e^{rx}$ est ou non solution de l'équation homogène.
\begin{itemize}
\item Si $x \mapsto e^{rx}$ n'est pas solution de l'équation homogène, c'est-à-dire si $a + r \neq 0$, on peut trouver une solution de la forme $f: x \mapsto Be^{rx}$
\item En revanche, si $a + r = 0$ (et que $A \neq 0$), on voit que la méthode ci-dessus échoue. On peut en revanche trouver une solution sous la forme $g: x \mapsto Cxe^{rx}$
\end{itemize}
Cette disjonction de cas correspond au phénomène de la \uline{résonance}.

\subsection{Solution particulière: variation de la constante}
\begin{theorem}
Soit $A$ une primitive de $a$, de sorte que $\mathcal{S}_\text{hom} = \vect\left( x \mapsto e^{-A(x)} \right) = \left\{ x \mapsto \lambda e^{-A(x)} \, \middle| \, \lambda \in \mathbb{C} \right\}$ \\
Alors l'équation $y' + a(x)y = b(x)$ a une solution particulière de la forme $x \mapsto \lambda(x)e^{-A(x)}$, pour une certaine fonction $x \mapsto \lambda(x)$ dérivable.
\end{theorem}
\begin{corollaire}
Grâce à l'expression intégrale de la primitive, on trouve qu'une solution particulière de (É) est
\[ x \mapsto \left(\int_{x_0}^x b(t)e^{A(t)} \, dt \right) e^{-A(x)} \]
où $x_0 \in I$ est un point quelconque.
\end{corollaire}

\subsection{Résultat général}
\begin{theorem}
Les solutions de (É) sont les fonctions de la forme
\[ x \mapsto \left( \int_{x_0}^x b(t) e^{A(t)} \, dt \right) e^{-A(x)} + \lambda e^{-A(x)} = \left( \lambda + \int_{x_0}^x b(t) e^{A(t)} \, dt \right) e^{-A(x)} \]
où $\lambda$ décrit l'ensemble des nombres complexes.
\end{theorem}
\begin{corollaire}
Soit $s_0 \in \mathbb{C}$. L'équation (É) a une unique solution valant $s_0$ et $x_0$
\end{corollaire}

\section{Équations différentielles linéaires à coefficients constants d'ordre $2$}
\noindent Dans cette section, on considère une équation
\[ y'' + \alpha y' + \beta y = b(x) \tag{É} \]
où $\alpha, \beta$ et $b \in C^0(\mathbb{R}; \mathbb{C})$. Il s'agit donc d'une équation différentielle linéaire à coefficients constants.

\pagebreak

\subsection{Équation homogène}
\noindent Intéressons-nous d'abord à l'équation homogène associée.
\[ y'' + \alpha y' + \beta y = 0 \tag{Éh} \]
Le théorème de Cauchy linéaire entraîne que l'ensemble $\mathcal{S}_\text{hom}$ de ses solutions est un sous-espace vectoriel possédant une base à deux éléments de $D^2(\mathbb{R}; \mathbb{C})$ - et même de $C^\infty(\mathbb{R}; \mathbb{C})$, comme les coefficients sont (constants donc) lisses. \medskip

\noindent Comme dans le cas des suites récurrentes, le comportement dépend des solutions du \uline{polynôme caractéristique} $P = X^2 + \alpha X + \beta$
\begin{theorem}
\hfill
\begin{itemize}
\item Si le polynôme caractéristique $P$ possède deux racines simples $\rho$ et $\sigma \in \mathbb{C}$, alors
\[ \mathcal{S}_\text{hom} = \{ x \mapsto \lambda e^{\rho x} + \mu e^{\sigma x} \mid (\lambda, \mu) \in \mathbb{C}^2 \} = \vect( x \mapsto e^{\rho x}, x \mapsto e^{\sigma x}) \]
\item Si le polynôme caractéristique $P$ possède une racine double $\rho \in \mathbb{C}$, alors
\[ \mathcal{S}_\text{hom} = \{ x \mapsto (\lambda + \mu x) e^{\rho x} \mid (\lambda, \mu) \in \mathbb{C}^2 \} = \vect( x \mapsto e^{\rho x}, x \mapsto x e^{\rho x}) \]
\end{itemize}
\end{theorem}
\begin{lemme}
Soit $\alpha, \beta, \tau \in \mathbb{C}$ et $f \in D^2(\mathbb{R}; \mathbb{C})$ \\
Alors $f$ est solution de (ÉH) si et seulement si $g: x \mapsto e^{\tau x}f(x)$ est solution de
\[ y'' + (\alpha - 2\tau)y' + (\beta - \alpha \tau + \tau^2)y = 0 \tag{ÉH$_\tau$} \]
\end{lemme}
\begin{corollaire}
Soit $s_0, s_1 \in \mathbb{C}$ et $x_0 \in \mathbb{R}$ \\
Il existe une unique solution $f$ de (ÉH) telle que $f(x_0) = s_0$ et $f'(x_0) = s_1$
\end{corollaire}
\begin{theorem}
On suppose ici les coefficients $\alpha, \beta$ réels. Les solutions de l'équation différentielle sont
\begin{itemize}
\item Si $P$ a deux racines réelles $\rho \neq \sigma$:
\[ \mathcal{S}_\text{hom} = \{ x \mapsto \lambda e^{\rho x} + \mu e^{\sigma x} \mid (\lambda, \mu) \in \mathbb{C}^2 \} = \vect(x \mapsto e^{\rho x}, x \mapsto e^{\sigma x}) \]
\item Si $P$ a une racine double (nécessairement réelle) $\rho$:
\[ \mathcal{S}_\text{hom} = \{ x \mapsto (\lambda x + \mu) e^{\rho x} \mid (\lambda, \mu) \in \mathbb{C}^2 \} = \vect(x \mapsto e^{\rho x}, x \mapsto x e^{\rho x}) \]
\item Si $P$ a deux racines imaginaires conjuguées $r \pm is$, avec $r \in \mathbb{R}$ et $s \in \mathbb{R}_+^*$:
\begin{align*}
\mathcal{S}_\text{hom} &= \{ x \mapsto e^{r x}(\lambda \cos(sx) + \mu \sin(sx)) \mid (\lambda, \mu) \in \mathbb{C}^2 \} \\
&= \vect( x \mapsto e^{r x} \cos(sx),\, x \mapsto e^{r x} \sin(sx))
\end{align*}
\end{itemize}
\end{theorem}
\begin{theorem}
Les solutions \uline{à valeurs réelles} de (ÉH) sont les fonctions décrites par le théorème précédente, où le couple $(\lambda, \mu)$ appartient à $\mathbb{R}^2$
\end{theorem}

\pagebreak

\subsection{Solution particulière: quelques heuristiques}
\noindent Le principe de superposition entraîne que les solutions de (É) forment l'ensemble
\[ \mathcal{S} = \{ f_0 + h \mid h \in \mathcal{S}_\text{hom} \} \]
Pour résoudre complètement (É), il suffit donc maintenant d'en trouver une solution particulière. Deux cas sont au programme: celui d'un second membre exponentiel et celui d'un second membre sinusoïdal, qui s'y ramène.
\begin{proposition}
Considérons l'équation
\[ y'' + \alpha y' + \beta y = Ae^{rx} \tag{É} \]
où $A, r \in \mathbb{C}$, de polynôme caractéristique $P = X^2 + \alpha X + \beta$. Alors
\begin{itemize}
\item Si $r$ n'est pas racine de $P$, (É) a une solution particulière de la forme
\[ x \mapsto Ce^{rx} \]
\item Si $r$ est une racine simple de $P$, (É) a une solution simple de la forme
\[ x \mapsto Cxe^{rx} \]
\item Si $r$ est une racine double de $P$, (É) a une solution simple de la forme
\[ x \mapsto Cx^2e^{rx} \]
\end{itemize}
\end{proposition}
\end{document}