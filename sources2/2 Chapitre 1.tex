\documentclass[10pt,a4paper]{article}
\usepackage[utf8]{inputenc}
\usepackage[french]{babel}
\usepackage[T1]{fontenc}
\usepackage{amsmath}
\usepackage{amsfonts}
\usepackage{amssymb}
\usepackage{graphicx}
\usepackage[left=2cm,right=2cm,top=2cm,bottom=2cm]{geometry}
\usepackage{setspace}
\usepackage{ulem}
\usepackage{stmaryrd}
\usepackage{amsthm}
\usepackage{dsfont}
\usepackage{mathpazo}


\onehalfspacing

\theoremstyle{definition}
\newtheorem{proposition}{Proposition}[section]
\newtheorem{theorem}[proposition]{Théorème}
\newtheorem{corollaire}[proposition]{Corollaire}
\newtheorem{lemme}[proposition]{Lemme}
\newtheorem{definition}[proposition]{Définition}
\newtheorem*{rappel}{Rappel}

\DeclareMathOperator{\com}{com}
\DeclareMathOperator{\re}{Re}
\DeclareMathOperator{\im}{Im}

\begin{document}
\begin{center}
{\Large \textbf{Chapitre 1: Structures fondamentales}}
\end{center}
Dans la suite, $\mathbb{K}$ désigne $\mathbb{R}$ ou $\mathbb{C}$

\section{Groupes, anneaux, corps, espaces vectoriels}
\subsection{Structures algébriques usuelles}
\uline{lci} $*$: $\begin{cases}
E \times E \to E \\
(x, y) \mapsto x * y
\end{cases}$
\begin{definition}
Soit $M$ un ensemble muni d'une lci $*$ \\
$(M, *)$ est un monoïde si:
\begin{enumerate}
\item $*$ est associative.
\item $*$ possède un élément neutre $e_M$
\end{enumerate}
\end{definition}
\begin{definition}
Un groupe est un monoïde dont tous les éléments sont inversibles.
\end{definition}
\begin{definition}
Soit $A$ un ensemble avec 2 lci: $+$ et $*$ \\
$A$ est un anneau si:
\begin{enumerate}
\item $(A, +)$ est un groupe abélien.
\item $(A, *)$ est un monoïde.
\item $\forall a, x, y \in A \qquad \begin{cases}
a * (x + y) = a * x + a * y \\
(x + y) * a = x * a + y * a
\end{cases}$
\end{enumerate}
\end{definition}
\begin{definition}
Un anneau commutatif $\neq \{ 0 \}$ dont tous les éléments non nuls sont inversibles est un corps.
\end{definition}
\begin{definition}
Soit $(E, + , \bullet)$ avec $E$ ensemble, $*$ lci et $\bullet$: $\begin{cases}
\mathbb{K} \times E \to E \\
(\lambda, x) \mapsto \lambda \bullet x
\end{cases}$
(l.c. externe) \\
$(E, +, \bullet)$ est un $\mathbb{K}$ espace vectoriel si:
\begin{enumerate}
\item $(E, +)$ groupe abélien.
\item $\forall x \in E \qquad 1 \bullet x = x$
\item $\forall \lambda \in \mathbb{K}$, $\forall x, y \in E \qquad \lambda \bullet (x + y) = \lambda \bullet x + \lambda \bullet y$
\item $\forall \lambda, \mu \in \mathbb{K}$, $\forall x \in E \qquad (\lambda + \mu) \bullet x = \lambda \bullet x + \mu \bullet x$
\item $\forall \lambda, \mu \in \mathbb{K}$, $\forall x \in E \qquad (\lambda \bullet \mu) \bullet x = \lambda \bullet (\mu \bullet x) = \mu \bullet (\lambda \bullet x)$
\end{enumerate}
\end{definition}

\subsection{Sous-structures}
\begin{rappel} \hfill
\begin{enumerate}
\item $G$ groupe, $H \subset G$ \\
$H$ sous-groupe $\iff \begin{cases}
1_G \in H \\
\forall x, y \in H,\, xy \in H \\
\forall x \in H,\, x^{-1} \in H
\end{cases}$ \\
$H$ est un groupe aussi pour la restriction.
\item $A$ anneau, $B \subset A$ \\
$B$ sous-anneau $\iff \begin{cases}
\forall x, y \in B,\, x + y \in B,\, xy \in B \\
1_A \in B \\
\forall x \in B,\, -x \in B
\end{cases}$ \\
Le sous-anneau $B$ est en particulier un anneau.
\item $K$ un corps, $L \subset K$ \\
$L$ sous-corps de $K$ $\iff \begin{cases}
L \text{ sous-anneau de } K \\
\forall x \in L \setminus \{ 0 \},\, x^{-1} \in L
\end{cases}$
\item $E$ un $\mathbb{K}$-ev, $F \subset E$ \\
$F$ sous-espace vectoriel de $E$ $\iff \begin{cases}
\forall x, y \in F,\, x + y \in F \\
\forall \lambda \in \mathbb{K},\, \forall x \in F,\, \lambda x \in F \\
0 \in F
\end{cases}$
\end{enumerate}
\end{rappel}
\noindent \uline{Démarche}: Pour montrer qu'un ensemble est un machin \footnote{monoïde, groupe, anneau, corps ou $\mathbb{K}$-ev}, on pourra le réaliser comme un sous-machin d'un machin connu.

\begin{lemme}
Soit $M = \begin{pmatrix}
\alpha & \beta \\
\gamma & \delta
\end{pmatrix} \in GL_2(K)$ \\
Alors
\[ M^{-1} = \frac{1}{\det M} \begin{pmatrix}
\delta & - \beta \\
- \gamma & \alpha
\end{pmatrix}\]
\end{lemme}
\begin{proposition}
Soit $M$ un machin \footnotemark[1] et $(M_i)_{i \in I}$ une famille de sous-machins de $M$ \\
Alors $\bigcap\limits_{i \in I} M_i$ est un sous-machin de $M$
\end{proposition}

\subsection{Morphismes}
\begin{rappel} \hfill
\begin{enumerate}
\item $f: G \to H$, $G, H$ groupes. \\
$f$ morphisme de groupes $\iff \forall x, y \in G$, $f(x * y) = f(x) + f(y)$ \\
Dans ces conditions: $\begin{cases}
f(e_G) = e_H \\
\forall x \in G,\, f(x^{-1}) = f(x)^{-1}
\end{cases}$
\item Soit $f: A \to B$, $A, B$ anneaux. \\
$f$ morphisme d'anneaux $\iff \begin{cases}
f(1_A) = 1_B \\
\forall x, y \in A,\, \begin{cases}
f(x * y) = f(x) + f(y) \\
f(xy) = f(x) f(y)
\end{cases}
\end{cases}$ \\
Automatiquement: $\begin{cases}
f(0) = 0 \\
f(-x) = -f(x) \\
x \text{ inversible } \implies f(x) \text{ inversible et } f(x^{-1}) = f(x)^{-1}
\end{cases}$
\item Un morphisme de corps c'est un morphisme d'anneaux.
\item $u: E \to F$ linéaire $\iff \begin{cases}
\forall x, y \in E,\, u(x + y) = u(x) + u(y) \\
\forall x \in E, \forall \lambda \in \mathbb{K},\, u(\lambda x) = \lambda u(x)
\end{cases}$
\end{enumerate}
\end{rappel}
\begin{rappel}
Isomorphisme = morphisme bijectif. \\
La composée de $2$ morphismes est un morphisme. La réciproque d'un isomorphisme est un isomorphisme. \\
$G$ et $H$ sont dits isomorphes s'il existe $f: G \to H$ isomorphe. On note alors $G \underset{f}{\simeq} H$ ou $G \simeq H$
\end{rappel}

\pagebreak

\begin{rappel}
\hfill
\begin{enumerate}
\item $f: G \to H$ morphisme de groupes. \\
$\ker f = \{ x \in G \mid f(x) = e_H \}$ \\
( Respectivement, $\ker f = \{ x \in B \mid f(x) = 1 \}$ )
\item Si $f: A \to B$ morphisme d'anneaux. \\
$\ker f = \{ x \in A \mid f(x) = 0 \}$
\item Si $u \in \mathcal{L}(E, F)$ \\
$\ker u = \{x \in E \mid u(x) = 0\}$ \\
$f$ injective $\iff \ker f = \{ \text{ neutre } \}$
\end{enumerate}
\end{rappel}
\begin{rappel}
\hfill
\begin{enumerate}
\item Soit $f: G \to H$ morphisme de machins\footnotemark[1]. \\
Alors $f(G)$ est un sous-machin de $H$
\item Si $f$ est $\begin{cases}
\text{ un morphisme de groupes } \\
\text{ une application linéaire} 
\end{cases}$ alors $\ker f$ est $\begin{cases}
\text{ un sous-groupe } \\
\text{ un sous-espace vectoriel} 
\end{cases}$
\end{enumerate}
\end{rappel}
\begin{definition}
Soit $E$ un $\mathbb{K}$-ev. \\
Un hyperplan de $E$ est le noyau d'une forme linéaire non nulle de $E$, ie. d'un élément de $\mathcal{L}(E, K) \setminus \{ 0 \}$
\end{definition}

\subsection{Structure de $\mathbb{K}$-algèbre}
\begin{definition}
Soit: \begin{itemize}
\item $\mathbb{K}$ un corps.
\item $A$ un ensemble.
\item $+, *$ deux lci sur $A$
\item $\bullet$ une lce sur $A$ à opérateurs dans $\mathbb{K}$
\end{itemize}
On dit que $A$ est une $\mathbb{K}$-algèbre si: 
\begin{enumerate}
\item $(A, +,  *)$ est un anneau.
\item $(A, +, \bullet)$ est un $\mathbb{K}$-ev.
\item $\forall \lambda \in \mathbb{K}$, $\forall A, B \in A \qquad \lambda \bullet (ab) = (\lambda \bullet a) b = a (\lambda \bullet b)$
\end{enumerate}
\end{definition}
\begin{proposition}
\hfill
\begin{enumerate}
\item Si $X$ est un ensemble, $\mathcal{F}(X, K) = K^X$ est une $\mathbb{K}$-algèbre commutative.
\item $\mathbb{K}[X]$ est une $\mathbb{K}$-algèbre commutative.
\item $M_n(\mathbb{K})$ est une $\mathbb{K}$-algèbre.
\item Si $E$ est un $\mathbb{K}$-ev, $(\mathcal{L}(E), +, \circ, *)$ est une $\mathbb{K}$-algèbre.
\end{enumerate}
\end{proposition}
\begin{proposition}
Soit $L$ un surcorps de $\mathbb{K}$ \\
Alors $L$ est une $\mathbb{K}$-algèbre.
\end{proposition}
\begin{definition}
Soit $A$ une $\mathbb{K}$-algèbre. Soit $B \subset A$ \\
$B$ est une sous-algèbre de $A$ si:
\begin{itemize}
\item $1_A \in B$
\item $\forall x, y \in B \qquad x + y \in B \qquad xy \in B$
\item $\forall \lambda \in \mathbb{K}$, $\forall x \in B \qquad \lambda x \in B$
\end{itemize}
\end{definition}

\pagebreak

\begin{definition}
Soit $f: A \to B$, $A, B$ deux $\mathbb{K}$-algèbres. \\
On dit que $f$ est un morphisme d'algèbres si $f$ est un morphisme d'anneaux linéaire, ie:
\begin{enumerate}
\item $f(1_A) = 1_B$ 
\item $\forall x, y \in A \qquad f(x + y) = f(x) + f(y)$
\item $\forall x, y \in A \qquad f(xy) = f(x) f(y)$ 
\item $\forall \lambda \in \mathbb{K}$, $\forall x \in A \qquad f(\lambda x) = \lambda f(x)$
\end{enumerate}
Si de plus $f$ est bijective, on dit que $f$ est un isomorphisme. On écrit alors $A \simeq B$
\end{definition}
\begin{proposition}
L'image d'une $\mathbb{K}$-algèbre par $f: A \to B$ morphisme est une sous-algèbre de $B$
\end{proposition}

\section{Ensembles quotients}
\subsection{Généralités}
\end{document}