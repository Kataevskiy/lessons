\documentclass[10pt,a4paper]{article}
\usepackage[utf8]{inputenc}
\usepackage[french]{babel}
\usepackage[T1]{fontenc}
\usepackage{amsmath}
\usepackage{amsfonts}
\usepackage{amssymb}
\usepackage{graphicx}
\usepackage[left=2cm,right=2cm,top=2cm,bottom=2cm]{geometry}
\usepackage{setspace}
\usepackage{ulem}
\usepackage{stmaryrd}
\usepackage{amsthm}
\usepackage{dsfont}
\usepackage{mathpazo}


\onehalfspacing

\theoremstyle{definition}
\newtheorem{proposition}{Proposition}[section]
\newtheorem{theorem}[proposition]{Théorème}
\newtheorem{corollaire}[proposition]{Corollaire}
\newtheorem{lemme}[proposition]{Lemme}
\newtheorem{definition}[proposition]{Définition}

\DeclareMathOperator{\vect}{Vect}

\newcommand{\vp}[2]{\left< #1 \mid #2 \right>}

\begin{document}
\renewcommand{\labelitemi}{$*$}
\begin{center}
{\Large \textbf{Chapitre 26: Espaces euclidiens}}
\end{center}
Dans tout le chapitre, le corps des scalaires est $\mathbb{R}$

\section{Généralités}
\subsection{Produit scalaire}
\begin{definition}
Soit $E$ un espace vectoriel (réel). \\
Un \uline{produit scalaire} sur $E$ est une application
\[ \vp{\cdot}{\cdot} : \begin{cases}
E^2 \to \mathbb{R} \\
(u, v) \to \vp{u}{v}
\end{cases}\]
\begin{itemize}
\item \uline{linéaire}
\item \uline{symétrique} (càd $\forall u, v \in E$, $\vp{u}{v} = \vp{v}{u}$)
\item et \uline{définie positive} (càd $\forall u \in E, \vp{u}{u} \geq 0$ et $\forall u \in E$, $\vp{u}{u} = 0 \implies u = 0_E$)
\end{itemize}
Un \uline{espace préhilbertien} (réel) est la donnée d'une ev E et d'un produit scalaire sur $E$ \\
Un \uline{espace euclidien} est un espace préhilbertien de dimension finie.
\end{definition}

\subsection{Norme euclidienne}
\begin{definition}
Soit $E$ un espace préhilbertien.
\begin{itemize}
\item La \uline{norme (euclidienne)} de $u \in E$ est $\lVert u \rVert = \sqrt{\vp{u}{v}}$
\item Le \uline{distance} de $u$ à $v \in E$ est $\text{d}(u, v) = \lVert v - u \rVert$
\end{itemize}
\end{definition}
\begin{theorem}[Inégalité de Cauchy-Schwarz]
Soit $E$ un espace préhilbertien et $u, v \in E$ \\
On a
\[ \vp{u}{v} \leq \left| \vp{u}{v} \right| \leq \lVert u \rVert \cdot \lVert v \rVert \]
"Le produit scalaire est inférieur au produit des normes"
\end{theorem}
\begin{theorem}
La norme $\lVert \cdot \rVert$ est une \uline{norme}, càd qu'on a: \\
\uline{Positivité}: $\forall u \in E$, $\lVert u \rVert \geq 0$ \\
\uline{Séparation}: $\forall u \in E$, $\lVert u \rVert = 0 \implies u = 0_E$ \\
\uline{Homogénéité}: $\forall u \in E$, $\forall \lambda \in \mathbb{R}$, $\lVert \lambda u \rVert = |\lambda| \cdot \lVert u \rVert$ \\
\uline{Inégalité triangulaire}: $\forall u, v \in E$, $\lVert u + v \rVert \leq \lVert u \rVert + \lVert v \rVert$
\end{theorem}

\section{Orthogonalité}
\subsection{Définition}
Dans toute cette section, $E$ est un espace préhilbertien.
\begin{definition}
\hfill
\begin{itemize}
\item Deux vecteurs $u, v \in E$ sont dits \uline{orthogonaux} ( et on note $u \perp v$ ) si $\vp{u}{v} = 0$
\item Un vecteur $u \in E$ est \uline{orthogonal} à une partie $X$ de $E$ ( et on note $u \perp X$ ) si $\forall v \in X$, $u \perp v$
\item Deux parties $X$ et $Y$ de $E$ sont \uline{orthogonales} ( et on note $X \perp Y$ ) si $\forall u \in X$, $\forall v \in Y$, $u \perp v$
\end{itemize}
\end{definition}
\begin{theorem}[Pythagore]
Soit $u, v \in E$ \\
Alors $u \perp v$ ssi $\lVert u + v \rVert^2 = \lVert u \rVert^2 + \lVert v \rVert^2$
\begin{definition}
Soit $X$ une partie de $E$ \\
On définit \uline{l'orthogonal de $X$}
\[ X^\perp = \left\{ u \in E \mid u \perp X \right\} = \left\{ u \in E \mid \forall v \in X,\, \vp{u}{v} = 0 \right\} \]
\end{definition}
\end{theorem}
\begin{proposition}
Soit $X$ une partie de $E$ \\
On a:
\begin{itemize}
\item $X^\perp$ est une sev de $E$
\item $X^\perp = \vect(X)^\perp$
\end{itemize}
\end{proposition}
\begin{theorem}[de représentation de Riesz]
Soit $E$ un espace \uline{euclidien} et $\varphi \in E^*$ \\
Alors il existe $u \in E$ tel que $\varphi: \begin{cases}
E \to \mathbb{R} \\
v \mapsto \vp{u}{v}
\end{cases}$
\end{theorem}

\subsection{Familles et bases orthonormées}
\begin{definition}
Soit $E$ un espace préhilbertien.
\begin{itemize}
\item Une famille $(x_i)_{i \in I}$ de vecteurs de $E$ est dite \uline{orthogonale} si $\forall i \neq j \in I$, $\vp{x_i}{x_j} = 0$
\item La famille $(x_i)_{i \in I}$ est dite \uline{orthonormée} (ou \uline{orthonormale}) si les vecteurs sont en outre de norme $1$, càd $\forall i, j \in I$, $\vp{x_i}{x_j} = \delta_{ij}$
\end{itemize}
\end{definition}
\begin{proposition}
Toute famille orthogonale de vecteurs non nuls (en particulier, toute famille orthonormée) \\
est libre.
\end{proposition}
\begin{definition}
Une \uline{base orthogonale} (resp. \uline{orthonormée}) (BON) d'un espace préhilbertien $E$ est une base de $E$ qui est également une famille orthogonale (resp. orthonormée).
\end{definition}
\begin{theorem}
Tout espace euclidien $E$ possède une base orthonormée.
\end{theorem}
\noindent \uline{Remarque}: On utilise l'algorithme d'orthonormalisation: \\
Pour $k \in \llbracket 1, n \rrbracket$ on remplace $v_k$ par 
\[ \frac{v_k - \sum\limits_{j = 1}^{k - 1} \vp{v_k}{e_j} e_j}{\lVert v_k - \sum\limits_{j = 1}^{k - 1} \vp{v_k}{e_j} e_j \rVert} \]
\begin{corollaire}[Théorème de la base orthonormée incomplète]
\hfill \\
Soit $E$ un espace euclidien et $(e_1, ...\,, e_r)$ une famille orthonormée. \\
Alors il existe $(e_{r + 1}, ...\,, e_n)$ telle que $(e_1, ...\,, e_n)$ soit une base orthonormée de $E$
\end{corollaire}
\begin{proposition}
Soit $E$ un espace euclidien et $(e_1, ...\,, e_n)$ une BON de $E$ \\
Alors, pour tous $x, y \in E$ on a:
\begin{itemize}
\item $x = \sum\limits_{i = 1}^n \vp{x}{e_i} e_i$
\item $\vp{x}{y} = \sum\limits_{i = 1}^n \vp{x}{e_i} \vp{y}{e_i}$
\item $\lVert x \rVert^2 = \sum\limits_{i = 1}^n \vp{x}{e_i}^2$
\end{itemize}
Autrement dit, dans une BON, tous les calculs se font comme dans $\mathbb{R}^n$ muni du produit scalaire canonique. \\
Plus conceptuellement, tout espace euclidien de dimension $n$ est isomorphe (en tant qu'espace euclidien) à $\mathbb{R}^n$
\end{proposition}

\section{Projection orthogonale}
Dans toute la section, $E$ est un espace préhilbertien et $F$ un sev \uline{de dimension finie} de $E$
\subsection{Définition}
\begin{proposition}
Avec ces notations ($F$ de dimension finie!) on a:
\begin{itemize}
\item $E = F \oplus F^\perp$
\item $\left(F^\perp\right)^\perp = F$
\end{itemize}
\end{proposition}
\begin{definition}
On note $p_F$ et on appelle \uline{projection orthogonale sur $F$} le projecteur sur $F$ parallèlement à $F^\perp$
\end{definition}
\begin{proposition}
Si $F$ possède une base orthonormée $(e_1, ...\,, e_r)$, on a 
\[\forall x \in E ,\, p_f(x) = \sum\limits_{i = 1}^r \vp{x}{e_i} e_i \]
\end{proposition}
\begin{proposition}
Soit $x \in E$
\begin{itemize}
\item Le projeté $p_F(x)$ est l'unique vecteur de $F$ tel que $\forall y \in F$, $\vp{p_F(x)}{y} = \vp{x}{y}$
\item Si $F$ possède une base (pas nécessairement ON) $(v_1, ...\, v_r)$, cette condition équivaut à \\
$\forall j \in \llbracket 1, n \rrbracket$, $\vp{p_F(x)}{e_j} = \vp{x}{e_j}$
\end{itemize}
\end{proposition}
\begin{proposition}[Inégalité de Bessel]
On a $\forall x \in E$, $\lVert p_F(x) \rVert \leq \lVert x \rVert$
\end{proposition}

\subsection{Distance à un sev de dimension finie}
\begin{proposition}
Soit $E$ un espace préhilbertien et $F$ un sev de dimension finie de $E$. Soit $x \in E$ \\
On a $\forall y \in F$, $\lVert x - y \rVert \geq \lVert x - p_F(x) \rVert$ avec égalité ssi $y = p_F(x)$
\end{proposition}
\begin{definition}
Avec les mêmes notations, $\lVert x - p_F(x) \rVert$ est la \uline{distance de $x$ à $F$}, notée $d(x, F)$
\end{definition}

\subsection{Cas d'un hyperplan}
Dans cette section, $E$ est un espace euclidien et $F$ est in hyperplan de $E$. On fixe un vecteur normal $n$ de $F$ (càd $F$ = $\vect(n)^\perp$)
\begin{proposition}
On a:
\[ p_F(x) = x - \frac{\vp{x}{n}}{\lVert n \rVert^2} n \text{\quad \quad et \quad \quad} d(x, F) = \frac{\left| \vp{x}{n} \right|}{\lVert n \rVert} \]
\end{proposition}
\end{document}