\documentclass[10pt,a4paper]{article}
\usepackage[utf8]{inputenc}
\usepackage[french]{babel}
\usepackage[T1]{fontenc}
\usepackage{amsmath}
\usepackage{amsfonts}
\usepackage{amssymb}
\usepackage{graphicx}
\usepackage[left=2cm,right=2cm,top=2cm,bottom=2cm]{geometry}
\usepackage{setspace}
\usepackage{ulem}
\usepackage{stmaryrd}
\usepackage{amsthm}
\usepackage{dsfont}
\usepackage{mathpazo}

\onehalfspacing{}
\theoremstyle{definition}
\newtheorem{proposition}{Proposition}[section]
\newtheorem{theorem}[proposition]{Théorème}
\newtheorem{corollary}[proposition]{Corollaire}
\newtheorem{lemme}[proposition]{Lemme}
\newtheorem{definition}[proposition]{Définition}

\begin{document}
\renewcommand{\labelitemi}{\textbullet}

\renewcommand{\labelenumi}{(\roman{enumi})}

\begin{center}
{\Large \textbf{Chapitre 16. Intégrales à paramètre}}
\end{center}

\noindent \(I, J\) intervalles d'intérieur non vide, \(E\) un evn.

\renewcommand{\labelenumi}{\arabic{enumi}.}

\section{Le théorème de convergence dominée}
\subsection{Le théorème de convergence dominée}
\begin{theorem}[ Théorème de convergence monotone ]
    Soit \(g_n: I \to [0, +\infty]\) mesurables et \(g_n: I \to [0, +\infty]\) \\
    On suppose que:
    \begin{enumerate}
        \item \((g_n)_{n \in \mathbb{N}}\) converge simplement vers \(g\)
        \item \(\forall n \in \mathbb{N}\) \, \(0 \leq g_n \leq g_{n + 1}\)
    \end{enumerate}
    Alors dans \([0, +\infty]\)
    \[\lim_{n \to +\infty}\int_{I} g_n = \int_{I} g \in [0, +\infty]\]
\end{theorem}
\begin{corollary}
    Soit \(F_0 : I \to \mathbb{R}_+\) intégrable et \(F_n: I \to \mathbb{R}_+\) \((n \geq 1)\) mesurable. \\
    On suppose:
    \begin{enumerate}
        \item \(\forall n \in \mathbb{N} \, 0 \leq F_{n + 1} \leq F_n\)
        \item \((F_n)\) converge simplement vers 0
    \end{enumerate}
    Alors
    \[\int_{I} F_n \xrightarrow[n \to +\infty]{} 0\]
\end{corollary}

\subsection{Énoncé du théorème de convergence dominée}
\begin{theorem}[ Théorème de convergence dominée ]
    Soit \(f_n: I \to \mathbb{K}\) continue par morceaux \((n \in \mathbb{N})\) \\
    On suppose:
    \begin{enumerate}
        \item \((f_n)_{n \geq 0}\) converge simplement vers \(f: I \to \mathbb{K}\) continue par morceaux sur \(I\)
        \item Il existe \(\varphi: I \to \mathbb{R}_+\) \uline{intégrable} telle que \(\forall n \in \mathbb{N}, \forall t \in I \quad |f_n(x)| \leq \varphi(t)\) ( Hypothèse de domination )
    \end{enumerate}
    Alors les \(f_n\) sont intégrables et \(f\) aussi et
    \[\boxed{\int_{I}f_n \xrightarrow[n \to +\infty]{} \int_{I}f}\]
    On a même
    \[\lVert f_n - f \rVert_1 = \int_{I}|f_n - f| \xrightarrow[n \to +\infty]{} 0\]
\end{theorem}

\subsection{Premiers exemples d'application}
\noindent Quelques exercices classiques:
\begin{enumerate}
    \item Montrer que \[I_n = \int_{0}^{\frac{\pi}{2}} \sin^{n}t dt \xrightarrow[n \to +\infty]{} 0\]
    \item Soit \(f: [0, 1] \to \mathbb{K}\) continue. \\ Montrer que \[I_n = \int_{0}^{1} f\left(\frac{x}{n}\right) dx \xrightarrow[n \to +\infty]{} f(0)\]
    \item Soit \(f: [0, 1] \to \mathbb{K}\) continue avec \(\lim\limits_{+\infty}f = 0\). \\ Montrer que \[I_n = \int_{0}^{1} f(nx) dx \xrightarrow[n \to +\infty]{} f(0)\]
    \item Soit \(n \geq 1\) et \[I_n = \int_{1}^{+\infty}e^{-x^{n}} dx\] Montrer que \(I_n \xrightarrow[n \to +\infty]{} 0\) et \(I_n \underset{+\infty}{\sim} \frac{\alpha}{n}\) avec \(\alpha\) à exprimer avec une intégrale.
    \item Montrer que \[I_n = \int_{0}^{n}\left(1 - \frac{x^2}{n^2}\right)^{n^2} dx \xrightarrow[n \to +\infty]{} \int_{0}^{+\infty}e^{-x^2} dx\]
\end{enumerate}

\subsection{Théorème de convergence dominée appliquée à l'interversion série / suite}
\begin{corollary}[ Théorème de convergence dominée ]
    Soit \(f_n: I \to \mathbb{K}\) continues par morceaux \((n \in \mathbb{N})\) \\
    On suppose que:
    \begin{enumerate}
        \item \(\sum f_n\) converge simplement vers \(\sum\limits_{n = 0}^{+\infty} f_n\) \(\mathcal{C}^0\) par morceaux.
        \item Il existe \(\varphi: I \to \mathbb{R}_+\) intégrable telle que \(\forall n \in \mathbb{N}\, \left| \sum\limits_{k = 0}^{n} f_k \right| \leq \varphi\) ( Domination )
    \end{enumerate}
    Alors les \(f_n\) sont intégrables, \(\sum f_n\) est intégrable et
    \[\boxed{\int_{I} \sum_{n = 0}^{+\infty} f_n = \sum_{n = 0}^{+\infty}\int_{I} f_n}\]
\end{corollary}

\subsection{Le théorème d'intégration terme à terme}
\end{document}