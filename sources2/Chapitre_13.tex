\documentclass[10pt,a4paper]{article}
\usepackage[utf8]{inputenc}
\usepackage[french]{babel}
\usepackage[T1]{fontenc}
\usepackage{amsmath}
\usepackage{amsfonts}
\usepackage{amssymb}
\usepackage{graphicx}
\usepackage[left=2cm,right=2cm,top=2cm,bottom=2cm]{geometry}
\usepackage{setspace}
\usepackage{ulem}
\usepackage{stmaryrd}
\usepackage{amsthm}
\usepackage{dsfont}
\usepackage{mathpazo}
\usepackage{empheq}

\onehalfspacing

\theoremstyle{definition}
\newtheorem{proposition}{Proposition}[section]
\newtheorem{theorem}[proposition]{Théorème}
\newtheorem{corollary}[proposition]{Corollaire}
\newtheorem{lemma}[proposition]{Lemme}
\newtheorem{definition}[proposition]{Définition}

\begin{document}
\renewcommand{\labelitemi}{\textbullet}
\begin{center}
{\Large \textbf{Chapitre 13. Équations différentielles linéaires (1\textsuperscript{ère} partie)}}
\end{center}

\section{Équations scalaires d'ordre 1}
\noindent $\mathbb{K} = \mathbb{R}$ ou $\mathbb{C}$ et $I$ intervalle tel que $\overset{\circ}{I} \neq \emptyset$

\subsection{Généralités}
\begin{definition}
    Une équation différentielle linéaire scalaire d'ordre 1 est une équation du type
    \[(E) \quad a(x)y' + b(x)y = c(x)\]
    avec \(a, b, c: I \to \mathbb{K}\) continues, données et \(y: I \to \mathbb{K}\) dérivable, inconnue. \\
    On dit que \((E)\) est régulière si \(a\) ne s'annule pas: \((E)\) peut être mis sous la forme
    \[(E) \quad y' + \varphi(x)y = \psi(x)\]
\end{definition}

\subsection{Solutions des équation régulières}
\begin{theorem}
    Soit \(\varphi, \psi : I \to \mathbb{K}\) continues et
    \begin{align*}
        (E)& \quad y' + \varphi y = \psi \\
        (E_0)& \quad y' + \varphi y = 0
    \end{align*}
    Soit \(x_0 \in I\) \\
    Les solutions de \((E_0) \quad y' + \varphi y = 0\) sont les fonctions
    \[\boxed{x \mapsto C\exp\left(-\int_{x_0}^{x} \varphi(t) \,dt \right)}\]
    où \(C\) est une constante arbitraire dans \(\mathbb{K}\)
\end{theorem}
\begin{theorem}[ Méthode de variation de la constante ]
    \hfill \\
    Soit \[(E) \quad y' + \varphi y = \psi\]
    avec \(\varphi, \psi: I \to \mathbb{K}\) continues. \\
    Soit \(Y\) une solution non nulle de \((E_0) \quad y' + \varphi y = 0\) \\
    Les solutions non nulles de \((E)\) sont les fonctions \(x \mapsto \lambda(x) Y(x)\) avec \(\lambda: I \to \mathbb{K}\) \(\mathcal{C}^1\) vérifiant \(\forall x \in I\)
    \[\boxed{\lambda'(x) Y(x) = \psi(x)}\]
    ie.
    \[\boxed{\lambda' Y = \psi}\]
\end{theorem}
\begin{theorem}[ Théorème de Cauchy-Lipschitz ]
    \hfill \\
    Soit \[(E) \quad y' + \varphi y = \psi\]
    avec \(\varphi, \psi \in \mathcal{C}(I, \mathbb{K})\) \\
    L'espace des solutions de \((E)\) est une droite affine de \(\mathcal{C}^1(I, \mathbb{K})\) \\
    De plus, si on se donne \(x_0 \in I,\, y_0 \in \mathbb{K}\) alors il existe une unique solution \(y\) de \((E)\) \\
    vérifiant \(y(x_0) = y_0\) ( Condition fe Cauchy )
\end{theorem}

\subsection{Exemples d'équations non régulières}
\noindent Pour les équations non régulières on procède par analyse-synthèse. \medskip

\noindent \uline{Exercice}: Résoudre sur \(\mathbb{R}\) \[(E) \quad x^2 y' + y = 1\]

\section{Équations scalaires d'ordre 2}
\begin{definition}
    Une équation différentielle linéaire scalaire d'ordre 2 est une équation du type
    \[(E) \quad a(x)y'' + b(x)y' + c(x)y = d(x)\]
    avec \(a, b, c, d: I \to \mathbb{K}\) continues, données et \(y: I \to \mathbb{K}\) dérivable 2 fois, inconnue. \\
    On dit que \((E)\) est régulière si \(a\) ne s'annule pas: \((E)\) peut être mis sous la forme
    \[(E) \quad y'' + \varphi(x)y' + \psi(x)y = \theta(x)\]
\end{definition}

\subsection{Cas des équations régulières}
\begin{theorem}[ Théorème de Cauchy-Lipschitz ]
    Soit \(\varphi, \psi, \theta: I \to \mathbb{K}\) continues, \(x_0 \in I\) et \(y_0, y_1 \in \mathbb{K}\) \\
    Soit \[(E) \quad y'' + \varphi(x)y' + \psi(x)y = \theta(x)\]
    avec \(x \in I\) \\
    Alors il existe une uniques solution \(y\) de \((E)\) vérifiant \(y(x_0) = y_0,\, y'(x_0) = y_1\) ( conditions de Cauchy )
\end{theorem}
\begin{corollary}[ Théorème de Cauchy-Lipschitz ]
    Soit \(\varphi, \psi, \theta: I \to \mathbb{K}\) continues. \\
    Soit \begin{align*}
        (E)& \quad y'' + \varphi y' + \psi y = \theta \\
        (E_0)& \quad y'' + \varphi y' + \psi y = 0
    \end{align*}
    Alors les solutions de \((E_0)\) constituent un sec de dimension 2 \(S_0\) de \(\mathcal{C}^2(I, \mathbb{R})\) \\
    Les solutions de l'équation complète constituent un sous-espace affine de dimension 2 de \(\mathcal{C}^2(I, \mathbb{K})\) dont la direction est \(S_0\)
\end{corollary}

\subsection{Méthode de variations de la constante}
\begin{definition}
    Soit 
    \[(E_0) \quad y'' + \varphi y' + \psi y = 0\]
    avec \(\varphi, \psi: I \to \mathbb{K}\) continues. \\
    Soit \(y, z\) deux solutions de \((E_0)\)
    \begin{enumerate}
        \item Si \((y, z)\) est libre, on dit que \((y, z)\) est un couple de solutions indépendantes \\
        ( c'est alors la base de \(S_0\) ens. des solutions )
        \item On définit le wronskien de \((y, z)\) par
        \[W_{(y, z)}: \begin{cases}
            I \to \mathbb{K} \\
            x \mapsto \begin{vmatrix}
                y(x) & z(x) \\
                y'(x) & z'(x)
            \end{vmatrix}
        \end{cases}\]
    \end{enumerate}
\end{definition}
\begin{theorem}
    Soit 
    \[(E_0) \quad y'' + \varphi y + \psi y = 0\]
    avec \(\varphi, \psi: I \to \mathbb{K}\) continues. \\
    Soit \(y, z\) deux solutions de \((E_0)\) \\
    Si \((y, z)\) sont indépendantes alors \(\forall x \in I\), \(W_{(y, z)}(x) \neq 0\) \\
    S'il existe \(x_0 \in I\) avec \(W_{(y, z)}(x_0) \neq 0\) alors \((y, z)\) sont indépendantes ( et le wronskien ne s'annule jamais ). \\
    En particulier un wronskien est identiquement nul ou il ne s'annule jamais.
\end{theorem}
\begin{theorem}[ Méthode de variations des constantes ]
    Soit \(\varphi, \psi, \theta: I \to \mathbb{K}\) \(\mathcal{C}^0\) et
    \begin{align*}
        (E)& \quad y'' + \varphi y' + \psi y = \theta \\
        (E_0)& \quad y'' + \varphi y + \psi y = 0
    \end{align*}
    Soit \((y, z)\) un système fondamental de solutions de \((E_0)\) \\
    Alors les solutions de \((E)\) sont les fonctions
    \[\boxed{\lambda y + \mu z}\] 
    avec \(\lambda, \mu: I \to \mathbb{K}\) \(\mathcal{C}^1\) vérifiant
    \[\boxed{\begin{pmatrix}
        y & z \\
        y' & z'
    \end{pmatrix} \begin{pmatrix}
        \lambda ' \\
        \mu'
    \end{pmatrix} = \begin{pmatrix}
        0 \\
        \theta
    \end{pmatrix}}\]
\end{theorem}

\subsection{Étude qualitative de solutions d'équation d'ordre 2}
\begin{lemma}[ Lemme de Gromwall (HP) ]
    Soit \(a \in \mathbb{R} \), \(C \in \mathbb{R}_+\), \(u, v: \left[ a, +\infty \right[ \to \mathbb{R}_+\) continues \\
    On suppose que \(\forall x \geq a\)
    \[u(x) \leq C + \int_{a}^{x} u(t) v(t) \, dt\]
    Alors
    \[\boxed{u(x) \leq C \exp \int_{a}^{x} v(t) \, dt}\]
\end{lemma}

\noindent \uline{Exercice}: Soit \(y\) une solution sur \(\mathbb{R}_+\) de
\[(E) \quad y'' + xy = 0\]
Montrer que \(y\) est bornée.

\subsection{Exemples d'équations non régulières}
\noindent Pour les équations non régulières on procède par analyse-synthèse. \medskip

\noindent \uline{Exercice}: Résoudre sur \(\mathbb{R}\)
\[(E) \quad (x + 1)y'' + (x - 1)y' - 2y = 0\]

\pagebreak

\section{Exercices classiques}
\subsection{Caractère isolé des zéros d'un système d'équations d'ordre 2 - Entrelacement}
\noindent Soit 
\[(E) \quad y'' + q(x)y = 0\]
avec \(q: I \to \mathbb{R}\) continue.
\begin{enumerate}
    \item Que peut-on dire d'une solution \(y\) de \(E\) telle que \(y(x_0) = y'(x_0) = 0\)? \\
    Soit \(y\) une solution non identiquement nulle.
    \item Montrer que les zéros de \(y\) sont isolés ie. sur un voisinage de \(x_0\) avec \(y(x_0) = 0\) \(y\) ne s'annule qu'en \(x_0\) \\ 
    En déduire que sur un segment \(y\) ne possède qu'un nombre fini de zéros. \\
    Montrer qu'il y a une quantité au plus dénombrable de zéros.
    \item On suppose que \(y\) possède deux zéros \(\alpha < \beta\) dans \(I\) \\
    Soit \(z\) solution tel que \((y, z)\) sont indépendantes. Montrer que \(z\) s'annule entre \(\alpha\) et \(\beta\)
\end{enumerate}

\subsection{Exercice type: Le changement de variables}
\noindent Pour une équation du second ordre on peut tenter un changement de variables pour se ramener à une équation à coefficients constantes. \\
Exemple:
\[(1 + x^2)y'' + xy' + k^2y = 0\]
avec \(k > 0\)
( indication: faire un changement de variable \(x = \sinh(t)\) )
\end{document}