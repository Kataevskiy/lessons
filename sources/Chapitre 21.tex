\documentclass[10pt,a4paper]{article}
\usepackage[utf8]{inputenc}
\usepackage[french]{babel}
\usepackage[T1]{fontenc}
\usepackage{amsmath}
\usepackage{amsfonts}
\usepackage{amssymb}
\usepackage{graphicx}
\usepackage[left=2cm,right=2cm,top=2cm,bottom=2cm]{geometry}
\usepackage{setspace}
\usepackage{ulem}
\usepackage{stmaryrd}
\usepackage{amsthm}
\usepackage{dsfont}
\usepackage{mathpazo}

\onehalfspacing

\theoremstyle{definition}
\newtheorem{proposition}{Proposition}[section]
\newtheorem{theorem}[proposition]{Théorème}
\newtheorem{corollaire}[proposition]{Corollaire}
\newtheorem{lemme}[proposition]{Lemme}
\newtheorem{definition}[proposition]{Définition}

\begin{document}
\renewcommand{\labelitemi}{$*$}
\begin{center}
{\Large \textbf{Chapitre 21: Intégration}}
\end{center}
Dans tout le chapitre: $I$ désigne un intervalle de $\mathbb{R}$

\section{Fonctions en escalier, fonctions continues par morceau}
\subsection{Subdivisions d'un segment}
\begin{definition}
\hfill
\begin{itemize}
\item Une \uline{subdivision} du segment $[a, b]$ est une famille $\sigma = (x_{0}, ... , x_{n})$, où \\ $a = x_{0} < x_{1} < ... < x_{n} = b$
\item les $x_{i}$ sont les \uline{points} de la subdivision
\item les intervalles $[x_{i}, x_{i+1}]$ (resp. $]x_{i}, x_{i+1}[$) sont les \uline{composantes fermées} (resp. \uline{ouverte}) de $\sigma$
\item le \uline{pas} de la subdivision $\sigma$ est $\max{\{x_{i+1} - x_{i} \mid i \in \llbracket 0, n-1 \rrbracket\}}$
\end{itemize}
\end{definition}
\begin{definition}
Soit $\sigma$, $\sigma'$ deux subdivisions d'un segment $[a, b]$. \\
On dit que $\sigma'$ \uline{raffine} $\sigma$ (ou: est plus fine que $\sigma$) si toute composante (ouverte) de $\sigma'$ est incluse dans une composante (ouverte) de $\sigma$.
\end{definition}
\begin{proposition}
Deux subdivisions $\sigma_{1}$, $\sigma_{2}$ de $[a, b]$ possèdent toujours un raffinement commun.
\end{proposition}

\subsection{Fonctions en escalier}
\begin{definition}
\hfill
\begin{itemize}
\item Une fonction $\varphi: [a, b] \to \mathbb{R}$ sera dite \uline{en escalier} s'il existe une subdivision $\sigma$ de $[a, b]$ telle que $\varphi$ soit constante sur chaque composante de $\sigma$
\item On dit alors que $\sigma$ est \uline{adaptée} à $\varphi$
\end{itemize}
\end{definition}
\begin{proposition}
L'ensemble $\mathcal{E}([a, b])$ des fonctions en escalier sur $[a, b]$ est une sous-algèbre de $\mathbb{R}^{[a, b]}$ \\
et, $\forall f \in \mathcal{E}([a, b])$, $|f| \in \mathcal{E}([a, b])$
\end{proposition}

\subsection{Fonctions continues par morceaux}
\begin{definition}
Une fonction $f: [a,b] \to \mathbb{R}$ est \uline{continue par morceaux} s'il existe une subdivision $\sigma = (x_{0}, ... ,x_{n})$ de $[a, b]$ telle que:
\begin{itemize}
\item la restriction $f_{|]x_{i}, x_{i+1}[}$ de $f$ à chaque composante ouverte est continue
\item $f$ admet des limites à gauche (resp. à droite) en tout point de la subdivision, sauf $a = x_{0}$ (resp. $b = x_{0}$)
\end{itemize}
\end{definition}
\begin{lemme}
L'ensemble $C^{\circ}_{pm}([a, b])$ des fonctions continues par morceaux est la somme $C^{\circ}([a, b]) + \mathcal{E}([a, b])$.
\end{lemme}
\begin{corollaire}
Toute fonction continue par morceaux est bornée.
\end{corollaire}

\section{Convergence uniforme}
\subsection{Convergence simple}

\begin{definition}
Soit $(f_n)_{n \in \mathbb{N}}$ une suite de fonctions $I \rightarrow \mathbb{R}$. \\
On dit que $(f_n)_{n \in \mathbb{N}}$ \uline{converge simplement} vers $f: I \to \mathbb{R}$ si $\forall x \in I$, $f_n(x) \xrightarrow[n \to +\infty]{} f(x)$ \\
On notera $f_n \xrightarrow[n \to +\infty]{CS} f$
\end{definition}
\begin{definition}
Si $f: I \to \mathbb{R}$ est bornée, on définit sa \uline{norme uniforme}: $ \lVert f \rVert_\infty$ = $\sup{\{ |f(t)| \mid t \in I \}}$
\end{definition}
\begin{proposition}
La norme uniforme $\lVert . \rVert_\infty$ est une \uline{norme} sur l'espace vectoriel $L^\infty(I)$ des fonctions $I \to \mathbb{R}$ bornées:
\begin{itemize}
\item Positivité: $\forall f \in L^\infty(I)$, $\lVert f \rVert_\infty \geq 0$
\item Séparation: $\forall f \in L^\infty(I)$, $\lVert f \rVert_\infty = 0 \implies f = 0$
\item Homogénéité: $\forall f \in L^\infty(I)$, $\forall \lambda \in \mathbb{R}$, $\lVert \lambda f \rVert_\infty = |\lambda| \cdot \lVert f \rVert_\infty$
\item Inégalité triangulaire: $\forall f, g \in L^\infty(I)$, $\lVert f + g \rVert_\infty \leq \lVert f \rVert_\infty + \lVert g \rVert_\infty$
\end{itemize}
\end{proposition}
\begin{definition}
Soit $(f_n)_{n \in \mathbb{N}}$ une suite de fonctions $I \to \mathbb{R}$ bornées et $f: I \to \mathbb{R}$ bornée. \\
On dit que $(f_n)_{n \in \mathbb{N}}$ \uline{converge uniformément} vers $f$ si $\lVert f_n - f \rVert_\infty \xrightarrow[n \to +\infty]{} 0$ \\
On note alors $f_n \xrightarrow[n \to +\infty]{CU} f$
\end{definition}
\begin{proposition}
Soit $(\varphi_n)_{n \in \mathbb{N}}$ et $(\psi_n)_{n \in \mathbb{N}}$  deux suites de fonctions bornées sur $I$ et $f, g: I \to \mathbb{R}$ bornées telles que
\[\begin{cases}\varphi_n \xrightarrow[n \to +\infty]{CU} f \\
\psi_n \xrightarrow[n \to +\infty]{CU} g
\end{cases}\]
Alors:
\begin{itemize}
\item $\forall \lambda \in \mathbb{R}$, $\varphi_n + \lambda \psi_n \xrightarrow[n \to +\infty]{CU} f + \lambda g$
\item $|\varphi_n| \xrightarrow[n \to +\infty]{CU} f$ 
\end{itemize} 
\end{proposition}
\begin{theorem}
Soit $(\varphi_n)_{n \in \mathbb{N}}$ une suite de fonctions bornées et $f: I \to \mathbb{R}$ bornée telle que $\varphi_n \xrightarrow[n \to +\infty]{CU} f$. \\
Alors, si pour tout $n \in \mathbb{N}$, $\varphi_n$ est continue, $f$ l'est aussi. 
\end{theorem}

\subsection{Approximation uniforme}
\begin{theorem}
Soit $f \in C^{\circ}_{pm}([a, b])$. \\
Alors il existe une suite $(\varphi_n)_{n \in \mathbb{N}}$ de fonctions en escalier telle que $\varphi_n \xrightarrow[n \to +\infty]{CU} f$
\end{theorem}

\section{Définition de l'intégrale}
\subsection{Intégrale des fonctions en escalier}
\begin{definition}
Soit $\varphi \in \mathcal{E}([a, b])$ et $\sigma = (a = x_0, ... , x_n = b)$ une subdivision adaptée à $\varphi$. \\
On peut donc écrire 
\[\varphi = \sum\limits_{i=0}^{n} \lambda_i \mathds{1}_{x_i} + \sum\limits_{j=0}^{n-1} \mu_j \mathds{1}_{]x_j, x_{j+1}[}\]
On définit alors \uline{l'intégrale de $\varphi$}: 
\[\int\limits_a^b \varphi = \sum\limits_{j=0}^{n-1} \mu_j (x_{j+1} - x_j)\]
\end{definition}
\begin{proposition}
\hfill
\begin{itemize}
\item Cette intégrale est bien définie.
\item L'intégrale est une forme linéaire $\int\limits_a^b : \mathcal{E}([a, b]) \to \mathbb{R}$
\item (Inégalité triangulaire \textnormal{\&} contrôle uniforme): $ \forall f \in \mathcal{E}([a, b])$, $|\int\limits_a^b \varphi \,| \leq \int\limits_a^b |\varphi| \leq (b - a)\lVert \varphi \rVert_\infty$
\item Relation de Chasles: si $a < b < c$, on a $\forall \varphi \in \mathcal{E}([a, b])$, $\int\limits_a^c \varphi = \int\limits_a^b \varphi + \int\limits_b^c \varphi$
\end{itemize}
\end{proposition}

\subsection{Lemme fondamental et définition}
\begin{theorem}
Soit $f \in C_{pm}^{\circ}([a, b])$ et $(\varphi_n)_{n \in \mathbb{N}}$ une suite d'éléments de $\mathcal{E}([a, b])$ \\
convergent uniformément vers $f$. \\
Alors:
\begin{itemize}
\item La suite $(\int\limits_a^b \varphi_n)_{n \in \mathbb{N}}$ converge.
\item Si $(\psi)_{n \in \mathbb{N}} \in \mathcal{E}([a, b])^{\mathbb{N}}$ vérifie également $\psi_n \xrightarrow[n \to +\infty]{CU} f$, alors $\lim\limits_{n \to +\infty} \int\limits_a^b \varphi_n = \lim\limits_{n \to +\infty} \int\limits_a^b \varphi_n$
\end{itemize}
\end{theorem}
\begin{definition}
Soit $f \in C_{pm}^{\circ}([a, b])$ \\
On définit \uline{l'intégrale de $f$}: 
$\int\limits_a^b f = \int\limits_a^b f(t)dt$ comme la limite $\lim\limits_{n \to +\infty} \int\limits_a^b \varphi_n$ où $(\varphi_n)_{n \in \mathbb{N}}$ est une suite de fonctions en escalier convergeant uniformément vers f.
\end{definition}

\subsection{Propriétés de base}
\begin{theorem}
\hfill
\begin{itemize}
\item L'intégrale est une forme linéaire $\int\limits_a^b : C_{pm}^{\circ}([a, b]) \to \mathbb{R}$
\item Inégalité triangulaire et contrôle uniforme: $\forall f \in C_{pm}^{\circ}([a, b])$, $|\int\limits_a^b f \,| \leq \int\limits_a^b |f| \leq (b-a) \lVert f \rVert_\infty$
\item si $f, g \in C_{pm}^{\circ}([a, c])$ et que $f$ et $g$ coïncident sur le complémentaire d'un ensemble fini, alors $\int\limits_a^b f = \int\limits_a^b g$
\item \uline{Positivité}: Soit $f \in C_{pm}^{\circ}([a, b])$ positive. Alors $\int\limits_a^b f \geq 0$
\item \uline{Croissance}: Soit $f, g \in C_{pm}^{\circ}([a, b])$ telles que $f \leq g$. Alors $\int\limits_a^b f \leq \int\limits_a^b g$
\end{itemize}
\end{theorem}
\end{document}