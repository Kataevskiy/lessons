\documentclass[10pt,a4paper]{article}
\usepackage[utf8]{inputenc}
\usepackage[french]{babel}
\usepackage[T1]{fontenc}
\usepackage{amsmath}
\usepackage{amsfonts}
\usepackage{amssymb}
\usepackage{graphicx}
\usepackage[left=2cm,right=2cm,top=2cm,bottom=2cm]{geometry}
\usepackage{setspace}
\usepackage{ulem}
\usepackage{stmaryrd}
\usepackage{amsthm}
\usepackage{dsfont}
\usepackage{mathpazo}

\onehalfspacing

\theoremstyle{definition}
\newtheorem{proposition}{Proposition}[section]
\newtheorem{theorem}[proposition]{Théorème}
\newtheorem{corollaire}[proposition]{Corollaire}
\newtheorem{lemme}[proposition]{Lemme}
\newtheorem{definition}[proposition]{Définition}

\DeclareMathOperator{\re}{Re}
\DeclareMathOperator{\im}{Im}

\usepackage{array}
\newcolumntype{M}[1]{>{\centering\arraybackslash}m{#1}}

\begin{document}
\renewcommand{\labelitemi}{$*$}
\renewcommand{\labelenumi}{(\roman{enumi})}
\begin{center}
{\Large \textbf{Chapitre 12. Suites réelles et complexes}}
\end{center}

\section{Convergence}
\subsection{Définition}
\begin{definition}
Soit $u \in \mathbb{R}^\mathbb{N}$ une suite réelle.
\begin{itemize}
\item Soit $l \in \mathbb{R}$. On dit que $u$ \uline{converge} (ou \uline{tend}) vers $l$ si $\forall \varepsilon > 0$, $\exists N \in \mathbb{N}: \forall n \geq N$, $\left| u_n - l \right| \leq \varepsilon$  \\
Dans ce cas, on note $u_n \xrightarrow[n \to +\infty]{} l$ ou $u \to l$ ou $\lim\limits_{n \to +\infty} u_n = l$
\item On dit que $u$ \uline{diverge} si elle ne converge vers aucun $l \in \mathbb{R}$
\end{itemize}
\end{definition}

\subsection{Premières propriétés}
\begin{proposition}[Unicité de la limite]
Soit $u \in \mathbb{R}^\mathbb{N}$. Soit $l, l' \in \mathbb{R}$ tels que $\begin{cases}
u_n \xrightarrow[n \to +\infty]{} l \\ u_n \xrightarrow[n \to +\infty]{} l'
\end{cases}$ \\
Alors $l = l'$
\end{proposition}
\begin{proposition}
Soit $u \in \mathbb{R}^\mathbb{N}$ et $l \in \mathbb{R}$ \\
On a $u_n \xrightarrow[n \to +\infty]{} l \iff |u_n - l| \xrightarrow[n \to \infty]{} 0$
\end{proposition}
\begin{proposition}
Toute suite convergente est bornée.
\end{proposition}
\begin{lemme}
Toute suite bornée \uline{à partir d'un certain rang} (àpcr) est bornée.
\end{lemme}
\begin{proposition}[Caractère asymptotique de la limite]
\hfill \\
La convergence d'une suite ne dépend pas de ses premiers termes. \\
Plus précisément, soit $u, v \in \mathbb{R}^\mathbb{N}$ égales àpcr. \\
Alors $u$ converge si et seulement si $v$ converge. Si c'est le cas, $\lim\limits_{n \to +\infty} u_n = \lim\limits_{n \to +\infty} v_n$
\end{proposition}

\subsection{Limites et inégalités}
\begin{theorem}[Passage à la limite dans les inégalités larges]
\hfill \\
Soit $u, v \in \mathbb{R}^\mathbb{N}$ et $l, l' \in \mathbb{R}$ tels que $u_n \xrightarrow[n \to +\infty]{} l$ et $v_n \xrightarrow[n \to +\infty]{} l'$. On suppose $\forall n \in \mathbb{N}$, $u_n \leq v_n$ \\
Alors $l \leq l'$
\end{theorem}
\begin{theorem}[$\mathbb{R}_+^*$ est ouvert]
Soit $u \in \mathbb{R}^\mathbb{N}$ telle que $u_n \xrightarrow[n \to +\infty]{} l > 0$ \\
Alors $u$ est strictement positive àpcr, càd $\exists N \in \mathbb{N}: \forall n \geq N$, $u_n > 0$
\end{theorem}

\subsection{Limite infinie}
\begin{definition}
Soit $u \in \mathbb{R}^\mathbb{N}$
\begin{itemize}
\item On dit que $u$ \uline{tend} (ou \uline{diverge}) vers $+\infty$ si $\forall A \in \mathbb{R}$, $\exists N \in \mathbb{N}: \forall n \geq N$, $u_n \geq A$ \\
Dans ce cas, on note $u_n \xrightarrow[n \to +\infty]{} +\infty$ ou $u \to +\infty$ ou $\lim\limits_{n \to +\infty} u_n = +\infty$
\item On dit que $u$ \uline{tend} (ou \uline{diverge}) vers $-\infty$ si $\forall A \in \mathbb{R}$, $\exists N \in \mathbb{N}: \forall n \geq N$, $u_n \leq A$
\end{itemize}
\end{definition}
\begin{definition}
La \uline{droite numérique achevée} est l'ensemble $\overline{\mathbb{R}} = \mathbb{R} \cup \left\{ -\infty; +\infty \right\}$
\end{definition}
\begin{proposition}[Unicité de la limite dans $\overline{\mathbb{R}}$]
Soit $n \in \mathbb{R}^\mathbb{N}$ et $l, l' \in \overline{\mathbb{R}}$ tels que $u_n \xrightarrow[n \to +\infty]{} l$ et $u_n \xrightarrow[n \to +\infty]{} l'$ \\
Alors $l = l'$
\end{proposition}

\section{Théorèmes de convergence}
\subsection{Opérations}
On munit $\overline{\mathbb{R}}$ d'une addition et d'une multiplication "partielles", càd qu'elles ne sont pas définies pour tous les couples d'éléments de $\overline{\mathbb{R}}$
\begin{center}
\begin{tabular}{ M{5em} | M{5em} | M{5em} | M{5em} }
$+$ & $-\infty$ & $b \in \mathbb{R}$ & $+\infty$ \\
\hline
$-\infty$ & $-\infty$ & $-\infty$ & X\\
\hline
$a \in \mathbb{R}$ & $-\infty$ & $a + b$ & $+\infty$ \\
\hline
$+\infty$ & X & $+\infty$ & $+\infty$
\end{tabular}
\end{center}
\begin{center}
\begin{tabular}{ M{5em} | M{5em} | M{5em} | M{5em} | M{5em} | M{5em}}
$\times$ & $-\infty$ & $b \in \mathbb{R}_-^*$ & $0$ & $b \in \mathbb{R}_+^*$ & $+\infty$ \\
\hline
$-\infty$ & $+\infty$ & $+\infty$ & X & $-\infty$ & $-\infty$ \\
\hline
$a \in \mathbb{R}_-^*$ & $+\infty$ & $a b$ & $0$ & $a b$ & $-\infty$ \\
\hline
$0$ & X & $0$ & $0$ & $0$ & X \\
\hline
$a \in \mathbb{R}_+^*$ & $-\infty$ & $a b$ & $0$ & $a b$ & $+\infty$ \\
\hline
$+\infty$ & $-\infty$ & $-\infty$ & X & $+\infty$ & $+\infty$
\end{tabular}
\end{center}
\begin{theorem}
Soit $u, v \in \mathbb{R}^\mathbb{N}$ telles que $\begin{cases} u_n \to l_1 \in \overline{\mathbb{R}} \\ v_n \to l_2 \in \overline{\mathbb{R}} \end{cases}$ et $\lambda \in \mathbb{R}$
\begin{itemize}
\item On a $|u_n| \xrightarrow[n \to +\infty]{} |l_1|$
\item Si $\lambda \in \mathbb{R}^*$, $\lambda u_n \xrightarrow[n \to +\infty]{} \lambda l_1$
\item Si $l_1 + l_2$ est bien définie, $u_n + v_n \xrightarrow[n \to +\infty]{} l_1 + l_2$
\item Si $l_1 l_2$ est bien définie, $u_n v_n \xrightarrow[n \to +\infty]{} l_1 l_2$
\end{itemize}
\end{theorem}
\begin{lemme}
Soit $u, v \in \mathbb{R}^\mathbb{N}$ telles que $\begin{cases}
u \text{ bornée} \\ v_n \xrightarrow[n \to +\infty]{} +\infty \end{cases}$ \\
Alors $u_n + v_n \xrightarrow[n \to +\infty]{} +\infty$
\end{lemme}
\begin{lemme}
Soit $u, v \in \mathbb{R}^\mathbb{N}$ telles que $\begin{cases}
u \text{ bornée} \\ v_n \xrightarrow[x \to +\infty]{} 0 \end{cases}$ \\
Alors $u_n v_n \xrightarrow[n \to +\infty]{} 0$
\end{lemme}
\begin{theorem}
Soit $u \in \mathbb{R}^\mathbb{N}$ qui ne s'annule pas.
\begin{itemize}
\item Si $u_n \xrightarrow[n \to +\infty]{} l \in \mathbb{R}^*$, alors $\frac{1}{u_n} \xrightarrow[n \to +\infty]{} \frac{1}{l}$
\item Si $u_n \xrightarrow[n \to +\infty]{} \pm\infty$, alors $\frac{1}{u_n} \xrightarrow[n \to +\infty]{} 0$
\item Si $u_n \xrightarrow[n \to +\infty]{} 0$ et que $\forall n \in \mathbb{N}$, $u_n > 0$, alors $\frac{1}{u_n} \xrightarrow[n \to +\infty]{} +\infty$
\end{itemize}
\end{theorem}

\subsection{Théorème de la limite monotone}
\begin{theorem}
Soit $u \in \mathbb{R}^\mathbb{N}$ croissante.
\begin{itemize}
\item Si $u$ est majorée, elle converge: on peut trouver $l \in \mathbb{R}$ tel que $\begin{cases}\forall n \in \mathbb{N}, \, u_n \leq l \\ u_n \xrightarrow[n \to +\infty]{} l \end{cases}$
\item Si $u$ n'est pas majorée, alors $u_n \xrightarrow[n \to +\infty]{} +\infty$
\end{itemize}
Soit $u \in \mathbb{R}^\mathbb{N}$ décroissante.
\begin{itemize}
\item Si $u$ est minorée, il existe $l \in \mathbb{R}$ tel que $\begin{cases} \forall n \in \mathbb{N}, \, u_n \geq l \\ u_n \xrightarrow[n \to +\infty]{} l \end{cases}$
\item Si $u$ n'est pas minorée, alors $u_n \xrightarrow[n \to +\infty]{} -\infty$
\end{itemize}
\end{theorem}
\begin{definition}[Suites adjacentes]
Soit $u, v \in \mathbb{R}^\mathbb{N}$. On dit que $u$ et $v$ sont \uline{adjacents} si:
\begin{itemize}
\item $u$ croît et $v$ décroît.
\item $v_n - u_n \xrightarrow[n \to +\infty]{} 0$
\end{itemize}
\end{definition}
\begin{theorem}[des suites adjacentes]
Soit $u, v \in \mathbb{R}^\mathbb{N}$ deux suites adjacentes. \\
Alors $u$ et $v$ convergent (et leurs limites sont égales).
\end{theorem}
\begin{corollaire}[Théorème des segments emboîtés]
Soit $\left([a_n, b_n]\right)_{n \in \mathbb{N}}$ une suite de segments (non vides) \\
emboîtés, càd telle que $\forall n \in \mathbb{N}$, $[a_{n + 1}, b_{n + 1}] \subseteq [a_n, b_n]$. On suppose en outre $b_n - a_n \xrightarrow[n \to +\infty]{} 0$ \\
Alors $\bigcap\limits_{n \in \mathbb{N}} [a_n, b_n]$ est un singleton.
\end{corollaire}

\subsection{Théorèmes de minoration, de majoration, d'encadrement}
\begin{theorem}[d'encadrement / des gendarmes]
Soit $u, v, w \in \mathbb{R}^\mathbb{N}$, $l \in \mathbb{R}$ telles que \[\begin{cases} \forall n \in \mathbb{N}, \, u_n \leq v_n \leq w_n \\ u_n \xrightarrow[n \to +\infty]{} l, \, w_n \xrightarrow[n \to +\infty]{} l\end{cases}\]
Alors $v_n \xrightarrow[n \to +\infty]{} l$
\end{theorem}
\begin{corollaire}
Soit $u, h \in \mathbb{R}^\mathbb{N}$, $l \in \mathbb{R}$ telles que:
\[\begin{cases}
\forall n \in \mathbb{N},\, |u_n - l| \leq h_n \\
h_n \xrightarrow[n \to +\infty]{} 0
\end{cases}\]
Alors $u_n \xrightarrow[n \to +\infty]{} l$
\end{corollaire}
\begin{theorem}[de minoration]
Soit $u, v \in \mathbb{R}^\mathbb{N}$ telles que $\forall n \in \mathbb{N}$, $u_n \leq v_n$ et $u_n \xrightarrow[n \to +\infty]{} +\infty$ \\
Alors $v_n \xrightarrow[n \to +\infty]{} +\infty$
\end{theorem}
\begin{theorem}[de majoration]
Soit $u, v \in \mathbb{R}^\mathbb{N}$ telles que $\forall n \in \mathbb{N}$, $u_n \leq v_n$ et $v_n \xrightarrow[n \to +\infty]{} -\infty$ \\
Alors $u_n \xrightarrow[n \to +\infty]{} -\infty$
\end{theorem}

\subsection{Théorème de Cesàro}
\begin{theorem}
Soit $u \in \mathbb{R}^\mathbb{N}$ et $l \in \mathbb{R}$ tels que $u_n \xrightarrow[n \to +\infty]{} l$ \\
Soit $(c_n)_{n \in \mathbb{N}} = \left( \frac{1}{n + 1} \sum\limits_{k = 0}^n u_k \right)_{n \in \mathbb{N}}$ Alors $c_n \xrightarrow[n \to +\infty]{} l$
\end{theorem}
\begin{corollaire}[Lemme de l'escalier]
Soit $u \in \mathbb{R}^\mathbb{N}$ telle que $u_{n + 1} - u_n \xrightarrow[n \to +\infty]{} l \in \mathbb{R}$ \\
Alors $\frac{u_n}{n} \xrightarrow[n \to +\infty]{} l$
\end{corollaire}

\section{Suites extraites}
\subsection{Définitions et premières propriétés}
\begin{definition}
\hfill
\begin{itemize}
\item Une \uline{extractrice} est une fonction $\varphi: \mathbb{N} \to \mathbb{N}$ strictement croissante.
\item Soit $u \in \mathbb{R}^\mathbb{N}$. Une suite extrictrice (ou une sous-suite) de $u$ est une suite de la forme $\left(u_{\varphi(k)}\right)_{k \in \mathbb{N}}$
\end{itemize}
\end{definition}
\begin{proposition}
Soit $u \in \mathbb{R}^\mathbb{N}$ et $l \in \overline{\mathbb{R}}$ tels que $u_n \xrightarrow[n \to +\infty]{} l$. Soit $\varphi$ une extractrice. \\
Alors $u_{\varphi(k)} \xrightarrow[k \to +\infty]{} l$
\end{proposition}
\begin{lemme}
$\forall k \in \mathbb{N}$, $\varphi(k) \geq k$
\end{lemme}
\begin{proposition}
Soit $u \in \mathbb{R}^\mathbb{N}$ et $l \in \overline{\mathbb{R}}$ tels que $u_{2k} \xrightarrow[k \to +\infty]{} l$ et $u_{2k + 1} \xrightarrow[k \to +\infty]{} l$ \\
Alors $u_n \xrightarrow[n \to +\infty]{} l$
\end{proposition}

\subsection{Construction de sous-suites particulières}
\begin{proposition}
Soit $u \in \mathbb{R}^\mathbb{N}$ une suite non majorée. \\
Alors il existe une extractrice $\varphi$ telle que $u_{\varphi(k)} \xrightarrow[k \to +\infty]{} +\infty$
\end{proposition}
\begin{proposition}
Soit $u \in \mathbb{R}^\mathbb{N}$, $\left(\varepsilon_k\right)_{k \in \mathbb{N}} \in \left(\mathbb{R}_+^*\right)^\mathbb{N}$ et $l \in \mathbb{R}$ tels que $u_n \xrightarrow[n \to +\infty]{} l$ \\
Alors il existe une extractrice $\varphi$ telle que $\forall k \in \mathbb{N}$, $\left| u_{\varphi(k)} - l \right| \leq \varepsilon_k$
\end{proposition}

\subsection{Théorème de Bolzano-Weierstrass}
\begin{theorem}[Bolzano-Weierstrass]
Toute suite réelle bornée admet une sous-suite convergente.
\end{theorem}

\subsection{Valeurs d'adhérence}
\begin{definition}
Soit $u \in \mathbb{R}^\mathbb{N}$ \\
Un réel $l \in \mathbb{R}$ est une \uline{valeur d'adhérence} de $u$ s'il existe une extractrice $\varphi$ telle que $u_{\varphi(k)} \xrightarrow[k \to +\infty]{} l$
\end{definition}
\begin{proposition}
Une suite bornée possédant une unique valeur d'adhérence converge.
\end{proposition}
\begin{definition}
Soit $u \in \mathbb{R}^\mathbb{N}$ bornée. On définit:
\begin{itemize}
\item La \uline{limite supérieure} de $u$, notée $\lim \sup(u)$ (ou $\lim\limits_{n \to +\infty} \sup u_n$ ou $\overline{\lim\limits_{n \to +\infty}} u_n$)
\[\lim \sup(u) = \lim\limits_{n \to +\infty} \sup\limits_{p \geq n} u_p = \sup \{u_p \mid p \geq n\} \quad (= \inf\limits_{n \in \mathbb{N}} \sup\limits_{p \geq n} u_n)\]
\item La \uline{limite inférieure} de $u$, notée $\lim \inf(u)$ (ou $\lim\limits_{n \to +\infty} \inf u_n$ ou $\underline{\lim\limits_{n \to +\infty}} u_n$)
\[\lim \inf(u) = \lim\limits_{n \to +\infty} \inf\limits_{p \geq n} u_p = \inf \{u_p \mid p \geq n\} \quad (= \sup\limits_{n \in \mathbb{N}} \inf\limits_{p \geq n} u_p)\]
\end{itemize}
\end{definition}
\begin{proposition}
Soit $u \in \mathbb{R}^\mathbb{N}$ bornée.
\begin{itemize}
\item $\lim \sup(u)$ est la plus grande valeur d'adhérence de $u$
\item $\lim \inf(u)$ est la plus petite valeur d'adhérence de $u$
\item La suite $u$ converge si et seulement si $\lim \sup(u) = \lim \inf(u)$
\end{itemize}
\end{proposition}

\section{Caractérisation séquentielle}
\subsection{Caractérisation séquentielle de l'adhérence}
\begin{proposition}
Soit $A \subseteq \mathbb{R}$ et $x \in \mathbb{R}$ \\
Alors $x \in \overline{A}$ si et seulement s'il existe une suite $(a_n)_{n \in \mathbb{N}} \in A^\mathbb{N}$ telle que $a_n \xrightarrow[n \to +\infty]{} x$
\end{proposition}
\begin{corollaire}
Soit $A \subseteq \mathbb{R}$ non vide et majorée et $S \in \mathbb{R}$ \\
Alors $S = \sup(A)$ ssi $S$ majore $A$ et il existe $(a_n)_{n \in \mathbb{N}} \in A^\mathbb{N}$ telle que $a_n \xrightarrow[n \to +\infty]{} S$
\end{corollaire}

\subsection{Caractérisation séquentielle de la densité}
On sait déjà qu'une partie $A \subseteq \mathbb{R}$ est \uline{dense} ssi $\overline{A} = \mathbb{R}$. On obtient la caractérisation suivante:
\begin{proposition}
Soit $A \subseteq \mathbb{R}$. \\
$A$ est dense dans $\mathbb{R}$ ssi $\forall x \in \mathbb{R}$, $\exists (a_n)_{n \in \mathbb{N}} \in A^\mathbb{N}: a_n \xrightarrow[n \to +\infty]{} x$
\end{proposition}

\section{Extension aux suites complexes}
\subsection{Généralités}
\begin{definition}
Soit $u \in \mathbb{C}^\mathbb{N}$ et $l \in \mathbb{C}$ \\
On dit que $u$ \uline{converge vers $l$} si $\forall \varepsilon > 0$, $\exists N \in \mathbb{N}: \forall n \geq N$, $\left| u_n - l \right| \leq \varepsilon$
\end{definition}
\begin{proposition}
Soit $u \in \mathbb{C}^\mathbb{N}$ et $l \in \mathbb{C}$ \\
On a $u_n \xrightarrow[n \to +\infty]{} l \iff \left| u_n - l \right| \xrightarrow[n \to +\infty]{} 0$
\end{proposition}
\begin{definition}
Une suite $u \in \mathbb{C}^\mathbb{N}$ est \uline{bornée} si $\forall C \in \mathbb{R}_+: \forall n \in \mathbb{N}$, $|u_n| \leq C$
\end{definition}
\begin{proposition}
Toute suite complexe convergente est bornée.
\end{proposition}
\begin{definition}
Soit $u \in \mathbb{C}^\mathbb{N}$ \\
On dit que \uline{$u$ tend vers l'infini} si $|u_n| \xrightarrow[n \to +\infty]{} +\infty$
\end{definition}

\subsection{Lien avec le cas réel}
\begin{theorem}
Soit $u \in \mathbb{C}^\mathbb{N}$ et $l \in \mathbb{C}$ \\
Alors
\[ u_n \xrightarrow[n \to +\infty]{} l \iff \begin{cases}
\re u_n \xrightarrow[n \to +\infty]{} \re l \\
\im u_n \xrightarrow[n \to +\infty]{} \im l
\end{cases}\]
\end{theorem}
\begin{theorem}[Bolzano-Weierstrass, cas complexe]
\hfill \\
Toute suite complexe bornée possède une sous-suite convergente.
\end{theorem}

\subsection{Suite géométrique}
\begin{theorem}
Soit $a \in \mathbb{C}$ \\
Alors $(a^n)_{n \in \mathbb{N}}$ converge ssi $|a| < 1$ (auquel cas $a^n \xrightarrow[n \to +\infty]{} 0$) ou $a = 1$ (auquel cas $a^n \xrightarrow[n \to +\infty]{} 1$) \\
et $(a^n)_{n \in \mathbb{N}}$ tend vers l'infini ssi $|a| > 1$
\end{theorem}

\pagebreak

\section{Suites récurrentes}
On étudie des suites récurrentes (d'ordre $1$), càd des suites $u \in \mathbb{R}^\mathbb{N}$ vérifiant $\forall n \in \mathbb{N}$, $u_{n + 1} = f(u_n)$, où $f: I \to \mathbb{R}$ est une certaine fonction, que l'on appelle \uline{itératrice}

\subsection{Itératrice croissante: un exemple}
\noindent Étudions la suite $(u_n)_{n \in \mathbb{N}}$ définie par $\begin{cases}
u_n = 3 \\ \forall n \in \mathbb{N},\, u_{n + 1} = \sqrt{2 + u_n}
\end{cases}$ \\
On observe que $\left[-2, +\infty\right]$ est un intervalle stable donc la suite est bien définie. \medskip

1) La suite décroit.\\
Pour tout $n \in \mathbb{N}$ notons $P(n): "n_{n + 1} \leq u_n"$ \\
\uline{Initialisation}: On a $u_1 = f(3) \leq 3 = u_0$ \\
\uline{Hérédité}: Soit $n \in \mathbb{N}$ tel que $P(n)$ \\
On a donc $u_{n + 1} \leq u_n$ \\
Par croissance de $f$, $f(u_{n + 1} \leq f(u_n)$, càd $u_{n + 2} \leq u_{n + 1}$, d'où $P(n + 1)$, ce qui clôt la récurrence. $\qed$ \medskip

2) $u$ est minorée. \\
En effet, $\left[-2, +\infty\right[$ est stable donc $\forall n \in \mathbb{N}$, $u_n \in [-2, +\infty[$ (récurrence immédiate) \\
D'après le théorème de la limite monotone, on peut don trouver $l \in \mathbb{R}$ tel que $u_n \xrightarrow[ n \to +\infty]{} l$ \\
Par passage à la limite dans les inégalités larges, $l \geq 2$ (càd $l \in [-2, +\infty[$) \medskip

3) Montrons que $l$ est nécessairement un point fixe de $f$, càd $f(l) = l$. \\
On a $(u_{n + 1})_{n \in \mathbb{N}} = (f(u_n))_{n \in \mathbb{N}}$ \\
Par extraction, $u_{n + 1} \xrightarrow[n \to +\infty]{} l$ \\
Par continuité de $f$, $f(u_n) \xrightarrow[n \to +\infty]{} f(l)$ \\
Par unicité de la limite, $f(l) = l$\\
Puisque $2$ est le seul point fixe de $f$, on a $u_n \xrightarrow[n \to +\infty]{} 2$ \medskip

\noindent \uline{Variante}: Cas de la suite $(v_n)_{n \in \mathbb{N}}$ définie par $\begin{cases}
v_0 = 1 \\ \forall n \in \mathbb{N},\, v_{n + 1} = f(v_n)
\end{cases}$ \\
Comme dans (1): \medskip

1)' La suite $v$ est monotone car $f$ est croissante. \\
Comme $v_1 \geq v_0$, cette fois \uline{$v$ croît}. \medskip

2)' On doit trouver un intervalle stable plus petit. \\
Ici, $[-2, 2]$ est stable et contient $v_0$ donc $v$ est à valeurs dans $[-2, 2]$, donc majorée. \medskip

3)' La suite converge nécessairement vers un point fixe. \\
$v_n \xrightarrow[n \to +\infty]{} 2$

\pagebreak

\subsection{Itératrice décroissante: un exemple}
\noindent Étudions la suite $u$ définie par $\begin{cases}
u_0 = 1 \\ \forall n \in \mathbb{N},\, u_{n + 1} = \frac{1}{2 + u_n}
\end{cases}$ \\
Posons $f: \begin{cases}
\mathbb{R} \setminus \{ 2 \} \to \mathbb{R} \\ x \mapsto \frac{1}{2 + x}
\end{cases}$ \\
Le segment $[0, 1]$ est stable et inclus dans le domaine de $f$. On en déduit que la suite $u$ est bien définie et à valeurs dans $[0, 1]$ \medskip

1) Les sous-suites $(u_{2k})_{k \in \mathbb{N}}$ et $(u_{2k + 1})_{k \in \mathbb{N}}$ sont monotones, de monotonies opposées. \\
En effet, $\forall k \in \mathbb{N}; \begin{cases}
u_{2(k + 1)} = (f \circ f)(u_{2k}) \\ u_{2(k + 1) + 1} = (f \circ f)(u_{2k + 1})
\end{cases}$ \\
Comme $f \circ f$ croît (par opération), les sous-suites sont monotones. \\
Comme $u_0 = 1, u_1 = \frac{1}{3}$ et $u_2 = \frac{3}{7}$, on a $u_2 \leq u_0$, donc $(u_{2k})_{k \in \mathbb{N}}$ décroît d'où $u_3 \geq u_1$ (en appliquant $f$), ce que donne $(u_{2k + 1})_{k \in \mathbb{N}}$ croît.\medskip

2)Les deux sous-suites sont à valeurs dans $[0, 1]$ donc bornées. \\
D'après le théorème de la limite monotone \\
$u_{2k} \xrightarrow[k \to +\infty]{} l_0 \in [0, 1]$ \\
$u_{2k + 1} \xrightarrow[k \to +\infty]{} l_1 \in [0, 1]$ \medskip

3)On a $\underbrace{u_{2k + 1}}_{\xrightarrow[k \to +\infty]{} l_1} = \underbrace{f(u_{2k})}_{\xrightarrow[k \to +\infty]{} f(l_0)}$ (par continuité de $f$)\\
Donc $f(l_0) = l_1$ par unicité de la limité. \\
$\underbrace{u_{2k + 2}}_{\xrightarrow[k \to +\infty]{} l_0} = \underbrace{f(u_{2k+1})}_{\xrightarrow[k \to +\infty]{} f(l_1)}$ \\
Donc $f(l_1) = l_0$ par unicité de la limite. \\
On obtient alors que $\begin{cases}
f(f(l_0)) = l_0 \\ f(f(l_1)) = l_1
\end{cases}$ \\
\uline{Déterminons les points fixes de $f$}: \\
Soit $x \in \mathbb{R} \setminus \{ 2 \}$. On a:
\begin{align*}
f(x) = x &\iff \frac{1}{2 + x} = x \\
&\iff 1 = 2x + x^2 \\
&\iff (x + 1)^2 = 2 \\
&\iff x = -1 \pm \sqrt{2}
\end{align*} \\
\uline{Déterminons les points fixes de $f \circ f$}: \\
Soit $x \in \mathbb{R}$ tel que $f(f(x))$ sont bien définit.
\begin{align*}
f(f(x)) = x &\iff \frac{1}{2 + \frac{1}{2 + x}} = x \\
&\iff \frac{x + 2}{2x + 5} = x \\
&\iff 2x^2 + 4x - 1 = 0 \\
&\iff x^2 + 2x - 1 = 0 \\
&\iff x = -1 \pm \sqrt{2}
\end{align*} \\
Comme $l_0, l_1 \in [0, 1]$, on a $l_0 = l_1 = \sqrt{2} - 1$ \\
Donc $\begin{cases}
u_{2k} \xrightarrow[k \to +\infty]{} \sqrt{2} - 1 \\ u_{2k + 1} \xrightarrow[k \to +\infty]{} \sqrt{2} - 1\end{cases}$\\
Ainsi, $u_n \xrightarrow[n \to +\infty]{} \sqrt{2} - 1$

\subsection{Résumé des résultats utiles}
\noindent Soit $u \in \mathbb{R}^\mathbb{N}$ telle que $\forall n \in \mathbb{N}$, $u_{n + 1} = f(u_n)$ où $f: I \to \mathbb{R}$
\begin{itemize}
\item Si $S \subseteq I$ est stable et $u_0 \in S$, alors $u$ (est bien définie et) à valeurs dans $S$.
\item Si l'itératrice $f$ est croissante, $u$ est monotone.
\item Si l'itératrice $f$ est décroissante, $(u_{2k})_{k \in \mathbb{N}}$ et $(u_{2k + 1})_{k \in \mathbb{N}}$ sont monotones, de monotonies opposées.
\item Si $u$ converge vers $l$ et que $f$ est continue (en $l$), alors $f(l) = l$.
\end{itemize}
\end{document}